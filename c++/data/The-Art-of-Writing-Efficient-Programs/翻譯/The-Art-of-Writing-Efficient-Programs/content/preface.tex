\begin{flushright}
\zihao{0} 前言
\end{flushright}

高性能编程的艺术仍在归来途中。我还是编程菜鸟时,那时候的开发者必须知道每一个数据位的去向(有时的确要这样——用面板上的开关)。现在,计算机已经有能力完成日常工作,就没必要在意有些细节。不过,还是有一些领域没有足够的计算能力。其实,大多数开发者都可以写出高效的代码,这不是一件坏事。在不受性能限制的前提下,这样开发者就可以更专注于如何使代码变的更好。

本书首先阐明了,为什么越来越多的开发者需要关注性能和效率。这将为整本书定下基调,因为其定义了我们在后续章节中所使用的方法论。关于性能的知识最终必须来自于测试,并且每个与性能相关的声明都必须有数据支撑。

其中,有五个要素决定了程序的性能。

首先,我们深入研究所有性能的底层基础——用于计算的硬件(没有交换机——那些日子已经一去不复返了)。从单个部件——处理器和内存——到多处理器计算系统。在此过程中,我们会了解了内存模型、数据共享的成本,还有无锁编程。

高性能编程的第二个要素是对编程语言的使用,这一点正是本书针对C++(其他语言也有低效率)进行的说明。而后的是第三个要素,即是帮助编译器提高程序性能的技能。

第四个要素是设计。如果设计没有将性能作为其明确目标,那么几乎不可能在之后再为程序添加良好的性能。然而,因为这是一个高级概念,它汇集了之前所学到的所有知识,所以我们最后再来了解性能设计。

高性能编程的最后,也是第五个要素就是——正在阅读本书的你,你的知识和技能水平将最终决定结果。为了帮助读者进行学习,这本书提供了许多例子,可以用于动手探索和自学。学习是一项终身的运动,切勿在看完本书后停止学习。

\hspace*{\fill} \\ %插入空行
\noindent\textbf{适读人群}

这本书是为有经验的开发人员编写,从事对性能至关重要的项目,并希望学习不同的技术来提高代码的性能。算法交易、游戏、生物信息学、基因组学或流体动力学社区的开发者,都可以从这本书中学习各种技术,并将其应用到他们的工作领域中。

虽然本书使用的是C++,但本书的概念可以转移或应用到其他编译语言,如C、Java、Rust、Go等。

\hspace*{\fill} \\ %插入空行
\textbf{本书内容}

\textit{第1章,性能和并发性}。讨论了重视程序性能的原因,特别是性能好的应用为什么不会凭空出现。为了实现最佳性能,甚至是提升性能,理解影响性能的不同因素和程序特定行为的原因(无论是快执行还是慢执行)。

\textit{第2章,性能测试}。本章内容都是关于性能测试的。性能有时并不直观,所有涉及效率的决策,从设计选择到具体优化策略,都应该以可靠的数据为指导。本章描述了不同类型的性能评估方式,解释了它们的不同之处,以及应该在什么时候使用哪种方式,并了解如何在不同的情况下正确地评估性能。

\textit{第3章,CPU架构、资源和性能}。研究硬件,以及如何高效地使用硬件资源,从而达到最佳性能。本章将了解CPU资源和能力,以及使用方法。了解CPU资源没有得到充分利用的原因,以及如何解决这些问题。

\textit{第4章,内存架构与性能}。了解现代内存架构,以及其缺点,以及避免或规避这些缺点的方法。对于许多程序来说,性能完全依赖于开发者是否可以利用硬件特性来提高内存性能。

\textit{第5章,线程、内存和并发}。继续学习内存系统,及其对性能的影响,但现在会扩展到多核系统和多线程领域。事实证明,内存已经是性能的“关键”,当添加了并发时,内存管理可能会很棘手。虽然硬件带来的物理限制无法克服,但大多数程序的性能其实还未触及这些限制,而且对于资深的开发者来说,代码效率还有很大的提高空间。所以,本章为读者们提供了必要的知识和工具。

\textit{第6章,并发性和性能}。了解如何为线程安全的程序开发高性能并发算法和数据结构。一方面,为了充分利用并发性,必须从更高的角度看待问题和解决方案策略:数据组织、工作分区,甚至是解决方案的定义都会对程序性能产生重大影响。另一方面,性能受到底层的极大影响,比如:缓存中数据的分布,即使是最好的设计也可能因糟糕的实现,达不到预期的性能。

\textit{第7章,用于并发的数据结构}。解释并发程序中数据结构的本质,以及在多线程上下文中使用数据结构,比如:定义常用的数据结构(如“堆栈”和“队列”等)。

\textit{第8章,C++中的并发}。介绍了C++17和C++20标准中最近添加的并发特性。虽然,现在谈论使用这些特性获得最佳性能的方式还为时过早,但我们可以描述其作用,以及编译器当前的支持情况。

\textit{第9章,高性能C++}。重点从硬件资源的使用,转移到特定编程语言的应用。虽然,我们学到的所有东西都可以直接应用到任何语言中,但本章将讨论C++的特性。读者将了解C++语言的哪些特性可能会导致性能问题,以及如何规避这些问题。

\textit{第10章,C++中的编译器优化}。本章还会讨论编译器优化的问题,以及开发者如何帮助编译器生成更高效的代码。

\textit{第11章,未定义的行为和性能}。这里有两个重点:一方面,解释了开发者在试图最大化代码性能时,经常忽略的未定义行为的危险性。另一方面,解释了如何利用未定义的行为来提高性能,以及如何正确地控制和记录这些情况。总的来说,与“任何事情都可能发生”相比,这一章提供了一些常见方式来理解未定义行为的问题。

\textit{第12章,为性能而设计}。回顾本书中所学到的所有与性能相关的因素和特性,并总结所获得的知识,了解如何在开发新软件系统或重新架构现有软件系统时,如何做出的设计决策。

\hspace*{\fill} \\ %插入空行
\textbf{编译环境}

除了特定于C++效率的章节,不依赖于任何C++知识。所有的例子都是用C++编写(但是有关硬件性能、高效数据结构和性能设计的部分适用于任何编程语言)。要看懂这些示例,至少需要具备中等程度的C++知识。

\begin{table}[H]
	\begin{tabular}{|l|l|}
		\hline
		C++ compiler(GCC, Clang, Visual Studio等)                                                                                                                  & 操作系统                                                             \\ \hline
		LLVM版本高于或等于12.x                                                                                                                  & \begin{tabular}[c]{@{}l@{}}Windows, macOS或Linux\end{tabular}                                                             \\ \hline
		\begin{tabular}[c]{@{}l@{}} Profiler(VTune, Perf, GoogleProf等)\end{tabular} &  \\ \hline
		Benchmark Library(GoogleBench)                                                                                                                                  &                                                                                  \\ \hline
	\end{tabular}
\end{table}

每一章都会提到编译和执行示例所需的软件(如果有的话)。大多数情况下,现代C++编译器都可以与示例一起使用,除了第8章(C++中的并发),该章要求支持C++20,从而可以展示\textbf{协程}的相关示例。

\textbf{如果你正在使用这本书的数字版本,我们建议自己输入代码或通过GitHub库访问代码(链接在下一节中提供)。这样做将帮助您避免与复制和粘贴代码相关的任何潜在错误}

\hspace*{\fill} \\ %插入空行
\textbf{下载示例}

可以从GitHub网站\url{https://	github.com/PacktPublishing/The-Art-of-Writing-Efficient-Programs}下载本书的示例代码文件。如果代码有更新,会在现有的GitHub库中更新。

我们还有其他的代码包,还有丰富的书籍和视频目录,都在\url{https://github.com/PacktPublishing/}。去看看吧!

\hspace*{\fill} \\ %插入空行
\textbf{联系方式}

我们欢迎读者的反馈。

\textbf{反馈}:如果你对这本书的任何方面有疑问,需要在你的信息的主题中提到书名,并给我们发邮件到\url{customercare@packtpub.com}。

\textbf{勘误}:尽管我们谨慎地确保内容的准确性,但错误还是会发生。如果您在本书中发现了错误,请向我们报告,我们将不胜感激。请访问\url{www.packtpub.com/support/errata},选择相应书籍,点击勘误表提交表单链接,并输入详细信息。

\textbf{盗版}:如果您在互联网上发现任何形式的非法拷贝,非常感谢您提供地址或网站名称。请通过\url{copyright@packt.com}与我们联系,并提供材料链接。

\textbf{如果对成为书籍作者感兴趣}:如果你对某主题有专长,又想写一本书或为之撰稿,请访问\url{authors.packtpub.com}。

\hspace*{\fill} \\ %插入空行
\textbf{欢迎评论}

请留下评论。当您阅读并使用了本书,为什么不在购买网站上留下评论呢?其他读者可以看到您的评论,并根据您的意见来做出购买决定。我们在Packt可以了解您对我们产品的看法,作者也可以看到您对他们撰写书籍的反馈。谢谢你!

想要了解Packt的更多信息,请访问\url{packt.com}。










