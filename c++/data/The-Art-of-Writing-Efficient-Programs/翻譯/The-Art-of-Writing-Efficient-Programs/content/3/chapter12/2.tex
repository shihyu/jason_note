优秀的设计是否有助于实现良好的性能,或者确定是否需要牺牲最佳设计实践来实现最佳性能?这些问题在编程界已经争论了很久。通常,设计倡导者会认为,若需要在良好的设计和良好的性能之间做选择,那么设计者的能力太水。另一方面,黑客(我们使用这个术语的传统意义,将解决方案组合在一起的开发者,与犯罪方面无关)通常将设计指南视为最佳优化的约束条件。 

这两种观点在一定程度上都是正确的。如果把它们视为“全部的真相”,也不对。许多设计实践在应用于特定软件系统时,会限制性能,否认这一点就很蠢。另外,许多实现和维护有效代码的指南也是可靠的设计建议,可以提高性能和设计质量。 

我们对设计和性能之间的关系采取了更微妙的看法。对于特定的系统(开发者最感兴趣的是还是正在开发的系统),一些设计指南和实践确实会导致效率和性能低下。很难想出一个总是与效率相对立的设计规则,但是对于特定的系统,也许在某些特定的环境中,这样的规则和实践就很常见。若采用了遵循这些规则的设计,那么可能会将效率低下的设计嵌入到软件系统的核心架构中,这将很难通过“优化”进行弥补,很可能就是要重写程序的关键部分。任何忽视或粉饰这个陷阱的潜在严重性的开发者,都无法获得最佳性能。另外,声称这是放弃可靠设计实践的理由的人都是错误的,他们的选择过于二元化。 

如果意识到特定的设计方法可以遵循的实践,就可以提高设计的清晰度和可维护性,虽然降低了性能,但也很好的设计方法。换句话说,虽然一些好的设计产生了糟糕的性能,但是对于一个给定的软件系统来说,每一个好的设计都会导致低效,听起来就很荒谬。所需要做的就是从几种可能的高质量设计中,选择一种具有良好性能的设计。 

当然,说起来容易做起来难,但希望这本书能有所帮助。本章的其余部分,我们将关注问题的两个方面。首先,当考虑性能时,建议采用什么设计实践?其次,当没有可以运行和测试的程序,但有一个(可能不完整的)设计时,如何评估可能的性能影响?

如果仔细阅读了最后两段,会发现性能是一种设计考虑因素,就像在设计中考虑“支持许多用户”或“在磁盘上存储TB级数据”等需求一样,性能目标也是需求的一部分,应该在设计阶段明确考虑。这将我们引向设计高性能系统,它是……


































































