本书的最后一章回顾了我们所了解到的关于性能和决定性能的因素的所有知识,然后使用这些知识来提出高性能软件系统的设计指南。提供了一些关于设计接口、数据组织、组件和模块的建议,并描述了在有一个性能可以测试的实现之前,如何根据良好的测量结果做出设计决策的方法。

必须再次强调,为性能而设计不会自动产生良好的性能,但允许实现具有高性能。另一种选择是一种对性能不利的设计,并固化决策,限制并阻止有效的代码和数据结构。

这本书是一个旅程:从学习单个硬件组件的性能开始,然后研究它们之间的相互作用,以及它们如何影响对编程语言的使用。这条路最终将我们引向了性能设计的理念。这是本书的最后一章,但不是各位读者旅程的最后一程:现在可以将知识应用于有待解决的实际问题,想想就令人兴奋。