本章有两个重点。一方面,解释了开发者在试图最大化代码性能时经常忽略的未定义行为的危险。另一方面,解释了如何利用未定义行为来提高性能,以及如何正确地指定和记录这种情况。与通常的“都可能发生”相比,这一章提供了一种有点不同寻常,但更相关的方式来理解未定义行为的问题。

本章将讨论以下内容:

\begin{itemize}
\item 
理解未定义行为及其存在的原因

\item 
理解关于未定义行为的真相和传说

\item 
如何利用未定义行为

\item 
内存带宽和延迟

\item 
学习未定义行为和效率之间的联系,以及如何利用它
\end{itemize}

将了解在(别人的)代码中遇到未定义行为时,如何识别并了解未定义的行为如何与性能相关。本章还描述如何,通过有意地允许、记录,并在其周围设置保护措施,从而使用未定义行为。