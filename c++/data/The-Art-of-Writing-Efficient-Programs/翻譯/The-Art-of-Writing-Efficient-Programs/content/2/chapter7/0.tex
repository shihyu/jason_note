上一章中,详细地探讨了可用于确保并发程序正确性的同步。还研究了这些程序的构建块,线程安全的计数器和指针。

本章中,将继续研究并发程序的数据结构。本章的目标有两个:一方面,将了解如何设计基本数据结构的线程安全版本。另一方面,给出一些一般原则和观察,这些对于设计用于并发的数据结构非常重要,对于评估组织方式和存储数据的最佳方法也非常重要。

本章将讨论以下内容:

\begin{itemize}
\item
理解线程安全的数据结构,包括线性容器:堆栈和队列;和基于节点的容器:链表

\item
提高并发性、性能和访问顺序的保证

\item
设计线程安全数据结构的建议

\end{itemize}















