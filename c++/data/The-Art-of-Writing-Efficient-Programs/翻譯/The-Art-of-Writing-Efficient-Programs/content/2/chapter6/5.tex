本章中,我们了解了并发程序的基本构件的性能。对共享数据的所有访问都必须进行保护或同步,但是在实现同步时,有很多选择。虽然互斥锁是最常用和最简单的方法,但我们已经了解了其他几个性能更好的选项:自旋锁及其变体,以及无锁同步。

高效并发程序的关键是使尽可能多的数据位于线程本地,并尽量减少对共享数据的操作。特定于每个问题的需求通常不能完全消除此类操作,因此本章主要讨论如何提高并发数据访问的效率。

我们研究了如何跨多个线程计数或累积,尝试了使用和不使用锁的方法。通过理解数据依赖关系,发现发布协议可以使用线程安全的智能指针实现,并适用于不同的应用程序。

现在,我们已经做好了充分的准备,将本章中几个构建块以更复杂的线程安全数据结构的形式放在一起。下一章中,将了解如何使用这些技术为并发程序设计数据结构。