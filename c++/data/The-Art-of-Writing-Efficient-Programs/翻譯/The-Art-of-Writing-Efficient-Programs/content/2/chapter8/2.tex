C++11之前,C++标准没有并发。当然,开发者们早在2011年之前就用C++编写多线程和分布式程序了。使C++中使用并发的原因可能是编译器作者进行了额外的限制和保证,通常是通过遵守C++标准(对于该语言)和另一个标准(如POSIX)来支持并发。

C++11通过引入C++内存模型,内存模型描述了线程如何通过内存进行交互。自此,C++第一次有了原生并发。内存模型的没有副作用,因为新的C++内存模型与大多数编译器支持的内存模型非常相似。这些模型之间有一些细微的差别,新标准保证了这些差异的可移植性。

一些直接支持多线程的语言特性更有用。首先,标准引入了线程的概念。对于线程的行为,很少有明确的说明,但是大多数实现只是简单地使用系统线程来支持C++线程。在实现的最底层上这是没问题的,但对于简单的程序来说,这就不够的。为程序必须执行的每个独立任务创建一个新线程,就肯定会失败。启动新线程需要时间,而且很少有操作系统能够有效地处理数百万个线程。另一方面,对于开发线程调度器的开发者来说,C++线程接口没有为线程行为提供足够的控制(大多数线程属性特定于操作系统)。

接下来,标准引入了几个同步原语来控制对内存的并发访问。标准提供了\texttt{std::mutex},通常使用常规的系统互斥锁来实现。在POSIX平台上,这通常是POSIX互斥锁。标准提供了计时和递归互斥锁(紧随POSIX)。为了简化异常处理,应该避免直接锁定和解锁互斥对象,使用RAII模板\texttt{std::lock\_guard}。

为了安全锁定多个互斥对象,避免死锁的风险,标准提供了\texttt{std::lock()}(虽然保证没有死锁,但使用的算法未指定,而且特定实现的性能差别很大)。另一个常用的同步原语是一个条件变量\texttt{std::condition\_variable},以及相应的等待和信号操作。这个功能也非常接近于POSIX相应的特性。

然后,支持底层的原子操作:\texttt{std::atomic}、比较-交换之类的原子操作,以及内存序。我们已经在第5章,第6章和第7章,了解了它们的行为和应用。

最后,语言增加了对异步执行的支持,可以使用\texttt{std::async}异步调用函数(可能在另一个线程上)。虽然这可能支持并发编程,但该特性对于高性能应用几乎完全没用。大多数实现要么提供非常有限的并行性,要么在自己的线程上执行异步函数。大多数操作系统在创建和汇入线程时有相当大的开销(我所见过的唯一一个使并发编程简单到为每个任务启动一个线程的操作系统是AIX,如果需要的操作系统可能会启用数百万个线程。在其他操作系统上,这会造成混乱)。

总的来说,谈到并发性时,C++11在概念上是一个重大的进步,C++14的改进集中在其他地方,所以在并发性方面没有什么值得注意的变化。然后,再了解一下C++17带来了哪些新的特性。




































