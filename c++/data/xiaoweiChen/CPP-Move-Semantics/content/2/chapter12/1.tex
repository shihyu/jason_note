泛型代码中,经常会计算一个值,然后返回给调用者。问题是,如何完美地返回值,但仍然保留类型和值类别?换句话说:应该如何声明以下函数的返回类型:\par

\begin{lstlisting}[caption={}]
template<typename T>
??? callFoo(T&& arg)
{
	return foo(std::forward<T>(arg));
}
\end{lstlisting}

这个函数中,调用了名为\textit{foo()}的函数,形参是完美转发的\textit{arg}。不知道这种类型的\textit{foo()}返回什么;可能是临时值(prvalue)、lvalue引用或rvalue引用。返回类型可以是\textit{const}或非\textit{const}。\par

那么,如何完美地将\textit{foo()}的返回值返回给\textit{callFoo()}的调用者呢?先说有几个种不起作用的方式:\par

\begin{itemize}
	\item 返回类型auto将删除\textit{foo()}返回类型的引用。例如,如果提供对容器元素的访问权限(将\textit{foo()}视为\textit{at()}成员函数或vector的索引操作符),\textit{callFoo()}将不再提供对该元素的访问权限。此外,可能会创建不必要的副本(如果没有优化掉的话)。
	\item 任何作为引用的返回类型(auto\&, \textit{const} auto\&,和auto\&\&)将返回对局部对象的引用,如果\textit{foo()}按值返回一个临时对象。幸运的是,编译器在检测到此类bug时会发出警告。
\end{itemize}

也就是说,需要一种表示方式:\par

\begin{itemize}
	\item 如果有一个值,则按值返回
	\item 如果得到/有一个引用,则按引用返回
\end{itemize}

但仍然保留返回的类型和值类别。\par

C++14为此引入了一个新的占位符类型:decltype(auto)。\par

\begin{lstlisting}[caption={}]
template<typename T>
decltype(auto) callFoo(T&& arg) // since C++14
{
	return foo(std::forward<T>(arg));
}
\end{lstlisting}

有了这个声明,如果\textit{foo()}按值返回,\textit{callFoo()}就能按值返回;如果\textit{foo()}按引用返回,\textit{callFoo()}按引用返回,类型和值类别都可以保留。\par











































