IOStreams在C++98中引入,作为一种可以读写的抽象(标准I/O,文件,甚至字符串)。这是早期的设定,不可能复制这些对象(复制打开的文件的对象意味着什么,对相同的文件有两个句柄或复制文件,以及如何同步访问?)\par

自从C++11以来,移动语义允许我们移动IOStream对象,并使用临时流。\par

\hspace*{\fill} \par %插入空行
\textbf{15.5.1 移动IOStream对象}

考虑如下示例:\par

{\color{red}{lib/outfile.cpp}}\par

\begin{lstlisting}[caption={}]
#include <iostream>
#include <fstream>
#include <stream>

std::ofstream openToWrite(const std::string& name)
{
	std::ofstream file(name); // open a file to write to
	if (!file) {
		std::cerr << "can't open file '" << name << "'\n";
		std::exit(EXIT_FAILURE);
	}
	return file; // return ownership (open file)
}

void storeData(std::ofstream fstrm) // takes ownership of file (but this might change)
{
	fstrm << 42 << '\n';
} // closes the file

int main()
{
	auto outFile{openToWrite("iostream.tmp")}; // open file
	storeData(std::move(outFile)); // store data
	
	// better ensure that the file is closed:
	if (outFile.is_open()) {
		outFile.close();
	}
}
\end{lstlisting}

这里,函数openToWrite()打开并返回输出文件流:\par

\begin{lstlisting}[caption={}]
std::ofstream openToWrite(const std::string& name)
{
	std::ofstream file(name); // open a file to write to
	...
	return file; // return ownership (open file)
}
\end{lstlisting}

使用返回值初始化\textit{outFile},并将其传递给\textit{storeData()}:\par

\begin{lstlisting}[caption={}]
auto outFile{openToWrite("iostream.tmp")}; // open file
storeData(std::move(outFile)); // store data
\end{lstlisting}

因为\textit{storeData()}按值接受参数,所以它有打开文件的所有权。因此,在\textit{storeData()}的末尾,文件关闭:\par

\begin{lstlisting}[caption={}]
void storeData(std::ofstream fstrm) // takes ownership of file (but this might change)
{
	...
} // closes the file
\end{lstlisting}

但是,\textit{storeData()}也可以通过引用来获取参数,这意味着它不一定需要获取所有权。这种情况下,需要再次检查传递参数\textit{outFile}的状态:\par

\begin{lstlisting}[caption={}]
// better ensure that the file is closed:
if (!outFile.is_closed()) {
	outFile.close();
}
\end{lstlisting}

调用\textit{outFile.close()}通常就足够了,但如果文件流已经关闭,则会设置该文件流的\textit{failbit}。\par

\hspace*{\fill} \par %插入空行
\textbf{15.5.2 使用临时IOStreams}

自C++11开始,IOStreams库也提供了函数重载来接受rvalue引用,允许接受临时对象。例如:\par

\begin{lstlisting}[caption={}]
std::string s = "hello, world";
std::ofstream("fstream1.tmp") << s << '\n'; // OK since C++11
\end{lstlisting}

甚至可以这样向流中写一个字符串字面值(使用C++11标准之前编译,其使用operator<<(const void*)输出字符串字面值的地址):\par

\begin{lstlisting}[caption={}]
std::ofstream("fstream1.tmp") << "hello, world\n"; // correct since C++11
// (wrote address before)
\end{lstlisting}

同样,可以用临时字符串流来解析给定的字符串:\par

\begin{lstlisting}[caption={}]
std::string name, firstname, lastname;
...
name = "Tina Turner";
std::istringstream{name} >> firstname >> lastname; // OK since C++11
\end{lstlisting}

最后,可以使用std::getline()来解析临时流的第一行:\par

\begin{lstlisting}[caption={}]
std::string multiLineString, firstLine;
...
std::getline(std::stringstream{multiLineString}, // read from temporary string stream
firstLine);
\end{lstlisting}



























