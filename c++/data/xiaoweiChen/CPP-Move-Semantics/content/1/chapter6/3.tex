已移动对象可能会破坏“有效但未定义的状态”,这比破坏可销毁要容易得多。我们可以将对象带入一种状态,从而破坏它们的不变量。\par

幸运的是,只有在显式请求移动语义时才会出现这个问题,因为临时对象无论如何都会立即销毁。然而,明确的移动请求不仅仅是用\textit{std::move()}标记对象。可以用以下方法创建已移动对象:\par

\begin{itemize}
	\item 对对象使用\textit{std::move()}
	\item 移动算法(\textit{std::move()}和\textit{std::move\_backward()})
	\item 通过将“未删除”的元素移到前面来“删除”元素的算法(例如:\textit{std::remove()},\textit{std::remove\_if()},\textit{std::unique()})
	\item 移动迭代器
\end{itemize}

如果类的不变量被(生成的)移动操作破坏,有以下解决方法:\par

\begin{itemize}
	\item 修复移动操作,使已移动对象进入不变量稳定的状态。
	\item 禁用移动语义。
	\item 将定义已移动状态的不变量放宽为有效。特别是,成员函数和使用对象的函数必须以不同的方式实现,以处理新的可能状态。
	\item 提供一个成员函数来检查“不变量”的状态,这样类型的用户在用\textit{std::move()}标记后就不会使用这种类型的对象了(或者只使用一组有限制的操作)。
\end{itemize}

来看一些不变量破坏的例子,并讨论如何修复它们。\par

\hspace*{\fill} \par %插入空行
\textbf{6.3.1 破坏由移动值破坏的不变量}

移动操作破坏不变量的第一个原因,与成员的已移动状态就是一个问题,因为该状态是有效的,但不应该发生。\par

考虑纸牌游戏中的Card类。假设每个对象都是一张有效的牌,比如红桃8或方块K。还假设由于某种原因,该值是字符串,并且该类的不变式是每个对象都有一个表示有效卡片的状态。这意味着可能没有默认构造函数,而初始化构造函数会断言该值是有效的。例如:\par

\begin{lstlisting}[caption={}]
class Card {
private:
	std::string value; // rank + "-of-" + suit
public:
	Card(const std::string& v)
	: value{v} {
		assertValidCard(value); // ensure the value is always valid
	}
	std::string getValue() const {
		return value;
	}
};
\end{lstlisting}

这个类中,生成的特殊移动成员函数创建了一个无效状态,它破坏了类的不变条件,即花色和数值(比如“红心皇后”)。\par

只要我们不使用\textit{std::move()}或其他移动操作,这就不是问题(该类型的析构函数可以从字符串中移动),但是当调用\textit{std::move()}时,就会遇到麻烦。给新值赋值没什么问题:\par

\begin{lstlisting}[caption={}]
std::vector<Card> deck;
... // initialize deck
Card c{std::move(deck[0])}; // deck[0] has invalid state
deck[0] = Card{"ace-of-hearts"}; // deck[0] is valid again
\end{lstlisting}

然而,打印已移动卡片的值可能会失败:\par

\begin{lstlisting}[caption={}]
std::vector<Card> deck;
... // initialize deck
Card c{std::move(deck[0])}; // deck[0] has invalid state
print(deck[0]); // passing an object with broken invariant
\end{lstlisting}

如果print函数的不变量没有破坏,我们会得到一个core dump:\par

\begin{lstlisting}[caption={}]
void print(const Card& c) {
	std::string val{c.getValue()};
	auto pos = val.find("-of-"); // find position of substring (no check)
	std::cout << val.substr(0, pos) << ' '
	<< val.substr(pos+4) << '\n'; // OOPS: possible core dump
}
\end{lstlisting}

此代码可能在运行时失败,因为对于从移动的卡片,不再保证该值包含“-of-”。在这种情况下,\textit{find()}用std::string::npos初始化pos,当pos+4用作substr()的第一个参数时,会抛出类型为std::out\_of\_range的异常。\par

完整示例请参见basics/card.hpp和basics/card.cpp。\par

修复方式如下:\par

\begin{itemize}
	\item 禁用移动语义:\par
	\begin{lstlisting}[caption={}]
	class Card {
		...
		Card(const Card&) = default; // disable move semantics
		Card& operator=(const Card&) = default; // disable move semantics
	};
	\end{lstlisting}
	这让移动操作(例如,由std::sort()调用)的开销更大。
	\item 完全禁用复制和移动:\par
	\begin{lstlisting}[caption={}]
	class Card {
		...
		Card(const Card&) = delete; // disable copy/move semantics
		Card& operator=(const Card&) = delete; // disable copy/move semantics
	};
	\end{lstlisting}
	然而,不能再洗牌或对卡片进行排序。
	\item 修复损坏的特殊移动成员函数。\par
	然而,什么是有效的修复(总是分配一个“默认值”,如“梅花A”)?如何确保具有默认值的对象,在不分配内存的情况下性能良好?
	\item 内部允许新状态,但不允许调用getValue()或其他成员函数。\par
	可以对此进行记录(“对于已移动对象,只允许赋值。所有其他成员函数都以对象不处于已移动状态为前提”),甚至在成员函数内部检查这一点并触发断言或异常。
	\item 纸牌的方式可能没有新状态来扩展不变量。\par
	这意味着必须实现移动特殊成员函数,必须确保对于一个已移动对象,成员值处于这种状态。\par
	通常,已移动状态等同于默认构造状态。因此,这也是提供默认构造函数的机会。理想情况下,还可以提供成员函数来检查这个状态。\par
	这个变化中,这个类的用户必须考虑到字符串的值可能是空,并相应地更新代码。例如:\par
	\begin{lstlisting}[caption={}]
	void print(const Card& c) {
		std::string val{c.getValue()};
		auto pos = val.find("-of-"); // find position of substring
		If (pos != std::string::npos) { // check whether it exists
			std::cout << val.substr(0, pos) << ' '
			<< val.substr(pos+4) << '\n';
		}
		else {
			std::cout << "no value\n";
		}
	}
	\end{lstlisting}
\end{itemize}

另一方面,getValue()可能返回一个std::optional(自C++17支持)。似乎没有明显的完美解决方案。必须考虑这些修复对程序更大的不变量意味着什么。比如只有一张牌,或者所有的牌都有效等),然后决定用哪一张。\par

这个类在C++11之前运行良好,C++11之前的标准不支持移动语义(这可能意味着第一个选项是最好的)。因此,C++11可能会为实现类时不可能的类引入状态。这是一种罕见的情况,但移动语义的引入,确实会破坏现有的代码。\par

请参阅Email类的另一个例子,在这个类中,内部标记了已移动状态,以便单独处理,并通过“移除”算法使元素处于已移动状态后,使该状态可见。\par

\hspace*{\fill} \par %插入空行
\textbf{6.3.2 移动的一致性会破坏成员的不变量}

移动操作破坏不变量的第二个原因,与两个成员必须一致但可能被移动破坏的对象有关。正如在线程数组的例子中所看到的,这可能导致破坏析构函数的不一致。通常,析构函数没问题,但自移状态破坏了一个不变量。在类中,有两个不同的值表示形式,一个整数和一个字符串:\par

{\color{red}{basics/intstring.hpp}}

\begin{lstlisting}[caption={}]
#include <iostream>
#include <string>
class IntString
{
	private:
	int val; // value
	std::string sval; // cached string representation of the value
	public:
	IntString(int i = 0)
	: val{i}, sval{std::to_string(i)} {
	}
	void setValue(int i) {
		val = i;
		sval = std::to_string(i);
	}
	...
	void dump() const {
		std::cout << " [" << val << "/'" << sval << "']\n";
	}
};
\end{lstlisting}

这个类中,确保成员\textit{val}和成员\textit{sval}只是相同值的两种不同表示。这意味着在该类的实现和使用中,通常期望其状态的int和string表示一致。如果在这里使用移动操作,将保留值\textit{val},但是\textit{sval}不再保证可用\textit{val}的字符串表示。\par

请看下面程序代码:\par

{\color{red}{basics/intstring.cpp}}\par

\begin{lstlisting}[caption={}]
#include "intstring.hpp"
#include <iostream>

int main()
{
	IntString is1{42};
	IntString is2;
	std::cout << "is1 and is2 before move:\n";
	is1.dump();
	is2.dump();
	
	is2 = std::move(is1);
	
	std::cout << "is1 and is2 after move:\n";
	is1.dump();
	is2.dump();
}
\end{lstlisting}

程序有以下输出(已移动符串可能会变成空):\par

\begin{tcolorbox}[colback=white,colframe=black]
is1 and is2 before move: \\
\ [42/'42'] \\
\ [0/'0'] \\
is1 and is2 after move: \\
\ [42/''] \\
\ [42/'42']
\end{tcolorbox}

也就是说,自动生成的移动操作打破了两个成员匹配的不变量。\par

这个问题有多大?这至少是个陷阱。同样,您可能会认为在移动之后不应该再使用该值(直到再次设置)。然而,开发者希望使用针对C++标准库对象的策略,该策略声明对象处于有效但未定义的状态。\par

事实是,有了相应的getter,类不再保证int和string的匹配,这可能是类的不变量(隐式或显式声明)。您可能认为最坏的结果是值(现在是未指定的)看起来不同,这取决于如何使用,但使用它没有问题,因为它仍然是有效的int或字符串。\par

然而,依赖于此不变量的代码可能会破坏。该代码可能假设字符串表示至少有一个数字。例如,如果搜索第一个或最后一个数字,肯定会找到。已移动的字符串(通常是空的)不再是这样了。因此,代码不重复检查字符串值中是否有任何字符,可能会遇到未定义行为。\par

同样,如何处理这个问题取决于类的设计者。但是,如果遵循C++标准库的规则,应该使已移动对象处于有效状态,这可能要表示为“没有任何值”的状态。\par

通常,当一个对象的状态具有以某种方式相互依赖时,必须显式地确保已移动状态属于有效状态。但有些情况会出现例外:\par

\begin{itemize}
	\item 对相同值有不同的表示,但是其中一些已移走。
	\item 像\textit{counter}这样的成员对应于成员中的元素数量。
	\item 用布尔值声明字符串的值已验证,但是该验证值已移走。
	\item 所有元素的平均值的缓存值仍然存在,但是值(在容器成员中)已移走。
\end{itemize}

注意,这个类在C++11之前没问题,因为不支持移动语义。当切换到C++11或更高版本,并使用已移动对象时,不变量就破坏了。\par

\hspace*{\fill} \par %插入空行
\textbf{6.3.3 移动的类指针成员破坏不变量}

移动操作破坏不变量的第三个原因,与具有类似指针语义(如(智能)指针)成员的对象有关。\par

考虑以下类的例子,其中对象使用std::shared\_ptr<>共享整型值:\par

\begin{lstlisting}[caption={}]
class SharedInt {
private:
	std::shared_ptr<int> sp;
public:
	explicit SharedInt(int val)
	: sp{std::make_shared<int>(val)} {
	}
	std::string asString() const {
		return std::to_string(*sp); // OOPS: assume there is always an int value
	}
};
\end{lstlisting}

这个类的对象与副本共享的初始整数值。只要新对象是复制的,一切都好:\par

\begin{lstlisting}[caption={}]
SharedInt si1{42};
SharedInt si2{si1}; // si1 and si2 share the value 42

std::cout << si1.asString() << '\n'; // OK
\end{lstlisting}

由于只是复制,\textit{SharedInt}成员\textit{sp}总是为它的值分配内存(从std::make\_shared<>()或从已分配内存的现有共享指针复制)。\par

然而,当使用移动语义时,如果使用已移动对象,就会遇到未定义行为:\par

\begin{lstlisting}[caption={}]
SharedInt si1{42};
SharedInt si3{std::move(si1)}; // OOPS: moves away the allocated memory in si1

std::cout << si1.asString() << '\n'; // undefined behavior (probably core dump)
\end{lstlisting}

问题是在类内部,没有正确处理值可能已移走,因为默认生成的移动操作调用了共享指针的移动操作,该操作将所有权从原始对象移走。这意味着从\textit{SharedInt}的状态移动到成员\textit{sp}不再拥有对象的情况,其成员函数\textit{asString()}无法正确处理。\par

您可能会争辩说,为具有已移动状态的对象调用\textit{asString()}没有意义,因为使用的是未定义的值,但至少标准库保证已移动类型的对象处于有效状态,因此可以调用没有限制的所有操作。如果在用户定义的类型中不提供相同的保证,该类型的用户可能会感到非常惊讶。\par

从鲁棒编程(避免意外、陷阱和未定义行为)的角度来看,我建议您遵循C++标准库的规则。也就是说:移动操作不应该将对象带入破坏不变量的状态。\par

本例中,必须执行以下操作之一:\par

\begin{itemize}
	\item 通过正确处理所有可能的已移动状态来修复类的所有错误操作
	\item 禁用移动语义,在复制对象时没有优化
	\item 显式实现移动操作
	\item 调整并记录类或特定操作的不变量(约束/先决条件)(例如“为已移动对象调用\textit{asString()}是未定义的行为”)。
\end{itemize}

分配内存非常昂贵,最好的解决办法是正确处理所有权移走的情况。这将使用默认构造函数创建对象应具有的状态,这样就可以引入这个状态了。\par

下面的小节将演示这一点和其他修缮的代码。\par

\hspace*{\fill} \par %插入空行
\textbf{修复损坏的成员函数}

第一个选项,修复所有损坏的操作,扩展类的不变量(所有对象的可能状态),以便所有操作都可以处理已移动状态。我们仍然需要做出设计决策,例如:为已移动对象(或更一般地说,共享指针不拥有整型值的对象)调用asString()时,可以这样:\par

\begin{itemize}
	\item 返回备选值:\par
	\begin{lstlisting}[caption={}]
	class SharedInt {
		...
		std::string asString() const {
			return sp ? std::to_string(*sp) : "";
		}
		...
	};
	\end{lstlisting}
	\item 抛出异常:\par
	\begin{lstlisting}[caption={}]
	class SharedInt {
		...
		std::string asString() const {
			if (!sp) throw ...
			return std::to_string(*sp);
		}
		...
	};
	\end{lstlisting}
	\item 在调试模式下强制出现运行时错误:\par
	\begin{lstlisting}[caption={}]
	class SharedInt {
		...
		std::string asString() const {
			assert(sp);
			return std::to_string(*sp);
		}
		...
	};
	\end{lstlisting}
\end{itemize}

\hspace*{\fill} \par %插入空行
\textbf{禁用移动语义}

第二个选项是禁用移动语义,只使用复制语义。我们前面描述了如何禁用移动语义,必须声明另一个特殊成员函数。通常,对复制特殊成员函数使用\textit{=default}:\par

\begin{lstlisting}[caption={}]
class SharedInt {
	...
	SharedInt(const SharedInt&) = default; // disable move semantics
	SharedInt& operator=(const SharedInt&) = default; // disable move semantics
	...
};
\end{lstlisting}

\hspace*{\fill} \par %插入空行
\textbf{实现移动语义}

第三种选择是实现移动操作,这样就不会破坏类的不变量。\par

为此,必须确定已移动对象的状态应该是什么。为了支持\textit{asString()}可以调用operator*(解引用)而不检查值是否存在,所以必须提供一个值。例如,可以有一个静态的已移动的值,把它赋给那些已移动的对象:\par

{\color{red}{basics/sharedint.hpp}}

\begin{lstlisting}[caption={}]
#include <memory>
#include <string>
class SharedInt {
	private:
	std::shared_ptr<int> sp;
	// special “value” for moved-from objects:
	inline static std::shared_ptr<int> movedFromValue{std::make_shared<int>(0)};
	public:
	explicit SharedInt(int val)
	: sp{std::make_shared<int>(val)} {
	}
	std::string asString() const {
		return std::to_string(*sp); // OOPS: unconditional deref
	}
	// fix moving special member functions:
	SharedInt (SharedInt&& si)
	: sp{std::move(si.sp)} {
		si.sp = movedFromValue;
	}
	SharedInt& operator= (SharedInt&& si) noexcept {
		if (this != &si) {
			sp = std::move(si.sp);
			si.sp = movedFromValue;
		}
		return *this;
	}
	// enable copying (deleted with user-declared move operations):
	SharedInt (const SharedInt&) = default;
	SharedInt& operator= (const SharedInt&) = default;
};
\end{lstlisting}

注意,我们必须对复制特殊成员的函数使用\textit{=default},因为当我们有声明特殊成员移动函数时,这些函数的默认版本会删除。\par


















































