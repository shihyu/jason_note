与移动语义一起,C++11引入了一个新的关键字decltype。这个关键字的主要目标是获得声明对象的确切类型,也可以用于确定表达式的值类型。\par

\hspace*{\fill} \par %插入空行
\textbf{8.6.1 使用decltype检查名称的类型}

在接受rvalue引用形参的函数中,可以使用decltype查询并使用形参的确切类型。只需将参数的名称传递给decltype。例如:\par

\begin{lstlisting}[caption={}]
void rvFunc(std::string&& str)
{
	std::cout << std::is_same<decltype(str), std::string>::value; // false
	std::cout << std::is_same<decltype(str), std::string&>::value; // false
	std::cout << std::is_same<decltype(str), std::string&&>::value; // true
	std::cout << std::is_reference<decltype(str)>::value; // true
	std::cout << std::is_lvalue_reference<decltype(str)>::value; // false
	std::cout << std::is_rvalue_reference<decltype(str)>::value; // true
}
\end{lstlisting}

decltype(str)表达式总是表示\textit{str}的类型,即std::string\&\&。在表达式中任何需要该类型的地方都可以使用该类型。类型特征(类型函数如std::is\_same<>)会帮助我们处理这些类型。\par

例如,要声明传递的形参类型不是引用的新对象,可以声明:\par

\begin{lstlisting}[caption={}]
void rvFunc(std::string&& str)
{
	std::remove_reference<decltype(str)>::type tmp;
	...
}
\end{lstlisting}

\textit{tmp}在这个函数中是std::string类型(也可以显式地声明,如果使它成为T类型对象的泛型函数,代码仍可以工作)。\par

\hspace*{\fill} \par %插入空行
\textbf{8.6.2 使用decltype检查值类型}

目前为止,只向decltype传递了名称来查询类型。但是,也可以将表达式(不仅仅是名称)传递给decltype,会根据以下约定生成值类型:\par

\begin{itemize}
	\item 对于prvalue,产生值类型:type
	\item 对于lvalue,将其类型作为lvalue引用:type\&
	\item 对于xvalue,将其类型作为rvalue引用:type\&\&
\end{itemize}

例如:\par

\begin{lstlisting}[caption={}]
void rvFunc(std::string&& str)
{
	decltype(str + str) // yields std::string because s+s is a prvalue
	decltype(str[0]) // yields char& because the index operator yields an lvalue
	...
}
\end{lstlisting}

这意味着,如果只是传递一个放在圆括号内的名称(这是一个表达式,而不再只是名称),decltype将生成其类型。行为如下:\par

\begin{lstlisting}[caption={}]
void rvFunc(std::string&& str)
{
	std::cout << std::is_same<decltype((str)), std::string>::value; // false
	std::cout << std::is_same<decltype((str)), std::string&>::value; // true
	std::cout << std::is_same<decltype((str)), std::string&&>::value; // false
	std::cout << std::is_reference<decltype((str))>::value; // true
	std::cout << std::is_lvalue_reference<decltype((str))>::value; // true
	std::cout << std::is_rvalue_reference<decltype((str))>::value; // false
}
\end{lstlisting}

将此函数与不使用括号的前一个函数实现进行比较。这里,decltype(str)的结果是std::string\&,因为str是lvalue的std::string类型。\par

对于decltype,当传递的名称周围加上圆括号时,会产生不同的结果,这在稍后讨论decltype(auto)时会很重要。\par

\hspace*{\fill} \par %插入空行
\textbf{检查值类型内部代码}

现在可以在代码中检查特定的值类别,如下所示:\par

\begin{itemize}
	\item !std::is\_reference\_v<decltype((expr))>
	检查expr是否为prvalue。
	\item std::is\_lvalue\_reference\_v<decltype((expr))>
	检查expr是否为lvalue。
	\item std::is\_rvalue\_reference\_v<decltype((expr))>
	检查expr是否为xvalue。
	\item !std::is\_lvalue\_reference\_v<decltype((expr))>
	检查expr是否为rvalue。
\end{itemize}

请再次注意这里使用的括号,以确保即使只传递名称\textit{expr},也使用decltype的值-类别检查形式。\par

C++20之前,必须使用::value来替代后缀\_v。\par















