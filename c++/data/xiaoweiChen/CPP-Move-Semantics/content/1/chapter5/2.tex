从C++98开始,可以重载成员函数来实现const和非const版本。例如:\par

\begin{lstlisting}[caption={}]
class C {
public:
	...
	void foo(); // foo() for non-const objects
	void foo() const; // foo() for const objects
};
\end{lstlisting}

圆括号后面的限定符允许限定一个没有传递给形参的对象:可以调用此成员函数的对象。\par

有了移动语义,就有了用限定符重载函数的新方。考虑以下代码:\par

{\color{red}{basics/refqual.cpp}}

\begin{lstlisting}[caption={}]
#include <iostream>
class C {
public:
	void foo() const& {
		std::cout << "foo() const&\n";
	}
	void foo() && {
		std::cout << "foo() &&\n";
	}
	void foo() & {
		std::cout << "foo() &\n";
	}
	void foo() const&& {
		std::cout << "foo() const&&\n";
	}
};

int main()
{
	C x;
	x.foo(); // calls foo() &
	C{}.foo(); // calls foo() &&
	std::move(x).foo(); // calls foo() &&
	
	const C cx;
	cx.foo(); // calls foo() const&
	std::move(cx).foo(); // calls foo() const&&
}
\end{lstlisting}

这个程序演示了所有可能的引用限定符,以及何时调用。通常,只有两到三个这样的重载,比如对getter使用\&\&和const\&(和\&)。\par

还要注意,不允许引用和非引用限定符的重载:\par

\begin{lstlisting}[caption={}]
class C {
public:
	void foo() &&;
	void foo() const; // ERROR: can’t overload by both reference and value qualifiers
};
\end{lstlisting}

























































