如前所述,变量模板也有可变参数。但变量不能递归定义,也不能像类模板那样特化。折叠表达式简化了从可变数量的参数生成表达式的过程,对于创建可变参数的变量模板非常方便。

下面的例子中,定义了一个名为Sum的可变变量模板,在编译时初始化为所有作为非类型模板参数提供的整数的和:

\begin{lstlisting}[style=styleCXX]
template <int... R>
constexpr int Sum = (... + R);
int main()
{
	std::cout << Sum<1> << '\n';
	std::cout << Sum<1,2> << '\n';
	std::cout << Sum<1,2,3,4,5> << '\n';
}
\end{lstlisting}

这类似于用折叠表达式写的和函数,要添加的数字作为函数参数提供。这里,作为模板参数提供给变量template。区别主要是句法上的;启用优化后,最终结果在生成的汇编代码和性能方面可能相同。

可变参数变量模板与所有其他类型的模板遵循相同的模式,尽管不像其他模板那么常用。结束这个主题后,目前已经完成了对C++中可变参数模板的了解。















































