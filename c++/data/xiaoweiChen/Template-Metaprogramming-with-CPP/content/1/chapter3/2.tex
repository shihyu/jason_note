可变参数函数模板是具有可变数量参数的模板函数,用省略号(…)的用法来指定一组参数,这些参数可以有不同的语法。

为了理解可变参数函数模板的基本原理,我们重写一下之前min函数:

\begin{lstlisting}[style=styleCXX]
template <typename T>
T min(T a, T b)
{
	return a < b ? a : b;
}

template <typename T, typename... Args>
T min(T a, Args... args)
{
	return min(a, min(args...));
}

int main()
{
	std::cout << "min(42.0, 7.5)=" << min(42.0, 7.5)
	          << '\n';
	std::cout << "min(1,5,3,-4,9)=" << min(1, 5, 3, -4, 9)
	          << '\n';
}
\end{lstlisting}

这里有两个min函数的重载。第一个是带有两个形参的函数模板,返回两个形参中最小的一个。第二个是带有可变数量参数的函数模板,通过扩展形参包递归地调用自身。尽管可变函数模板实现看起来像是使用了某种编译时递归机制(带有两个参数的重载作为结束),但只依赖于重载函数,从模板和提供的参数集实例化。

可变参数变量函数模板的实现中,省略号(…)在三个不同的地方使用,具有不同的含义:

\begin{itemize}
\item
要在模板参数列表中指定一组参数,如typename…Args,这称为模板参数包。可以为类型模板、非类型模板和双重模板参数定义的模板参数包。

\item
函数参数列表中指定一组参数,如Args…args,这称为函数参数包。

\item
函数体中展开包,如在args…中,可在min(args…)中看到,这称为参数包展开。这种展开的结果是一个由零个或多个值(或表达式)组成的逗号分隔列表。
\end{itemize}

从min(1,5,3,-4,9)开始,编译器实例化了一组带有5,4,3和2个参数的重载函数。理论上,其拥有以下一组重载函数:

\begin{lstlisting}[style=styleCXX]
int min(int a, int b)
{
	return a < b ? a : b;
}

int min(int a, int b, int c)
{
	return min(a, min(b, c));
}

int min(int a, int b, int c, int d)
{
	return min(a, min(b, min(c, d)));
}

int min(int a, int b, int c, int d, int e)
{
	return min(a, min(b, min(c, min(d, e))));
}
\end{lstlisting}

min(1,5,3,-4,9)会扩展为min(1, min(5, min(3, min(- 4,9)))))。这可能会引发关于可变参数模板性能的问题。实际中,编译器会进行大量的优化,例如:尽可能多地内联。当启用优化时,将不会有实际函数的调用。可以使用在线资源,例如Compiler Explorer(\url{https:// godbolt.org/})),以查看不同编译器使用不同选项生成的代码(例如优化设置)。考虑下面的代码片段(min是前面所示实现的可变参数函数模板):

\begin{lstlisting}[style=styleCXX]
int main()
{
	std::cout << min(1, 5, 3, -4, 9);
}
\end{lstlisting}

GCC 11.2中使用-O标志进行编译以进行优化,会生成以下汇编代码:

\begin{lstlisting}[style=styleCXX]
sub rsp, 8
mov esi, -4
mov edi, OFFSET FLAT:_ZSt4cout
call std::basic_ostream<char, std::char_traits<char>>
        ::operator<<(int)
mov eax, 0
add rsp, 8
ret
\end{lstlisting}

这里,不需特别了解汇编。对min(1,5,3,-4,9)的计算在编译时完成,结果-4直接加载到ESI寄存器中。没有运行时调用,也没有计算,一切在编译时皆为已知。当然,情况并非总是如此。

下面的代码段显示了对min函数模板的调用,不能在编译时求值,因为参数只有在运行时才明了:

\begin{lstlisting}[style=styleCXX]
int main()
{
	int a, b, c, d, e;
	std::cin >> a >> b >> c >> d >> e;
	std::cout << min(a, b, c, d, e);
}
\end{lstlisting}

这一次,生成的汇编代码如下所示(只显示了调用min函数的代码):

\begin{lstlisting}[style=styleCXX]
mov esi, DWORD PTR [rsp+12]
mov eax, DWORD PTR [rsp+16]
cmp esi, eax
cmovg esi, eax
mov eax, DWORD PTR [rsp+20]
cmp esi, eax
cmovg esi, eax
mov eax, DWORD PTR [rsp+24]
cmp esi, eax
cmovg esi, eax
mov eax, DWORD PTR [rsp+28]
cmp esi, eax
cmovg esi, eax
mov edi, OFFSET FLAT:_ZSt4cout
call std::basic_ostream<char, std::char_traits<char>>
        ::operator<<(int)
\end{lstlisting}

编译器已经内联了所有对min重载的调用。只有一系列的指令用于将值加载到寄存器中,比较寄存器值,并根据比较结果进行跳转,但是没有函数调用。

当禁用优化时,确实会发生函数调用。可以通过使用特定于编译器的宏来跟踪在调用min函数期间发生的这些调用。GCC和Clang提供了一个名为\_\_PRETTY\_FUNCTION\_\_的宏,包含函数的签名及其名称。类似地,Visual C++提供了一个名为\_\_FUNCSIG\_\_的宏,做了同样的事情。这些可以在函数体中使用,以打印其名称和签名:


\begin{lstlisting}[style=styleCXX]
template <typename T>
T min(T a, T b)
{
#if defined(__clang__) || defined(__GNUC__) || defined(__GNUG__)
	std::cout << __PRETTY_FUNCTION__ << "\n";
#elif defined(_MSC_VER)
	std::cout << __FUNCSIG__ << "\n";
#endif
	return a < b ? a : b;
}

template <typename T, typename... Args>
T min(T a, Args... args)
{
#if defined(__clang__) || defined(__GNUC__) || defined(__GNUG__)
	std::cout << __PRETTY_FUNCTION__ << "\n";
#elif defined(_MSC_VER)
	std::cout << __FUNCSIG__ << "\n";
#endif
	return min(a, min(args...));
}

int main()
{
	min(1, 5, 3, -4, 9);
}
\end{lstlisting}

当用Clang编译这个程序时,执行的结果如下所示:

\begin{lstlisting}[style=styleCXX]
T min(T, Args...) [T = int, Args = <int, int, int, int>]
T min(T, Args...) [T = int, Args = <int, int, int>]
T min(T, Args...) [T = int, Args = <int, int>]
T min(T, T) [T = int]
T min(T, T) [T = int]
T min(T, T) [T = int]
T min(T, T) [T = int]
\end{lstlisting}

当用Visual C++编译时,输出如下所示:

\begin{lstlisting}[style=styleCXX]
int __cdecl min<int,int,int,int,int>(int,int,int,int,int)
int __cdecl min<int,int,int,int>(int,int,int,int)
int __cdecl min<int,int,int>(int,int,int)
int __cdecl min<int>(int,int)
int __cdecl min<int>(int,int)
int __cdecl min<int>(int,int)
int __cdecl min<int>(int,int)
\end{lstlisting}

尽管Clang/GCC和VC++之间签名的格式有很大的不同,但也有相同的显示:首先,调用一个带有五个参数的重载函数;然后是一个带有四个参数的重载函数;再然后是一个带有三个参数的重载函数;最后,有四个带有两个参数的重载函数调用(这标志着扩展的结束)。

理解参数包的扩展是理解可变参数模板的关键,我们将在下一节中详细探讨这个主题。



