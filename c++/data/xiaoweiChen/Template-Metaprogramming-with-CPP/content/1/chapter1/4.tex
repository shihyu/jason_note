模板元编程是泛型编程的C++实现。这种范式在20世纪70年代首次出现,在20世纪80年代上半叶出现了第一批支持泛型的语言:Ada和Eiffel。David Musser和Alexander Stepanov在1989年的一篇论文中定义了泛型编程:

\begin{center}
\textit{
泛型编程的核心思想是从具体、有效的算法中抽象出来,以获得与不同数据表示相结合的泛型算法,进而生成各种软件。
}
\end{center}

这就是编程范式的定义,算法根据稍后指定的类型定义,并根据使用进行实例化。

模板最初是\textbf{C with Classes}语言的一部分,并不是由Bjarne Stroustrup开发的。Stroustrup描述C++模板的第一篇论文出现在1986年,也就是《C++程序设计语言第一版》出版一年后。1990年,在ANSI和ISO C++标准化委员会成立之前,模板已经存在于C++中了。

20世纪90年代早期,Alexander Stepanov, David Musser和Meng Lee尝试在C++中实现各种泛型概念,这就是\textbf{标准模板库(STL)}的第一个实现。当ANSI/ISO委员会在1994年发现到这个库时,就很快的将其添加到规范中了。1998年,STL与C++语言一起标准化,也就是C++98标准。

C++标准的新版本,统称为\textbf{现代C++},引入了对模板元编程的各种改进。下表简要列了一下:

\begin{table}[H]
\centering
	\begin{tabular}{|l|l|l|}
		\hline
		\textbf{版本} &
		\textbf{特性} &
		\textbf{描述} \\ \hline
		\multirow{4}{*}{C++11} &
		可变参模板 &
		
		有可变数量的模板参数。\\ \cline{2-3} 
		&
		模板别名 &
		能够使用声明定义模板类型的别名。 \\ \cline{2-3} 
		&
		外部模板 &
		告诉编译器不要实例化模板 \\ \cline{2-3} 
		&
		类型特征 &
		\begin{tabular}[c]{@{}l@{}}新头文件 \textless{}type\_traits\textgreater 包含标识对象类别和\\ 类型特征。\end{tabular} \\ \hline
		C++14 &
		变量模板 &
		支持定义变量或静态数据成员。 \\ \hline
		\multirow{4}{*}{C++17} &
		折叠表达式 &
		\begin{tabular}[c]{@{}l@{}}用二进制运算符减少可变参数模板的参数包。\end{tabular} \\ \cline{2-3} 
		&
		\begin{tabular}[c]{@{}l@{}}模板参数\\ typename\end{tabular} &
		typename关键字可以用来代替模板参数中的class。 \\ \cline{2-3} 
		&
		\begin{tabular}[c]{@{}l@{}}非类型模板\\ 参数auto\end{tabular} &
		关键字auto可以用于非类型的模板参数。\\ \cline{2-3} 
		&
		\begin{tabular}[c]{@{}l@{}}类模板的\\ 参数推导\end{tabular} &
		编译器从对象初始化的方式推导模板参数的类型。 \\ \hline
		\multirow{4}{*}{C++20} &
		模板Lambda &
		Ladmbd表达式可作为模板。 \\ \cline{2-3} 
		&
		\begin{tabular}[c]{@{}l@{}}字符串字面值\\ 作为模板参数\end{tabular} &
		\begin{tabular}[c]{@{}l@{}}字符串字面量可以用作非类型模板参数,\\ 以及用户定义的字符串字面操作符的新形式。\end{tabular} \\ \cline{2-3} 
		&
		约束 &
		明确模板参数的需求。 \\ \cline{2-3} 
		&
		概念 &
		命名的约束集。 \\ \hline
	\end{tabular}
\end{table}

\begin{center}
表 1.1
\end{center}

这些特性,以及模板元编程的其他方面,是本书的唯一主题,并将在接下来的章节中进行详细介绍。现在,来看看模板的优缺点。






















