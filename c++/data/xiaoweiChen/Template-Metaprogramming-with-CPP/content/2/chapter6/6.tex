作为类模板成员的非模板函数,可以用相应的方式进行约束,模板类只能为满足某些需求的类型定义成员函数。下面的例子中,operator==是受约束的:

\begin{lstlisting}[style=styleCXX]
template <typename T>
struct wrapper
{
	T value;
	
	bool operator==(std::string_view str)
	requires std::is_convertible_v<T, std::string_view>
	{
		return value == str;
	}
};
\end{lstlisting}

wrapper类保存一个T类型的值,并仅为可转换为std::string\_view的类型定义operator==成员:

\begin{lstlisting}[style=styleCXX]
wrapper<int> a{ 42 };
wrapper<char const*> b{ "42" };

if(a == 42) {} // error
if(b == "42") {} // OK
\end{lstlisting}

这里有wrapper类的两个实例化,一个用于int,一个用于char const*。尝试将a对象与字面值42进行比较会产生编译器错误,因为没为这种类型定义operator==。但将b对象与字符串字面值"42"进行比较是可能的,因为相等操作符是为可以隐式转换为std::string\_view的类型定义的,char const*就是这样一种类型。

约束非模板成员很有用,相较于强制成员为模板并使用SFINAE,这是一种更清晰的解决方案。为了更好地进行理解,看看如下wrapper类的实现:

\begin{lstlisting}[style=styleCXX]
template <typename T>
struct wrapper
{
	T value;
	
	wrapper(T const & v) :value(v) {}
};
\end{lstlisting}

这个类模板可以进行如下的实例化:

\begin{lstlisting}[style=styleCXX]
wrapper<int> a = 42; //OK

wrapper<std::unique_ptr<int>> p =
	std::make_unique<int>(42); //error
\end{lstlisting}

第一行编译成功,但第二行会产生编译器错误。不同的编译器会发出不同的消息,但错误的核心是调用隐式删除的std::unique\_ptr复制构造函数。

我们要做的是限制包装器的复制构造来自T类型的对象,以便它只适用于可复制构造的T类型。C++20之前可用的方法是将复制构造函数转换为模板,并使用SFINAE:

\begin{lstlisting}[style=styleCXX]
template <typename T>
struct wrapper
{
	T value;
	template <typename U,
		typename = std::enable_if_t<
			std::is_copy_constructible_v<U> &&
			std::is_convertible_v<U, T>>>
	wrapper(U const& v) :value(v) {}
};
\end{lstlisting}

当试图从std::unique\_ptr<int>值初始化wrapper<std::unique\_ptr<int>{}>时,也会收获一个错误,但错误是不同的。例如,下面是Clang生成的错误消息:

\begin{tcblisting}{commandshell={}}
prog.cc:19:35: error: no viable conversion from 'typename __
unique_if<int>::__unique_single' (aka 'unique_ptr<int>') to
   'wrapper<std::unique_ptr<int>>'
wrapper<std::unique_ptr<int>> p = std::make_
unique<int>(42); // error
                                        ^ ~~~~~~~~~~~~~~~~~~~~~~~~~
prog.cc:6:8: note: candidate constructor (the implicit copy
constructor) not viable: no known conversion from 'typename
__unique_if<int>::__unique_single' (aka 'unique_ptr<int>') to
'const wrapper<std::unique_ptr<int>> &' for 1st argument
struct wrapper
         ^
prog.cc:6:8: note: candidate constructor (the implicit move
constructor) not viable: no known conversion from 'typename
__unique_if<int>::__unique_single' (aka 'unique_ptr<int>') to
'wrapper<std::unique_ptr<int>> &&' for 1st argument
struct wrapper
         ^
prog.cc:13:9: note: candidate template ignored: requirement
'std::is_copy_constructible_v<std::unique_ptr<int,
std::default_delete<int>>>' was not satisfied [with U =
std::unique_ptr<int>]
        wrapper(U const& v) :value(v) {}
        ^
\end{tcblisting}

帮助理解问题原因的最重要的信息是最后一条,说U替换为std::unique\_ptr<int>的要求不满足布尔条件。C++20中,可以更好地实现对T模板参数的相同限制。这一次,我们可以使用约束,复制构造函数不再需要是模板。C++20中的实现如下所示:

\begin{lstlisting}[style=styleCXX]
template <typename T>
struct wrapper
{
	T value;
	wrapper(T const& v)
		requires std::is_copy_constructible_v<T>
		:value(v)
	{}
};
\end{lstlisting}

不仅代码更少,不需要复杂的SFINAE机制,而且更简单,更容易理解,还可能生成更好的错误消息。Clang的情况下,前面的错误信息会替换为以下内容:

\begin{tcblisting}{commandshell={}}
prog.cc:9:5: note: candidate constructor not viable:
constraints not satisfied
    wrapper(T const& v)
    ^

prog.cc:10:18: note: because 'std::is_copy_constructible_
v<std::unique_ptr<int> >' evaluated to false
        requires std::is_copy_constructible_v<T>
\end{tcblisting}

值得一提的是,不仅属于类成员的非模板函数可以受到约束,独立函数也可以。非模板函数的用例很少见,可以用其他简单的解决方案来实现,比如constexpr if:

\begin{lstlisting}[style=styleCXX]
void handle(int v)
{ /* do something */ }

void handle(long v)
	requires (sizeof(long) > sizeof(int))
{ /* do something else */ }
\end{lstlisting}

这个代码片段中,我们有两个handle函数的重载。第一个重载采用int,第二个重载采用long。这些重载函数体并不重要,但当long的大小与int的大小不同时,就应该做不同的事情。标准规定int的大小至少是16位,尽管在大多数平台上它是32位,long长度至少为32位。但有一些平台,比如LP64,其中int是32位,long是64位。这些平台上,两个重载都应该是可用的。在所有其他平台上,两种类型具有相同的大小,只有第一个重载可用。可以用前面所示的形式定义,在C++17中使用constexpr也可以实现同样的功能:

\begin{lstlisting}[style=styleCXX]
void handle(long v)
{
	if constexpr (sizeof(long) > sizeof(int))
	{
		/* do something else */
	}
	else
	{
		/* do something */
	}
}
\end{lstlisting}

下一节中,我们将学习如何使用约束来定义类模板的模板参数的限制。


