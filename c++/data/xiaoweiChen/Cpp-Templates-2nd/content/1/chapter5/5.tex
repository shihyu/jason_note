
类成员可以是模板,对于嵌套类和成员函数都有可能。Stack<>类模板可以再次展示这种能力的应用和优势。通常,只有具有相同的类型时,才能相互分配堆栈,这意味着元素具有相同的类型。但不能将其他类型的元素赋值给堆栈,即使对定义的元素类型有隐式的类型转换:

\begin{lstlisting}[style=styleCXX]
Stack<int> intStack1, intStack2; // stacks for ints
Stack<float> floatStack; // stack for floats
...
intStack1 = intStack2; // OK: stacks have same type
floatStack = intStack1; // ERROR: stacks have different types
\end{lstlisting}

默认赋值操作符要求赋值操作符的两端具有相同的类型。

将赋值操作符定义为模板,可以使用定义了适当类型转换的元素对堆栈赋值。需要进行如下声明:

\hspace*{\fill} \\ %插入空行
\noindent
\textit{basics/stack5decl.hpp}
\begin{lstlisting}[style=styleCXX]
template<typename T>
class Stack {
private:
	std::deque<T> elems; // elements
	
public:
	void push(T const&); // push element
	void pop(); // pop element
	T const& top() const; // return top element
	bool empty() const { // return whether the stack is empty
		return elems.empty();
	}
	
	// assign stack of elements of type T2
	template<typename T2>
	Stack& operator= (Stack<T2> const&);
};
\end{lstlisting}

进行了以下两处修改:

\begin{enumerate}
\item 
为另一种类型T2的元素堆栈添加了赋值操作符声明。

\item 
栈现在使用std::deque<>作为元素的内部容器。同样,这也是新赋值操作符实现的结果。
\end{enumerate}

新赋值操作符的实现如下所示:

\hspace*{\fill} \\ %插入空行
\noindent
\textit{basics/stack5assign.hpp}
\begin{lstlisting}[style=styleCXX]
template<typename T>
template<typename T2>
Stack<T>& Stack<T>::operator= (Stack<T2> const& op2)
{
	Stack<T2> tmp(op2); // create a copy of the assigned stack
	
	elems.clear(); // remove existing elements
	while (!tmp.empty()) { // copy all elements
		elems.push_front(tmp.top());
		tmp.pop();
	}
	return *this;
}
\end{lstlisting}

\begin{tcolorbox}[colback=webgreen!5!white,colframe=webgreen!75!black]
\hspace*{0.75cm}这只是一个模板特性的基本实现。像异常处理的实现,这里没有添加。
\end{tcolorbox}

首先看看定义成员模板的语法。模板参数为T的模板内部,定义了一个模板参数为T2的内部模板:

\begin{lstlisting}[style=styleCXX]
template<typename T>
template<typename T2>
...
\end{lstlisting}

成员函数中,可能只希望访问所赋堆栈op2的必要数据。但这个堆栈有不同的类型(若为两种不同的参数类型实例化一个类模板,将得到两种不同的类类型),因此只能使用公共接口。所以,访问元素的唯一方法是top()。然而,每个元素都必须成为顶部元素。因此,必须创建op2的副本,以便通过调用pop()从该副本中获取元素。因为top()返回压入堆栈的最后一个元素,所以可能更倾向于使用支持在集合的另一端插入元素的容器。所以,这里使用std::deque<>的push\_front(),将元素放在集合的另一端。

要访问op2的所有成员,可以将所有其他堆栈实例声明为友元:

\hspace*{\fill} \\ %插入空行
\noindent
\textit{basics/stack6decl.hpp}
\begin{lstlisting}[style=styleCXX]
template<typename T>
class Stack {
private:
	std::deque<T> elems; // elements
	
public:
	void push(T const&); // push element
	void pop(); // pop element
	T const& top() const; // return top element
	bool empty() const { // return whether the stack is empty
		return elems.empty();
	}

	// assign stack of elements of type T2
	template<typename T2>
	Stack& operator= (Stack<T2> const&);
	// to get access to private members of Stack<T2> for any type T2:
	template<typename> friend class Stack;
};
\end{lstlisting}

因为模板参数的名字没有使用,所以可以省略:

\begin{lstlisting}[style=styleCXX]
template<typename> friend class Stack;
\end{lstlisting}

现在,模板赋值操作符的实现可能是这样:

\hspace*{\fill} \\ %插入空行
\noindent
\textit{basics/stack6assign.hpp}
\begin{lstlisting}[style=styleCXX]
template<typename T>
template<typename T2>
Stack<T>& Stack<T>::operator= (Stack<T2> const& op2)
{
	elems.clear(); // remove existing elements
	elems.insert(elems.begin(), // insert at the beginning
				op2.elems.begin(), // all elements from op2
				op2.elems.end());
	return *this;
}
\end{lstlisting}

无论实现什么样,有了这个成员模板,现在可以将一堆int型对象赋值给一堆float型对象:

\begin{lstlisting}[style=styleCXX]
Stack<int> intStack; // stack for ints
Stack<float> floatStack; // stack for floats
...
floatStack = intStack; // OK: stacks have different types,
						// but int converts to float
\end{lstlisting}

当然,这个赋值不会改变堆栈及其元素的类型。赋值之后,floatStack的元素仍然是浮点数,因此top()仍然返回一个浮点数。

这个函数可能会禁用类型检查,这样就可以将任何类型的元素赋给堆栈,但事实并非如此。当源堆栈(副本)的元素移动到目标堆栈时,就会进行类型检查:

\begin{lstlisting}[style=styleCXX]
elems.push_front(tmp.top());
\end{lstlisting}

若一堆string赋值给一堆float数,这一行的编译会产生一个错误信息,表示tmp.top()返回的字符串不能作为参数传递给elems.push\_front()(错误信息会随编译器而变化,但这是其含义的重点):

\begin{lstlisting}[style=styleCXX]
Stack<std::string> stringStack; // stack of strings
Stack<float> floatStack; // stack of floats
...
floatStack = stringStack; // ERROR: std::string doesn’t convert to float
\end{lstlisting}

同样,可以改变实现来参数化内部容器类型:

\hspace*{\fill} \\ %插入空行
\noindent
\textit{basics/stack7decl.hpp}
\begin{lstlisting}[style=styleCXX]
template<typename T, typename Cont = std::deque<T>>
class Stack {
private:
	Cont elems; // elements
	
public:
	void push(T const&); // push element
	void pop(); // pop element
	T const& top() const; // return top element
	bool empty() const { // return whether the stack is empty
		return elems.empty();
	}

	// assign stack of elements of type T2
	template<typename T2, typename Cont2>
	Stack& operator= (Stack<T2,Cont2> const&);
	// to get access to private members of Stack<T2> for any type T2:
	template<typename, typename> friend class Stack;
};
\end{lstlisting}

所以模板赋值操作符就可以这样实现:

\hspace*{\fill} \\ %插入空行
\noindent
\textit{basics/stack7assign.hpp}
\begin{lstlisting}[style=styleCXX]
template<typename T, typename Cont>
template<typename T2, typename Cont2>
Stack<T,Cont>&
Stack<T,Cont>::operator= (Stack<T2,Cont2> const& op2)
{
	elems.clear(); // remove existing elements
	elems.insert(elems.begin(), // insert at the beginning
				op2.elems.begin(), // all elements from op2
				op2.elems.end());
	return *this;
}
\end{lstlisting}

对于类模板,只有那些调用的成员函数才会实例化。因此,若需要避免在堆栈中分配不同类型的元素,可以使用vector作为内部容器:

\begin{lstlisting}[style=styleCXX]
// stack for ints using a vector as an internal container
Stack<int,std::vector<int>> vStack;
...
vStack.push(42);
vStack.push(7);
std::cout << vStack.top() << '\n';
\end{lstlisting}

因为赋值操作符模板不是必需,所以不会出现缺少成员函数push\_front()的错误消息。

要获得上一个示例的完整实现,请参阅子目录基础中以stack7开头的文件。

\hspace*{\fill} \\ %插入空行
\noindent
\textbf{成员函数模板的特化}

成员函数模板也可以全特化,但不能偏特化。例如,对于以下类:

\hspace*{\fill} \\ %插入空行
\noindent
\textit{basics/boolstring.hpp}
\begin{lstlisting}[style=styleCXX]
class BoolString {
private:
	std::string value;
public:
	BoolString (std::string const& s)
	: value(s) {
	}
	template<typename T = std::string>
	T get() const { // get value (converted to T)
		return value;
	}
};
\end{lstlisting}

可以为如下所示的成员函数模板提供全特化:

\hspace*{\fill} \\ %插入空行
\noindent
\textit{basics/boolstringgetbool.hpp}
\begin{lstlisting}[style=styleCXX]
// full specialization for BoolString::getValue<>() for bool
template<>
inline bool BoolString::get<bool>() const {
	return value == "true" || value == "1" || value == "on";
}
\end{lstlisting}

不需要也不能对特化进行声明,只需要定义。因为它是全特化的,且位于头文件中,所以若定义包含在不同的转译单元中,必须使用内联声明,以避免错误。

可以使用类的全特化,如下所示:

\begin{lstlisting}[style=styleCXX]
std::cout << std::boolalpha;
BoolString s1("hello");
std::cout << s1.get() << '\n'; // prints hello
std::cout << s1.get<bool>() << '\n'; // prints false
BoolString s2("on");
std::cout << s2.get<bool>() << '\n'; // prints true
\end{lstlisting}

\hspace*{\fill} \\ %插入空行
\noindent
\textbf{特殊成员函数的模板}

只要特殊成员函数允许复制或移动对象,就可以使用模板成员函数。与上面定义的赋值操作符类似,也可以是构造函数。但模板构造函数或模板赋值操作符,不能取代预定义构造函数或赋值操作符,成员模板不作为复制或移动对象的特殊成员函数。本例中,对于相同类型的堆栈赋值,仍然使用默认赋值操作符。

这种影响有好有坏:

\begin{itemize}
\item 
模板构造函数或赋值操作符,可能比预定义的复制/移动构造函数或赋值操作符更匹配,尽管只提供了用于初始化其他类型的模板版本。请参见6.2节。

\item 
要对复制/移动构造函数进行“模板化”以约束其存在是非常困难的。详见章节6.4。
\end{itemize}

\subsubsubsection{5.5.1\hspace{0.2cm}.template构造}

有时,在调用成员模板时,需要显式限定模板参数。这种情况下,必须使用template关键字来确保<是模板参数列表的开头。考虑下面使用标准bitset类型的例子:

\begin{lstlisting}[style=styleCXX]
template<unsigned long N>
void printBitset (std::bitset<N> const& bs) {
	std::cout << bs.template to_string<char, std::char_traits<char>,
			std::allocator<char>>();
}
\end{lstlisting}

对于bitset bs,使用to\_string()的成员函数模板,同时显式指定字符串类型的信息。如果没有使用.template,编译器就不知道后面的小于标记(<)是模板参数列表的开头。注意,只有在句点之前的构造依赖于模板参数时才会出现问题。在例子中,参数bs依赖于模板参数N。

.template表示法(以及类似的表示法,如\texttt{->}template和::template)应该只在模板内部使用,只有当它们遵循依赖于模板的某些参数时才会使用。详细信息请参见13.3.3。

\subsubsubsection{5.5.2\hspace{0.2cm}泛型Lambda和成员模板}

注意,C++14中引入的泛型Lambda是成员模板的快捷方式。计算任意类型的两个参数的“加和”的简单Lambda表达式:

\begin{lstlisting}[style=styleCXX]
[] (auto x, auto y) {
	return x + y;
}
\end{lstlisting}

是以下类默认构造对象的快捷方式:

\begin{lstlisting}[style=styleCXX]
class SomeCompilerSpecificName {
	public:
	SomeCompilerSpecificName(); // constructor only callable by compiler
	template<typename T1, typename T2>
	auto operator() (T1 x, T2 y) const {
		return x + y;
	}
};
\end{lstlisting}

详细信息请参见15.10.6。







