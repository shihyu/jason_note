

可以接受任意数量的模板参数,具有这种能力的模板称为可变参数模板。

\subsubsubsection{4.1.1\hspace{0.2cm}示例}

可以使用print()来获取可变类型的参数:

\hspace*{\fill} \\ %插入空行
\noindent
\textit{basics/varprint1.hpp}
\begin{lstlisting}[style=styleCXX]
#include <iostream>

void print ()
{
}

template<typename T, typename... Types>
void print (T firstArg, Types... args)
{
	std::cout << firstArg << '\n'; // print first argument
	print(args...); // call print() for remaining arguments
}
\end{lstlisting}

若传递了一个或多个参数,则使用函数模板,通过指定第一个参数,可以在递归调用其余参数的print()前,打印第一个参数。剩下的args是一个函数参数包:

\begin{lstlisting}[style=styleCXX]
void print (T firstArg, Types... args)
\end{lstlisting}

使用由模板参数包指定的不同“类型”:

\begin{lstlisting}[style=styleCXX]
template<typename T, typename... Types>
\end{lstlisting}

要结束递归,需要提供print()的非模板重载,该重载在参数包为空时使用。

例如,

\begin{lstlisting}[style=styleCXX]
std::string s("world");
print (7.5, "hello", s);
\end{lstlisting}

将会输出以下内容:

\begin{tcblisting}{commandshell={}}
7.5
hello
world
\end{tcblisting}

调用首先展开为

\begin{lstlisting}[style=styleCXX]
print<double, char const*, std::string> (7.5, "hello", s);
\end{lstlisting}

\begin{itemize}
\item 
firstArg的值为7.5,因此类型T是一个double类型

\item 
args是可变参数模板参数,其值为char const*型的"hello"和std::string型的"world"。
\end{itemize}

将7.5作为firstArg打印后,再次对其余参数调用print(),然后展开为:

\begin{lstlisting}[style=styleCXX]
print<char const*, std::string> ("hello", s);
\end{lstlisting}

\begin{itemize}
\item 
firstArg的值为"hello",所以T的类型是char const*

\item 
args是可变参数,其值类型为std::string。
\end{itemize}

将"hello"作为firstArg打印后,再次对其余参数调用print(),然后展开为:

\begin{lstlisting}[style=styleCXX]
print<std::string> (s);
\end{lstlisting}

\begin{itemize}
\item 
firstArg 的值为“world”,所以类型T现在是std::string

\item 
args是空的可变参数模板参数。
\end{itemize}

因此,在将"world"打印作为firstArg之后,调用print()时不带参数,这将调用print()的非模板重载。

\subsubsubsection{4.1.2\hspace{0.2cm}重载可变和非可变模板}

也可以这样实现上面的例子:

\hspace*{\fill} \\ %插入空行
\noindent
\textit{basics/varprint2.hpp}
\begin{lstlisting}[style=styleCXX]
#include <iostream>

template<typename T>
void print (T arg)
{
	std::cout << arg << '\n'; // print passed argument
}

template<typename T, typename... Types>
void print (T firstArg, Types... args)
{
	print(firstArg); // call print() for the first argument
	print(args...); // call print() for remaining arguments
}
\end{lstlisting}

若两个函数模板的区别仅在于末尾参数包的不同,则首选没有末尾参数包的函数模板。

\begin{tcolorbox}[colback=webgreen!5!white,colframe=webgreen!75!black]
\hspace*{0.75cm}起初在C++11和C++14中,这是一个不明确的问题,后来得到了解决(参见[CoreIssue1395]),并且所有版本的编译器都会这样处理。
\end{tcolorbox}

C.3.1节解释了这里应用的重载解析规则。

\subsubsubsection{4.1.3\hspace{0.2cm}操作符sizeof...}

C++11还为可变参模板引入了一种新的操作符:sizeof...,其值为参数包包含的元素数量。因此,

\begin{lstlisting}[style=styleCXX]
template<typename T, typename... Types>
void print (T firstArg, Types... args)
{
	std::cout << sizeof...(Types) << '\n'; // print number of remaining types
	std::cout << sizeof...(args) << '\n'; // print number of remaining args
	...
}
\end{lstlisting}

传递给print()的第一个参数之后,输出剩余两次参数的数量。对于模板参数包和函数参数包可以使用sizeof...。

我们可能会认为,递归结束时可以跳过该函数,并在没有参数的情况下不调用它:

\begin{lstlisting}[style=styleCXX]
template<typename T, typename... Types>
void print (T firstArg, Types... args)
{
	std::cout << firstArg << '\n';
	if (sizeof...(args) > 0) { // error if sizeof...(args)==0
		print(args...); // and no print() for no arguments declared
	}
}
\end{lstlisting}

但因为函数模板中所有if的两个分支都要实例化。这里,是否运行是运行时确定,而调用的实例化是编译时确定。因此,如果为一个(最后一个)参数使用print()函数模板,则调用print(args...)的语句在没有参数的情况下仍然会实例化,若没有参数的情况下不提供print()函数,则会报错。

但请注意,由于C++17提供了编译时if,就可以实现了这里的期望,但在语法上略有不同,这将在第8.5节中讨论。
























