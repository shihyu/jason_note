还可以为函数模板定义非类型参数。例如,下面的函数模板定义了一组函数,可以为其加一个数:

\hspace*{\fill} \\ %插入空行
\noindent
\textit{basics/addvalue.hpp}
\begin{lstlisting}[style=styleCXX]
template<int Val, typename T>
T addValue (T x)
{
	return x + Val;
}
\end{lstlisting}

如果函数或操作当作参数,这些类型的函数可能很有用。例如使用C++标准库,可以通过函数模板的实例化来给集合的每个元素加一个数:

\begin{lstlisting}[style=styleCXX]
std::transform (source.begin(), source.end(), // start and end of source
dest.begin(), // start of destination
addValue<5,int>); // operation
\end{lstlisting}

最后一个参数实例化函数模板addValue<>(),将5加到传入的int值上。对source中的每个元素调用生成的函数,同时将其转换为目标dest。

注意,必须为模板参数addValue<>()指定为int。推导仅适用于即时调用,而std::transform()需要完整的类型来推导其第四个参数的类型。不支持只替换/推导一些模板参数,并查看哪些参数适合,然后推导剩下的参数。

同样,也可以指定模板参数是从前面的参数推导出来的。例如,从传递的非类型派生返回类型:

\begin{lstlisting}[style=styleCXX]
template<auto Val, typename T = decltype(Val)>
T foo();
\end{lstlisting}

或者确保传递的值与传递的类型相同:

\begin{lstlisting}[style=styleCXX]
template<typename T, T Val = T{}>
T bar();
\end{lstlisting}































