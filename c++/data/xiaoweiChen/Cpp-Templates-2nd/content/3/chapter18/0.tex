多态性是将不同的行为与单个泛型表示法关联起来的能力。

\begin{tcolorbox}[colback=webgreen!5!white,colframe=webgreen!75!black]
\hspace*{0.75cm}多态性字面上指的是具有多种形式或形状的情况(来自希腊语polymorphos)。
\end{tcolorbox}

多态性也是面向对象泛型编程的基础,C++中通过类继承和虚函数来支持多态。因为这些机制(至少部分)在运行时处理,所以这里讨论动态多态性。通常所说的多态性,指的就是动态多态性。然而,模板还允许将不同的特定行为与单个泛型表示关联起来,但这种关联通常在编译时进行处理,称之为静态多态性。本章中,我们回顾了多态的两种形式,并讨论哪种形式适合于哪种情况。

注意,介绍和讨论了一些设计问题之后,第22章将讨论一些处理多态性的方法。