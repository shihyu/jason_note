FunctorBridge的每个实例都是抽象类,因此其派生类负责提供虚函数的实现。为了支持潜在函数对象的完整范围(无界集合),需要无界的派生类。可以通过对派生类存储的函数对象类型,进行参数化来实现:

\hspace*{\fill} \\ %插入空行
\noindent
\textit{bridge/specificfunctorbridge.hpp}
\begin{lstlisting}[style=styleCXX]
template<typename Functor, typename R, typename... Args>
class SpecificFunctorBridge : public FunctorBridge<R, Args...> {
	Functor functor;
	public:
	template<typename FunctorFwd>
	SpecificFunctorBridge(FunctorFwd&& functor)
	: functor(std::forward<FunctorFwd>(functor)) {
	}
	virtual SpecificFunctorBridge* clone() const override {
		return new SpecificFunctorBridge(functor);
	}
	virtual R invoke(Args... args) const override {
		return functor(std::forward<Args>(args)...);
	}
};
\end{lstlisting}

每个SpecificFunctorBridge实例存储一个函数对象的副本(类型是Functor),可以调用、复制或销毁(析构函数中隐式进行)。当一个FunctionPtr初始化为新的函数对象时,特定的functorbridge实例将创建,完成FunctionPtr示例:

\hspace*{\fill} \\ %插入空行
\noindent
\textit{bridge/functionptr-init.hpp}
\begin{lstlisting}[style=styleCXX]
template<typename R, typename... Args>
template<typename F>
FunctionPtr<R(Args...)>::FunctionPtr(F&& f)
: bridge(nullptr)
{
	using Functor = std::decay_t<F>;
	using Bridge = SpecificFunctorBridge<Functor, R, Args...>;
	bridge = new Bridge(std::forward<F>(f));
}
\end{lstlisting}

虽然FunctionPtr构造函数在函数对象类型F上模板化的,但该类型只有特定的SpecificFunctorBridge特化才知晓(由Bridge类型别名描述)。当为新Bridge实例分配数据成员bridge,因为类型从Bridge *到FunctorBridge<R, Args...> *,所以关于特定类型F的信息将丢失。

\begin{tcolorbox}[colback=webgreen!5!white,colframe=webgreen!75!black]
\hspace*{0.75cm}虽然类型可以通过dynamic\_cast查询,FunctionPtr类使bridge指针私有化,所以FunctionPtr的外部代码不能访问类型本身。
\end{tcolorbox}

这种类型信息的丢失解释了,为什么使用术语类型擦除,来描述静态和动态多态之间的桥接技术。

该实现的特点是使用std::decay(请参阅第D.4节)来产生Functor类型,这使得推导类型F适合存储,例如:通过将对函数类型的引用转换成函数指针类型,并移除const、volatile和引用类型的限定。







































