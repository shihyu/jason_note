每个大项目都有自己的一组依赖关系,处理这些依赖关系的最简单方法是使用包管理器,例如Conan或vcpkg。但是由于公司策略、项目需求或缺乏资源,使用包管理器并不总是可行。因此,项目作者可能会参考传统的、老式的方法来处理依赖性。处理这些依赖关系的通常方法可能包括将所有的依赖关系嵌入到存储库的构建代码中。或者,项目作者可能决定让最终用户从头处理依赖项。这两种方法都不优雅,有自己的缺点。

所以呢?

可以使用超级构建。

超级构建是一种方法,可用于从项目代码中分离满足依赖关系所需的逻辑,类似于包管理器的工作方式,我们可以称这个方法为穷人版的包管理器。将依赖逻辑从项目代码中分离出来,可以使我们拥有更灵活和可维护的项目结构。本章中,将详细了解如何实现。

我们将讨论以下主题:

\begin{itemize}
\item 
超级构建的要求和需求

\item 
跨多个代码库的构建

\item 
超级构建中的版本一致性
\end{itemize}


















