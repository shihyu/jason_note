
对于交叉编译工具链很难的误解可能源于这样一个事实,互联网上有许多过于复杂的工具链文件。其中许多是为CMake的早期版本编写的,因此实现需要许多额外的测试和检查,现在这些测试和检查现在已经内置为CMake的一部分。CMake工具链文件基本上做以下几件事:

\begin{itemize}
\item 
定义目标系统和架构。

\item 
为平台构建软件提供所需工具的路径,通常是编译器。

\item 
为编译器和链接器设置默认标志。

\item 
若是交叉编译,则指向sysroot和暂存目录。

\item 
为CMake的\texttt{find\_}指令设置搜索顺序提示。更改搜索顺序是项目可以定义的,这属于工具链文件,还是应该由项目处理信息,目前具有争议。
\end{itemize}

执行这些操作的工具链示例如下所示:

\begin{lstlisting}[style=styleCMake]
set(CMAKE_SYSTEM_NAME Linux)
set(CMAKE_SYSTEM_PROCESSOR arm)

set(CMAKE_C_COMPILER /usr/bin/arm-linux-gnueabi-gcc-9)
set(CMAKE_CXX_COMPILER /usr/bin/arm-linux-gnueabihf-g++-9)

set(CMAKE_C_FLAGS_INIT -pedantic)
set(CMAKE_CXX_FLAGS_INIT -pedantic)

set(CMAKE_SYSROOT /home/builder/raspi-sysroot/)
set(CMAKE_STAGING_PREFIX /home/builder/raspi-sysroot-staging/)

set(CMAKE_FIND_ROOT_PATH_MODE_PROGRAM NEVER)
set(CMAKE_FIND_ROOT_PATH_MODE_LIBRARY ONLY)
set(CMAKE_FIND_ROOT_PATH_MODE_INCLUDE ONLY)
set(CMAKE_FIND_ROOT_PATH_MODE_PACKAGE BOTH)
\end{lstlisting}

本例将定义一个工具链,目标是在Advanced RISC Machine (ARM)处理器上运行的Linux操作系统的构建。要使用的C和C++编译器使用一个版本的GCC,并且安装在主机系统的/usr/bin/文件夹中。编译器用-pedantic标记严格的输出,输出国际标准组织(ISO) C和ISO C++标准化要求的所有警告。接下来,用于查找所需库的sysroot设置为/home/builder/raspi-sysroot/,交叉编译时安装程序的暂存目录设置为/home/builder/raspi-sysroot-staging/。最后,更改CMake的搜索行为,以便在主机系统上搜索程序,而在sysroot中搜索库(包括文件和包)。工具链文件是否应该影响搜索行为,目前存在争议。因为只有项目知道要寻找什么,所以在工具链文件中做假设可能会破坏项目结构。但只有工具链知道要使用哪个系统根目录,以及其中包含什么类型的文件,所以让工具链来定义可能比较方便。一种中间方法是使用CMake预设来定义工具链和搜索行为,而不是将其放入项目或工具链文件中。

\subsubsubsection{12.3.1\hspace{0.2cm}定义目标系统}

交叉编译的目标系统由以下三个变量定义:CMAKE\_SYSTEM\_NAME、CMAKE\_SYSTEM\_PROCESSOR和CMAKE\_SYSTEM\_VERSION。与它们对应的是CMAKE\_HOST\_SYSTEM\_NAME、CMAKE\_HOST\_SYSTEM\_PROCESSOR和CMAKE\_HOST\_SYSTEM\_VERSION变量,这些变量描述了执行构建的平台。

CMAKE\_SYSTEM\_NAME变量描述要为其构建软件的目标操作系统,会让CMake将CMAKE\_CROSSCOMPILING设置为true。典型的值是Linux、Windows、Darwin、Android或QNX,也可以使用更具体的平台名称,如WindowsPhone、WindowsCE、WindowsStore等。对于嵌入式设备,CMAKE\_SYSTEM\_NAME变量可设置为Generic。在撰写本书时,CMake文档中还没有支持系统的官方列表,可以检查本地CMake安装中的/Modules/Platform文件夹中的文件。

CMAKE\_SYSTEM\_PROCESSOR变量用于描述平台的硬件架构,没有指定则设置为CMAKE\_HOST\_SYSTEM\_PROCESSOR变量的值。当64位平台交叉编译32位平台时,也应该设置目标处理器架构,即使处理器的类型相同。对于Android和Apple平台,处理器通常不指定。当对Apple目标进行交叉编译时,实际的设备由所使用的SDK定义,该SDK由CMAKE\_OSX\_SYSROOT变量指定。当交叉编译Android,有专门的变量,如CMAKE\_ANDROID\_ARCH\_ABI,CMAKE\_ANDROID\_ARM\_MODE和可选的CMAKE\_ANDROID\_ARM\_NEON来控制目标架构。

定义目标系统的最后一个变量是CMAKE\_SYSTEM\_VERSION,其取决于所构建的系统。对于WindowsCE、WindowsStore和WindowsPhone,它将用于定义要使用的Windows SDK的哪个版本。在Linux上经常省略,或者可能包含目标系统的内核修订版(若与此相关)。

使用CMAKE\_SYSTEM\_NAME、CMAKE\_SYSTEM\_PROCESSOR和CMAKE\_SYSTEM\_VERSION变量,目标平台通常是完全指定的。然而,一些生成器(如Visual Studio)直接支持原生平台,可以使用CMake的-A命令行选项设置体系结构,如下所示:

\begin{tcblisting}{commandshell={}}
cmake -G "Visual Studio 2019" -A Win32 -T host=x64
\end{tcblisting}

当使用预设时,架构设定可以在配置预设中使用,以达到同样的效果。当定义了目标系统,就等于定义了实际构建软件的工具。

一些编译器,如Clang和QNX(非Unix GCC (QNX GCC)),本质上是交叉编译器以平台为参数。为了将参数传递给这些编译器,使用了CMAKE\_<LANG>\_COMPILER\_TARGET变量。对于Clang来说,该值是一个目标三元组,如arm-linux-gnueabihf,而对于QNX GCC,编译器名称和目标的值为gcc\_ntoarmv7le。Clang支持的三元组在其官方文档\url{https://clang.llvm.org/docs/CrossCompilation.html}中有描述。

有关QNX的可用选项,可以参考\url{https://www.qnx.com/developers/docs/}的QNX文档。

因此,使用Clang的工具链文件可能像这样:

\begin{lstlisting}[style=styleCMake]
set(CMAKE_SYSTEM_NAME Linux)
set(CMAKE_SYSTEM_PROCESSOR arm)

set(CMAKE_C_COMPILER /usr/bin/clang)
set(CMAKE_C_COMPILER_TARGET arm-linux-gnueabihf)

set(CMAKE_CXX_COMPILER /usr/bin/clang++)
set(CMAKE_CXX_COMPILER_TARGET arm-linux-gnueabihf)
\end{lstlisting}

本例中,Clang用于为运行在ARM处理器上的Linux系统编译C和C++代码,该系统支持硬件浮点。定义目标系统通常对要使用的构建工具有直接影响。下一节中,我们将研究如何选择编译器和相关工具进行交叉编译。

\subsubsubsection{12.3.2\hspace{0.2cm}选择构建工具}

构建软件时,编译器是主要工具。大多数情况下,设置编译器就足够了。编译器的路径由CMAKE\_<LANG>\_COMPILER缓存变量设置,该变量可以在工具链文件中设置,也可以手动传递给CMAKE。若路径是绝对的,将直接使用;否则,将使用与\texttt{find\_program()}相同的搜索顺序进行搜素,这就是为什么必须谨慎对待更改工具链文件中搜索行为的原因。若工具链文件和用户都没有指定编译器,CMake将根据指定的目标平台和生成器自动选择一个编译器。此外,编译器可以设置在以<LANG>命名的环境变量上。因此,C将设置C编译器,CXX设置C++编译器,ASM设置汇编器等。

一些生成器(如Visual Studio)支持的不同的工具集定义。可以使用-T命令行选项进行设置。以下命令将告诉CMake为Visual Studio生成代码,为32位系统生成二进制文件,但要使用64位编译器来编译:

\begin{tcblisting}{commandshell={}}
cmake -G "Visual Studio 2019" -A Win32 -T host=x64
\end{tcblisting}

这些值也可以通过工具链文件中的CMAKE\_GENERATOR\_TOOLSET变量来设置。这不应该在项目内部设置,因为它破坏了CMake项目文件不受生成器和平台影响的想法。

对于Visual Studio用户来说,通过安装相同版本的预览版和正式发布版,可以安装同一Visual Studio版本的多个实例。若是这种情况,CMAKE\_GENERATOR\_INSTANCE变量可以设置为工具链文件中Visual Studio的绝对安装路径。

通过指定要使用的编译器,CMake将为编译器和链接器选择默认标志,并通过设置CMAKE\_<LANG>\_FLAGS和CMAKE\_<LANG>\_FLAGS\_<CONFIG>使其在项目中可用,其中<LANG>代表编程语言,<CONFIG>表示构建配置,如Debug或Release等。默认的链接器标志是由CMAKE\_<TARGETTYPE>\_LINKER\_FLAGS和CMAKE\_<TARGETTYPE>\_LINKER\_FLAGS\_<CONFIG>变量设置的,其中<TARGETTYPE>是EXE、STATIC、SHARED或MODULE。

要将自定义标志添加到默认标志中,每个变量都有一个\_INIT的变量——例如,CMAKE\_<LANG>\_FLAGS\_INIT。使用工具链文件时,使用\_INIT变量设置必要的标志。用GCC从64位主机为32位目标编译的工具链文件如下所示:

\begin{lstlisting}[style=styleCMake]
set(CMAKE_SYSTEM_NAME Linux)
set(CMAKE_SYSTEM_PROCESSOR i686)

set(CMAKE_C_COMPILER gcc)
set(CMAKE_CXX_COMPILER g++)

set(CMAKE_C_FLAGS_INIT -m32)
set(CMAKE_CXX_FLAGS_INIT -m32)

set(CMAKE_EXE_LINKER_FLAGS_INIT -m32)
set(CMAKE_SHARED_LINKER_FLAGS_INIT -m32)
set(CMAKE_STATIC_LINKER_FLAGS_INIT -m32)
set(CMAKE_MODULE_LINKER_FLAGS_INIT -m32)
\end{lstlisting}

对于简单的项目,设置目标系统和工具链可能已经足够创建二进制文件了,但是对于更复杂的项目,可能需要访问目标系统的库和头文件,需要在工具链文件中指定sysroot。

\subsubsubsection{12.3.3\hspace{0.2cm}设置sysroot}

交叉编译时,所有链接的依赖项必须与目标平台匹配,处理此问题的常用方法是创建sysroot,它是文件夹中目标系统的根文件系统。虽然sysroot可能包含完整的系统,但它们只提供所需的内容。

将CMAKE\_SYSROOT设置为sysroot的路径。若设置了该值,CMake将默认情况下首先在sysroot中查找库和头文件,除非另有指定。大多数情况下,CMake还会自动设置必要的编译器和链接器标志,使工具与sysroot一起工作。

不应该直接在sysroot中安装构建构件的情况下,可以设置CMAKE\_STAGING\_PREFIX变量来提供另一种安装路径。sysroot应该保持干净或者将其挂载为只读时,通常会出现这种情况。CMAKE\_STAGING\_PREFIX设置不会将这个目录添加到CMAKE\_SYSTEM\_PREFIX\_PATH中,所以只有当工具链中的CMAKE\_FIND\_ROOT\_PATH\_MODE\_PACKAGE变量设置为BOTH或NEVER时,才会使用\texttt{find\_package()}来查找安装在暂存目录中的东西。

定义目标系统和设置工具链配置、sysroot和暂存目录通常是交叉编译所需的全部内容。两个例外是对Android和Apple的iOS、tvOS或watchOS的交叉编译。

\subsubsubsection{12.3.4\hspace{0.2cm}交叉编译——Android}

以前,Android的NDK和各种CMake版本之间的兼容性有时很不愉快,因为新版本的NDK突然之间不再像以前的版本那样与CMake工作。然而,现在已经有了很大的改善,因为r23版本的Android NDK现在将使用CMake(3.21或更高的版本)的内部支持来处理Android NDK。Android NDK与CMake集成的官方文档可以在这里看到:\url{https://developer.android.com/ndk/guides/cmake}。

从r23起,NDK提供CMake工具链文件,位于<NDK\_ROOT>/build/cmake/android.toolchain.cmake,可以像常规的工具链文件一样使用。NDK还包括基于clang的工具链所需的所有工具,因此通常不需要定义进一步的工具。为了控制目标平台,以下CMake变量应该通过命令行或CMake预设设置:

\begin{itemize}
\item 
ANDROID\_ABI: 指定要使用的目标应用程序二进制接口(ABI)。有效值为armeabi-v7a、arm64-v8a、x86和x86\_64。为Android交叉编译时,应该始终设置这个值。

\item 
ANDROID\_ARM\_NEON: 启用armeabi-v7a的NEON支持。此变量对其他ABI版本没有影响。当使用r21版本以上的NDK时,NEON支持默认启用,很少需要禁用。

\item 
ANDROID\_ARM\_MODE: 指定是否为armeabi-v7a生成ARM或Thumb指令。有效值是thumb或arm。此变量对其他ABI版本没有影响。

\item 
ANDROID\_LD: 选择是使用默认链接器,还是使用llvm中的实验性的lld。有效值是默认值或lld,但由于lld处于实验状态,这个变量通常在生产构建中忽略。

\item 
ANDROID\_PLATFORM: 指定应用程序支持的最小API级别,格式为\$API\_LEVEL, android-\$API\_LEVEL,或android-\$API\_LETTER,其中\$API\_LEVEL是一个数字,\$API\_LETTER是平台的版本代码,ANDROID\_NATIVE\_API\_LEVEL是变量的别名。虽然没有必要严格地设置API级别,但通常都会这样做。

\item 
ANDROID\_STL: 指定用于此应用程序的标准模板库(STL)。这可以是c++\_static(这是默认值),c++\_shared,none或system。无论是c++\_shared还是c++\_static都需要支持现代C++。系统库只为C库头文件提供new、delete和C++包装器,而没有库提供STL支持。
\end{itemize}

使用CMake来配置Android版本可能像这样:

\begin{tcblisting}{commandshell={}}
cmake -S . -B build --toolchain <NDK_DIR>/build/cmake/android
  .toolchain.cmake -DANDROID_ABI=armeabi-v7a -DANDROID_PLATFORM=23
\end{tcblisting}

这个调用将指定一个需要API级别23或更高的构建,这对应于32位ARM中央处理单元(CPU)的Android 6.0或更高。

使用NDK提供工具链的另一种选择是将CMake指向Android NDK的位置,对于r23版本或更高版本的NDK,这是推荐的方法。然后使用各自的CMake变量进行目标平台的配置,将CMAKE\_SYSTEM\_NAME变量设置为android,将CMAKE\_ANDROID\_NDK变量设置为android NDK的位置,就可以在CMake中使用NDK了。这既可以通过命令行进行,也可以在工具链文件中进行。若ANDROID\_NDK\_ROOT或ANDROID\_NDK环境变量已经设置,将作为CMAKE\_ANDROID\_NDK的值。

以这种方式使用NDK时,配置在CMAKE\_变量上定义,相当于直接调用NDK的工具链文件时使用的变量:

\begin{itemize}
\item 
CMAKE\_ANDROID\_API或CMAKE\_SYSTEM\_VERSION用于指定要构建的最低API级别。

\item 
CMAKE\_ANDROID\_ARCH\_ABI用于指定要使用的ABI模式。

\item
CMAKE\_ANDROID\_STL\_TYPE指定要使用的STL。
\end{itemize}

用Android NDK配置CMake的工具链文件示例如下:

\begin{lstlisting}[style=styleCMake]
set(CMAKE_SYSTEM_NAME Android)
set(CMAKE_SYSTEM_VERSION 21)
set(CMAKE_ANDROID_ARCH_ABI arm64-v8a)
set(CMAKE_ANDROID_NDK /path/to/the/android-ndk-r23b)
set(CMAKE_ANDROID_STL_TYPE c++_static)
\end{lstlisting}

使用Visual Studio生成器为Android交叉编译时,CMake需要安装NVIDIA Nsight Tegra的Visual Studio插件,或使用Android NDK的Visual Studio工具。当使用Visual Studio构建Android二进制文件时,可以通过将CMAKE\_ANDROID\_NDK变量设置为NDK的位置来使用CMake的Android NDK的内置支持。

CMake 3.20起,NDK的最新版本和CMake版本使得交叉编译Android本地代码变得更加容易。交叉编译的另一个特殊情况是针对Apple的iOS、tvOS或watchOS。

\subsubsubsection{12.3.5\hspace{0.2cm}交叉编译——iOS、tvOS或watchOS}

推荐交叉编译Apple的iPhone、Apple TV或Apple Watche的方法是使用Xcode生成器。Apple在为这些设备构建应用程序时使用的工具非常有限,因此使用macOS或运行macOS的虚拟机(VM)是必要的。虽然也可以使用Makefiles或Ninja文件,但需要对Apple系统有更深入的了解才能正确配置。

为了对这些设备进行交叉编译,需要使用Apple设备SDK,并将CMAKE\_SYSTEM\_NAME变量设置为iOS、tvOS或watchOS,如下例所示:

\begin{tcblisting}{commandshell={}}
cmake -S <SourceDir> -B <BuildDir> -G Xcode -DCMAKE_SYSTEM_NAME=iOS
\end{tcblisting}

对于相对现代的sdk和3.14或更高版本的CMake,这通常就够了。默认情况下,使用系统上可用的最新设备SDK,但若确实需要,可以通过将CMAKE\_OSX\_SYSROOT变量设置为SDK的路径来选择不同的SDK。最小目标平台版本可以通过CMAKE\_OSX\_DEPLOYMENT\_TARGET变量指定。

在为iPhone、Apple TV或Apple watch交叉编译时,目标可以是真实的设备,也可以是不同sdk附带的设备模拟器。然而,Xcode内置支持在构建部分切换,所以CMake不需要运行两次。如果选择了Xcode生成器,CMake内部使用xcodebuild命令行工具,它支持-sdk选项来选择所需的SDK。当通过CMake构建时,选项可以像这样使用:

\begin{tcblisting}{commandshell={}}
cmake -build <BuildDir> -- -sdk <sdk>
\end{tcblisting}

把带有指定值的-sdk选项传递给xcodebuild。iOS允许的值为iphoneos或iphoneonessimulator,Apple TV设备允许的值为appletvos或appletvsimulator,Apple watch允许的值为watchos或watchsimulator。Apple嵌入式平台要求对某些构建构件进行强制签名。

对于Xcode生成器,开发团队标识符(ID),通常是大约10个字符的短字符串,可以通过CMAKE\_XCODE\_ATTRIBUTE\_DEVELOPMENT\_TEAM缓存变量来指定。

为Apple嵌入式设备构建时,模拟器可以方便地测试代码,而不需要每次都部署到设备上。这种情况下,最好通过Xcode或xcodebuild进行测试,但对于其他平台,交叉编译的代码可以直接通过CMake和CTest进行测试。



























