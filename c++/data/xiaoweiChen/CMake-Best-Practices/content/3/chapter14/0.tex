软件项目会有一定的生命周期,对于一些项目来说,在10年或更长时间内处于活跃的开发也有可能。但即使项目没有那么长的生命,也会随着时间出现混乱和难以维护的部分。通常,维护一个项目不仅是重构代码或添加特性,这需要保持构建信息和依赖关系的更新。

随着项目变得越来越复杂,构建时间通常也会增加,以至于开发可能会因为漫长的等待而变得乏味。长时间的构建不仅不方便,还可能迫使开发人员走捷径,会让尝试使用项目变得困难。若每次构建都需要数小时才能完成,若每次推送到CI/CD流水中都需要数小时才能返回,那么就很难尝试新的东西。

除了选择一个好的、模块化的项目结构来提高增量构建的效率外,CMake还有一些特性来帮助分析和优化构建时间。若使用CMake还不够,那么使用诸如编译器缓存(ccache)等技术来缓存构建结果,或预编译头文件可以进一步帮助加快增量构建。

优化构建时间可以产生良好的结果,极大地改善开发人员的日常工作,甚至是可以节省很多成本,因为CI/CD流水可能需要更少的资源来构建项目。然而,过度优化的系统可能变得脆弱和易崩溃,并且优化构建时间可能需要与易维护程度进行的权衡。

本章中,将介绍一些维护项目的技巧,以及如何组织项目以控制维护工作。然后,我们将深入分析构建性能,并了解如何加快构建速度。

本章将将讨论以下内容:

\begin{itemize}
\item 
保持CMake项目的可维护性

\item 
对CMake构建进行性能分析

\item 
优化构建性能
\end{itemize}