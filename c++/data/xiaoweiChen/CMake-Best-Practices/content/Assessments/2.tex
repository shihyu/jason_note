

\begin{enumerate}
\item 
以下是答案:
\begin{enumerate}[label=\Alph*]
\item 

\begin{tcblisting}{commandshell={}}
cmake –S . -B ./build -DCMAKE_CXX_COMPILER:STRING=
"/usr/bin/clang++ "
\end{tcblisting}

\item

\begin{tcblisting}{commandshell={}}
cmake –S . -B ./build -G "Ninja"
\end{tcblisting}

\item 

\begin{tcblisting}{commandshell={}}
cmake –S . -B ./build 
-DCMAKE_BUILD_FLAGS_DEBUG:STRING="-Wall"
\end{tcblisting}
\end{enumerate}

\item 
前面在问题1中配置的项目可以通过命令行使用CMake构建:

\begin{enumerate}[label=\Alph*]
\item 
\begin{tcblisting}{commandshell={}}
cmake –S . -B ./build 
-DCMAKE_CXX_COMPILER:STRING= "/usr/bin/clang++ "
\end{tcblisting}

\item

\begin{tcblisting}{commandshell={}}
cmake --build ./build --parallel 8
\end{tcblisting}

\item 

\begin{tcblisting}{commandshell={}}
cmake --build ./build -- VERBOSE=1
\end{tcblisting}
\end{enumerate}

\item 
\begin{tcblisting}{commandshell={}}
cmake --install ./build --prefix=/opt/project
\end{tcblisting}

\item 
\begin{tcblisting}{commandshell={}}
cmake --install ./build --component ch2.libraries
\end{tcblisting}

\item 
它是一个CMake缓存变量,可以通过\texttt{mark\_as\_advanced()}指令将其标记为高级,并在GUI中隐藏。
\end{enumerate}