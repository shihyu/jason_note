
如果对自己的CI/CD流水有足够的信心,则可以进行更进一步的了解。可以部署系统的构件,而不是部署应用程序的构件。有什么区别呢?在接下来的章节中来了解这一点。

\subsubsubsection{9.8.1\hspace{0.2cm}不可变的基础设施}

之前,重点讨论了如何使应用程序的代码在目标基础设施上的可部署性。CI系统创建软件包(如容器),然后由CD进程部署这些软件包。每次流水运行时,基础设施保持不变,但软件不同。

如果正在使用云计算,可以像对待任何其他工件一样对待基础设施。不需要部署容器,可以部署整个虚拟机(VM),例如AWS EC2实例。可以预先构建这样一个VM镜像,作为CI过程的另一个元素。这样,版本化的VM镜像以及部署所需的代码将成为工件,而不是容器本身。

有两个工具(都是由HashiCorp编写的)可以精确地处理这种场景。Packer帮助以一种可重复的方式创建VM镜像,将所有指令存储为代码(通常以JSON文件的形式)。Terrform是一个基础设施作为代码的工具,用来提供所有必要的基础设施资源。这里将使用Packer的输出作为Terraform的输入。通过这种方式,Terraform将创建一个包含以下内容的完整系统:

\begin{itemize}
\item 
实例组

\item 
负载平衡器

\item 
VPC

\item 
包含代码的VM时,使用的其他云元素
\end{itemize}

这一节的标题可能会让你感到困惑。什么是“不可变的基础设施”?而内容中却主张在每次提交后都要改变整个基础设施?如果了解过函数式语言,可能会更清楚不变性的概念。

可变对象是可以改变其状态的对象。基础架构中,这很容易理解:可以登录到VM,并下载最新版本的代码。现在的状态已经和介入之前不一样了。

不可变对象是指无法改变其状态的对象。这意味着无法登录到机器上并更改内容。当从映像部署VM,它就会一直保持原样,直到销毁。这听起来可能非常麻烦,但解决了软件维护的问题。

\subsubsubsection{9.8.2\hspace{0.2cm}不可变的优势}

首先,不可变的基础设施使配置出错的概念过时。没有配置管理,所以也不会有出错的机会。升级也更安全,因为不能以半生不熟的状态结束。这种状态既不是上一个版本也不是下一个版本,而是介于两者之间。部署过程提供二进制信息:机器是否已创建并运行。

为了让不可变基础设施在不影响正常运行时间的情况下工作,还需要以下几点:

\begin{itemize}
\item 
负载平衡

\item 
(一定程度的)冗余
\end{itemize}

毕竟,升级过程包括关闭整个实例。不能依赖这台机器的地址或特定于这台机器的东西。相反,需要至少有两个机器来处理工作负载,同时用最新的版本替换另一个。当完成对一台机器的升级后,可以对另一台机器重复相同的过程。这样,将拥有两个升级的实例,而不会丢失服务,这种策略称为滚动升级。

可以从流程中认识到,在处理无状态服务时,不可变基础结构的工作效果最好。当服务具有某种形式太久时,就很难正确地实现。这时,必须将持久性级别拆分为一个独立对象,例如包含所有应用程序数据的NFS卷。这样的卷可以在实例组中的所有计算机之间共享,每台新机都可以访问前面运行的应用程序留下的公共状态。

\subsubsubsection{9.8.3\hspace{0.2cm}使用Packer构建实例镜像}

考虑到示例应用程序已经是无状态的,可以继续在其上构建不可变的基础设施。由于Packer生成的工件是VM镜像,必须决定想要使用的格式和构建器。这里将示例的重点放在Amazon Web Services上,类似的方法也可以用于其他受支持的提供商。简单的Packer模板可能是这样的:

\begin{tcblisting}{commandshell={}}
{
  "variables": {
    "aws_access_key": "",
    "aws_secret_key": ""
  },
  "builders": [{
    "type": "amazon-ebs",
    "access_key": "{{user `aws_access_key`}}",
    "secret_key": "{{user `aws_secret_key`}}",
    "region": "eu-central-1",
    "source_ami": "ami-0f1026b68319bad6c",
    "instance_type": "t2.micro",
    "ssh_username": "admin",
    "ami_name": "Project's Base Image {{timestamp}}"
  }],
  "provisioners": [{
    "type": "shell",
    "inline": [
      "sudo apt-get update",
      "sudo apt-get install -y nginx"
    ]
  }]
}
\end{tcblisting}

前面的代码将使用EBS构建器为Amazon Web服务构建一个镜像。该镜像将驻留在eu-central-1区域,并将基于ami-5900cc36镜像,这是一个Debian Jessie镜像。这里想让构建器是t2.micro实例(在AWS中相当于一个虚拟机大小)。为了准备镜像,需要运行两个apt-get命令。

也可以重用之前定义的Ansible代码,而不是使用Packer来提供我们的应用程序,可以将Ansible替换为提供程序。代码如下所示:

\begin{tcblisting}{commandshell={}}
{
  "variables": {
    "aws_access_key": "",
    "aws_secret_key": ""
  },
  "builders": [{
    "type": "amazon-ebs",
    "access_key": "{{user `aws_access_key`}}",
    "secret_key": "{{user `aws_secret_key`}}",
    "region": "eu-central-1",
    "source_ami": "ami-0f1026b68319bad6c",
    "instance_type": "t2.micro",
    "ssh_username": "admin",
    "ami_name": "Project's Base Image {{timestamp}}"
  }],
  "provisioners": [{
    "type": "ansible",
    "playbook_file": "./provision.yml",
    "user": "admin",
    "host_alias": "baseimage"
  }],
  "post-processors": [{
    "type": "manifest",
    "output": "manifest.json",
    "strip_path": true
  }]
}
\end{tcblisting}

更改在供应程序块中,还添加了一个新的块,即后处理器。这一次,不使用shell,而是一个不同的供应程序来运行Ansible。后处理器在这里以机器可读的格式产生构建结果。当Packer完成构建所需的工件,就会返回它的ID并将其保存在manifest.json中。对于AWS来说,这就是一个AMI ID,可以提供给Terraform。

\subsubsubsection{9.8.4\hspace{0.2cm}使用Terraform编排基础设施}

使用Packer创建图像是第一步。之后,希望部署镜像来使用它。可以使用Terraform基于Packer模板中的镜像构建AWS EC2实例。

Terraform代码示例如下所示:

\begin{tcblisting}{commandshell={}}
# Configure the AWS provider
provider "aws" {
  region = var.region
  version = "~> 2.7"
}

# Input variable pointing to an SSH key we want to associate with the
# newly created machine
variable "public_key_path" {
  description = <<DESCRIPTION
Path to the SSH public key to be used for authentication. 
Ensure this keypair is added to your local SSH agent so 
provisioners can connect.
Example: ~/.ssh/terraform.pub
DESCRIPTION

  default = "~/.ssh/id_rsa.pub"
}

# Input variable with a name to attach to the SSH key
variable "aws_key_name" {
  description = "Desired name of AWS key pair"
  default = "terraformer"
}

# An ID from our previous Packer run that points to the custom base image
variable "packer_ami" {
}

variable "env" {
  default = "development"
}
\end{tcblisting}
\begin{tcblisting}{commandshell={}}
variable "region" {
}

# Create a new AWS key pair cotaining the public key set as the input
# variable
resource "aws_key_pair" "deployer" {
  key_name = var.aws_key_name
  public_key = file(var.public_key_path)
}

# Create a VM instance from the custom base image that uses the previously 
# created key
# The VM size is t2.xlarge, it uses a persistent storage volume of 60GiB,
# and is tagged for easier filtering
resource "aws_instance" "project" {
  ami = var.packer_ami
  
  instance_type = "t2.xlarge"
  
  key_name = aws_key_pair.deployer.key_name
  
  root_block_device {
    volume_type = "gp2"
    volume_size = 60
  }

  tags = {
    Provider = "terraform"
    Env = var.env
    Name = "main-instance"
  }
}
\end{tcblisting}

这将创建一个密钥对和一个使用该密钥对的EC2实例,其基于作为变量提供的AMI。在调用Terraform时,将该变量设置为指向Packer生成的镜像。



