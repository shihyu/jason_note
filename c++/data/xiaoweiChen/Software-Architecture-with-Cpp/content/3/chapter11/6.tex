本章中,了解了哪些类型的工具可以帮助代码获得更好的性能。了解了如何进行实验、编写性能测试以及寻找性能瓶颈,使用Google Benchmark编写微基准测试。此外,还讨论了如何分析代码,以及如何(以及为什么)实现系统的分布式跟踪。还讨论了使用标准库实用程序和外部解决方案并行计算。最后,介绍了协程。现在知道了C++20为协程给C++带来了什么,以及可以在cppcoro库中找到什么。还了解了如何编写自己的协程。

本章最重要的是:当涉及到性能时,首先测试,然后优化。这将有助于最大限度地发挥工作的影响。

这就是性能——在书中讨论的最后一个质量属性。下一章中,将开始进入服务和云的世界。我们将从讨论面向服务的架构开始。
























