
本节中,将讨论几种并行计算的不同方法。将从比较线程和进程开始,之后将向展示C++标准中可用的工具,最后,将简单介绍OpenMP和MPI框架。

开始之前,简单介绍一下如何估计代码并行化可能带来的最大收益,这里有两条规律可以给予我们帮助。首先是阿姆达尔定律,如果想通过添加更多的核心来加快程序的速度,那么代码中必须保持顺序(不能并行化)的部分将限制可扩展性。例如,如果代码有90\%是可并行的,那么即使核数无限,仍然只能获得10倍的加速。即使将90\%的执行时间减少到零,10\%的代码仍然会留在那里。

第二定律是古斯塔夫森定律,每个足够大的任务都可以并行化。通过增加问题的规模,可以获得更好的并行化(假设有免费的计算资源可以使用)。换句话说,有时添加更多的功能以在同一时间段内运行,而不是试图减少现有代码的执行时间,这是更好的做法。如果可以通过加倍核心来减少一半的任务时间,在某些情况下,一次次地加倍会让性能回报递减,所以处理能力可以花在其他地方。

\subsubsubsection{11.4.1\hspace{0.2cm}理解线程和进程的区别}

为了有效地并行化计算,还需要了解何时使用进程来执行计算,以及何时线程是这项工作的更好工具。长话短说,如果目标是实际并行工作,那么最好从添加额外的线程开始。此时,在网络中的其他机器上添加更多进程,每个机器也有多个线程。

这是为什么呢?因为进程比线程更重量级。生成一个进程并在它们之间切换所花费的时间,比创建线程和在线程之间切换要长。每个进程需要自己的内存空间,而同一个进程中的线程可以共享内存。而且,进程间通信比线程之间传递变量要慢。使用线程比使用进程更容易,开发速度也会更快。

然而,进程在单个应用程序的范围内也有其用途。非常适合隔离那些可以独立运行和崩溃的组件,而不会破坏整个应用程序。拥有独立的内存还意味着一个进程不能窥探另一个进程的内存,当运行可能是恶意的第三方代码时,这是非常好。这两个原因就是为什么它们用于网络浏览器和其他应用程序。除此之外,还可以运行具有不同操作系统权限或特权的不同进程,这是多线程无法实现的。

现在讨论在单个机器范围内并行工作的一种简单方法。

\subsubsubsection{11.4.2\hspace{0.2cm}标准的并行算法}

如果执行的计算可以并行化,有两种方法可以使用。一种是用可并行的算法替换对标准库算法的常规调用。如果不熟悉并行算法,它们是在C++17中添加,本质上是相同的算法,但可以为传递一个执行策略。有三种执行策略:

\begin{itemize}
\item 
\texttt{std::execution::seq}:以非并行方式执行算法的顺序策略。

\item
\texttt{std::execution::par}:一种并行策略,表明执行可能并行化,通常在后台使用线程池。

\item
\texttt{std::execution::par\_unseq}: 一种并行策略,表示执行可以并行化和向量化。

\item
\texttt{std::execution::unseq}:C++20的新成员。该策略表示可以向量化执行,但不能并行化执行。
\end{itemize}

如果上述策略都还不够,那么标准库实现可能会提供其他策略。可能的未来添加可能包括CUDA,SyCL,OpenCL,甚至人工智能处理器的支持。

现在来看看并行算法的实际应用。要对一个数组进行并行排序,可以这样写:

\begin{lstlisting}[style=styleCXX]
std::sort(std::execution::par, v.begin(), v.end());
\end{lstlisting}

简单且容易!尽管在许多情况下,这将产生更好的性能,但在某些情况下,以传统的方式执行算法可能会更好。为什么呢?因为在更多的线程上调度工作需要额外的工作和同步。此外,根据应用程序的架构,可能会影响其他已经存在的线程的性能,并刷新核心数据缓存。这里同样也需要先进行测试。

\subsubsubsection{11.4.3\hspace{0.2cm}使用OpenMP和MPI并行计算}

使用标准并行算法的另一种选择是利用OpenMP的实用程序,是一种简单的方法,只需添加几行代码,就可以并行处理许多类型的计算。如果想在集群中分发代码,可能想了解MPI能做些什么。这两者也可以连接在一起。

使用OpenMP,可以使用各种\texttt{pragmas}来轻松地并行化代码。例如,可以在for循环之前添加\texttt{\#pragma openmp parallel for},以便使用并行线程执行它。这个库可以做更多的事情,比如在GPU和其他加速器上执行计算。

将MPI集成到项目中比添加适当的\texttt{pragma}更困难。这里,需要在代码库中使用MPI API来发送或接收进程之间的数据(使用如\texttt{MPI\_Send}和\texttt{MPI\_Recv}的调用),或执行各种收集和减少操作(调用\texttt{MPI\_Bcast}和\texttt{MPI\_Reduce})。通信可以使用通信器对象进行点对点,或对所有集群的通讯。

根据算法实现的不同,MPI节点可以全部执行相同的代码,也可以根据需要不同。节点将根据排名知道应该如何行为:计算开始时分配的唯一数字。说到这里,使用MPI启动一个进程,应该通过一个包装器运行它,像这样:

\begin{tcblisting}{commandshell={}}
$ mpirun --hostfile my_hostfile -np 4 my_command --with some ./args
\end{tcblisting}

这将从上述文件逐一读取主机,连接到每个主机,并在每个主机上运行\texttt{my\_command}的四个实例,并传递参数。

MPI有很多实现。其中最值得注意的是OpenMPI(不要将其与OpenMP混淆),一些特性提供了容错功能。毕竟,节点宕机的情况并不少见。

本节中,最后一个工具是GNU Parallel,如果想通过生成并行进程来轻松地跨越执行工作的进程,就会发现它很有用。它既可以在一台机器上使用,也可以跨计算集群使用。

说到执行代码的不同方式,那就要说一下C++20中的另一个大主题:协程。



















