
设计软件系统时,通常要处理几十个或数百个不同的需求。为了理解并想出一个好的设计,需要知道哪些是重要的,哪些是可以实现的。不管设计决策是什么,应该学会如何识别最重要的需求,这样就可以首先关注它们,并在尽可能在短时间内交付具有更大价值的产品。

\begin{tcolorbox}[colback=webgreen!5!white,colframe=webgreen!75!black, title=TIP]
\hspace*{0.75cm}应该使用两个指标对需求进行优先级划分:业务价值和对架构的影响。在两个比重排名较高的项目是最重要的,应作优先处理。如果提出了太多这样的需求,应该重新考虑这些需求的优先级。如果无法对优先级进行排序,那么可能是这个系统无法实现。
\end{tcolorbox}

ASR对系统架构有影响,其可以是功能性的,也可以是非功能性的。那如何确定哪些是重要的呢?如果将某个特定需求的去掉,就需要创建一个不同的架构,那么这个需求就是ASR。因为需要重新设计系统的某些部分,所以延迟发现这些需求通常会花费时间和金钱。如果不是整个解决方案的话,只能希望这不会再损失其他资源和/或声誉。

\begin{tcolorbox}[colback=webgreen!5!white,colframe=webgreen!75!black, title=TIP]
\hspace*{0.75cm}从架构工作的一开始就将具体的技术应用到架构中,这是一个常见的错误。我们强烈建议首先收集所有的需求,专注于那些对架构重要的需求,然后再决定架构项目所依赖的技术和/或技术栈。
\end{tcolorbox}

既然识别ASR很重要,那就来了解一些有帮助的模式。

\subsubsubsection{3.3.1\hspace{0.2cm}架构指标的意义}

如果有与外部系统集成的需求,这很可能会影响架构。先来了解需求是ASR的一些常见指标:

\begin{itemize}
\item 
\textbf{需要创建软件组件进行处理}: 包括发送电子邮件、推送通知、与公司的SAP服务器交换数据或使用特定的数据存储。

\item 
\textbf{对系统影响较大}: 核心功能通常会定义系统应该如何。减少关注点,授权、审核或具有事务性行为,也是很好的例子。

\item 
\textbf{难以实现}: 低延迟是一个很好的例子:除非在开发早期就考虑到这一点,否则实现这可能会是一场鏖战,特别是意识到在热路径上无法真正的进行垃圾收集时。

\item 
\textbf{强制对特定架构进行权衡}: 如果成本太高,设计决策可能需要对某些需求进行折衷,以支持其他更重要的需求。将这些决策记录在某个地方,并注意到这里处理的是ASR,这是一个很好的实践。如果需求以某种方式限制了架构师或产品,那么它很可能对架构产生重大影响。如果想要得到最好的架构,就一定要阅读\textbf{架构权衡分析方法(ATAM)},可以在扩展阅读中找到它。
\end{itemize}

应用程序运行的约束和环境也会影响架构。嵌入式应用程序,需要以不同于云中应用的方式进行设计,经验较少的开发者在开发应用的过程中,应该使用简单、安全的框架,而不是使用陡峭的学习曲线或自己开发的框架。

\subsubsubsection{3.3.2\hspace{0.2cm}ASR的识别及处理方法}

与直觉相反,许多架构的重要需求不明显,这是由两个因素造成的。它们可能很难定义,即使它们被描述了,也可能是模糊的。客户可能不清楚他们需要什么,但作为架构师应该主动询问,避免任何假设。如果系统要发送通知,必须知道这些通知是实时的,还是每天发送一封就足够了,因为前者可能需要建立发布者-订阅者架构。 

大多数情况下,需要做一些假设,因为不是所有事情都已知。如果发现一个需求挑战了之前的假设,那可能是一个ASR。如果假设只在凌晨3点到4点之间维持服务,并且意识到来自不同时区的客户仍需要使用它,那么事实将会挑战假设,并可能改变产品的架构。

人们往往倾向于在项目的早期阶段模糊地对待质量属性,特别是缺乏经验或技术的个人。另一方面,这是解决ASR的最佳时机,因为现在在系统中实现它们,是成本最低的时刻。

许多人在指定需求时,喜欢在没有仔细考虑的情况下使用模糊的短语。如果正在设计一个类似于Uber的服务,一些例子可以是:\textit{当收到一个DriverSearchRequest,系统必须快速回复一个AvailableDrivers消息,或系统必须是可用的24/7}。

提出问题后,会发现每月具有99.9\%的完美可用性,而快速实际上只需几秒钟。这些短语总是需要确认,了解它们背后的基本原理。也许这只是某人的主观意见,没有任何数据或业务需求的支持。另外,在请求和响应的情况下,质量属性隐藏在另一个需求中,这使得它更难捕获。

最后,即使某些系统服务于类似的目的,对于这个系统架构上重要的需求,对于另一个系统并不一定同样重要。随着时间的推移,当系统开始发展,并与越来越多的其他系统进行通信,其中一些服务会变得更加重要。当产品需求发生变化,原来不重要的需求可能会变得很重要。这就是为什么没有什么诀窍可以明确说明哪些需求是ASR,哪些不是。 

了解了如何区分重要需求,就知道要寻找\textit{什么}了。接下来,让我们了解一下去\textit{哪里}了解需求。




























