
已经知道了需要关注哪些需求,那就聊一下收集这些需求的技巧吧。

\subsubsubsection{3.4.1\hspace{0.2cm}了解背景}

挖掘需求时,应该考虑背景。必须确定哪些可能出现的问题可能对产品产生负面影响,这些风险通常来自外部。重新审视一下类似Uber的服务场景,服务可能面临的风险是政策的变化:应该意识到,一些国家可能试图改变政策,将这个产品从他们的市场中移除。Uber缓解这种风险的方法,是让当地的合作伙伴来应对地区性限制。

除了未来的风险之外,还必须了解当前的问题,例如公司中缺乏相关领域的专家,或市场上激烈的竞争。可以这样做:

\begin{itemize}
\item 
关注所有的假设,最好有一个专门的文件来跟踪这些。

\item 
如果可能的话,通过问题来确认或消除假设。

\item 
需要考虑项目内部的依赖关系,可能会影响开发进度。另外是要塑造公司日常的业务规则,产品需要遵守这些规则,并且需要强化这些规则。

\item 
此外,如有足够多的与用户或业务相关的数据,应该尝试挖掘这些数据,以获得深刻的理解,并找到有用的模式,这些模式有助于对产品及其架构的决策。如果已经有一些用户,但无法挖掘数据,那么观察其行为即可。
\end{itemize}

理想情况下,可以使用部署系统执行日常任务时进行记录。通过这种方式,不仅可以自动化部分工作,还可以使工作流程更高效。但没人会心甘情愿的改变习惯,所以最好是逐步引入改变。

\subsubsubsection{3.4.2\hspace{0.2cm}了解现有文档}

现有文档是一个很好的信息来源,尽管它们可能存在问题。应该预留一些时间,至少熟悉所有与当前工作相关的文档,其可能隐藏了一些需求。另一方面,文档从来不是完美的,很可能会缺少一些重要的信息。同样,也需要为它已经过时做好心理准备。当涉及到架构时,除了阅读文档之外,应该与相关人员进行咨询,阅读文档是为咨询做准备的一种方式。

\subsubsubsection{3.4.3\hspace{0.2cm}了解责任相关方}

要成为一名成功的架构师,必须学会在需求直接或间接地来自业务人员时与他们进行沟通。无论他们是来自你的公司还是客户,都应该了解他们的业务背景。例如,必须知道以下内容:

\begin{itemize}
\item 
业务的驱动力是什么?

\item 
公司的目标是什么?

\item 
产品将帮助实现哪些目标?
\end{itemize}

这就需要与来自管理或执行人员等人达成了共识,收集关于软件的需求将会容易得多,例如公司需要保密用户的隐私,可以要求存储尽可能少的用户数据,并使用只存储在用户设备上的密钥对数据进行加密。若这些要求来自于公司文化,那么对一些员工来说就是早已心知肚明的,甚至可以直接进行表达。了解业务背景有助于提出适当的问题,并反过来帮助公司。

话虽如此,责任相关方的需求不一定直接反映在公司目标。对于要提供的功能或软件应该实现的指标,他们也可以有自己的想法,也许会有一个经理承诺给他的员工一个学习新技术或使用特定技术的机会。如果这个项目对他们的职业生涯很重要,则可以成为架构师强有力的盟友,甚至说服别人接受架构师的决定,成为架构师的说客。

另一组重要的责任相关方是负责部署软件的人员。它们可以带有自己的需求子组,称为过渡需求。这些例子包括用户和数据库迁移、基础设施转换或数据转换,所以不要忘记和他们尽早的进行沟通,并收集这些信息。

\subsubsubsection{3.4.4\hspace{0.2cm}从责任相关方处收集需求}

此时,应该有一个责任相关方列表及其角色和联系信息的清单了吧。现在是时候使用它的时候了,一定要花时间与每个相关方的负责人讨论他们从系统中需要什么,以及他们的想法。可以进行面谈,如面对面的会议或小组会议。当与负责人交谈时,也可以帮助他们做出明智的决定——告知他们对最终产品期望结果的可能性。

相关方的负责人通常会说他们所有的需求都同等重要。试着从业务价值的角度说服他们,从而根据已给定的需求,制定优先级。当然,会有一些关键的任务需求,但最可能的情况是,如果一堆需求都无法支持,项目也不会获批,更不用说需求清单中的东西了。

除了面谈,也可以开会组织研讨,就像头脑风暴一样。当有共识达成,并且每个人都知道为什么要参与这样的项目时,就可以开始向每个人了解尽可能多的使用场景。完成了这些,就可以梳理需求和应用场景并形成故事,确定哪些需要优先考虑。最后,需要对所有的故事进行完善的处理。研讨不仅仅是关于功能需求,每个使用场景也可以分配相应的质量属性。进行提炼之后,所有的质量属性都必须可测量。注意,不需要将所有的相关方的负责人都牵扯到此类事件中,因为根据系统的规模,这些事件有时可能需要很长的时间才能完成。

已经了解了如何使用技术和资源来挖掘需求。接下来,来了解一下如何将现有的发现记录到精心制作的文档中。







