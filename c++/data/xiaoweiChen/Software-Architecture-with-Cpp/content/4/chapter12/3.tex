
消息传递有许多不同的用例,从物联网和传感器网络到运行在云中基于微服务的分布式应用程序。

消息传递的好处之一是,连接使用不同技术实现服务的中立。在开发SOA时,每个服务通常由专门的团队开发和维护。团队可以选择他们觉得舒服的工具,这适用于编程语言、第三方库和构建系统。

维护统一的工具集可能会适得其反,因为不同的服务可能有不同的需求。kiosk应用程序可能需要图形用户界面(GUI)库,例如Qt。作为同一应用程序一部分的硬件控制器可能有其他需求,可能链接到硬件制造商的第三方组件。这些依赖可能会施加一些不能同时满足两个组件的限制(例如,GUI应用程序可能需要一个最新的编译器,而硬件对应的可能固定在一个旧的编译器上),使用消息传递系统来解耦这些组件使它们具有独立的生命周期。

消息传递系统的一些用例包括:

\begin{itemize}
\item 
金融业务

\item 
车辆监控

\item 
物流捕获

\item 
处理传感器

\item 
执行数据订单

\item 
任务队列
\end{itemize}

以下部分将重点介绍为低开销设计的消息传递系统,以及用于分布式系统的代理的消息系统。

\subsubsubsection{12.3.1\hspace{0.2cm}低开销的消息传递系统}

低开销的消息传递系统通常用于占用空间小或延迟低的环境中,这些通常是传感器网络、嵌入式解决方案和物联网设备。它们在基于云的分布式服务中不太常见,但也可以在此类解决方案中使用它们。

\hspace*{\fill} \\ %插入空行
\noindent
\textbf{MQTT}

MQTT表示消息队列遥测传输。它是OASIS和ISO下的开放标准。MQTT通常在TCP/IP上使用PubSub模型,也可以与其他传输协议一起工作。

顾名思义,MQTT的设计目标是低代码占用和在低带宽位置运行的可能性。有一个独立的规范称为MQTT-SN,代表传感器网络的MQTT,专注于没有TCP/IP栈的电池供电的嵌入式设备。

MQTT使用消息代理接收来自客户机的所有消息,并将这些消息路由到目的地。服务质素分为三个级别:

\begin{itemize}
\item 
最多发一次(不保证交付)

\item 
至少发一次(确认交付)

\item 
一次发送(保证交付)
\end{itemize}

MQTT在各种物联网应用程序中特别受欢迎,这并不奇怪。它由OpenHAB、Node-RED、Pimatic、Microsoft Azure IoT Hub和Amazon IoT支持。在即时通讯中也很流行,在ejabberd和Facebook messenger中使用。其他用例包括汽车共享平台、物流和运输。

支持该标准的两个最流行的C++库是基于C++ 14和Boost.Asio的Eclipse Paho和mqtt\_cpp。对于Qt应用程序,还有qmqtt。

\hspace*{\fill} \\ %插入空行
\noindent
\textbf{ZeroMQ}

ZeroMQ是一个无代理消息队列。它支持常见的消息传递模式,如PubSub、客户机/服务器和其他几种模式。它独立于特定的传输,可以与TCP、WebSockets或IPC一起使用。

其主要思想包含在名称中,即ZeroMQ不需要任何代理和管理。还提倡提供零延迟,这意味着代理的存在不会增加延迟。

底层库是用C编写的,它有各种流行编程语言的实现,包括C++。主流C++实现是cppzmq,是一个使用C++11的头文件库。

\subsubsubsection{12.3.2\hspace{0.2cm}代理消息传递系统}

主流的两种不关注低开销的消息系统是基于AMQP的RabbitMQ和Apache Kafka。两者都是成熟的解决方案,在许多不同的设计中都非常受欢迎。很多文章关注RabbitMQ或Apache Kafka在某一特定领域的优势。

这是一个稍微不正确的观点,因为两个消息传递系统都基于不同的范例。Apache Kafka专注于将大量数据流化,并将流存储在持久内存中,以允许未来重用。另一方面,RabbitMQ经常用作不同微服务之间的消息代理或处理后台任务的任务队列。因为这个原因,RabbitMQ中的路由比Apache Kafka中的路由要先进得多。Kafka的主要用来进行数据分析和实时处理。

而RabbitMQ使用AMQP协议(也支持其他协议,如MQTT和STOMP), Kafka使用自己的基于TCP/IP的协议。RabbitMQ可以与基于这些受支持协议的其他现有解决方案互操作。如果写了一个使用AMQP与RabbitMQ交互的应用程序,应该可以将其迁移到Apache Qpid, Apache ActiveMQ,或者AWS或Microsoft Azure的托管解决方案。

扩展方面的考虑也可能促使选择一个消息代理而不是另一个。Apache Kafka的架构允许水平扩展,可以向现有的工作池添加更多的机器。另一方面,RabbitMQ在设计时考虑了垂直扩展,这意味着可以向现有机器添加更多的资源,而不是添加更多机器。














