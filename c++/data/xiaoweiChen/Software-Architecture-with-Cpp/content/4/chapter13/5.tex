
微服务之所以如此有用,是因为它们可以通过许多不同的方式与其他服务连接起来,从而创造新的价值。然而,由于微服务没有标准,所以没有单一的方式进行连接。

大多数情况下,想要使用一个特定的微服务时,必须学会如何与它交互。好消息是,尽管在微服务中实现任何通信方法都是可能的,但有一些流行的方法是大多数微服务所遵循的。

如何连接微服务只是围绕设计架构时的相关问题。另一个是连接什么,以及连接在哪里,就是服务发现发挥作用的地方。通过服务发现,让微服务使用自动的方法来发现和连接应用程序中的其他服务。

这三个问题,\textit{如何做,做什么,何处做},将是下一个主题。我们将介绍现代微服务中使用的一些主流的通信和发现方法。

\subsubsubsection{13.5.1\hspace{0.2cm}应用程序编程接口}

就像软件库一样,微服务经常公开API。这些API使得与微服务通信成为可能。由于典型的通信方式利用计算机网络,最流行的API形式是Web API。

前一章中,已经介绍了一些使用Web服务的可能方法。如今,微服务通常使用基于具象状态传输(REpresentational State Transfer,REST)的Web服务。

\subsubsubsection{13.5.2\hspace{0.2cm}远程过程调用}

虽然REST等Web API允许轻松调试和良好的互操作性,但与数据转换和使用HTTP进行传输相关的开销很大。

对于一些微服务来说,这种开销可能太大了,这就是轻量级远程过程调用(Remote Procedure call,RPC)的原因。

\hspace*{\fill} \\ %插入空行
\noindent
\textbf{Apache Thrift}

Apache Thrift是一个接口描述语言和二进制通信协议。它用作RPC方法,允许创建以各种语言构建的分布式和可扩展的服务。

支持多种二进制协议和传输方法。每种编程语言都使用本机数据类型,因此即使在现有的代码库中也很容易引入本机的数据。

\hspace*{\fill} \\ %插入空行
\noindent
\textbf{gRPC}

如果真的关心性能,通常会发现基于文本的解决方案并不适合。REST无论多么优雅和容易理解,对于需求来说可能太慢了。如果是这种情况,应该尝试围绕二进制协议构建API,其中一种是gRPC。

gRPC是最初由Google开发的RPC系统。使用HTTP/2进行传输,并使用协议缓冲区作为接口描述语言(IDL)来实现多种编程语言之间的互操作性和数据序列化。这可以使用替代技术,例如FlatBuffers。gRPC可以以同步和异步的方式使用,并允许创建简单的服务和流服务。 

假设决定使用protobufs,我们的Greeter服务定义如下:

\begin{lstlisting}[style=styleCXX]
service Greeter {
	rpc Greet(GreetRequest) returns (GreetResponse);
}
message GreetRequest {
	string name = 1;
}
message GreetResponse {
	string reply = 1;
}
\end{lstlisting}

使用协议编译器,从这个定义创建数据访问代码。假设想为Greeter创建一个同步服务器,可以按以下方式创建服务:

\begin{lstlisting}[style=styleCXX]
class Greeter : public Greeter::Service {
		Status sendRequest(ServerContext *context, const GreetRequest
	*request,
							GreetReply *reply) override {
		auto name = request->name();
		if (name.empty()) return Status::INVALID_ARGUMENT;
		reply->set_result("Hello " + name);
		return Status::OK;
	}
};
\end{lstlisting}

然后,必须为它构建并运行服务器:

\begin{lstlisting}[style=styleCXX]
int main() {
	Greeter service;
	ServerBuilder builder;
	builder.AddListeningPort("localhost", grpc::InsecureServerCredentials());
	builder.RegisterService(&service);
	
	auto server(builder.BuildAndStart());
	server->Wait();
}
\end{lstlisting}

就这么简单。现在看看使用这个服务的客户端:

\begin{lstlisting}[style=styleCXX]
#include <grpcpp/grpcpp.h>

#include <string>

#include "grpc/service.grpc.pb.h"

using grpc::ClientContext;
using grpc::Status;

int main() {
	std::string address("localhost:50000");
	auto channel =
		grpc::CreateChannel(address, grpc::InsecureChannelCredentials());
	auto stub = Greeter::NewStub(channel);
	
	GreetRequest request;
	request.set_name("World");
	
	GreetResponse reply;
	ClientContext context;
	Status status = stub->Greet(&context, request, &reply);
	
	if (status.ok()) {
		std::cout << reply.reply() << '\n';
	} else {
		std::cerr << "Error: " << status.error_code() << '\n';
	}
}
\end{lstlisting}

这是一个简单的同步示例。为了让它异步工作,需要添加标签和CompletionQueue,如gRPC网站上描述的那样。

gRPC的一个有趣特性是,可用于Android和iOS上的移动应用程序。如果在内部使用gRPC,不需要提供额外的服务器来转换来自移动应用程序的流量。

本节中,了解了微服务使用的最流行的通信和发现方法。接下来,将了解到微服务是如何扩展的。




