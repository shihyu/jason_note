

容器最近引起了很多关注。有人可能会认为它们是一种全新的技术,以前没有。然而,在Docker和Kubernetes崛起之前,已经有像LXC这样的解决方案,它提供了很多类似的功能。

可以通过自1979年以来在UNIX系统中使用的chroot机制,来追溯将一个执行环境与另一个执行环境分离的起源。类似的概念也在FreeBSD jails和Solaris专区中使用。

容器的主要任务是隔离一个执行环境与另一个执行环境。这个隔离的环境可以拥有自己的配置、不同的应用程序,甚至与主机环境不同的用户帐户。

尽管容器与主机是隔离的,但通常共享相同的操作系统内核。这是与虚拟化环境的主要区别。虚拟机有专用的虚拟资源,这意味着在硬件级别的分离。容器在流程级别上是分开的,这意味着运行它们的开销更少。

能够打包并运行已经为运行应用程序进行了优化和配置的另一个操作系统,这是容器的一个强大优势。如果没有容器,构建和部署过程通常包括以下几个步骤:

\begin{enumerate}
\item 
构建应用程序。

\item 
提供配置文件示例。

\item 
准备已安装脚本和相关文档。

\item 
该应用程序是针对目标操作系统(如Debian或Red Hat)打包的。

\item 
这些包部署到目标平台。

\item 
安装脚本为应用程序的运行准备了基础。

\item 
必须调整配置以适应现有的系统。
\end{enumerate}

当切换到容器时,不太需要健壮的安装脚本。应用程序将只针对一个已知的操作系统——容器中存在的操作系统。配置也是如此:与准备许多可配置选项不同,应用程序是为目标操作系统预先配置的,并与之一起分发。部署流程只包括解包容器映像并在其中运行应用程序流程。

虽然集成和微型服务通常是一回事,但它们并非如此。此外,容器可能意味着应用程序容器或操作系统容器,只有应用程序容器适合微服务。将描述可能遇到的不同容器类型,展示与微服务的关系,并解释何时使用(以及何时不使用)。

\subsubsubsection{14.2.1\hspace{0.2cm}容器类型}

目前所描述的容器中,操作系统容器与目前由Docker、Kubernetes和LXD所引领的容器趋势不同。应用程序容器关注的是在容器中运行单个进程——仅是应用程序,而不是使用syslog和cron等服务重新创建整个操作系统。

专有解决方案取代了所有常见的操作系统级服务,这些解决方案提供了统一的方法来管理容器内的应用程序。例如,不使用syslog来处理日志,而将PID为1的进程的标准输出视为应用程序日志。而不是使用init.d等机制或systemd,应用程序容器的生命周期由运行时应用程序处理。

由于Docker目前是应用程序容器的主要解决方案,将在本书中主要使用它作为例子。为了使情况更完整,将提供可行的替代方案,因为它们可能更适合某些需要。由于项目和规范是开源的,这些替代方案与Docker兼容,可以作为替代品使用。

本章的后面,将解释如何使用Docker来构建、部署、运行和管理应用程序容器。

\subsubsubsection{14.2.2\hspace{0.2cm}微服务的兴起}

Docker的成功恰逢微服务的兴起,因为微服务和应用程序容器可以很自然地结合在一起。

如果没有应用程序容器,就没有简单而统一的方式来打包、部署和维护微服务。尽管个别公司开发了一些解决方案来解决这些问题,但没有一个可以成为行业标准。

如果没有微服务,应用程序容器就非常有限。软件架构的重点是构建针对运行在那里的给定服务集显式配置的整个系统。用另一个服务替换一个服务需要改变架构。

当应用程序容器组合在一起时,它为微服务的分发提供了一种标准方式。每个微服务器都自带嵌入式配置,因此自动扩展或自修复等操作不再需要了解底层应用。

仍然可以在没有应用程序容器的情况下使用微服务,也可以在没有托管微服务的情况下使用应用程序容器。例如,尽管PostgreSQL数据库和Nginx Web服务器都没有设计成微服务,但它们通常会用在应用程序容器中。

\subsubsubsection{14.2.3\hspace{0.2cm}何时使用容器}

容器方法有几个好处,操作系统容器和应用程序容器也有它们的优势所在。

\hspace*{\fill} \\ %插入空行
\noindent
\textbf{容器的好处}

与虚拟机(隔离环境的另一种流行方式)相比,容器在运行时需要的开销更少。与虚拟机不同,不需要运行操作系统内核的单独版本,也不需要使用硬件或软件虚拟化技术。应用程序容器也不运行虚拟机中常见的其他操作系统服务,如syslog、cron或init。此外,应用程序容器提供较小的映像,因为它们通常不需要携带整个操作系统副本。在极端的例子中,应用程序容器可以由单个静态链接的二进制文件组成。

如果容器中只有一个二进制文件,为什么还要费心处理容器呢?有一种统一和标准化的方式来构建和运行容器有一个特别的好处。由于容器必须遵循特定的约定,因此它比常规的二进制文件更容易进行编排,后者在日志记录、配置、开放端口等方面可能有不同的期望。

另外,容器提供了一种内置的隔离方法。每个容器都有自己的进程名称空间和用户帐户名称空间。这意味着来自一个容器的进程(或多个进程)没有主机上或其他容器中的进程的概念。沙箱甚至可以更进一步,可以用相同的标准用户界面(无论是Docker、Kubernetes还是其他东西)为容器分配内存和CPU配额。

标准化的运行时还意味着更高的可移植性。构建容器之后,通常可以在不同的操作系统上运行,而无需进行修改。在操作中运行的内容,与在开发中运行的内容非常接近或相同。问题复现更加轻松,调试也是。

\hspace*{\fill} \\ %插入空行
\noindent
\textbf{容器的缺点}

由于现在将工作负载迁移到容器的压力很大,作为架构师,希望了解与这种迁移相关的所有风险,并到处都在吹捧它的好处。

采用容器的主要问题是,并不是所有的应用程序都可以轻松地迁移到容器。对于在设计时考虑到微服务的应用程序容器尤其如此。如果应用程序不是基于微服务体系结构的,那么将其放入容器可能会带来比它能解决的更多的问题。 

如果应用程序已经可以很好地扩展,使用基于TCP/IP的IPC,并且大多数情况下是无状态的,那么迁移到容器应该不会很困难。否则,这些方面中的每一个都将带来挑战,并需要对现有设计进行重新思考。

与容器相关的另一个问题是持久存储。理想情况下,容器应该没有自己的持久存储。这使得快速创建、易于扩展和灵活调度成为可能。问题在于,如果没有持久存储,提供业务价值的应用程序就不可能存在。

通过使大多数容器无状态,并依赖外部非容器化组件来存储数据和状态,这一缺陷通常可以得到缓解。这样的外部组件可以是传统的自托管数据库,也可以是来自云提供商的托管数据库。无论选择哪个方向,都需要重新考虑架构并进行相应地修改。

由于应用程序容器遵循特定的约定,因此必须修改应用程序以遵循这些约定。对于某些应用程序来说,这将是一项简单的任务。对于其他组件,例如使用内存内进程间通信(IPC)的多进程组件,就会很复杂。

经常忽略的是,只要应用程序容器中的应用程序是本机Linux应用程序,应用程序容器就可以工作得很好。虽然支持Windows容器,但它们既不方便,也不像Linux容器那样受支持,还需要作为主机运行授权的Windows机器。

如果从头构建一个新的应用程序,并且可以基于这种技术进行设计,就更容易享受应用程序容器的好处。将现有的应用程序移动到应用程序容器,特别是在复杂的情况下,将需要大量的工作,而且可能还需要修改整个架构。这种情况下,建议考虑所有的好处和坏处。做出错误的决定可能会影响产品的交付时间、可用性和预算。














