
微服务和原生云设计也有自己的问题。即使服务数量有限,不同服务之间的通信、可观察性、调试、速率限制、认证、访问控制和A/B测试也可能具有挑战性。当服务的数量增加时,上述需求的复杂性也会增加。

这就是服务网格进入的地方。简而言之,服务网格权衡了一些资源(运行控制平面和sidecars所必需的),以实现对上述挑战的自动化和集中控制的解决方案。

\subsubsubsection{15.5.1\hspace{0.2cm}引入服务网格}

本章的介绍中,提到的所有需求过去都是在应用程序本身中编码的。事实证明,许多可能可以抽象,因为它们在许多不同的应用程序之间共享。当应用程序包含许多服务时,为所有服务添加新特性的成本将会很高。使用服务网格,可以控制这些特性。

因为容器化的工作流已经对一些运行时和网络进行了抽象,服务网格将这种抽象提升到了另一个层次。这样,容器中的应用程序只知道OSI网络模型的应用程序级别发生了什么,而服务网格处理的任务较为底层。

设置服务网格可以以一种新的方式控制所有的网络流量,并更好地了解这些流量,依赖关系变得可见。

不仅是由服务网格处理的流量。其他流行的模式,如电路断路、速率限制或重试,不必由每个应用程序实现并单独配置,这也是一个可以外包给服务网格的特性。类似地,A/B测试或金丝雀部署是服务网格能够实现的用例。

如前所述,服务网格的好处之一是更好的控制。架构通常由外部流量的可管理边缘代理和内部代理组成,这些代理通常部署在每个微服务上。通过这种方式,可以将网络策略编写为代码,并与所有其他配置一起存储在一个位置。只需在服务网格配置中启用该功能一次,而不必为想要连接的两个服务开启相互TLS加密。

\subsubsubsection{15.5.2\hspace{0.2cm}网格服务解决方案}

这里描述的所有解决方案都是自托管的。

\hspace*{\fill} \\ %插入空行
\noindent
\textbf{Istio}

Istio是一个强大的服务网格工具集合,可以通过将Envoy代理部署为sidecar容器来连接微服务。因为Envoy是可编程的,所以Istio控制平面的配置更改会与所有代理通信,然后这些代理会相应地重新配置自己.

Envoy代理还负责提供加密和身份验证。使用Istio,在服务之间启用相互TLS在大多数时间需要在配置中添加一个交换机。如果不希望在所有服务之间使用mTLS,也可以选择那些需要这种额外保护的服务,同时允许在其他所有服务之间使用未加密的通信。

Istio还可以提供可观察性。首先,Envoy代理导出与Prometheus兼容的代理级指标,Istio导出了服务级别指标和控制台指标。接下来,有描述网格内流量的分布式轨迹。Istio可以为不同的后端提供跟踪:Zipkin,Jaeger,Lightstep和Datadog。最后,还有Envoy访问日志,类似于Nginx的格式显示每个调用。

这是可能的可视化网格使用Kiali,一个交互式的网络界面。通过这种方式,可以看到服务的图表,包括加密是否启用、不同服务之间的流大小或每个服务的健康检查状态等信息。

Istio的作者声称这个服务网格应该与不同的技术兼容。在撰写本文时,具有最好的文档、最好的集成和最好的测试是与Kubernetes的集成。其他受支持的环境有on-premises、通用云、Mesos和Nomad with Consul。

如果在与合规相关的行业(如金融机构)工作,那么Istio可以在这些方面提供帮助。

\hspace*{\fill} \\ %插入空行
\noindent
\textbf{Envoy}

Envoy本身并不是一个服务网格,但由于它在Istio中使用,所以在本节进行介绍。

Envoy是一个服务代理,作用很像Nginx或HAProxy,主要的区别在于可以动态地重新配置。这是通过API以编程方式实现的,不需要更改配置文件,也不需要重新加载守护进程。

关于特使有趣的事实是它的性能和知名度。根据SolarWinds的测试,Envoy在作为服务代理的性能方面优于其他竞争对手,这包括HAProxy,Nginx,Traefik和AWS应用负载均衡器。Envoy比Nginx、HAProxy、Apache和Microsoft IIS等已经建立的领导者要年轻得多,但根据Netcraft的数据,这并不妨碍Envoy进入最常用Web服务器的前10名。

\hspace*{\fill} \\ %插入空行
\noindent
\textbf{Linkerd}

在Istio成为服务网格的代名词之前,这个领域的代表是Linkerd。其在命名方面有一些混淆,因为最初的Linkerd项目设计为平台无关的,并且以Java VM为目标。这意味着它的重资源,而且经常行动迟缓,新版本Linkerd2已经重写以解决这些问题。与最初的Linkerd不同,Linkerd2只关注Kubernetes。

Linkerd和Linkerd2都使用自己的代理解决方案,而不是依赖于像Envoy这样的现有项目。这样做的理由是专用代理(与通用Envoy相比)提供更好的安全性和性能。Linkerd2的一个有趣特性是,开发它的公司还提供付费支持版本。

\hspace*{\fill} \\ %插入空行
\noindent
\textbf{领事服务网格}

服务网格空间最近增加了Consul服务网格。这是HashiCorp的产品,HashiCorp是一家知名的云计算公司,以terrform、Vault、Packer、Nomad和Consul等工具而闻名。

就像其他解决方案一样,具有mTLS和流量管理功能。宣传为多云、多数据中心和多区域网格。集成了不同的平台、数据平面产品和可观察性提供商。在撰写本文时,实际情况要好一些,因为主要支持的平台是Nomad和Kubernetes,而支持的代理要么是内置代理,要么是Envoy。

如果考虑在应用程序中使用Nomad,那么Consul服务网格可能是一个很好的选择,因为它们都是HashiCorp的产品。






















