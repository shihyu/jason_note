
云原生设计描述的是应用程序的架构,首先是在云中运行。它不是由单一的技术或语言定义的,而是利用了现代云平台提供的所有优势。

这可能意味着在必要时结合使用平台即服务(PaaS)、多云部署、边缘计算、功能即服务(FaaS)、静态文件托管、微服务和托管服务,超越了传统操作系统的界限。云本地开发人员使用boto3、Pulumi或Kubernetes等库和框架构建更高级的概念,而不是针对POSIX API和类UNIX操作系统。

本章将讨论以下内容:

\begin{itemize}
\item 
了解原生云

\item 
使用Kubernetes协调云原生工作负载

\item 
使用服务网格连接服务

\item 
分布式系统中的可观测性

\item 
使用GitOps
\end{itemize}

本章结束时,将很好地理解如何在应用程序中使用现代软件架构,及趋势。