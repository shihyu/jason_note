
已经展示了如何使用Conan安装依赖项。现在,尝试创建自己的Conan包。

先在项目中创建一个新的顶层目录,命名为conan,将在其中存储使用此工具打包所需的文件:用于构建包的脚本和测试环境。

\subsubsubsection{7.6.1\hspace{0.2cm}创建conanfile.py脚本}

所有Conan包需要的文件是conanfile.py。在本例中,希望使用CMake变量填充它的一些细节,因此来创建一个conanfile.py.in文件。使用它来创建文件,并添加以下文件到CMakeLists.txt文件中:

\begin{lstlisting}[style=styleCMake]
configure_file(${PROJECT_SOURCE_DIR}/conan/conanfile.py.in
	${CMAKE_CURRENT_BINARY_DIR}/conan/conanfile.py @ONLY)
\end{lstlisting}

文件将从Python的\texttt{import}开始,比如Conan在CMake项目中需要的那些:

\begin{lstlisting}[style=stylePython]
import os
from conans import ConanFile, CMake
\end{lstlisting}

需要创建一个定义包的类:

\begin{lstlisting}[style=stylePython]
class CustomerConan(ConanFile):
	name = "customer"
	version = "@PROJECT_VERSION@"
	license = "MIT"
	author = "Authors"
	description = "Library and app for the Customer microservice"
	topics = ("Customer", "domifair")
\end{lstlisting}

首先,从一堆通用变量开始,从CMake代码中获取项目版本。通常,描述将是一个多行字符串。这些主题对于在JFrog的Artifactory等网站上找到库很有用,并且可以告诉读者我们的包是干什么的。现在来看看其他变量:

\begin{lstlisting}[style=stylePython]
	homepage = "https://example.com"
	url =
	"https://github.com/PacktPublishing/Hands-On-Software-Architecture-with-Cpp
	/"
\end{lstlisting}

\texttt{homepage}应该指向项目的主页:文档、教程、FAQ和类似的东西都可以放在那里。另一方面,\texttt{url}是放置包存储库的位置。许多开放源码库在一个库中包含代码,而在另一个库中包含打包代码。还有种情况是,包是由中央的Conan包服务器构建的,这样\texttt{url}应该指向 \url{https://github.com/conan-io/conan-center-index}。

接下来,可以指定如何构建我们的包:

\begin{lstlisting}[style=stylePython]
	settings = "os", "compiler", "build_type", "arch"
	options = {"shared": [True, False], "fPIC": [True, False]}
	default_options = {"shared": False, "fPIC": True}
	generators = "CMakeDeps"
	keep_imports = True # useful for repackaging, e.g. of licenses
\end{lstlisting}

\texttt{settings}将确定是否需要构建包,或者可以下载已构建的版本。

\texttt{options}和\texttt{default\_options}可以是任何值。\texttt{shared}和\texttt{fPIC}是大多数包提供的两个功能,所以可以遵循这个约定。

现在已经定义了变量,就开始编写Conan用来打包软件的方法。首先,指定包的使用者应该链接的库:

\begin{lstlisting}[style=stylePython]
	def package_info(self):
		self.cpp_info.libs = ["customer"]
\end{lstlisting}

\texttt{self.cpp\_info}对象允许进行更多的设置,但这是最小值。可以查看Conan文档中的其他属性。

接下来,指定其他需要的包:

\begin{lstlisting}[style=stylePython]
	def requirements(self):
		self.requires.add('cpprestsdk/2.10.18')
\end{lstlisting}

这次,直接从Conan获取C++ REST SDK,而不是指定操作系统的包管理器应该依赖哪些包。现在,来指定CMake应该如何(以及在哪里)生成构建系统:

\begin{lstlisting}[style=stylePython]
	def _configure_cmake(self):
		cmake = CMake(self)
		cmake.configure(source_folder="@CMAKE_SOURCE_DIR@")
		return cmake
\end{lstlisting}

本例中,可以简单地将其指向源目录。当构建系统配置好后,就需要构建项目了:

\begin{lstlisting}[style=stylePython]
	def build(self):
		cmake = self._configure_cmake()
		cmake.build()
\end{lstlisting}

Conan还支持非CMake的构建系统。构建完包之后,就可以打包了,这需要提供另一种方法:

\begin{lstlisting}[style=stylePython]
	def package(self):
		cmake = self._configure_cmake()
		cmake.install()
		self.copy("license*", ignore_case=True, keep_path=True)
\end{lstlisting}

注意我们是如何使用相同的\texttt{\_configure\_cmake()}函数来构建和打包项目的。除了安装二进制文件之外,还需要指定在哪里存放部署许可证。最后,告诉Conan在安装包时应该复制哪些东西:

\begin{lstlisting}[style=stylePython]
	def imports(self):
		self.copy("license*", dst="licenses", folder=True,
		ignore_case=True)
		# Use the following for the cmake_multi generator on Windows
		and/or Mac OS to copy libs to the right directory.
		# Invoke Conan like so:
		# conan install . -e CONAN_IMPORT_PATH=Release -g cmake_multi
		dest = os.getenv("CONAN_IMPORT_PATH", "bin")
		self.copy("*.dll", dst=dest, src="bin")
		self.copy("*.dylib*", dst=dest, src="lib")
\end{lstlisting}

上面的代码指定在安装库时,在何处解压缩许可证文件、库和可执行文件。

现在,了解了如何构建一个Conan包,接下来来测试安装包。

\subsubsubsection{7.6.2\hspace{0.2cm}测试Conan包}

当Conan构建了程序包,就应该测试这个包否构建正确。首先在conan目录中创建一个\texttt{test\_package}子目录。

还需要包含一个conanfile.py脚本,但这次是一个更小的脚本。应该是这样开始的:

\begin{lstlisting}[style=stylePython]
import os
from conans import ConanFile, CMake, tools

class CustomerTestConan(ConanFile):
	settings = "os", "compiler", "build_type", "arch"
	generators = "CMakeDeps"
\end{lstlisting}

这里没什么特别的,提供构建测试包的逻辑:

\begin{lstlisting}[style=stylePython]
	def build(self):
		cmake = CMake(self)
		# Current dir is "test_package/build/<build_id>" and
		# CMakeLists.txt is in "test_package"
		cmake.configure()
		cmake.build()
\end{lstlisting}

马上就到写CMakeLists.txt文件的环节了。在写之前,还需要再完成两件事:导入方法和测试方法。导入方法可以这样写:

\begin{lstlisting}[style=stylePython]
	def imports(self):
		self.copy("*.dll", dst="bin", src="bin")
		self.copy("*.dylib*", dst="bin", src="lib")
		self.copy('*.so*', dst='bin', src='lib')
\end{lstlisting}

然后,就有了包测试逻辑的核心——测试方法:

\begin{lstlisting}[style=stylePython]
	def test(self):
		if not tools.cross_building(self.settings):
			self.run(".%sexample" % os.sep)
\end{lstlisting}

我们只希望在为本地架构构建时才运行它。否则,可能无法运行已编译的可执行文件。

现在,就可以来写CMakeLists.txt文件了:

\begin{lstlisting}[style=styleCMake]
cmake_minimum_required(VERSION 3.12)

project(PackageTest CXX)

list(APPEND CMAKE_PREFIX_PATH "${CMAKE_BINARY_DIR}")

find_package(customer CONFIG REQUIRED)

add_executable(example example.cpp)
target_link_libraries(example customer::customer)

# CTest tests can be added here
\end{lstlisting}

就这么简单,链接到所有提供的Conan库(在例子中,只有Customer库)。

最后,检查包是否成功创建了example.cpp文件:

\begin{lstlisting}[style=styleCXX]
#include <customer/customer.h>
int main() { responder{}.prepare_response("Conan"); }
\end{lstlisting}

运行这些前,需要在CMake的主构建树中做一些小的更改。现在来了解一下,如何正确地从CMake文件中导出Conan目标。 

\subsubsubsection{7.6.3\hspace{0.2cm}添加Conan打包代码到CMakeLists}

还记得在重用质量代码的章节中编写的安装逻辑吗?如果依赖于Conan进行打包,可能不需要运行裸CMake导出和安装逻辑。假设只需要在不使用Conan的情况下导出和安装,就需要修改CMakeLists:

\begin{lstlisting}[style=styleCMake]
if(NOT CONAN_EXPORTED)
	install(
		EXPORT CustomerTargets
		FILE CustomerTargets.cmake
		NAMESPACE domifair::
		DESTINATION ${CMAKE_INSTALL_LIBDIR}/cmake/Customer)
	
	configure_file(${PROJECT_SOURCE_DIR}/cmake/CustomerConfig.cmake.in
				   CustomerConfig.cmake @ONLY)
	
	include(CMakePackageConfigHelpers)
	write_basic_package_version_file(
		CustomerConfigVersion.cmake
		VERSION ${PACKAGE_VERSION}
		COMPATIBILITY AnyNewerVersion)
	
	install(FILES ${CMAKE_CURRENT_BINARY_DIR}/CustomerConfig.cmake
				${CMAKE_CURRENT_BINARY_DIR}/CustomerConfigVersion.cmake
			DESTINATION ${CMAKE_INSTALL_LIBDIR}/cmake/Customer)
endif()

install(
	FILES ${PROJECT_SOURCE_DIR}/LICENSE
	 	DESTINATION
	$<IF:$<BOOL:${CONAN_EXPORTED}>,licenses,${CMAKE_INSTALL_DOCDIR}>)
\end{lstlisting}

添加if语句和一个生成器表达式对于拥有干净的包来说是合理的,这就是所要做的事情了。

最后一件事情——可以构建一个目标来创建Conan包:

\begin{lstlisting}[style=styleCMake]
add_custom_target(
	conan
	COMMAND
		${CMAKE_COMMAND} -E copy_directory
		${PROJECT_SOURCE_DIR}/conan/test_package/
		${CMAKE_CURRENT_BINARY_DIR}/conan/test_package
	COMMAND conan create . customer/testing -s build_type=$<CONFIG>
	WORKING_DIRECTORY ${CMAKE_CURRENT_BINARY_DIR}/conan
	VERBATIM)
\end{lstlisting}

当运行\texttt{cmake -\,-build .-\,-target conan}(或\texttt{ninja conan}若已使用生成器并想命令行简单一些),CMake将复制\texttt{test\_package}目录到构建文件夹,构建conan包,并使用复制的文件对其进行测试。

全部完成!

\begin{tcolorbox}[colback=blue!5!white,colframe=blue!75!black, title=Note]
\hspace*{0.7cm}涉及到创建Conan包的内容远不止在这里描述到的,更多信息请参考Conan的文档。可以在扩展阅读找到相应的链接。
\end{tcolorbox}








