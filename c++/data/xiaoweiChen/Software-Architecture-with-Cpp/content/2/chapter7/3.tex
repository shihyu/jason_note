
本节中,将深入研究CMake脚本,它是全世界范围内用于C++项目的标准构建系统生成器。

\subsubsubsection{7.3.1\hspace{0.2cm}CMake}

CMake是一个构建系统生成器,而不是构建系统本身是什么意思呢?简单地说,CMake可以用来生成各种类型的构建系统。可以使用它来生成Visual Studio项目、Makefile项目、基于Ninja的项目、Sublime、Eclipse和其他一些项目。

CMake附带了一组其他工具,如用于执行测试的CTest和用于打包和创建安装程序的CPack。CMake本身也允许导出和安装目标。

CMake的生成器可以是单配置生成器,比如Make或NMAKE,也可以是多配置生成器,比如Visual Studio。对于单配置生成器,第一次在一个文件夹中运行生成时,应该传递\texttt{CMAKE\_BUILD\_TYPE}标志。例如,要配置一个调试版本,可以运行\texttt{cmake <project\_directory> -DCMAKE\_BUILD\_TYPE=Debug}。其他预定义的配置包括Release、RelWithDebInfo(带有调试符号的版本)和MinSizeRel(为最小二进制优化版本)。为了保持源目录的整洁,总是创建一个单独的构建文件夹,并从那里运行CMake生成。

尽管可以添加自己的构建类型,但尽量不要这样做,因为这会使某些IDE使用起来会变得更加困难,而且无法扩展。更好的选择是使用选项。

CMake文件可以用两种风格编写:过时的基于变量的风格,以及基于目标的现代CMake风格。这里我们只关注后者。尽量避免通过全局变量进行设置,在重用目标时,这会导致问题。

\hspace*{\fill} \\ %插入空行
\noindent
\textbf{创建CMake工程}

每个CMake项目在顶部CMakeLists.txt文件中应该有以下几行:

\begin{lstlisting}[style=styleCMake]
cmake_minimum_required(VERSION 3.15...3.19)

project(
	Customer
	VERSION 0.0.1
	LANGUAGES CXX)
\end{lstlisting}

设置支持的最小和最大版本很重要,因为它会通过设置策略影响CMake的行为。如果需要,也可以手动设置。

项目的定义指定了它的名称、版本(将用于填充一些变量)和CMake将用于构建项目(填充更多变量并找到所需的工具)的编程语言。

通常,一个C++项目有以下目录:

\begin{itemize}
\item 
cmake: CMake脚本

\item 
include: 公共头文件,通常带有以项目名称命名的子文件夹

\item 
src: 源文件和私有头文件

\item 
test: 测试
\end{itemize}

可以使用CMake目录来存储定制的CMake模块。为了方便地从这个目录访问脚本,可以添加到CMake的\texttt{include()}搜索路径中,如下所示:

\begin{lstlisting}[style=styleCMake]
list(APPEND CMAKE_MODULE_PATH "${CMAKE_CURRENT_LIST_DIR}/cmake"
\end{lstlisting}

当包含CMake模块时,可以省略\texttt{.cmake}后缀。这样,\texttt{include(CommonCompileFlags.cmake)}就等价于\texttt{include(CommonCompileFlags)}。

\hspace*{\fill} \\ %插入空行
\noindent
\textbf{CMake内置目录变量}

在CMake中目录有一个常见的陷阱,并不是每个人都知道。当编写CMake脚本时,试着区分以下内置变量:

\begin{itemize}
\item 
\texttt{PROJECT\_SOURCE\_DIR}: 项目命令CMake脚本调用的目录。

\item 
\texttt{PROJECT\_BINARY\_DIR}: 与前面的示例类似,但用于构建目录。

\item 
\texttt{CMAKE\_SOURCE\_DIR}: 顶层源目录(这可能在另一个项目中,只是将当前项目的作为依赖项/子目录添加)。

\item 
\texttt{CMAKE\_BINARY\_DIR}: 类似于\texttt{CMAKE\_SOURCE\_DIR},但用于构建目录。

\item 
\texttt{CMAKE\_CURRENT\_SOURCE\_DIR}: 对应于当前处理的CMakeLists.txt文件的源目录。

\item 
\texttt{CMAKE\_CURRENT\_BINARY\_DIR}: 与\texttt{CMAKE\_CURRENT\_SOURCE\_DIR}匹配的二进制(构建)目录。

\item 
\texttt{CMAKE\_CURRENT\_LIST\_DIR}: \texttt{CMAKE\_CURRENT\_LIST\_FILE}目录。如果当前的CMake脚本包含在另一个目录中(对于包含的CMake模块来说很常见),可以不同于当前的源目录。
\end{itemize}

在顶层的CMakeLists.txt文件中,可能需要调用\texttt{add\_subdirectory(src)}以便CMake处理该目录。

\hspace*{\fill} \\ %插入空行
\noindent
\textbf{指定CMake目标}

src目录中,应该有另一个CMakeLists.txt文件,这次可能定义了一个或两个目标。现在为前面提到的多米尼加博览会系统,添加一个客户微服务的可执行文件:

\begin{lstlisting}[style=styleCMake]
add_executable(customer main.cpp)
\end{lstlisting}

源文件可以在前面的代码行中指定,或者使用\texttt{target\_sources}添加。

常见的CMake反模式是使用globs指定源文件,这种方式的缺点是CMake在重新运行生成步骤之前,不会知道文件是否已经添加。这样做的后果是,如果从存储库中提取更改并简单地构建,可能会错过编译和运行新的单元测试或其他代码。即使使用带有\texttt{CONFIGURE\_DEPENDS}的globs,构建时间也会变长,因为globs必须作为每个构建的一部分进行检查。此外,该标志可能并不适用于所有生成器。甚至CMake的作者也不喜欢使用它,而是喜欢显式地声明源文件。

好的,定义了代码来源。现在指定目标需要编译器支持C++17:

\begin{lstlisting}[style=styleCMake]
target_compile_features(customer PRIVATE cxx_std_17)
\end{lstlisting}

\texttt{PRIVATE}关键字说明这是一个内部需求,也就是只对这个特定目标可见,但对依赖它的目标不可见。如果正在编写一个为用户提供C++17 API的库,那么可以使用\texttt{INTERFACE}关键字。要指定接口和内部需求,可以使用\texttt{PUBLIC}关键字。当用户链接到目标时,CMake也会自动要求支持C++ 17。如果正在编写一个未构建的目标(即仅包含头文件的库或导入的目标),使用\texttt{INTERFACE}关键字通常就足够了。

指定目标使用C++17特性,并不会强制执行C++标准,也不允许对目标进行编译器扩展。为此,应该这样使用:

\begin{lstlisting}[style=styleCMake]
set_target_properties(customer PROPERTIES
	CXX_STANDARD 17
	CXX_STANDARD_REQUIRED YES
	CXX_EXTENSIONS NO
)
\end{lstlisting}

若想要有一组编译器标志来传递给每个目标,可以将它们存储在一个变量中。若想创建一个目标,将这些标志设置为\texttt{INTERFACE},并且没有任何源文件,并在\texttt{target\_link\_libraries}中使用这个目标:

\begin{lstlisting}[style=styleCMake]
target_compile_options(customer PRIVATE ${BASE_COMPILE_FLAGS})
\end{lstlisting}

除了添加链接器标志外,该命令还自动传播包括目录、选项、宏和其他属性。说到链接,先创建一个要链接的库:

\begin{lstlisting}[style=styleCMake]
add_library(libcustomer lib.cpp)
add_library(domifair::libcustomer ALIAS libcustomer)
set_target_properties(libcustomer PROPERTIES OUTPUT_NAME customer)
# ...
target_link_libraries(customer PRIVATE libcustomer)
\end{lstlisting}

\texttt{add\_library}可用于创建静态、动态、对象和接口(仅考虑头文件)库,以及定义导入库。

它的\texttt{ALIAS}版本创建了一个命名空间目标,这有助于调试许多CMake问题,是现代CMake推荐的实践方式。

因已经给目标一个lib前缀,所以输出名称为ibcustomer.a,而不是liblibcustomer.a。

最后,将可执行文件与添加的库链接起来。尽量为\texttt{target\_link\_libraries}命令指定\texttt{PUBLIC}、\texttt{PRIVATE}或\texttt{INTERFACE}关键字,因为这对于CMake管理目标依赖项的可传递性至关重要。

\hspace*{\fill} \\ %插入空行
\noindent
\textbf{指定输出目录}

当使用\texttt{cmake -\,-build .}等命令构建代码,可能想知道在哪里可以找到构建好的组件。默认情况下,CMake将在与定义它们的源目录相匹配的目录中创建它们。例如,如果有一个\texttt{src/CMakeLists.txt}文件,其中有一个\texttt{add\_executable}指令,那么这个二进制文件将默认放在构建目录的src子目录中。也可以使用以下代码进行覆盖:

\begin{lstlisting}[style=styleCMake]
set(CMAKE_RUNTIME_OUTPUT_DIRECTORY ${PROJECT_BINARY_DIR}/bin)
set(CMAKE_ARCHIVE_OUTPUT_DIRECTORY ${PROJECT_BINARY_DIR}/lib)
set(CMAKE_LIBRARY_OUTPUT_DIRECTORY ${PROJECT_BINARY_DIR}/lib)
\end{lstlisting}

通过这种方式,二进制文件和DLL文件将放在项目的构建目录的bin子目录中,而静态和动态Linux库将放在lib子目录中。

\subsubsubsection{7.3.2\hspace{0.2cm}生成器表达式}

在设置编译标志方面,同时支持单配置和多配置生成器可能会很棘手,因为CMake在配置时执行\texttt{if}语句和许多其他构造,而不是在构建/安装时执行。

这意味着以下是CMake的反模式:

\begin{lstlisting}[style=styleCMake]
if(CMAKE_BUILD_TYPE STREQUAL Release)
	target_compile_definitions(libcustomer PRIVATE RUN_FAST)
endif()
\end{lstlisting}

相反,生成器表达式是实现相同目标的正确方法,因为它们将在稍后处理。若想为Release配置添加一个预处理器定义,可以这样写:

\begin{lstlisting}[style=styleCMake]
target_compile_definitions(libcustomer PRIVATE
"$<$<CONFIG:Release>:RUN_FAST>")
\end{lstlisting}

这将解析为在特定的配置下,对\texttt{RUN\_FAST}进行构建。对于其他配置,将解析为空值。它适用于单配置和多配置生成器。不过,这并不是生成器表达式的唯一用例。

当项目在构建期间使用目标,而其他项目在安装目标时,使用目标的方式可能会有所不同。比如\texttt{include}目录,在CMake中处理这个的常见方法如下:

\begin{lstlisting}[style=styleCMake]
target_include_directories(
	libcustomer PUBLIC $<INSTALL_INTERFACE:include>
		$<BUILD_INTERFACE:${PROJECT_SOURCE_DIR}/include>)
\end{lstlisting}

本例中,有两个生成器表达式。第一个在安装时可以在\texttt{include}目录中找到包含文件,相对于安装前缀(安装的根目录)。如果不安装,这个表达式将变成一个空表达式。这就是为什么有另一个表达“构建”的短语,这个将解析为找到最后一个使用的\texttt{project()}目录的\texttt{include}子目录。

\begin{tcolorbox}[colback=blue!5!white,colframe=blue!75!black, title=Note]
\hspace*{0.7cm}不要使用带有模块外部路径的\texttt{target\_include\_directories}。如果这样做,就是在窃取别人的头文件,而不是显式声明库/目标依赖。这是CMake的一个反模式。
\end{tcolorbox}

CMake定义了许多生成器表达式,可以使用它们来查询编译器和平台,以及目标(例如全名、目标文件列表、任何属性值等)。除此之外,还有运行布尔运算、\texttt{if}语句、字符串比较等的表达式。

现在,对于更复杂的例子,若想要在目标中使用一组编译标志,并且依赖于使用的编译器,可以这样定义:

\begin{lstlisting}[style=styleCMake]
list(
		APPEND
		BASE_COMPILE_FLAGS
	"$<$<OR:$<CXX_COMPILER_ID:Clang>,
	$<CXX_COMPILER_ID:AppleClang>,$<CXX_COMPIL
	ER_ID:GNU>>:-Wall;-Wextra;-pedantic;-Werror>"
		"$<$<CXX_COMPILER_ID:MSVC>:/W4;/WX>")
\end{lstlisting}

如果编译器是Clang、AppleClang或GCC,这将附加一组标志,如果使用MSVC则附加另一组标志。注意,这里用分号分隔标志,因为这是CMake分隔列表元素的方式。

接下来,来看看如何添加项目,并使用的外部代码。








