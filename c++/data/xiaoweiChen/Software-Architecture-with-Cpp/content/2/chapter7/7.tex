本章中,了解了很多关于构建和打包的知识。现在能够编写更快的构建模板代码,知道如何选择工具来更快地编译代码(将在下一章了解更多关于工具的知识),并且知道何时使用直接声明,而不是\texttt{\#include}。

除此之外,可以使用现代CMake来定义构建目标和测试套件,使用find模块和\texttt{FetchContent}来管理外部依赖,创建各种格式的包和安装程序,再使用Conan来安装依赖和创建自己的工件。

下一章中,将了解如何编写易于测试的代码。持续集成和持续部署在拥有良好的测试覆盖率时才有用。没有全面测试的持续部署,更容易向生产环境引入新的bug,但这并不是设计软件架构时的目标。