
本节中,将讨论与对象创建相关的问题和常见解决方案。将讨论各种类型的对象工厂,浏览构建器,并接触组合和原型。然而,在描述解决方案时,将采用与四人帮不同的方法。提出了复杂的、动态多态的类层结构作为模式的实现。C++许多模式都可以解决实际问题,而不需要引入太多的类和动态调度的开销。这就是为什么在例子中,实现是不同的,并且在许多情况下更简单或性能更好(四人帮会更特化和更不“通用”)。让我们开始吧!

\subsubsubsection{6.4.1\hspace{0.2cm}何为工厂}

第一种创造模式是工厂。当对象构造可以在单个步骤中完成时(如果不能在工厂之后立即覆盖该模式,则该模式很有用),但当构造函数不够方便时,这种方式就很有用。工厂有三种类型——工厂方法、工厂函数和工厂类。

\hspace*{\fill} \\ %插入空行
\noindent
\textbf{工厂方法}

工厂方法,也称为命名构造函数习惯性用法,基本上是调用私有构造函数的成员函数。什么时候使用?以下是一些场景:

\begin{itemize}
\item 
当有许多不同的方法来构造一个对象时,这很可能会导致错误。例如为给定的像素构造一个存储不同颜色通道的类,每个通道由一个字节值表示。只使用一个构造函数将很容易传递错误的通道顺序,或意味着完全不同的调色板的值。此外,切换像素的内部颜色表示也会变得棘手。有读者可能会争辩说,应该用不同的类型来表示这些不同格式的颜色,但使用工厂方法也是一种很有效的方法。

\item 
当希望强制在堆上或在另一个特定内存区域中创建对象时。如果对象占用了堆栈上的大量空间,并且担心会耗尽堆栈内存,那么使用工厂方法是一个解决方案。如果要求在设备的某个内存区域中创建所有实例,情况也是如此。

\item 
构造对象时可能会失败,但不能抛出异常。应该使用异常而不是其他错误处理方法。如果使用得当,可以生成更清晰、性能更好的代码。然而,一些项目或环境要求禁用异常。这种情况下,使用工厂方法可以报告在构造期间发生的错误。
\end{itemize}

描述的第一种情况的工厂方法如下:

\begin{lstlisting}[style=styleCXX]
class Pixel {
	public:
	static Pixel fromRgba(char r, char b, char g, char a) {
		return Pixel{r, g, b, a};
	}
	static Pixel fromBgra(char b, char g, char r, char a) {
		return Pixel{r, g, b, a};
    }
	// other members
	
private:
	Pixel(char r, char g, char b, char a) : r_(r), g_(g), b_(b), a_(a) {}
	char r_, g_, b_, a_;
}
\end{lstlisting}

这个类有两个工厂方法(实际上,C++标准不识别术语方法,而是将其称为成员函数):\texttt{fromRgba}和\texttt{fromBgra}。现在就很难以错误的顺序初始化通道了。

注意,拥有私有构造函数可以有效地阻止其他类进行继承,因为如果不访问其构造函数,就不能创建实例。然而,如果这是目标,而不是副作用,应该更倾向于把类型标记为\texttt{final}。

\hspace*{\fill} \\ %插入空行
\noindent
\textbf{工厂函数}

与使用工厂成员函数相反,也可以使用非成员函数来实现。这样,就可以提供更好的封装,正如Scott Meyers的文章中所述,文章链接在扩展阅读部分可以找到。

在\texttt{Pixel}类中,还可以创建一个自由函数来创建实例:

\begin{lstlisting}[style=styleCXX]
struct Pixel {
	char r, g, b, a;
};

Pixel makePixelFromRgba(char r, char b, char g, char a) {
	return Pixel{r, g, b, a};
}

Pixel makePixelFromBgra(char b, char g, char r, char a) {
	return Pixel{r, g, b, a};
}
\end{lstlisting}

这种方法会使设计符合第1章中描述的开闭原则。很容易为其他调色板添加更多的工厂功能,而不需要修改\texttt{Pixel}本身。

\texttt{Pixel}的这个实现允许用户手工初始化,而不是使用提供的函数,可以通过更改类声明来抑制这种行为。以下是修复后的处理方法:

\begin{lstlisting}[style=styleCXX]
struct Pixel {
	char r, g, b, a;
	
private:
	Pixel(char r, char g, char b, char a) : r(r), g(g), b(b), a(a) {}
	friend Pixel makePixelFromRgba(char r, char g, char b, char a);
	friend Pixel makePixelFromBgra(char b, char g, char r, char a);
};
\end{lstlisting}

这次,工厂函数是类的友元函数。但该类型不再是聚合类型,因此不能使用聚合初始化(\texttt{Pixel\{\}}),包括指定的初始化方式,所以也放弃了开闭原理。这两种方法提供了不同的权衡,所以要明智地选择。

\hspace*{\fill} \\ %插入空行
\noindent
\textbf{选择工厂的返回类型}

实现对象工厂时,应该选择的另一件事是应该返回的实际类型。

\texttt{Pixel}是一个值类型,而不是一个多态类型,最简单的方法效果最好——简单地按值返回。如果生成了一个多态类型,通过一个智能指针返回它(永远不要使用裸指针,可能在某些时候产生内存泄漏)。如果调用者拥有创建的对象,通常将其以\texttt{unique\_ptr}形式返回给基类是最好的方法。在不太常见的情况下,工厂和调用方都必须拥有对象,使用\texttt{shared\_ptr}或其他引用计数替代。有时候,工厂跟踪对象而不存储。这种情况下,在工厂内部存储\texttt{weak\_ptr},并在工厂外部返回\texttt{shared\_ptr}。

一些C++程序员可能会争辩说,应该使用\texttt{out}参数返回特定的类型,但在大多数情况下,这不是最好的方法。性能方面,按值返回通常是最好的选择,因为编译器不会生成对象的副本。如果问题是类型不可复制,C++17标准指定在哪些地方必须省略复制,因此按值返回此类类型通常不是问题。如果函数返回多个对象,请使用\texttt{pair}、\texttt{tuple}、\texttt{结构体}或\texttt{容器}。

如果在构建过程中出现问题,这里有几个选择:

\begin{itemize}
\item 
不需要向调用者提供错误消息,则返回类型为\texttt{std::optional}。

\item 
构造过程中很少出现错误并且应该传播错误,则抛出异常。

\item 
在构造过程中经常出现错误,则返回类型为\texttt{absl::StatusOr}(请参阅Abseil关于该模板的文档,详见扩展阅读部分)。
\end{itemize}

既然知道了该返回什么类型,接着讨论最后一个——工厂类。

\hspace*{\fill} \\ %插入空行
\noindent
\textbf{工厂类}

工厂类是可以制造对象的类型,可以将多态对象类型与其调用者解耦。允许使用对象池(其中保存了可重用对象,不需要经常分配和释放)或其他分配方案。仔细看看另一个种方式,假设需要根据输入参数创建不同的多态类型。某些情况下,一个多态工厂函数是不够的:

\begin{lstlisting}[style=styleCXX]
std::unique_ptr<IDocument> open(std::string_view path) {
	if (path.ends_with(".pdf")) return 	std::make_unique<PdfDocument>();
	if (name == ".html") return 			std::make_unique<HtmlDocument>();
	
	return nullptr;
}
\end{lstlisting}

如果想打开其他类型的文档,比如OpenDocument文本文件,怎么办?发现前面的开放式工厂并不是开放用于扩展,这非常具有讽刺意味。如果拥有代码库,这可能不是一个大问题,但如果库的使用者需要注册自己的类型,这可能就有问题了。为了解决这个问题,可以使用工厂类,它允许注册函数打开不同类型的文档:

\begin{lstlisting}[style=styleCXX]
class DocumentOpener {
public:
	using DocumentType = std::unique_ptr<IDocument>;
	using ConcreteOpener = DocumentType (*)(std::string_view);

private:
	std::unordered_map<std::string_view, ConcreteOpener> openerByExtension;
};
\end{lstlisting}

这个类还需要完成很多事情,但它有一个扩展到函数的map,这些函数可以用来打开给定类型的文件。现在添加两个公共成员函数,第一个将注册新的文件类型:

\begin{lstlisting}[style=styleCXX]
void Register(std::string_view extension, ConcreteOpener opener) {
	openerByExtension.emplace(extension, opener);
}
\end{lstlisting}

现在有了填充map的方法。第二个新的公共功能将使用适当的opener打开文件:

\begin{lstlisting}[style=styleCXX]
DocumentType open(std::string_view path) {
	if (auto last_dot = path.find_last_of('.');
	last_dot != std::string_view::npos) {
		auto extension = path.substr(last_dot + 1);
		return openerByExtension.at(extension)(path);
	} else {
		throw std::invalid_argument{"Trying to open a file with no
			extension"};
	}
}
\end{lstlisting}

从文件路径中提取扩展名,如果为空,则抛出异常。如果不是,则在map中查找一个opener。如果找到,则使用它打开给定的文件,如果没有找到,则map将抛出另一个异常。

现在可以实例化工厂,并注册自定义文件类型,例如OpenDocument文本格式:

\begin{lstlisting}[style=styleCXX]
auto document_opener = DocumentOpener{};

document_opener.Register(
	"odt", [](auto path) -> DocumentOpener::DocumentType {
		return std::make_unique<OdtDocument>(path);
	});
\end{lstlisting}

注意,这里注册了一个lambda,因为它可以转换为\texttt{ConcreteOpener}类型,这是一个函数指针。如果有状态,就不是这样了。这种情况下,需要一些包装。其中之一就是\texttt{std::function},但是这样做的缺点是每次运行该函数时吗,都需要类型擦除(性能降低)。在打开文件的情况下,这可能是可以的。但是,如果需要更好的性能,可以考虑使用\texttt{function\_ref}这样的类型。

在Sy Brand的GitHub repo上可以找到一个针对C++标准(尚未接受)的实现示例,参见扩展阅读部分。

现在在工厂中注册了opener,用它来打开一个文件并从中提取一些文本:

\begin{lstlisting}[style=styleCXX]
auto document = document_opener.open("file.odt");
std::cout << document->extract_text().front();
\end{lstlisting}

若希望为库的消费者提供一种注册自己类型的方式,那么必须在运行时访问map。可以提供一个API来进行访问,或者使工厂成为静态的,并允许在任何地方进行注册。

对于工厂和创建对象来说,这一步就完成了。然后,讨论一下若工厂不适合的话,可以使用的另一种方式。

\subsubsubsection{6.4.2\hspace{0.2cm}使用构建器}

构建器类似于工厂,这是来自“四人帮”的一种设计模式。与工厂不同,它们可以构建更复杂的对象:那些不能一步完成的对象,比如由许多独立部件组装而成的类型,还提供了一种自定义对象构造的方法。本例中,将跳过设计复杂构建器的层结构。从而,展示构建器如何提供帮助。这里会把层结构的实现留给读者们作为练习。

当无法在单个步骤中生成对象时,就需要构建器。若单步没法完成,那么使用连贯的接口可以让它们变得简单。让我们演示如何使用CRTP创建流畅的构建器层次结构。

在示例中,我们将创建一个CRTP,使用\texttt{GenericItemBuilder}作为基本构建器,以及使用\texttt{FetchingItemBuilder}作为一个专用的构建器,可以使用远程地址获取数据(如果远程地址是受支持的特性)。这种特化甚至可以存在于不同的库中,例如:使用不同的API,这些API在构建时可能可用,也可能不可用。

为了演示目的,将构建第5章中\texttt{Item}结构体的实例,利用C++语言特性:

\begin{lstlisting}[style=styleCXX]
struct Item {
	std::string name;
	std::optional<std::string> photo_url;
	std::string description;
	std::optional<float> price;
	time_point<system_clock> date_added{};
	bool featured{};
};
\end{lstlisting}

可以通过将默认构造函数设为私有,并将构造函数设为友元的方式,来强制使用构造函数来构建\texttt{Item}实例:

\begin{lstlisting}[style=styleCXX]
template <typename ConcreteBuilder> friend class GenericItemBuilder;
\end{lstlisting}

构建器的实现如下所示:

\begin{lstlisting}[style=styleCXX]
template <typename ConcreteBuilder> class GenericItemBuilder {
public:
	explicit GenericItemBuilder(std::string name)
		: item_{.name = std::move(name)} {}
protected:
	Item item_;
\end{lstlisting}

虽然不建议创建受保护的成员,但希望后代构建器能够访问父类。另一种方法是在派生类中只使用基类构建器的公共方法。

在构造器的构造函数中接受名称,因为它是来自用户的一个输入,需要在创建项目时设置。这样,可以确保名称会进行设置。另一种选择是在构建的最后阶段,即对象释放给用户时,检查它是否合适。在例子中,构建步骤的实现如下:

\begin{lstlisting}[style=styleCXX]
Item build() && {
	item_.date_added = system_clock::now();
	return std::move(item_);
}
\end{lstlisting}

当调用该方法时,强制构造器被“消耗”,必须是右值。这意味着既可以在一行中使用构建器,也可以在最后一步中移动它,以标记工作的结束。然后,设置项目的创建时间,并将其移动到构建器之外。

构建器API可以提供如下功能:

\begin{lstlisting}[style=styleCXX]
ConcreteBuilder &&with_description(std::string description) {
	item_.description = std::move(description);
	return static_cast<ConcreteBuilder &&>(*this);
}

ConcreteBuilder &&marked_as_featured() {
	item_.featured = true;
	return static_cast<ConcreteBuilder &&>(*this);
}
\end{lstlisting}

它们将具体(派生)构建器对象作为右值引用返回。也许与直觉相反,这次应该首选这样的返回类型,而不是按值返回。这是为了避免在构建时对项目进行不必要的复制。另一方面,通过左值引用返回可能导致悬空引用,并使调用\texttt{build()}更加困难,因为返回的左值引用与预期的右值引用不匹配。

最终的构建器类型如下所示:

\begin{lstlisting}[style=styleCXX]
class ItemBuilder final : public GenericItemBuilder<ItemBuilder> {
	using GenericItemBuilder<ItemBuilder>::GenericItemBuilder;
};
\end{lstlisting}

只是一个重用通用构造器中的构造函数的类。使用方式如下所示:

\begin{lstlisting}[style=styleCXX]
auto directly_loaded_item = ItemBuilder{"Pot"}
							.with_description("A decent one")
							.with_price(100)
							.build();
\end{lstlisting}

可以使用函数链调用最终的接口,方法名使整个调用可读。

如果不直接加载每一项,而是使用可以从远程端点加载部分数据的更专用构建器,会怎么样呢?可以这样定义:

\begin{lstlisting}[style=styleCXX]
class FetchingItemBuilder final
: public GenericItemBuilder<FetchingItemBuilder> {
public:
	explicit FetchingItemBuilder(std::string name)
		: GenericItemBuilder(std::move(name)) {}
	
	FetchingItemBuilder&& using_data_from(std::string_view url) && {
		item_ = fetch_item(url);
		return std::move(*this);
    }
};
\end{lstlisting}

还使用CRTP来继承通用构建器,并强制给一个名称。这一次,使用函数扩展了基本构建器,以获取内容并将其放入正在构建的项中。多亏了CRTP,当从基本构建器调用函数时,将返回派生函数,这使接口更容易使用。可以通过以下方式调用:

\begin{lstlisting}[style=styleCXX]
auto fetched_item =
	FetchingItemBuilder{"Linen blouse"}
		.using_data_from("https://example.com/items/linen_blouse")
		.marked_as_featured()
		.build();
\end{lstlisting}

一切都很好!

如果需要创建不可变对象,构建器也可以派上用场。因为构造器可以访问类的私有成员,所以可以修改它们(即使类没有为它们提供setter)。

\hspace*{\fill} \\ %插入空行
\noindent
\textbf{使用组合模式和原型模式}

需要使用构建器的情况是在创建组合时。组合是一种设计模式,一组对象视为一个对象,所有对象共享相同的接口(或相同的基类型)。例如,图(可以由子图组成)或文档(可以嵌套其他文档)。当对这样的对象调用\texttt{print()}时,所有子对象都将调用它们的\texttt{print()},以便打印整个组合。构建器模式可以用于创建每个子对象并将它们组合在一起。

原型是另一种可用于对象构造的模式。如果类型创建成本很高,或者只想在其基础上构建一个基对象,可能需要使用此模式。它可以归结为提供一种克隆对象的方法,可以单独使用该对象,也可以对其进行修改,使其成为它应该的样子。在多态层结构的情况下,可以这样添加\texttt{clone()}:

\begin{lstlisting}[style=styleCXX]
class Map {
	public:
	virtual std::unique_ptr<Map> clone() const;
	// ... other members ...
};

class MapWithPointsOfInterests {
public:
	std::unique_ptr<Map> clone() override const;
	// ... other members ...
private:
	std::vector<PointOfInterest> pois_;
};
\end{lstlisting}

\texttt{MapWithPointsOfInterests}对象也可以克隆,所以不需要手动重新添加。通过这种方式,可以在最终用户创建自己的map时向他们提供一些默认值。在某些情况下,使用简单的复制构造函数就足够了,而不是使用原型。

现在已经介绍了对象创建。在这个过程中接触了变体,所以为什么不重新访问它们(双关语),看看它们还能做些什么?






















