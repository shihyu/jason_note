本章讨论的C++的最后一个特性是模块。它们是对C++20的又一个补充,对代码的构建和划分有很大的影响。

C++使用\texttt{\#include}已经有很长时间了。然而,这种文本形式的依赖包含有其缺陷,如下所示:

\begin{itemize}
\item 
由于需要处理大量文本(即使是预处理后的Hello World也需要大约50万行代码),所以速度很慢。这将违反定义规则(ODR)。

\item 
顺序很重要,但不应该这样。这个问题是前一个问题的两倍,这还会导致循环依赖。

\item 
最后,很难封装只需要放在头文件中的内容。即使你把一些东西放在一个详细的命名空间中,也会有人使用它,正如Hyrum定律所预测的那样。
\end{itemize}

这正是模块入场的时候。它应该解决上述的缺陷,极大地加快构建速度,并在构建时提供更好的C++可扩展性。使用模块,可以导出想要导出的内容,这将带来良好的封装。因为导入的顺序并不重要,所以包含依赖项的特定顺序也不是问题。

\begin{tcolorbox}[colback=blue!5!white,colframe=blue!75!black, title=Note]
\hspace*{0.75cm}不幸的是,在编写本文时,编译器对模块的支持仍然只是部分完成。这就是为什么只展示GCC 11中已有的功能。遗憾的是,这意味着像模块分区这样的东西将不会在这里出现。
\end{tcolorbox}

每个模块编译后,不仅会编译成目标文件,还会编译成模块接口文件。编译器可以快速知道一个给定模块包含什么类型和函数,而不是解析一个文件的所有依赖项。开发者所需要做的就是输入以下内容:

\begin{lstlisting}[style=styleCXX]
import my_module;
\end{lstlisting}

当\texttt{my\_module}编译并可用,就可以使用了。模块本身应该定义在\texttt{.cppm}文件中,但CMake仍然不支持这些文件,所以最好暂时将它们的后缀改成\texttt{.cpp}。

言归正传,回到多米尼加博览会示例,并展示如何在实践中使用它们。

首先,为客户代码创建第一个模块,从下面的指令开始:

\begin{lstlisting}[style=styleCXX]
module;
\end{lstlisting}

这个语句标志着,从此以后,这个模块中的所有东西都是私有的。这是放置包含和其他不会导出的内容的好地方。

接下来,必须指定导出的模块名称:

\begin{lstlisting}[style=styleCXX]
export module customer;
\end{lstlisting}

这是稍后用于导入模块的名称,并且这一行必须在导出的内容之前。现在,让指定模块导出实际的内容,并在定义前加上\texttt{export}关键字:

\begin{lstlisting}[style=styleCXX]
export using CustomerId = int;

export CustomerId get_current_customer_id() { return 42; }
\end{lstlisting}

完成了!我们的第一个模块可以使用了。让为商人创建另一个实例:

\begin{lstlisting}[style=styleCXX]
module;
export module merchant;
export struct Merchant {
	int id;
};
\end{lstlisting}

与第一个模块非常相似,这里指定了要导出的名称和类型(与第一个模块的类型别名和函数相反)。也可以导出其他定义,比如模板。但是,使用宏时需要一些技巧,因为需要导入\texttt{<header\_file>}才能看到。

顺便说一下,模块的一个优点是它们不允许宏传播到导入的模块中。这意味着当你写下面的代码时,模块不会定义\texttt{MY\_MACRO}:

\begin{lstlisting}[style=styleCXX]
#define MY_MACRO
import my_module;
\end{lstlisting}

它有助于在模块中有决定论,因为它可以保护不破坏其他模块中的代码。

现在,为商店和商品定义第三个模块。这里不会讨论导出其他函数、枚举和其他类型,因为它与前两个模块没有区别。有趣的是模块文件的启动方式。首先,在私有模块部分包含需要的内容:

\begin{lstlisting}[style=styleCXX]
module;

#include <chrono>
#include <iomanip>
#include <optional>
#include <string>
#include <vector>
\end{lstlisting}

在C++20标准中,标准库头文件还不是模块,但这在不久的将来可能会改变。

现在,看看接下来会发生什么:

\begin{lstlisting}[style=styleCXX]
export module store;
export import merchant;
\end{lstlisting}

这是有趣的部分。商店模块导入前面定义的商人模块,然后将其作为商店接口的一部分重新导出。如果模块是其他模块的外观,在不久的将来(也是C++20的一部分)的模块分区中,这将非常方便,可以将模块拆分为多个文件。其中一个可以包含以下内容:

\begin{lstlisting}[style=styleCXX]
export module my_module:foo;
export template<typename T> foo() {}
\end{lstlisting}

正如前面所讨论的,然后模块的主文件导出:

\begin{lstlisting}[style=styleCXX]
export module my_module;
export import :foo;
\end{lstlisting}

这就完成了本章所规划的模块和主要的C++特性。现在,总结一下所了解的知识。






