
C++17中添加了std::clamp()函数,可将标量数值的范围限制在最小值和最大值之间。该函数经过优化,可以使用移动语义,以获得最大的速度和效率。

\subsubsection{How to do it…}

可以使用clamp()在循环中使用容器,或者使用transform()算法来限制容器的值。让我们来看一些例子。

\begin{itemize}
\item 
从一个简单的输出容器值的函数开始:

\begin{lstlisting}[style=styleCXX]
void printc(auto& c, string_view s = "") {
	if(s.size()) cout << format("{}: ", s);
	for(auto e : c) cout << format("{:>5} ", e);
	cout << '\n';
}
\end{lstlisting}

注意格式字符串“\{:>5\}”。对于表格视图,这将每个值右对齐到5个空格。

\item 
在main()函数中,定义一个用于容器的初始化式列表。这允许我们多次使用相同的值:

\begin{lstlisting}[style=styleCXX]
int main() {
	auto il = { 0, -12, 2001, 4, 5, -14, 100, 200,
		30000 };
	...
}
\end{lstlisting}

对于使用clamp()来说,这是一个值的范围。

\item 
先来定义一些常数作为极值:

\begin{lstlisting}[style=styleCXX]
constexpr int ilow{0};
constexpr int ihigh{500};
\end{lstlisting}

在调用clamp()时使用这些值。

\item 
现在可以在main()函数中定义一个容器,这里将使用int类型的vector:

\begin{lstlisting}[style=styleCXX]
vector<int> voi{ il };
cout << "vector voi before:\n";
printc(voi);
\end{lstlisting}

使用初始化器列表中的值,输出如下:

\begin{tcblisting}{commandshell={}}
vector voi before:
0 -12 2001 4 5 -14 100 200 30000
\end{tcblisting}

\item 
现在可以使用一个带有clamp()的for循环来限制值在0到500之间:

\begin{lstlisting}[style=styleCXX]
cout << "vector voi after:\n";
for(auto& e : voi) e = clamp(e, ilow, ihigh);
printc(voi);
\end{lstlisting}

这将clamp()函数应用于容器中的每个值,分别使用0和500作为下限和上限,输出是:

\begin{tcblisting}{commandshell={}}
vector voi before:
0 -12 2001 4 5 -14 100 200 30000
vector voi after:
0 0 500 4 5 0 100 200 500
\end{tcblisting}

clamp()操作后,负值为0,大于500的值为500。

\item 
可以用transform()算法做同样的事情,在lambda中使用clamp()。这次使用一个list容器:

\begin{lstlisting}[style=styleCXX]
cout << "list loi before:\n";
list<int> loi{ il };
printc(loi);
transform(loi.begin(), loi.end(), loi.begin(),
	[=](auto e){ return clamp(e, ilow, ihigh); });
cout << "list loi after:\n";
printc(loi);
\end{lstlisting}

输出结果与使用for循环的版本相同:

\begin{tcblisting}{commandshell={}}
list loi before:
0 -12 2001 4 5 -14 100 200 30000
list loi after:
0 0 500 4 5 0 100 200 500
\end{tcblisting}
\end{itemize}

\subsubsection{How it works…}

clamp()算法是一个简单的函数:

\begin{lstlisting}[style=styleCXX]
template<class T>
constexpr const T& clamp( const T& v, const T& lo,
		const T& hi ) {
	return less(v, lo) ? lo : less(hi, v) ? hi : v;
}
\end{lstlisting}

若v的值小于lo,则返回lo。若hi小于v,则返回hi。

我们的例子中,可以使用for循环对容器应用clamp():

\begin{lstlisting}[style=styleCXX]
for(auto& v : voi) v = clamp(v, ilow, ihigh);
\end{lstlisting}

这里还在lambda中使用了transform()算法和clamp():

\begin{lstlisting}[style=styleCXX]
transform(loi.begin(), loi.end(), loi.begin(),
	[=](auto v){ return clamp(v, ilow, ihigh); });
\end{lstlisting}

示例中,两个版本给出了相同的结果,并且都从GCC编译器生成了类似的代码。编译的大小略有不同(使用for循环的版本更小,正如预期的那样),性能上的差异可以忽略不计。

我更喜欢for循环,但transform()版本在其他应用中可能更灵活。
