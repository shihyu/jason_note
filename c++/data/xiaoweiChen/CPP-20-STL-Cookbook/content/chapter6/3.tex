
有时,库中没有现成的算法来完成任务。可以使用迭代器,使用与算法库相同的方式,编写一个迭代器。

例如,需要使用分隔符将容器中的元素连接到一个字符串中。常见的解决方案是使用简单的for()循环:

\begin{lstlisting}[style=styleCXX]
for(auto v : c) cout << v << ', ';
\end{lstlisting}

这个方案的问题是在尾部留下了一个分隔符:

\begin{lstlisting}[style=styleCXX]
vector<string> greek{ "alpha", "beta", "gamma",
	"delta", "epsilon" };
for(auto v : greek) cout << v << ", ";
cout << '\n';
\end{lstlisting}

输出为:

\begin{tcblisting}{commandshell={}}
alpha, beta, gamma, delta, epsilon,
\end{tcblisting}

这在测试环境中可能没问题,但在生产系统中,后面的逗号是无法接受的。

ranges::views库有一个join()函数,但不提供分隔符:

\begin{lstlisting}[style=styleCXX]
auto greek_view = views::join(greek);
\end{lstlisting}

views::join()函数的作用是:返回一个ranges::view对象。这需要一个单独的步骤来显示或转换为字符串。我们可以使用for()循环遍历视图:

\begin{lstlisting}[style=styleCXX]
for(const char c : greek_view) cout << c;
cout << '\n';
\end{lstlisting}

输出如下所示:

\begin{tcblisting}{commandshell={}}
alphabetagammadeltaepsilon
\end{tcblisting}

我们需要在元素之间设置一个适当的分隔符。

由于算法库没有满足需求的函数,就需要自己编写一个。

\subsubsection{How to do it…}

这个示例中,将一个容器的元素用分隔符连接成一个字符串:

\begin{itemize}
\item 
在main()函数中,声明了一个字符串vector:

\begin{lstlisting}[style=styleCXX]
int main() {
	vector<string> greek{ "alpha", "beta", "gamma",
		"delta", "epsilon" };
	...
}
\end{lstlisting}

\item 
现在,来写一个简单的join()函数,使用ostream对象用分隔符连接元素:

\begin{lstlisting}[style=styleCXX]
namespace bw {
	template<typename I>
	ostream& join(I it, I end_it, ostream& o,
				string_view sep = "") {
		if(it != end_it) o << *it++;
		while(it != end_it) o << sep << *it++;
		return o;
	}
}
\end{lstlisting}

我把它放在bw命名空间中,以避免名称冲突。

可以这样与cout一起使用:

\begin{lstlisting}[style=styleCXX]
bw::join(greek.begin(), greek.end(), cout, ", ") << '\n';
\end{lstlisting}

因为其返回ostream对象,所以可以在它后面加上<{}<来向流中添加换行符。

输出:

\begin{tcblisting}{commandshell={}}
alpha, beta, gamma, delta, epsilon
\end{tcblisting}

\item 
这里通常需要一个字符串,而不是直接写入到cout。对于返回string的版本,可以重载此函数:

\begin{lstlisting}[style=styleCXX]
template<typename I>
string join(I it, I end_it, string_view sep = "") {
	ostringstream ostr;
	join(it, end_it, ostr, sep);
	return ostr.str();
}
\end{lstlisting}

这也在bw名称空间中。这个函数创建一个ostringstream对象,传递给bw::join()的ostream版本,从ostringstream对象的str()方法返回一个字符串对象。

可以这样使用:

\begin{lstlisting}[style=styleCXX]
string s = bw::join(greek.begin(), greek.end(), ", ");
cout << s << '\n';
\end{lstlisting}

输出为:

\begin{tcblisting}{commandshell={}}
alpha, beta, gamma, delta, epsilon
\end{tcblisting}

\item 
来添加最后一个重载,让它更容易使用:

\begin{lstlisting}[style=styleCXX]
string join(const auto& c, string_view sep = "") {
	return join(begin(c), end(c), sep);
}
\end{lstlisting}

这个版本只需要一个容器和一个分隔符,就可以满足大多数用例:

\begin{lstlisting}[style=styleCXX]
string s = bw::join(greek, ", ");
cout << s << '\n';
\end{lstlisting}

输出为:

\begin{tcblisting}{commandshell={}}
alpha, beta, gamma, delta, epsilon
\end{tcblisting}
\end{itemize}

\subsubsection{How it works…}

这个示例中的大部分工作由迭代器和ostream对象完成:

\begin{lstlisting}[style=styleCXX]
namespace bw {
	template<typename I>
	ostream& join(I it, I end_it, ostream& o,
				string_view sep = "") {
		if(it != end_it) o << *it++;
		while(it != end_it) o << sep << *it++;
		return o;
	}
}
\end{lstlisting}

分隔符在第一个元素之后,在每个连续元素之间,并在最后一个元素之前停止。所以可以在每个元素之前添加分隔符,跳过第一个,也可以在每个元素之后添加分隔符,跳过最后一个。若测试并跳过第一个元素,逻辑会更简单。我们在while()循环之前的代码可以这样做:

\begin{lstlisting}[style=styleCXX]
if(it != end_it) o << *it++;
\end{lstlisting}

完成了第一个元素,就可以简单地在每个剩余元素之前添加分隔符:

\begin{lstlisting}[style=styleCXX]
while(it != end_it) o << sep << *it++;
\end{lstlisting}

方便起见,返回ostream对象。这允许用户可以向流中添加换行符或其他对象:

\begin{lstlisting}[style=styleCXX]
bw::join(greek.begin(), greek.end(), cout, ", ") << '\n';
\end{lstlisting}

输出为:

\begin{tcblisting}{commandshell={}}
alpha, beta, gamma, delta, epsilon
\end{tcblisting}

\subsubsection{There's more…}

与标准库算法一样,join()函数将适用于支持前向迭代器的容器。例如,下面是数字库中的double常量列表:

\begin{lstlisting}[style=styleCXX]
namespace num = std::numbers;
list<double> constants { num::pi, num::e, num::sqrt2 };
cout << bw::join(constants, ", ") << '\n';
\end{lstlisting}

输出为:

\begin{tcblisting}{commandshell={}}
3.14159, 2.71828, 1.41421
\end{tcblisting}

其可以与ranges::view对象一起工作,就像前面在这个示例中定义的greek\_view一样:

\begin{lstlisting}[style=styleCXX]
cout << bw::join(greek_view, ":") << '\n';
\end{lstlisting}

输出为:

\begin{tcblisting}{commandshell={}}
a:l:p:h:a:b:e:t:a:g:a:m:m:a:d:e:l:t:a:e:p:s:i:l:o:n
\end{tcblisting}

