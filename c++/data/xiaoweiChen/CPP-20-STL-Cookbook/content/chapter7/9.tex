
默认情况下,basic\_istream类每次读取一个单词。可以利用这个属性使用istream\_iterator来计算单词数。

\subsubsection{How to do it…}

这是一个使用istream\_iterator来计数单词的简单方法:

\begin{itemize}
\item 
使用istream\_iterator对象来计数单词:

\begin{lstlisting}[style=styleCXX]
size_t wordcount(auto& is) {
	using it_t = istream_iterator<string>;
	return distance(it_t{is}, it_t{});
}
\end{lstlisting}

distance()函数接受两个迭代器,并返回它们之间的距离。using语句为istream\_iterator类创建了一个带有字符串特化的别名it\_t。然后,用一个迭代器调用distance(),迭代器用输入流it\_t\{is\}初始化,另一个用默认构造函数调用distance(),作为流的结束哨点。

\item 
在main()中使用wordcount():

\begin{lstlisting}[style=styleCXX]
int main() {
	const char * fn{ "the-raven.txt" };
	std::ifstream infile{fn, std::ios_base::in};
	size_t wc{ wordcount(infile) };
	cout << format("There are {} words in the
		file.\n", wc);
}
\end{lstlisting}

这将使用fstream对象调用wordcount()并打印文件中的字数。当用其处理埃德加·爱伦·坡的《乌鸦》时,可得到这样的输出:

\begin{tcblisting}{commandshell={}}
There are 1068 words in the file.
\end{tcblisting}
\end{itemize}

\subsubsection{How it works…}

因为basic\_istream默认为逐字输入,文件中的步数将是字数。distance()函数将测量两个迭代器之间的距离,因此使用兼容对象的起始和末尾迭代器来计算文件中的字数。
