
map容器是一个关联容器,由按键值对组织的元素组成。键用于查找,并且必须是唯一的。

在这个示例中,我们将利用map容器的唯一键值来计算文本文件中每个单词的出现次数。

\subsubsection{How to do it…}

这个任务有几个部分可以分开解决:

\begin{enumerate}
\item 
要从一个文件中获取文本,使用cin流。

\item 
需要把单词和标点符号,以及其他非单词的内容分开,所以需要使用正则表达式库。

\item 
需要计算每个单词的出现频率,为此使用map容器。

\item 
最后,需要对结果进行排序,首先按频率排序,然后按频率内的单词的字母顺序排序,将对vector容器进行排序。
\end{enumerate}

\begin{itemize}
\item 
为了方便起见,先从别名开始:

\begin{lstlisting}[style=styleCXX]
namespace ranges = std::ranges;
namespace regex_constants = std::regex_constants;
\end{lstlisting}

对于std::空间中的命名空间,我喜欢使用短别名。在范围(ranges)命名空间中,会复用现有算法的名称。

\item 
将正则表达式存储在一个常量中。我不喜欢使全局名称空间混乱,因为这会导致冲突。我倾向于使用基于我名字首字母的命名空间,比如:

\begin{lstlisting}[style=styleCXX]
namespace bw {
	constexpr const char * re{"(\\w+)"};
}
\end{lstlisting}

稍后使用bw::re很容易获取它。

\item 
在main()的顶部,定义了数据结构:

\begin{lstlisting}[style=styleCXX]
int main() {
	map<string, int> wordmap{};
	vector<pair<string, int>> wordvec{};
	regex word_re(bw::re);
	size_t total_words{};
\end{lstlisting}

我们的主map叫做wordmap。我们有一个名为wordvec的vector,我们将对其进行排序。最后,是正则表达式类,word\_re。

\item 
for循环是大部分工作发生的地方,从cin流中读取文本,应用正则表达式,并将单词存储在map中:

\begin{lstlisting}[style=styleCXX]
for(string s{}; cin >> s; ) {
	auto words_begin{
		sregex_iterator(s.begin(), s.end(), word_re) };
	auto words_end{ sregex_iterator() };
	for(auto r_it{words_begin}; r_it != words_end;
	++r_it) {
		smatch match{ *r_it };
		auto word_str{match.str()};
		ranges::transform(word_str, word_str.begin(),
			[](unsigned char c){ return tolower(c); });
		auto [map_it, result] =
			wordmap.try_emplace(word_str, 0);
		auto & [w, count] = *map_it;
		++total_words;
		++count;
	}
}
\end{lstlisting}

我喜欢for循环,因为它可以控制变量s的作用域。

首先为正则表达式结果定义迭代器。这使我们能够区分多个单词,即使周围只有标点符号。for(r\_it…)循环返回cin字符串中的单个单词。

smatch类型是正则表达式字符串匹配类的特化,给出了正则表达式中的下一个单词。

然后,使用转换算法使单词小写——这样就可以不考虑大小写而计算单词。(例如,“The”和“the”是同一个词。)

接下来,使用try\_emplace()将单词添加到map中。若已经有了,就不会替换。

最后,使用++count增加map中单词的计数。

\item 
现在,map上有了单词和它们的频率计数,目前按字母顺序排列。不过,我们希望按词频降序排列。为此,会将其放在一个vector中,并对这个vector进行排序:

\begin{lstlisting}[style=styleCXX]
auto unique_words = wordmap.size();
wordvec.reserve(unique_words);
ranges::move(wordmap, back_inserter(wordvec));
ranges::sort(wordvec, [](const auto& a, const
auto& b) {
	if(a.second != b.second)
	return (a.second > b.second);
	return (a.first < b.first);
});
cout << format("unique word count: {}\n",
	total_words);
cout << format("unique word count: {}\n",
	unique_words);
\end{lstlisting}

Wordvec是一个vector,每个元素包含有单词和频率计数。

使用ranges::move()算法来填充vector,然后使用ranges::sort()算法对vector进行排序。注意,谓词Lambda函数首先按计数排序(降序),然后按单词排序(升序)。

\item 
最后,打印结果:

\begin{lstlisting}[style=styleCXX]
	for(int limit{20}; auto& [w, count] : wordvec) {
		cout << format("{}: {}\n", count, w);
		if(--limit == 0) break;
	}
}
\end{lstlisting}

目前,设置了只打印前20个条目的限制。读者们可以注释掉if(-limit == 0) break;,从而可以打印整个列表。

\item 
示例文件中,包含了埃德加·艾伦·坡的《乌鸦》的文本文件,我们可以用它来测试程序:

\begin{tcblisting}{commandshell={}}
$ ./word-count < the-raven.txt
total word count: 1098
unique word count: 439
56: the
38: and
32: i
24: my
21: of
17: that
17: this
15: a
14: door
11: chamber
11: is
11: nevermore
10: bird
10: on
10: raven
9: me
8: at
8: from
8: in
8: lenore
\end{tcblisting}

\end{itemize}

这首诗共有1098个单词,其中出现了439个单词。

\subsubsection{How it works…}

这个示例的核心是使用一个map对象来计数重复的单词。

我们使用cin流从标准输入中读取文本。默认情况下,cin在读入字符串对象时会跳过空格。通过将一个字符串对象放在>{}>操作符的右边(cin >{}> s),可以得到用空格分隔的文本块。对于许多目的来说,这是足够好的定义,但我们需要语言学上的单词。为此,需要使用正则表达式。

regex类提供了正则表达式语法的选择,它默认为ECMA语法。在ECMA语法中,正则表达式“(\verb|\|w+)”是“([a- za -z0-9\_]+)”的简写。这将选择包含这些字符的单词。

正则表达式本身就是一种语言。要了解更多关于正则表达式的知识,我推荐Jeffrey Friedl的《Mastering regular expressions》这本书。

当从正则表达式引擎获取每个单词时,可以使用map对象的try\_emplace()方法有条件地将单词添加到单词map中。若单词不在map中,则将其添加为0。若单词已经在map中,则计数不会受到影响。我们在后面的循环中增加计数,所以其内容总是正确的。

在用文件中的所有单词填充map后,我们使用ranges::move()算法将其传输到一个vector中,move()算法使这种传输快速而有效。然后,可以使用ranges::sort()对vector进行排序。用于排序的谓词lambda函数包括对对两边的比较,因此最终得到一个按字数(降序)和单词排序的结果。


