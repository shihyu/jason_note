
map类是保存键-值对的关联容器,其中键在容器内必须唯一。

填充map容器的方法有很多种,可以这样定义的map:

\begin{lstlisting}[style=styleCXX]
map<string, string> m;
\end{lstlisting}

使用[]操作符添加一个元素:

\begin{lstlisting}[style=styleCXX]
m["Miles"] = "Trumpet"
\end{lstlisting}

使用insert()成员函数:

\begin{lstlisting}[style=styleCXX]
m.insert(pair<string,string>("Hendrix", "Guitar"));
\end{lstlisting}

或者,使用emplace()成员函数:

\begin{lstlisting}[style=styleCXX]
m.emplace("Krupa", "Drums");
\end{lstlisting}

我倾向于使用emplace()函数,可以完全转发来为容器放置(在适当的位置创建)新元素。参数直接转发给元素构造函数,快速、高效且易于阅读。
 
尽管emplace()肯定是对其他选项的改进,但其问题在于,即使在不需要对象时,也会构造对象。这包括调用构造函数、分配内存、移动数据,然后丢弃临时对象。

为了解决这个问题,C++17提供了新的try\_emplace()函数,该函数只在需要时构造值对象,这对于大型对象尤为重要。

\begin{tcolorbox}[colback=webgreen!5!white,colframe=webgreen!75!black,title=Note]
map的每个元素都是一个键值对,其中元素命名为first和second,但在map中是键和值。我倾向于将值对象视为有效负载。要搜索一个现有的键,try\_emplace()函数必须构造键对象,但不需要构造有效负载对象,除非需要插入到map中。
\end{tcolorbox}

\subsubsection{How to do it…}

新的try\_emplace()函数避免了构造有效负载对象的开销,这在键值碰撞的情况下效率很高,特是在大有效载荷的情况下。

\begin{itemize}
\item 
首先,创建一个有效负载类。出于演示目的,该类有一个简单的std::string有效负载,并在构造时显示一条消息:

\begin{lstlisting}[style=styleCXX]
struct BigThing {
	string v_;
	BigThing(const char * v) : v_(v) {
		cout << format("BigThing constructed {}\n", v_);
	}
};
using Mymap = map<string, BigThing>;
\end{lstlisting}

这个BigThing类只有一个成员函数——构造函数,在构造对象时显示一条消息。我们将使用它来跟踪BigThing对象的构造频率。实际中,这个类会更大,使用更多的资源。

每个map元素将由pair对象组成,一个std::string用于键,一个BigThing对象用于负载。这里,Mymap只是一个别名,只是为了让我们能够更专注于功能。

\item 
再创建一个printm()函数来打印map的内容:

\begin{lstlisting}[style=styleCXX]
void printm(Mymap& m) {
	for(auto& [k, v] : m) {
		cout << format("[{}:{}] ", k, v.v_);
	}
	cout << "\n";
}
\end{lstlisting}

使用C++20的format()函数打印map,就可以在插入元素时观察它们了。

\item 
main()函数中,创建了map对象并插入了一些元素:

\begin{lstlisting}[style=styleCXX]
int main() {
	Mymap m;
	m.emplace("Miles", "Trumpet");
	m.emplace("Hendrix", "Guitar");
	m.emplace("Krupa", "Drums");
	m.emplace("Zappa", "Guitar");
	m.emplace("Liszt", "Piano");
	printm(m);
}
\end{lstlisting}

输出为:

\begin{tcblisting}{commandshell={}}
BigThing constructed Trumpet
BigThing constructed Guitar
BigThing constructed Drums
BigThing constructed Guitar
BigThing constructed Piano
[Hendrix:Guitar] [Krupa:Drums] [Liszt:Piano]
[Miles:Trumpet] [Zappa:Guitar]
\end{tcblisting}

输出显示了每个有效负载对象的构造,然后是printm()函数的输出。

\item 
使用emplace()函数将元素添加到map中,每个有效负载元素只构造一次。这里,也可以使用try\_emplace()函数,结果相同:

\begin{lstlisting}[style=styleCXX]
Mymap m;
m.try_emplace("Miles", "Trumpet");
m.try_emplace("Hendrix", "Guitar");
m.try_emplace("Krupa", "Drums");
m.try_emplace("Zappa", "Guitar");
m.try_emplace("Liszt", "Piano");
printm(m);
\end{lstlisting}

输出为:

\begin{tcblisting}{commandshell={}}
BigThing constructed Trumpet
BigThing constructed Guitar
BigThing constructed Drums
BigThing constructed Guitar
BigThing constructed Piano
[Hendrix:Guitar] [Krupa:Drums] [Liszt:Piano]
[Miles:Trumpet] [Zappa:Guitar]
\end{tcblisting}

\item 
emplace()和try\_emplace()之间的区别显示在尝试插入具有重复键的新元素时:

\begin{lstlisting}[style=styleCXX]
cout << "emplace(Hendrix)\n";
m.emplace("Hendrix", "Singer");
cout << "try_emplace(Zappa)\n";
m.try_emplace("Zappa", "Composer");
printm(m);
\end{lstlisting}

输出为:

\begin{tcblisting}{commandshell={}}
emplace(Hendrix)
BigThing constructed Singer
try_emplace(Zappa)
[Hendrix:Guitar] [Krupa:Drums] [Liszt:Piano]
[Miles:Trumpet] [Zappa:Guitar]
\end{tcblisting}

emplace()函数尝试添加一个具有重复键的元素(“Hendrix”)。但失败了,但仍然构造了有效负载对象(“Singer”)。try\_emplace()函数还尝试添加一个具有重复键的元素("Zappa")。也失败了,没有构造有效负载对象。
\end{itemize}

这个例子演示了emplace()和try\_emplace()之间的区别。

\subsubsection{How it works…}

try\_emplace()函数签名与emplace()函数签名相似,因此对代码的修改应该很容易。下面是try\_emplace()函数签名:

\begin{lstlisting}[style=styleCXX]
pair<iterator, bool> try_emplace( const Key& k,
Args&&... args );
\end{lstlisting}

乍一看,这与emplace()签名不同:

\begin{lstlisting}[style=styleCXX]
pair<iterator,bool> emplace( Args&&... args );
\end{lstlisting}

区别在于try\_emplace()为键参数使用了一个单独的形参,这允许在构造时隔离。从函数上讲,若正在使用模板参数推导,则可以使用try\_emplace()替换:

\begin{lstlisting}[style=styleCXX]
m.emplace("Miles", "Trumpet");
m.try_emplace("Miles", "Trumpet");
\end{lstlisting}

try\_emplace()的返回值与emplace()的返回值相同,是一个表示迭代器和bool的pair:

\begin{lstlisting}[style=styleCXX]
const char * key{"Zappa"};
const char * payload{"Composer"};
if(auto [it, success] = m.try_emplace(key, payload);
!success) {
	cout << "update\n";
	it->second = payload;
}
printm(m);
\end{lstlisting}

输出为:

\begin{tcblisting}{commandshell={}}
update
BigThing constructed Composer
[Hendrix:Guitar] [Krupa:Drums] [Liszt:Piano] [Miles:Trumpet]
[Zappa:Composer]
\end{tcblisting}

这里使用了结构化绑定(auto [it, success] =)和if初始化语句来测试返回值,并有条件地更新有效负载。注意,这仍然构造了有效负载对象。

try\_emplace()函数也适用于unordered\_map:

\begin{lstlisting}[style=styleCXX]
using Mymap = unordered_map<string, BigThing>;
\end{lstlisting}

try\_emplace()优点,只在准备将有效负载对象存储到map中时构造有效负载对象。实际中,可在运行时节省大量资源,所以应该首选try\_emplace(),而非emplace()。




















