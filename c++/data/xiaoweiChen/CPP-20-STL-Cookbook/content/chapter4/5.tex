
迭代器本质上是一种抽象,有一个特定的接口,并以特定的方式使用。除此之外,其代码可以用于其他目的。迭代器适配器是一个看起来像迭代器,但需要做其他事情的类。

STL附带了各种迭代器适配器,通常与算法库一起使用。STL迭代器适配器通常分为三类:

\begin{itemize}
\item 
插入迭代器或插入器用于在容器中插入元素。

\item 
流迭代器读取和写入流。

\item 
反向迭代器反转迭代器的方向。
\end{itemize}

\subsubsection{How to do it…}

在本节示例中,有一些STL迭代器适配器的例子:

\begin{itemize}
\item 
一个简单的函数,打印容器的内容:

\begin{lstlisting}[style=styleCXX]
void printc(const auto & v, const string_view s = "") {
	if(s.size()) cout << format("{}: ", s);
	for(auto e : v) cout << format("{} ", e);
	cout << '\n';
}
\end{lstlisting}

printc()函数可以轻松地查看算法的结果,其包含一个可选的string\_view参数用于描述。

\item 
main()函数中,将定义两个deque容器。使用的是deque容器,所以可以在两端进行插入:

\begin{lstlisting}[style=styleCXX]
int main() {
	deque<int> d1{ 1, 2, 3, 4, 5 };
	deque<int> d2(d1.size());
	copy(d1.begin(), d1.end(), d2.begin());
	printc(d1);
	printc(d2, "d2 after copy");
}
\end{lstlisting}

输出为:

\begin{tcblisting}{commandshell={}}
1 2 3 4 5
d2 after copy: 1 2 3 4 5
\end{tcblisting}

我们为相同数量的元素定义了带有5个int值的deque d1和带有空格的d2。copy()算法不会分配空间,因此d2必须为元素留有空间。

copy()算法接受三个迭代器:begin和end迭代器表示要复制的元素范围,begin迭代器表示目标范围,不检查迭代器以确保它们有效。(尝试不分配空间的vector,会出现段错误)

我们在两个容器上调用printc()来显示结果。

\item 
copy()算法对此并不总可行。有时想复制并添加元素到容器的末尾,若有一个算法可以为每个元素调用push\_back()就太好了。这就是迭代器适配器有用的地方。让我们在main()的末尾添加一些代码:

\begin{lstlisting}[style=styleCXX]
copy(d1.begin(), d1.end(), back_inserter(d2));
printc(d2, "d2 after back_inserter");
\end{lstlisting}

输出为:

\begin{tcblisting}{commandshell={}}
d2 after back_inserter: 1 2 3 4 5 1 2 3 4 5
\end{tcblisting}

back\_inserter()是一个插入迭代器适配器,为分配给它的每个项调用push\_back(),可以在需要输出迭代器的地方使用。

\item 
还有一个front\_inserter()适配器,用于在容器的前端插入元素:

\begin{lstlisting}[style=styleCXX]
deque<int> d3{ 47, 73, 114, 138, 54 };
copy(d3.begin(), d3.end(), front_inserter(d2));
printc(d2, "d2 after front_inserter");
\end{lstlisting}

输出为:

\begin{tcblisting}{commandshell={}}
d2 after front_inserter: 54 138 114 73 47 1 2 3 4 5 1 2 3 4 5
\end{tcblisting}

front\_inserter()适配器使用容器的push\_front()方法在前端插入元素。因为每个元素都插入到前一个元素之前,所以目标容器中的元素是反向的。

\item 
若想在中间插入,可以使用inserter()适配器:

\begin{lstlisting}[style=styleCXX]
auto it2{ d2.begin() + 2};
copy(d1.begin(), d1.end(), inserter(d2, it2));
printc(d2, "d2 after middle insert");
\end{lstlisting}

输出为:

\begin{tcblisting}{commandshell={}}
d2 after middle insert: 54 138 1 2 3 4 5 114 73 47 ...
\end{tcblisting}

inserter()适配器接受插入起始点的迭代器。

\item 
流迭代器可以方便地读写iostream对象,这是ostream\_iterator():

\begin{lstlisting}[style=styleCXX]
cout << "ostream_iterator: ";
copy(d1.begin(), d1.end(), ostream_iterator<int>(cout));
cout << '\n';
\end{lstlisting}

输出为:

\begin{tcblisting}{commandshell={}}
ostream_iterator: 12345
\end{tcblisting}

\item 
下面是istream\_iterator():

\begin{lstlisting}[style=styleCXX]
vector<string> vs{};
copy(istream_iterator<string>(cin),
	istream_iterator<string>(),
	back_inserter(vs));
printc(vs, "vs2");
\end{lstlisting}

输出为:

\begin{tcblisting}{commandshell={}}
$ ./working < five-words.txt
vs2: this is not a haiku
\end{tcblisting}

若没有流迭代器,istream\_iterator()适配器将默认返回一个结束迭代器。

\item 
反向适配器包含在大多数容器中,如函数成员rbegin()和rend():

\begin{lstlisting}[style=styleCXX]
for(auto it = d1.rbegin(); it != d1.rend(); ++it) {
	cout << format("{} ", *it);
}
cout << '\n';
\end{lstlisting}

输出为:

\begin{tcblisting}{commandshell={}}
5 4 3 2 1
\end{tcblisting}
\end{itemize}

\subsubsection{How it works…}

迭代器适配器通过包装现有容器来工作。当调用一个适配器时,比如用容器对象的back\_inserter():

\begin{lstlisting}[style=styleCXX]
copy(d1.begin(), d1.end(), back_inserter(d2));
\end{lstlisting}

适配器返回一个模拟迭代器的对象,在本例中是std::back\_insert\_iterator对象,在每次赋值给迭代器时调用容器对象上的push\_back()方法。可以使用适配器代替迭代器,同时执行其他任务。

istream\_adapter()也需要一个哨兵,哨兵表示长度不确定的迭代器的结束。当从一个流中读取时,并不知道流中有多少对象。当流到达结束时,哨兵将与迭代器进行相等比较,标志流的结束。istream\_adapter()在不带参数的情况下调用时会创建一个哨兵:

\begin{lstlisting}[style=styleCXX]
auto it = istream_adapter<string>(cin);
auto it_end = istream_adapter<string>(); // creates sentinel
\end{lstlisting}

这允许测试流的结束,就像测试其他容器一样:

\begin{lstlisting}[style=styleCXX]
for(auto it = istream_iterator<string>(cin);
		it != istream_iterator<string>();
		++it) {
	cout << format("{} ", *it);
}
cout << '\n';
\end{lstlisting}

输出为:

\begin{tcblisting}{commandshell={}}
$ ./working < five-words.txt
this is not a haiku
\end{tcblisting}

