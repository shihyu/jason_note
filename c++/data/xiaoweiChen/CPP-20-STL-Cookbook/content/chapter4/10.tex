
本节中是一个全功能连续/随机访问迭代器的例子,这是容器最完整的迭代器类型。随机访问迭代器包括所有其他类型的容器迭代器的所有特性,以及它的随机访问功能。

虽然在本章中包含一个完整的迭代器很重要,但这个示例的代码超过700行,比本书中的其他示例要大一些。我将在这里介绍代码的基本组件。请在此处查看完整的源代码\url{https://github.com/PacktPublishing/CPP-20-STL-Cookbook/blob/main/ chap04/container-iterator.cpp}。


\subsubsection{How to do it…}

迭代器需要一个容器。我们将使用一个简单的数组,并将其称为Container。迭代器类嵌套在Container类中。

所有这些设计会与STL容器的接口一致。

\begin{itemize}
\item 
容器定义为模板类,private部分只有两个元素:

\begin{lstlisting}[style=styleCXX]
template<typename T>
class Container {
	std::unique_ptr<T[]> c_{};
	size_t n_elements_{};
\end{lstlisting}

我们为数据使用unique\_pointer,让智能指针管理内存。这减少了对~Container()析构函数的需求。n\_elements\_变量保留了容器的大小。

\item 
public部分中,有构造函数:

\begin{lstlisting}[style=styleCXX]
Container(initializer_list<T> l) : n_elements_{l.size()}
{
	c_ = std::make_unique<T[]>(n_elements_);
	size_t index{0};
	for(T e : l) {
		c_[index++] = e;
	}
}
\end{lstlisting}

第一个构造函数使用initializer\_list为容器传递元素。使用make\_unique来分配空间,并用基于范围的for循环填充容器。

\item 
还有一个构造函数,分配空间但不填充元素:

\begin{lstlisting}[style=styleCXX]
Container(size_t sz) : n_elements_{sz} {
	c_ = std::make_unique<T[]>(n_elements_);
}
\end{lstlisting}

make\_unique()函数的作用是为element构造空对象。

\item 
函数的作用是:返回元素的个数:

\begin{lstlisting}[style=styleCXX]
size_t size() const {
	return n_elements_;
}
\end{lstlisting}

\item 
operator[]()函数,返回一个索引元素:

\begin{lstlisting}[style=styleCXX]
const T& operator[](const size_t index) const {
	return c_[index];
}
\end{lstlisting}

\item 
at()函数,对索引元素进行边界检查:

\begin{lstlisting}[style=styleCXX]
T& at(const size_t index) const {
	if(index > n_elements_ - 1) {
		throw std::out_of_range(
			"Container::at(): index out of range"
		);
	}
	return c_[index];
}
\end{lstlisting}

这与STL的用法一致,at()函数是首选。

\item 
begin()和end()函数使用容器数据的地址调用迭代器构造函数。

\begin{lstlisting}[style=styleCXX]
iterator begin() const { return iterator(c_.get()); }
iterator end() const {
	return iterator(c_.get() + n_elements_);
}
\end{lstlisting}

unique\_ptr::get()函数从智能指针返回地址。

\item 
迭代器类作为公共成员嵌套在Container类中。

\begin{lstlisting}[style=styleCXX]
class iterator {
	T* ptr_;
\end{lstlisting}

迭代器类有一个私有成员,在Container类的begin()和end()方法中初始化的指针。

\item 
迭代器构造函数接受一个指向容器数据的指针。

\begin{lstlisting}[style=styleCXX]
iterator(T* ptr = nullptr) : ptr_{ptr} {}
\end{lstlisting}

因为标准需要一个默认构造函数,所以这里有一个默认值。
\end{itemize}

\noindent
\textbf{运算符重载}

此迭代器为以下操作符提供操作符重载:前缀++,后缀++,前缀--,后缀--,[],默认比较<=>(C++20), ==, *, ->, +,非成员+,数字-,对象-,+=和-=。我们将在这里介绍几个值得注意的重载,具体请参阅这些重载的源代码。

\begin{itemize}
\item 
C++20默认的比较运算符<=>提供了全套比较运算符的功能,除了==运算符:

\begin{lstlisting}[style=styleCXX]
const auto operator<=>(const iterator& o) const {
	return ptr_ <=> o.ptr_;
}
\end{lstlisting}

这是C++20的特性,所以需要支持C++20的编译器和库。

\item 
有两个+操作符重载,支持it + n和n + it操作。

\begin{lstlisting}[style=styleCXX]
iterator operator+(const size_t n) const {
	return iterator(ptr_ + n);
}
// non-member operator (n + it)
friend const iterator operator+(
		const size_t n, const iterator& o) {
	return iterator(o.ptr_ + n);
}
\end{lstlisting}

友元声明是一种特殊情况。在模板类成员函数中使用时,相当于一个非成员函数。这允许在类上下文中定义非成员函数。

\item 
运算符也有两个重载,需要同时支持数值操作数和迭代器操作数。

\begin{lstlisting}[style=styleCXX]
const iterator operator-(const size_t n) {
	return iterator(ptr_ - n);
}
const size_t operator-(const iterator& o) {
	return ptr_ - o.ptr_;
}
\end{lstlisting}

这允许it - n和it - it运算。不需要非成员函数,因为(n -)并不是有效的操作。

\end{itemize}


\noindent
\textbf{验证}

C++20规范§23.3.4.13要求一个有效的随机访问迭代器有一组特定的操作和结果,我在源代码中写了一个unit\_tests()函数来验证这些需求。

main()函数创建一个Container对象并执行一些简单的验证函数。

\begin{itemize}
\item 
首先,创建一个Container<string>对象x,其中包含10个值。

\begin{lstlisting}[style=styleCXX]
Container<string> x{"one", "two", "three", "four",
	"five",
		"six", "seven", "eight", "nine", "ten" };
cout << format("Container x size: {}\n", x.size());
\end{lstlisting}

输出给出了元素的数量:

\begin{tcblisting}{commandshell={}}
Container x size: 10
\end{tcblisting}

\item 
用一个基于范围的for循环来显示容器的元素:

\begin{lstlisting}[style=styleCXX]
puts("Container x:");
for(auto e : x) {
	cout << format("{} ", e);
}
cout << '\n';
\end{lstlisting}

输出为:

\begin{tcblisting}{commandshell={}}
Container x:
one two three four five six seven eight nine ten
\end{tcblisting}

\item 
接下来,测试几个直接访问的方法:

\begin{lstlisting}[style=styleCXX]
puts("direct access elements:");
cout << format("element at(5): {}\n", x.at(5));
cout << format("element [5]: {}\n", x[5]);
cout << format("element begin + 5: {}\n",
	*(x.begin() + 5));
cout << format("element 5 + begin: {}\n",
	*(5 + x.begin()));
cout << format("element begin += 5: {}\n",
	*(x.begin() += 5));
\end{lstlisting}

输出为:

\begin{tcblisting}{commandshell={}}
direct access elements:
element at(5): six
element [5]: six
element begin + 5: six
element 5 + begin: six
element begin += 5: six
\end{tcblisting}

\item 
用ranges::views管道和views::reverse来测试容器:

\begin{lstlisting}[style=styleCXX]
puts("views pipe reverse:");
auto result = x | views::reverse;
for(auto v : result) cout << format("{} ", v);
cout << '\n';
\end{lstlisting}

输出为:

\begin{tcblisting}{commandshell={}}
views pipe reverse:
ten nine eight seven six five four three two one
\end{tcblisting}

\item 
最后,创建一个包含10个未初始化元素的Container对象y:

\begin{lstlisting}[style=styleCXX]
Container<string> y(x.size());
cout << format("Container y size: {}\n", y.size());
for(auto e : y) {
	cout << format("[{}] ", e);
}
cout << '\n';
\end{lstlisting}

输出为:

\begin{tcblisting}{commandshell={}}
Container y size: 10
[] [] [] [] [] [] [] [] [] []
\end{tcblisting}

\end{itemize}

\subsubsection{How it works…}

尽管有很多代码,但这个迭代器并不比一个较小的迭代器复杂。大多数代码都在操作符重载中,每个重载通常只有一两行代码。

容器本身由智能指针管理。这是因为它是一个“平面”数组,所以不需要展开或压缩。

当然,STL提供了一个std::array类,以及其他更复杂的数据结构。不过,揭开一个迭代器类的神秘面纱,还是挺有趣的。

