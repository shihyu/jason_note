
生成器是生成自己的值序列的迭代器,不使用容器。它动态地创建值,根据需要一次返回一个值。并且,C++生成器可以独立运行。

本节示例中,我们将为斐波那契数列构建一个生成器。这是一个数列,其中每个数字都是数列中前两个数字的和,从0和1开始:

\begin{equation*}
F(n)=
\begin{cases}
	0, n = 0 \\
	1, n = 1 \\
	F(n-1) + F(n-2), n > 1
\end{cases}
\end{equation*}

斐波那契数列的前十个值,不包括零,分别是:1,1,2,3,5,8,13,21,34,55。这与自然界的黄金比例非常接近。

\subsubsection{How to do it…}

斐波那契数列通常是用递归循环创建的。生成器中的递归可能会很困难,而且需要大量资源,因此只保存序列中的前两个值并将它们相加,这样更有效率更高。

\begin{itemize}
\item 
首先,定义一个函数来打印序列:

\begin{lstlisting}[style=styleCXX]
void printc(const auto & v, const string_view s = "") {
	if(s.size()) cout << format("{}: ", s);
	for(auto e : v) cout << format("{} ", e);
	cout << '\n';
}
\end{lstlisting}

之前使用过这个printc()函数,可打印一个可迭代范围,以及一个描述字符串。

\item 
我们的类从一个类型别名和一些对象变量开始,所有这些都在private部分中定义。

\begin{lstlisting}[style=styleCXX]
class fib_generator {
	using fib_t = unsigned long;
	fib_t stop_{};
	fib_t count_ { 0 };
	fib_t a_ { 0 };
	fib_t b_ { 1 };
\end{lstlisting}

stop\_变量将在后面用作哨兵,设置为要生成的值的数量。count\_用于跟踪生成了多少个值。a\_和b\_是前两个序列值,用于计算下一个值。

\item 
private部分,有一个简单的函数来计算斐波那契数列中的下一个值。

\begin{lstlisting}[style=styleCXX]
	constexpr void do_fib() {
		const fib_t old_b = b_;
		b_ += a_;
		a_ = old_b;
	}
\end{lstlisting}

\item 
public部分,有一个简单的构造函数,有一个默认值:

\begin{lstlisting}[style=styleCXX]
public:
	explicit fib_generator(fib_t stop = 0) : stop_{ stop
	} {}
\end{lstlisting}

该构造函数不带参数,用于创建哨兵。stop参数初始化stop\_变量以表示要生成多少值。

\item 
其余的公共函数是前向迭代器所期望的操作符重载:

\begin{lstlisting}[style=styleCXX]
	fib_t operator*() const { return b_; }
	constexpr fib_generator& operator++() {
		do_fib();
		++count_;
		return *this;
	}
	fib_generator operator++(int) {
		auto temp{ *this };
		++*this;
		return temp;
	}
	bool operator!=(const fib_generator &o) const {
		return count_ != o.count_;
	}
	bool operator==(const fib_generator&o) const {
		return count_ == o.count_;
	}
	const fib_generator& begin() const { return *this; }
	const fib_generator end() const {
		auto sentinel = fib_generator();
		sentinel.count_ = stop_;
		return sentinel;
	}
	fib_t size() { return stop_; }
};
\end{lstlisting}

还有一个简单的size()函数,若需要为复制操作初始化一个目标容器,这个函数会很有用。

\item 
现在可以在main函数中使用生成器

\begin{lstlisting}[style=styleCXX]
printc():
	int main() {
		printc(fib_generator(10));
	}
\end{lstlisting}

这将创建一个匿名的fib\_generator对象传递给printc()函数。

\item 
用前10个斐波那契数(不包括零)可得到如下输出:

\begin{tcblisting}{commandshell={}}
1 1 2 3 5 8 13 21 34 55
\end{tcblisting}
\end{itemize}

\subsubsection{How it works…}

fib\_generator类作为前向迭代器运行:

\begin{lstlisting}[style=styleCXX]
fib_generator {
public:
	fib_t operator*() const;
	constexpr fib_generator& operator++();
	fib_generator operator++(int);
	bool operator!=(const fib_generator &o) const;
	bool operator==(const fib_generator&o) const;
	const fib_generator& begin() const;
	const fib_generator end() const;
};
\end{lstlisting}

就基于范围的for循环而言,这是一个迭代器(看起来像一个迭代器)。

该值在do\_fib()函数中计算:

\begin{lstlisting}[style=styleCXX]
constexpr void do_fib() {
	const fib_t old_b = b_;
	b_ += a_;
	a_ = old_b;
}
\end{lstlisting}

这只是简单地添加了b\_ += a\_,将结果存储在b\_中,将旧的b\_存储在a\_中,为下一次迭代保存必要的值。

解引用操作符返回b\_的值,是序列中的下一个值:

\begin{lstlisting}[style=styleCXX]
fib_t operator*() const { return b_; }
\end{lstlisting}

end()函数创建了一个对象,其中count\_变量等于stop\_变量,从而创建了一个哨兵:

\begin{lstlisting}[style=styleCXX]
const fib_generator end() const {
	auto sentinel = fib_generator();
	sentinel.count_ = stop_;
	return sentinel;
}
\end{lstlisting}

现在,相等比较操作符可以很容易地检测序列的结束:

\begin{lstlisting}[style=styleCXX]
bool operator==(const fib_generator&o) const {
	return count_ == o.count_;
}
\end{lstlisting}

\subsubsection{There's more…}

若想让生成器与算法库一起工作,需要提供特性别名。这些放在public的顶部:

\begin{lstlisting}[style=styleCXX]
public:
	using iterator_concept = std::forward_iterator_tag;
	using iterator_category = std::forward_iterator_tag;
	using value_type = std::remove_cv_t<fib_t>;
	using difference_type = std::ptrdiff_t;
	using pointer = const fib_t*;
	using reference = const fib_t&;
\end{lstlisting}

现在可以使用生成器和算法:

\begin{lstlisting}[style=styleCXX]
fib_generator fib(10);
auto x = ranges::views::transform(fib,
	[](unsigned long x){ return x * x; });
printc(x, "squared:");
\end{lstlisting}

使用transform()算法的ranges::views版本对每个值进行平方。生成的对象可以用于可以使用迭代器的地方。

可以从printc()中获得以下输出:

\begin{tcblisting}{commandshell={}}
squared:: 1 1 4 9 25 64 169 441 1156 3025
\end{tcblisting}

