
从C++17开始,许多标准的STL算法可以并行执行。该特性允许算法将其工作拆分为子任务,以便在多个核上同时运行。这些算法接受一个执行策略对象,该对象指定应用于算法的并行度类型,该特性需要硬件支持。

\subsubsection{How to do it…}

执行策略在<execution>头文件中和std:: Execution命名空间中定义。本节中,我们将使用std::transform()算法测试可用的策略:

\begin{itemize}
\item 
出于计时的目的,将使用带有std::milli比率的duration对象,这样就可以以毫秒为单位进行测量:

\begin{lstlisting}[style=styleCXX]
using dur_t = duration<double, std::milli>;
\end{lstlisting}

\item 
出于演示目的,将从一个具有1000万个随机值的vector<unsigned>开始:

\begin{lstlisting}[style=styleCXX]
int main() {
	std::vector<unsigned> v(10 * 1000 * 1000);
	std::random_device rng;
	for(auto &i : v) i = rng() % 0xFFFF;
	...
\end{lstlisting}

\item 
现在,进行一个简单的变换:

\begin{lstlisting}[style=styleCXX]
auto mul2 = [](int n){ return n * 2; };
auto t1 = steady_clock::now();
std::transform(v.begin(), v.end(), v.begin(), mul2);
dur_t dur1 = steady_clock::now() - t1;
cout << format("no policy: {:.3}ms\n", dur1.count());
\end{lstlisting}

mul2 lambda只是将一个值乘以2。transform()算法将mul2应用于vector的每个成员。

此转换不指定执行策略。

输出为:

\begin{tcblisting}{commandshell={}}
no policy: 4.71ms
\end{tcblisting}

\item 
可以在算法的第一个参数中指定执行策略:

\begin{lstlisting}[style=styleCXX]
std::transform(execution::seq,
	v.begin(), v.end(), v.begin(), mul2);
\end{lstlisting}

seq策略意味着算法不能并行化,这与没有执行策略一样。

输出为:

\begin{tcblisting}{commandshell={}}
execution::seq: 4.91ms
\end{tcblisting}

注意,持续时间与没有策略时大致相同。因为它每次运行都会变化,所以永远不会是精确的。

\item 
execution::par策略允许算法并行化其工作负载:

\begin{lstlisting}[style=styleCXX]
std::transform(execution::par,
	v.begin(), v.end(), v.begin(), mul2);
\end{lstlisting}

输出为:

\begin{tcblisting}{commandshell={}}
execution::par: 3.22ms
\end{tcblisting}

注意,使用并行执行策略时,算法运行得稍微快一些。

\item 
execute::par\_unseq策略允许工作负载的非排序并行执行:

\begin{lstlisting}[style=styleCXX]
std::transform(execution::par_unseq,
	v.begin(), v.end(), v.begin(), mul2);
\end{lstlisting}

输出为:

\begin{tcblisting}{commandshell={}}
execution::par_unseq: 2.93ms
\end{tcblisting}

这里,我们注意到该策略的另一个性能提升。

execution::par\_unseq策略对算法有更严格的要求,算法不能执行需要并发或需要顺序执行的操作。
\end{itemize}

\subsubsection{How it works…}

执行策略接口没有指定如何并行算法工作负载,其设计为在不同的负载和环境下使用不同的硬件和处理器。其可以完全在库中实现,也可以依赖于编译器或硬件支持。

并行化将在运算量超过O(n)的算法上表现出最大的改进。例如,sort()显示了显著的改进。下面是一个没有并行化的sort():

\begin{lstlisting}[style=styleCXX]
auto t0 = steady_clock::now();
std::sort(v.begin(), v.end());
dur_t dur0 = steady_clock::now() - t0;
cout << format("sort: {:.3}ms\n", dur0.count());
\end{lstlisting}

输出为:

\begin{tcblisting}{commandshell={}}
sort: 751ms
\end{tcblisting}

使用execution::par,就可以看到显著的性能提升:

\begin{lstlisting}[style=styleCXX]
std::sort(execution::par, v.begin(), v.end());
\end{lstlisting}

输出为:

\begin{tcblisting}{commandshell={}}
sort: 163ms
\end{tcblisting}

execution::par\_unseq的改进会更好:

\begin{lstlisting}[style=styleCXX]
std::sort(execution::par_unseq, v.begin(), v.end());
\end{lstlisting}

输出为:

\begin{tcblisting}{commandshell={}}
sort: 152ms
\end{tcblisting}

在使用并行算法时,做大量的测试是个好主意。若算法或谓词本身不能很好地用于并行化,那么最终可能会获得最小的性能增益,或出现意想不到的副作用。

\begin{tcolorbox}[colback=webgreen!5!white,colframe=webgreen!75!black,title=Note]
撰写本文时,GCC对执行策略的支持很差,LLVM/Clang也不支持。这个示例是在运行Windows 10和Visual C++预览版的6核i7上测试的。
\end{tcolorbox}

















