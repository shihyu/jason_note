
术语“互斥”指的是对共享资源的互斥访问。互斥锁通常用于避免由于多个执行线程,访问相同的数据而导致的数据损坏和竞争条件。互斥锁通常使用锁来限制线程的访问。

STL在<mutex>头文件中提供了mutex和lock类。


\subsubsection{How to do it…}

这个示例中,我们将使用一个简单的Animal类来实验锁定和解锁互斥量:

\begin{itemize}
\item 
首先,创建一个互斥对象:

\begin{lstlisting}[style=styleCXX]
std::mutex animal_mutex;
\end{lstlisting}

互斥锁是在全局作用域中声明的,因此相关对象都可以访问它。

\item 
我们的Animal类有一个名字和一个朋友列表:

\begin{lstlisting}[style=styleCXX]
class Animal {
	using friend_t = list<Animal>;
	string_view s_name{ "unk" };
	friend_t l_friends{};
public:
	Animal() = delete;
	Animal(const string_view n) : s_name{n} {}
	...
}
\end{lstlisting}

添加和删除好友对互斥是一个有用的测试用例。

\item 
等式运算符是我们唯一需要的运算符:

\begin{lstlisting}[style=styleCXX]
bool operator==(const Animal& o) const {
	return s_name.data() == o.s_name.data();
}
\end{lstlisting}

s\_name成员是一个string\_view对象,因此可以测试其数据存储地址是否相等。

\item 
is\_friend()方法测试是否有其他动物在l\_friends列表中:

\begin{lstlisting}[style=styleCXX]
bool is_friend(const Animal& o) const {
	for(const auto& a : l_friends) {
		if(a == o) return true;
	}
	return false;
}
\end{lstlisting}

\item 
find\_friend()方法返回一个可选对象,若找到Animal,则带有指向Animal的迭代器:

\begin{lstlisting}[style=styleCXX]
optional<friend_t::iterator>
find_friend(const Animal& o) noexcept {
	for(auto it{l_friends.begin()};
	it != l_friends.end(); ++it) {
		if(*it == o) return it;
	}
	return {};
}
\end{lstlisting}

\item 
print()方法输出s\_name和l\_friends列表中每个Animal对象的名称:

\begin{lstlisting}[style=styleCXX]
void print() const noexcept {
	auto n_animals{ l_friends.size() };
	cout << format("Animal: {}, friends: ", s_name);
	if(!n_animals) cout << "none";
	else {
		for(auto n : l_friends) {
			cout << n.s_name;
			if(--n_animals) cout << ", ";
		}
	}
	cout << '\n';
}
\end{lstlisting}

\item 
add\_friend()方法将一个Animal对象添加到l\_friends列表中:

\begin{lstlisting}[style=styleCXX]
bool add_friend(Animal& o) noexcept {
	cout << format("add_friend {} -> {}\n", s_name,
		o.s_name);
	if(*this == o) return false;
	std::lock_guard<std::mutex> l(animal_mutex);
	if(!is_friend(o)) l_friends.emplace_back(o);
	if(!o.is_friend(*this))
		o.l_friends.emplace_back(*this);
	return true;
}
\end{lstlisting}

\item 
delete\_friend()方法从l\_friends列表中删除一个Animal对象:

\begin{lstlisting}[style=styleCXX]
bool delete_friend(Animal& o) noexcept {
	cout << format("delete_friend {} -> {}\n",
		s_name, o.s_name);
	if(*this == o) return false;
	if(auto it = find_friend(o))
		l_friends.erase(it.value());
	if(auto it = o.find_friend(*this))
		o.l_friends.erase(it.value());
	return true;
}
\end{lstlisting}

\item 
main()函数中,我们创建了一些Animal对象:

\begin{lstlisting}[style=styleCXX]
int main() {
	auto cat1 = std::make_unique<Animal>("Felix");
	auto tiger1 = std::make_unique<Animal>("Hobbes");
	auto dog1 = std::make_unique<Animal>("Astro");
	auto rabbit1 = std::make_unique<Animal>("Bugs");
	...
\end{lstlisting}

\item 
使用async()在对象上调用add\_friends(),在不同的线程中运行:

\begin{lstlisting}[style=styleCXX]
auto a1 = std::async([&]{ cat1->add_friend(*tiger1); });
auto a2 = std::async([&]{ cat1->add_friend(*rabbit1); });
auto a3 = std::async([&]{ rabbit1->add_friend(*dog1); });
auto a4 = std::async([&]{ rabbit1->add_friend(*cat1); });
a1.wait();
a2.wait();
a3.wait();
a4.wait();
\end{lstlisting}

可以使用wait()对线程进行阻塞式等待。

\item 
可以使用print()来查看动物们和他们之间的关系:

\begin{lstlisting}[style=styleCXX]
auto p1 = std::async([&]{ cat1->print(); });
auto p2 = std::async([&]{ tiger1->print(); });
auto p3 = std::async([&]{ dog1->print(); });
auto p4 = std::async([&]{ rabbit1->print(); });
p1.wait();
p2.wait();
p3.wait();
p4.wait();
\end{lstlisting}

\item 
最后,使用delete\_friend()来删除一个朋友:

\begin{lstlisting}[style=styleCXX]
auto a5 = std::async([&]{ cat1->delete_friend(*rabbit1);
});
a5.wait();
auto p5 = std::async([&]{ cat1->print(); });
auto p6 = std::async([&]{ rabbit1->print(); });
\end{lstlisting}

\item 
此时,输出是这样的:

\begin{tcblisting}{commandshell={}}
add_friend Bugs -> Felix
add_friend Felix -> Hobbes
add_friend Felix -> Bugs
add_friend Bugs -> Astro
Animal: Felix, friends: Bugs, Hobbes
Animal: Hobbes, friends: Animal: Bugs, friends:
FelixAnimal: Astro, friends: Felix
, Astro
Bugs
delete_friend Felix -> Bugs
Animal: Felix, friends: Hobbes
Animal: Bugs, friends: Astro
\end{tcblisting}

这个输出有点混乱。每次运行都是不同的。有时候可能没问题,但这就可能是问题。我们需要添加一些互斥锁来控制对数据的访问。

\item 
使用互斥的一种方法是使用它的lock()和unlock()方法,可以将其添加到add\_friend()函数中:

\begin{lstlisting}[style=styleCXX]
bool add_friend(Animal& o) noexcept {
	cout << format("add_friend {} -> {}\n", s_name, o.s_
	name);
	if(*this == o) return false;
	animal_mutex.lock();
	if(!is_friend(o)) l_friends.emplace_back(o);
	if(!o.is_friend(*this)) o.l_friends.emplace_
	back(*this);
	animal_mutex.unlock();
	return true;
}
\end{lstlisting}

lock()方法尝试获取互斥锁。若互斥锁已经锁定,会等待(块执行)互斥锁打开。

\item 
我们还需要添加一个锁来delete\_friend():

\begin{lstlisting}[style=styleCXX]
bool delete_friend(Animal& o) noexcept {
	cout << format("delete_friend {} -> {}\n",
		s_name, o.s_name);
	if(*this == o) return false;
	animal_mutex.lock();
	if(auto it = find_friend(o))
		l_friends.erase(it.value());
	if(auto it = o.find_friend(*this))
		o.l_friends.erase(it.value());
	animal_mutex.unlock();
	return true;
}
\end{lstlisting}

\item 
现在,需要为print()添加一个锁,以便在打印时不更改数据:

\begin{lstlisting}[style=styleCXX]
void print() const noexcept {
	animal_mutex.lock();
	auto n_animals{ l_friends.size() };
	cout << format("Animal: {}, friends: ", s_name);
	if(!n_animals) cout << "none";
	else {
		for(auto n : l_friends) {
			cout << n.s_name;
			if(--n_animals) cout << ", ";
		}
	}
	cout << '\n';
	animal_mutex.unlock();
}
\end{lstlisting}

现在,输出是合理的:

\begin{tcblisting}{commandshell={}}
add_friend Bugs -> Felix
add_friend Bugs -> Astro
add_friend Felix -> Hobbes
add_friend Felix -> Bugs
Animal: Felix, friends: Bugs, Hobbes
Animal: Hobbes, friends: Felix
Animal: Astro, friends: Bugs
Animal: Bugs, friends: Felix, Astro
delete_friend Felix -> Bugs
Animal: Felix, friends: Hobbes
Animal: Bugs, friends: Astro
\end{tcblisting}

由于异步操作,输出的行顺序可能不同。

\item 
lock()和unlock()方法很少直接使用。std::lock\_guard类使用适当的资源获取初始化(RAII)模式管理锁,该模式在销毁锁时自动释放锁。下面是带lock\_guard的add\_friend()方法:

\begin{lstlisting}[style=styleCXX]
bool add_friend(Animal& o) noexcept {
	cout << format("add_friend {} -> {}\n", s_name, o.s_
		name);
	if(*this == o) return false;
	std::lock_guard<std::mutex> l(animal_mutex);
	if(!is_friend(o)) l_friends.emplace_back(o);
	if(!o.is_friend(*this))
		o.l_friends.emplace_back(*this);
	return true;
}
\end{lstlisting}

lock\_guard对象创建并持有一个锁,直到它销毁。和lock()方法一样,lock\_guard也会阻塞至有锁可用为止。

\item 
让我们对delete\_friend()和print()方法使用lock\_guard。

delete\_friend():

\begin{lstlisting}[style=styleCXX]
bool delete_friend(Animal& o) noexcept {
	cout << format("delete_friend {} -> {}\n",
		s_name, o.s_name);
	if(*this == o) return false;
	std::lock_guard<std::mutex> l(animal_mutex);
	if(auto it = find_friend(o))
		l_friends.erase(it.value());
	if(auto it = o.find_friend(*this))
		o.l_friends.erase(it.value());
	return true;
}
\end{lstlisting}

print():

\begin{lstlisting}[style=styleCXX]
void print() const noexcept {
	std::lock_guard<std::mutex> l(animal_mutex);
	auto n_animals{ l_friends.size() };
	cout << format("Animal: {}, friends: ", s_name);
	if(!n_animals) cout << "none";
	else {
		for(auto n : l_friends) {
			cout << n.s_name;
			if(--n_animals) cout << ", ";
		}
	}
	cout << '\n';
}
\end{lstlisting}

输出保持一致:

\begin{tcblisting}{commandshell={}}
add_friend Felix -> Hobbes
add_friend Bugs -> Astro
add_friend Felix -> Bugs
add_friend Bugs -> Felix
Animal: Felix, friends: Bugs, Hobbes
Animal: Astro, friends: Bugs
Animal: Hobbes, friends: Felix
Animal: Bugs, friends: Astro, Felix
delete_friend Felix -> Bugs
Animal: Felix, friends: Hobbes
Animal: Bugs, friends: Astro
\end{tcblisting}

与前面一样,由于异步操作,输出的行顺序可能不同。

\end{itemize}

\subsubsection{How it works…}

互斥锁并不锁定数据,理解这一点很重要,它阻碍了执行。如本文所示,在对象方法中应用互斥锁时,可以使用它强制对数据进行互斥访问。

当一个线程用lock()或lock\_guard锁住一个互斥量时,就说这个线程拥有这个互斥量。其他试图锁定同一个互斥锁的线程都将阻塞,直到锁的所有者将其解锁。

互斥对象在被任何线程拥有时不能被销毁。同样,当线程拥有互斥时,不能销毁。与RAII兼容的包装器,比如lock\_guard,将确保这种情况不会发生。

\subsubsection{There's more…}

虽然,std::mutex提供了一个适用于许多排他性互斥锁,但STL确实提供了一些其他选择:

\begin{itemize}
\item 
shared\_mutex 允许多个线程同时拥有一个互斥量。

\item 
recursive\_mutex 允许一个线程在一个互斥锁上叠加多个锁。

\item 
timed\_mutex 为互斥锁提供超时。shared\_mutex和recursive\_mutex也有定时版本可用。
\end{itemize}












