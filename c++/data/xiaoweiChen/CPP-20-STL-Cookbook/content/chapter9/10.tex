
生产者-消费者问题实际上是一组问题。若缓冲区是有界的或无界的,或者有多个生产者、多个消费者,或者两者都有,解决方案将有所不同。

来考虑一个有多个生产者、多个消费者和一个有限(容量有限)缓冲区的情况。

\subsubsection{How to do it…}

在这个示例中,我们将看到一个有多个生产者和消费者的情况,以及一个有界缓冲区,并使用本章中介绍的各种技术:

\begin{itemize}
\item 
为了方便和可靠性,先从一些常量开始:

\begin{lstlisting}[style=styleCXX]
constexpr auto delay_time{ 50ms };
constexpr auto consumer_wait{ 100ms };
constexpr size_t queue_limit{ 5 };
constexpr size_t num_items{ 15 };
constexpr size_t num_producers{ 3 };
constexpr size_t num_consumers{ 5 };
\end{lstlisting}

\begin{itemize}
\item
delay\_time是一个duration对象,与sleep\_for()一起使用。

\item 
consumer\_wait是一个持续时间对象,与消费者条件变量一起使用。

\item 
queue\_limit是缓冲区限制——deque中的最大容量值。

\item 
num\_items是每个生产者生产的最大产品目数。

\item 
num\_producers是消费者的数量。
\end{itemize}

\item 
需要一些对象来控制这个过程:

\begin{lstlisting}[style=styleCXX]
deque<string> qs{};
mutex q_mutex{};
condition_variable cv_producer{};
condition_variable cv_consumer{};
bool production_complete{};
\end{lstlisting}

\begin{itemize}
\item 
qs是一个字符串deque,用于保存生成的对象。

\item 
q\_mutex可以对deque的访问进行控制。

\item 
queue\_limt是缓冲区限制——deque中项目的最大容量。

\item 
cv\_producer是协调生产者的条件变量。

\item 
cv\_consumer是协调使用者的条件变量。

\item 
production\_complete当所有生产者线程完成时,将其设置为true。
\end{itemize}

\item 
producer()线程运行这个函数:

\begin{lstlisting}[style=styleCXX]
void producer(const size_t id) {
	for(size_t i{}; i < num_items; ++i) {
		this_thread::sleep_for(delay_time * id);
		unique_lock<mutex> lock(q_mutex);
		cv_producer.wait(lock,
			[&]{ return qs.size() < queue_limit; });
		qs.push_back(format("pid {}, qs {},
			item {:02}\n", id, qs.size(), i + 1));
		cv_consumer.notify_all();
	}
}
\end{lstlisting}

传递值的id0用于标识生产者的连续数字。

主for循环重复num\_item次。sleep\_for()函数用于模拟生成一个项所需的一些工作。

然后,从q\_mutex中获得一个唯一的\_lock,并在cv\_producer上调用wait(),使用lambda根据queue\_limit常量检查deque的大小。若deque已经达到上限,生产者等待消费者线程来减小deque的体积。这表示生成器的有界缓冲区限制。

当条件满足,可以将一个产品推入deque。元素是一个格式化的字符串,包含生产者的id、qs的大小和来自循环控制变量的项目编号(i + 1)。

最后,在cv\_consumer条件变量上使用notify\_all()通知消费者有新数据可用。

\item 
consumer()线程运行这个函数:

\begin{lstlisting}[style=styleCXX]
void consumer(const size_t id) {
	while(!production_complete) {
		unique_lock<mutex> lock(q_mutex);
		cv_consumer.wait_for(lock, consumer_wait,
			[&]{ return !qs.empty(); });
		if(!qs.empty()){
			cout << format("cid {}: {}", id,
				qs.front());
			qs.pop_front();
		}
		cv_producer.notify_all();
	}
}
\end{lstlisting}

传递的id值是用于标识使用者的连续数字。

主while()循环继续进行,直到对production\_complete进行设置。

我们从q\_mutex中获得unique\_lock,并在cv\_consumer上调用wait\_for(),使用一个超时和一个lambda来测试deque是否为空。这里需要超时,因为当一些消费者线程仍在运行时,生产者线程可能已经完成,可使deque为空。

当有了一个非空的deque,就可以打印(消费)一个产品信息,并将其从deque中弹出。

\item 
在main()中,使用async()来生成生产者线程和消费者线程。async()符合RAII模式,所以我通常更喜欢使用async。async()返回一个future对象,可以保留一个future<void>对象列表用于进程管理:

\begin{lstlisting}[style=styleCXX]
int main() {
	list<future<void>> producers;
	list<future<void>> consumers;
	for(size_t i{}; i < num_producers; ++i) {
		producers.emplace_back(async(producer, i));
	}
	for(size_t i{}; i < num_consumers; ++i) {
		consumers.emplace_back(async(consumer, i));
	}
	...
\end{lstlisting}

使用for循环来创建生产者和消费者线程。

\item 
最后,可以使用future对象的列表来确定生产者和消费者线程何时完成:

\begin{lstlisting}[style=styleCXX]
for(auto& f : producers) f.wait();
production_complete = true;
cout << "producers done.\n";

for(auto& f : consumers) f.wait();
cout << "consumers done.\n";
\end{lstlisting}

循环生成器容器,调用wait()以允许生成器线程完成。然后,可以设置production\_complete标志。同样,循环遍历消费者容器,调用wait()以允许消费者线程完成。并且,可以在这里执行最终的分析或完成过程。

\item 
输出有点长,就不全部展示了:

\begin{tcblisting}{commandshell={}}
cid 0: pid 0, qs 0, item 01
cid 0: pid 0, qs 1, item 02
cid 0: pid 0, qs 2, item 03
cid 0: pid 0, qs 3, item 04
cid 0: pid 0, qs 4, item 05
...
cid 4: pid 2, qs 0, item 12
cid 4: pid 2, qs 0, item 13
cid 3: pid 2, qs 0, item 14
cid 0: pid 2, qs 0, item 15
producers done.
consumers done.
\end{tcblisting}

\end{itemize}


\subsubsection{How it works…}

这个配方的核心是使用两个condition\_variable对象来异步控制生产者和消费者线程:

\begin{lstlisting}[style=styleCXX]
condition_variable cv_producer{};
condition_variable cv_consumer{};
\end{lstlisting}

在producer()函数中,cv\_producer对象获得unique\_lock,等待deque可用,并在生成产品时通知cv\_consumer对象:

\begin{lstlisting}[style=styleCXX]
void producer(const size_t id) {
	for(size_t i{}; i < num_items; ++i) {
		this_thread::sleep_for(delay_time * id);
		unique_lock<mutex> lock(q_mutex);
		cv_producer.wait(lock,
			[&]{ return qs.size() < queue_limit; });
		qs.push_back(format("pid {}, qs {}, item {:02}\n",
			id, qs.size(), i + 1));
		cv_consumer.notify_all();
	}
}
\end{lstlisting}

相反,在consumer()函数中,cv\_consumer对象获得unqiue\_lock,等待deque中有产品,并在产品消费时通知cv\_producer对象:

\begin{lstlisting}[style=styleCXX]
void consumer(const size_t id) {
	while(!production_complete) {
		unique_lock<mutex> lock(q_mutex);
		cv_consumer.wait_for(lock, consumer_wait,
			[&]{ return !qs.empty(); });
		if(!qs.empty()) {
			cout << format("cid {}: {}", id, qs.front());
			qs.pop_front();
		}
		cv_producer.notify_all();
	}
}
\end{lstlisting}

这些锁、等待和通知,构成了多个生产者和消费者之间的动态平衡。







