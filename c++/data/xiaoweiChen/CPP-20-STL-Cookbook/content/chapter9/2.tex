
<thread>头文件提供了两个使线程进入休眠状态的函数,sleep\_for()和sleep\_until()。这两个函数都在std::this\_thread命名空间中。

本节探讨了这些函数的使用,我们将在本章后面使用它们。

\subsubsection{How to do it…}

看看如何使用sleep\_for()和sleep\_until()函数:

\begin{itemize}
\item 
与休眠相关的函数在std::this\_thread命名空间中。只有几个符号,可以为std::this\_thread和std::chrono\_literals使用using指令:

\begin{lstlisting}[style=styleCXX]
using namespace std::this_thread;
using namespace std::chrono_literals;
\end{lstlisting}

chrono\_literals命名空间具有表示持续时间的符号,例如1s表示一秒,100ms表示100毫秒。

\item 
在main()中,用steady\_clock::now()标记一个时间点,这样就可以计算测试时间了:

\begin{lstlisting}[style=styleCXX]
int main() {
	auto t1 = steady_clock::now();
	cout << "sleep for 1.3 seconds\n";
	sleep_for(1s + 300ms);
	cout << "sleep for 2 seconds\n";
	sleep_until(steady_clock::now() + 2s);
	duration<double> dur1 = steady_clock::now() - t1;
	cout << format("total duration: {:.5}s\n",
		dur1.count());
}
\end{lstlisting}

sleep\_for()函数的作用是:接受一个duration对象来指定睡眠时间。参数(1s + 300ms)使用chrono\_literal操作符返回一个表示1.3秒的duration对象。

sleep\_until()函数接受一个time\_point对象来指定从睡眠状态恢复的特定时间。使用chrono\_literal操作符,来修改从steady\_clock::now()返回的time\_point对象。

输出:

\begin{tcblisting}{commandshell={}}
sleep for 1.3 seconds
sleep for 2 seconds
total duration: 3.3005s
\end{tcblisting}

\end{itemize}

\subsubsection{How it works…}

sleep\_for(duration)和sleep\_until(time\_point)函数在指定的时间内暂停当前线程的执行,或者直到到达time\_point。若支持的话,sleep\_for()函数将使用steady\_clock实现。

否则,持续时间可能会有所调整。由于调度或资源延迟,这两个功能可能会阻塞更长的时间。

\subsubsection{There's more…}

一些系统支持POSIX函数sleep(),会在指定的秒数内暂停执行:

\begin{lstlisting}[style=styleCXX]
unsigned int sleep(unsigned int seconds);
\end{lstlisting}

sleep()函数是POSIX标准的一部分,而不是C++标准的一部分。







