
通过将lambda包装在函数中,可以轻松地创建具有不同捕获值的lambda的多个实例。可以使用相同的输入调用lambda的不同版本。

\subsubsection{How to do it…}

这是一个用不同类型的大括号包装值的简单例子:

\begin{itemize}
\item 
首先创建包装器函数braces():

\begin{lstlisting}[style=styleCXX]
auto braces (const char a, const char b) {
	return [a, b](const char v) {
		cout << format("{}{}{} ", a, v, b);
	};
}
\end{lstlisting}

braces()函数包装了一个lambda,该lambda返回一个三值字符串,其中第一个和最后一个值是作为捕获传递给lambda的字符,中间的值作为参数传递。

\item 
在main()函数中,使用braces()创建了四个lambda,使用了四组不同的括号:

\begin{lstlisting}[style=styleCXX]
auto a = braces('(', ')');
auto b = braces('[', ']');
auto c = braces('{', '}');
auto d = braces('|', '|');
\end{lstlisting}

\item 
现在,可以在简单的for()循环中调用lambda:

\begin{lstlisting}[style=styleCXX]
for( int i : { 1, 2, 3, 4, 5 } ) {
	for( auto x : { a, b, c, d } ) x(i);
	cout << '\n';
}
\end{lstlisting}

这是两个嵌套的for()循环。外部循环只是从1数到5,将一个整数传递给内部循环。内部循环调用带大括号的lambda函数。

这两个循环都使用初始化器列表作为基于范围的for()循环中的容器。这是一种方便的方法,可以循环使用一小组值。

\item 
程序的输出如下所示:

\begin{tcblisting}{commandshell={}}
(1) [1] {1} |1|
(2) [2] {2} |2|
(3) [3] {3} |3|
(4) [4] {4} |4|
(5) [5] {5} |5|
\end{tcblisting}

输出会显示了每个整数的括号组合。
\end{itemize}

\subsubsection{How it works…}

这是一个如何为lambda使用包装器的简单示例。braces()函数使用传递给它的括号构造了一个lambda:

\begin{lstlisting}[style=styleCXX]
auto braces (const char a, const char b) {
	return [a, b](const auto v) {
		cout << format("{}{}{} ", a, v, b);
	};
}
\end{lstlisting}

通过将括号()函数参数传递给lambda,可以返回具有该上下文的lambda。所以,main函数中的每个赋值操作都带有这些参数:

\begin{lstlisting}[style=styleCXX]
auto a = braces('(', ')');
auto b = braces('[', ']');
auto c = braces('{', '}');
auto d = braces('|', '|');
\end{lstlisting}

当使用数字调用这些lambda时,将返回一个字符串,对应的大括号中会包含该数字。









