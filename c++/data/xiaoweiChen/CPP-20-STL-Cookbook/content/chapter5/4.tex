
算法库中的某些函数需要使用谓词函数。谓词是测试条件并返回布尔true/false响应的函数(或函子或lambda)。

\subsubsection{How to do it…}

对于这个示例,我们将使用不同类型的谓词来测试count\_if()算法:

\begin{itemize}
\item 
首先,创建一个用作谓词的函数。谓词接受一定数量的参数并返回bool类型。count\_if()的谓词有一个参数:

\begin{lstlisting}[style=styleCXX]
bool is_div4(int i) {
	return i % 4 == 0;
}
\end{lstlisting}

该谓词检查int值是否能被4整除。

\item 
在main()中,定义一个int值的vector,并使用它来测试count\_if()的谓词:

\begin{lstlisting}[style=styleCXX]
int main() {
	const vector<int> v{ 1, 7, 4, 9, 4, 8, 12, 10, 20 };
	int count = count_if(v.begin(), v.end(), is_div4);
	cout << format("numbers divisible by 4: {}\n",
		count);
}
\end{lstlisting}

输出如下:

\begin{tcblisting}{commandshell={}}
numbers divisible by 4: 5
\end{tcblisting}

(5个可整除数分别是:4,4,8,12和20。)

count\_if()算法使用谓词来确定计算序列中的哪些元素,对每个元素作为参数调用谓词,并且仅在谓词返回true时计数元素。

本例中,我们使用函数作为谓词。

\item 
也可以将函子用作谓词:

\begin{lstlisting}[style=styleCXX]
struct is_div4 {
	bool operator()(int i) {
		return i % 4 == 0;
	}
};
\end{lstlisting}

这里唯一的变化是需要使用类的实例作为谓词:

\begin{lstlisting}[style=styleCXX]
int count = count_if(v.begin(), v.end(), is_div4());
\end{lstlisting}

函子的优点是可以携带上下文并访问类和实例变量。在C++11引入lambda表达式之前,这是使用谓词的常用方法。

\item 
使用lambda表达式,其具有两个优点:函数的简单性和函子的强大功能。可以使用lambda作为变量:

\begin{lstlisting}[style=styleCXX]
auto is_div4 = [](int i){ return i % 4 == 0; };
int count = count_if(v.begin(), v.end(), is_div4);
\end{lstlisting}

或者可以使用匿名lambda:

\begin{lstlisting}[style=styleCXX]
int count = count_if(v.begin(), v.end(),
	[](int i){ return i % 4 == 0; });
\end{lstlisting}

\item 
可以通过在函数中包装lambda来利用lambda捕获,并使用该函数上下文生成具有不同参数的相同lambda:

\begin{lstlisting}[style=styleCXX]
auto is_div_by(int divisor) {
	return [divisor](int i){ return i % divisor == 0; };
}
\end{lstlisting}

该函数从捕获上下文返回一个带除数的谓词lambda。

然后,count\_if()可以使用这个谓词

\begin{lstlisting}[style=styleCXX]
for( int i : { 3, 4, 5 } ) {
	auto pred = is_div_by(i);
	int count = count_if(v.begin(), v.end(), pred);
	cout << format("numbers divisible by {}: {}\n", i,
		count);
}
\end{lstlisting}

每次调用is\_div\_by()都会返回一个与i除数不同的谓词。

现在可得到这样的输出:

\begin{tcblisting}{commandshell={}}
numbers divisible by 3: 2
numbers divisible by 4: 5
numbers divisible by 5: 2
\end{tcblisting}
\end{itemize}

\subsubsection{How it works…}

函数指针的类型表示,为调用函数操作符的指针:

\begin{lstlisting}[style=styleCXX]
void (*)()
\end{lstlisting}

可以声明一个函数指针,并用函数名对其进行初始化:

\begin{lstlisting}[style=styleCXX]
void (*fp)() = func;
\end{lstlisting}

声明后,函数指针可以解引用,并像使用函数本身一样使用:

\begin{lstlisting}[style=styleCXX]
func(); // do the func thing
\end{lstlisting}

lambda表达式具有与函数指针相同的类型:

\begin{lstlisting}[style=styleCXX]
void (*fp)() = []{ cout << "foo\n"; };
\end{lstlisting}

无论在哪里使用具有特定签名的函数指针,都可以使用具有相同签名的lambda。这使得函数指针、函子和lambda的工作方式一致:

\begin{lstlisting}[style=styleCXX]
bool (*fp)(int) = is_div4;
bool (*fp)(int) = [](int i){ return i % 4 == 0; };
\end{lstlisting}

由于这种等价性,像count\_if()这样的算法接受函数、函子或lambda(在这些函数中期望具有特定函数签名的谓词)。

这适用于所有使用谓词的算法。





