
给定两个相似的向量,仅通过量化或分辨率不同,我们可以使用inner\_product()算法来计算误差和,定义为:

\begin{equation*}
e=\sum_{n}^{i=1}(a_i-b_i)^2
\end{equation*}

\begin{center}
图11.2 误差和的定义
\end{center}

其中e为误差和,即两个向量中一系列点的差的平方和。

可以使用<numeric>中的inner\_product()算法来计算两个向量之间的误差和。

\subsubsection{How to do it…}

在这个示例中,定义了两个向量,每个向量都有一个正弦波。一个vector的值是double类型,另一个是int类型。这给了我们量子化不同的向量,因为int类型不能表示分数值。然后我们使用inner\_product()来计算两个向量之间的误差和:

\begin{itemize}
\item 
在main()函数中,我们定义了vector和一个索引变量:

\begin{lstlisting}[style=styleCXX]
int main() {
	constexpr size_t vlen{ 100 };
	vector<double> ds(vlen);
	vector<int> is(vlen);
	size_t index{};
	...
\end{lstlisting}

ds是双正弦波的vector,是int正弦波的vector。每个向量有100个元素来保存一个正弦波,index变量用于初始化vector对象。

\item 
用一个循环和lambda在double的vector中生成正弦波:

\begin{lstlisting}[style=styleCXX]
auto sin_gen = [&index]{
	return 5.0 * sin(index++ * 2 * pi / 100);
};
for(auto& v : ds) v = sin_gen();
\end{lstlisting}

lambda捕获对索引变量的引用,因此可以对其进行递增。

pi常数来自std::numbers库。

\item 
我们现在有了一个双正弦波,可以用它来推导int版本:

\begin{lstlisting}[style=styleCXX]
index = 0;
for(auto& v : is) {
	v = static_cast<int>(round(ds.at(index++)));
}
\end{lstlisting}

这将从ds中取每个点,四舍五入,将其转换为int类型,并更新它在is容器中的位置。

\item 
用一个简单的循环来显示正弦波:

\begin{lstlisting}[style=styleCXX]
for(const auto& v : ds) cout << format("{:-5.2f} ", v);
cout << "\n\n";
for(const auto& v : is) cout << format("{:-3d} ", v);
cout << "\n\n";
\end{lstlisting}

输出是两个容器中作为数据点的正弦波:

\begin{tcblisting}{commandshell={}}
0.00 0.31 0.63 0.94 1.24 1.55 1.84 2.13 2.41
2.68 2.94 3.19 3.42 3.64 3.85 4.05 4.22 4.38
4.52 4.65 4.76 4.84 4.91 4.96 4.99 5.00 4.99
4.96 4.91 4.84 4.76 4.65 4.52 4.38 4.22 4.05
3.85 3.64 3.42 3.19 2.94 2.68 2.41 2.13 1.84
1.55 1.24 0.94 0.63 0.31 0.00 -0.31 -0.63 -0.94
-1.24 -1.55 -1.84 -2.13 -2.41 -2.68 -2.94 -3.19 -3.42
-3.64 -3.85 -4.05 -4.22 -4.38 -4.52 -4.65 -4.76 -4.84
-4.91 -4.96 -4.99 -5.00 -4.99 -4.96 -4.91 -4.84 -4.76
-4.65 -4.52 -4.38 -4.22 -4.05 -3.85 -3.64 -3.42 -3.19
-2.94 -2.68 -2.41 -2.13 -1.84 -1.55 -1.24 -0.94 -0.63
-0.31
0 0 1 1 1 2 2 2 2 3 3 3 3 4 4
4 4 4 5 5 5 5 5 5 5 5 5 5 5 5
5 5 5 4 4 4 4 4 3 3 3 3 2 2 2
2 1 1 1 0 0 0 -1 -1 -1 -2 -2 -2 -2 -3
-3 -3 -3 -4 -4 -4 -4 -4 -5 -5 -5 -5 -5 -5
-5 -5 -5 -5 -5 -5 -5 -5 -5 -4 -4 -4 -4 -4
-3 -3 -3 -3 -2 -2 -2 -2 -1 -1 -1 0
\end{tcblisting}

\item 
现在使用inner\_product()计算误差和:

\begin{lstlisting}[style=styleCXX]
double errsum = inner_product(ds.begin(), ds.end(),
	is.begin(), 0.0, std::plus<double>(),
	[](double a, double b){ return pow(a - b, 2); });
cout << format("error sum: {:.3f}\n\n", errsum);
\end{lstlisting}

lambda表达式返回公式的$ (a_i - b_i)^2 $的部分。std::plus()算法执行求和运算。

输出为:

\begin{tcblisting}{commandshell={}}
error sum: 7.304
\end{tcblisting}
\end{itemize}

\subsubsection{How it works…}

inner\_product()算法计算第一个输入范围内的乘积之和,签名为:

\begin{lstlisting}[style=styleCXX]
T inner_product(InputIt1 first1, InputIt1 last1,
	InputIt2 first2, T init, BinaryOperator1 op1,
	BinaryOperator2 op2)
\end{lstlisting}

该函数接受两个二元算子函子op1和op2。第一个op1用于求和,第二个op2用于乘积。使用std::plus()作为和运算符,lambda作为积运算符。

init参数可以用作起始值或偏置,将0.0传递给它。

返回值是产品的累计总和。

\subsubsection{There's more…}

可以通过在循环中放入inner\_product()来计算累积的误差和:

\begin{lstlisting}[style=styleCXX]
cout << "accumulated error:\n";
for (auto it{ds.begin()}; it != ds.end(); ++it) {
	double accumsum = inner_product(ds.begin(), it,
		is.begin(), 0.0, std::plus<double>(),
		[](double a, double b){ return pow(a - b, 2); });
	cout << format("{:-5.2f} ", accumsum);
}
cout << '\n';
\end{lstlisting}

输出为:

\begin{tcblisting}{commandshell={}}
accumulated error:
0.00 0.00 0.10 0.24 0.24 0.30 0.51 0.53 0.55 0.72
0.82 0.82 0.86 1.04 1.16 1.19 1.19 1.24 1.38 1.61
1.73 1.79 1.82 1.82 1.83 1.83 1.83 1.83 1.83 1.84
1.86 1.92 2.04 2.27 2.42 2.46 2.47 2.49 2.61 2.79
2.83 2.83 2.93 3.10 3.12 3.14 3.35 3.41 3.41 3.55
3.65 3.65 3.75 3.89 3.89 3.95 4.16 4.19 4.20 4.37
4.47 4.48 4.51 4.69 4.82 4.84 4.84 4.89 5.03 5.26
5.38 5.44 5.47 5.48 5.48 5.48 5.48 5.48 5.48 5.49
5.51 5.57 5.70 5.92 6.07 6.12 6.12 6.14 6.27 6.45
6.48 6.48 6.59 6.75 6.77 6.80 7.00 7.06 7.07 7.21
\end{tcblisting}

这在某些统计应用中可能有用。









