
只要添加了新特性,C++就会提供了某种形式的特性测试宏。C++20起这个过程标准化了,所有库特性的测试宏,都会放到<version>头文件中。这将使测试代码中的新特性变得更加容易。

这是一个非常有用的功能,使用起来非常简单。

\subsubsection{How to do it…}

所有特性测试宏都以前缀\_\_cpp\_开头。库特性以\_\_cpp\_lib\_开头。语言特性测试宏通常由编译器定义,库特性测试宏定义在新的<version>头文件中。可以像使用其他预处理器宏一样使用:

\begin{lstlisting}[style=styleCXX]
#include <version>
#ifdef __cpp_lib_three_way_comparison
# include <compare>
#else
# error Spaceship has not yet landed
#endif
\end{lstlisting}

某些情况下,可以使用\_\_has\_include预处理器操作符(C++17)来测试是否包含了某个文件。

\begin{lstlisting}[style=styleCXX]
#if __has_include(<compare>)
# include <compare>
#else
# error Spaceship has not yet landed
#endif
\end{lstlisting}

因为它是一个预处理器指令,所以可以使用\_\_has\_include来测试头文件是否存在。

\subsubsection{How it works…}

通常,可以通过使用\#ifdef或\#if定义测试非零值来使用特性测试宏。每个特性测试宏都有一个非零值,对应于标准委员会接受它的年份和月份。例如,\_\_cpp\_lib\_three\_way\_comparison宏的值为2011907,意味着其在2019年7月采纳。

\begin{lstlisting}[style=styleCXX]
#include <version>
#ifdef __cpp_lib_three_way_comparison
cout << "value is " << __cpp_lib_three_way_comparison << "\n"
#endif
\end{lstlisting}

输出为:

\begin{tcblisting}{commandshell={}}
$ ./working
value is 201907
\end{tcblisting}

宏的值可能在一些的情况下很有用,比如某个特性发生了变化,而程序会依赖于这些变化。通常,可以安全地忽略该值,只使用\#ifdef测试非零即可。

一些网站维护功能测试宏的完整列表,这里推荐cppreference(\url{https://j.bw.org/cppfeature})。








