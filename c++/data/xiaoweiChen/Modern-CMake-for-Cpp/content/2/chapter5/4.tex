
优化器将分析前一阶段的输出,并使用大量的技巧,开发者可能会认为这些技巧很脏,因为它们不遵守干净代码的原则。这没关系——优化器的关键作用是提高代码的性能(使用较少的CPU周期、较少的寄存器和较少的内存)。当优化器遍历源代码时,将对其进行大量转换,以至于几乎无法阅读,会变成为目标CPU专门准备的版本。优化器不仅将决定哪些函数可以删除或压缩,还会移动代码,甚至复制代码!若能够完全确定某些代码行没有意义,其将从一个重要函数的中间删除(开发者不太会注意)。它将重用内存,因此许多变量可以在不同的时间段占用相同的插槽。若可以在这里或那里减少一些指令周期,把相应的控制结构转变成完全不同的结构。

这里描述的技术,若由程序员手动应用到源代码中,将会造成源代码变成可怕的、不可读的混乱符号集。很难阅读的同时,这些符号也很难写,也就更难于人为推演了。

另一方面,若编译器应用这些代码,其将完全按照所写的顺序进行。优化器是一头无情的野兽,只服务于一个目的:使执行快速,无论输出将如何混乱。若在测试环境中运行这样的输出,那么可能包含一些调试信息,也可能不包含,使未经授权的人难以修改。

每个编译器都有自己的技巧,与所遵循的平台和原理保持一致。我们将看一看最常见的,在GNU GCC和LLVM Clang中可用的优化器,从而了解什么是有用的,什么是可能的。

许多编译器默认情况下不会启用任何优化(包括GCC),但在其他情况下就不是这样了。能快走为什么要慢走?要更改这些内容,可以使用target\_compile\_options()指令,并确切地指定想从编译器获得的内容。

该指令的语法与本章其他指令类似:

\begin{lstlisting}[style=styleCMake]
target_compile_options(<target> [BEFORE]
	<INTERFACE|PUBLIC|PRIVATE> [items1...]
	[<INTERFACE|PUBLIC|PRIVATE> [items2...] ...])
\end{lstlisting}  

我们提供要添加的目标命令行选项,并指定传播关键字。指令执行时,CMake将把给定的选项附加到目标的适当的COMPILE\_OPTIONS变量中。可选的BEFORE关键字可以用来指定把相应的选项放在前面。

\begin{tcolorbox}[colback=blue!5!white,colframe=blue!75!black,title=重要的Note]
target\_compile\_options()是一个通用指令。还可以用来为类似编译器的-D定义提供其他参数,为此CMake还提供了target\_compile\_definition()指令。总是建议尽可能使用CMake指令,因为其在所有支持的编译器上以相同的方式工作。
\end{tcolorbox}

后面的部分将介绍可以在大多数编译器中启用的各种优化。

\subsubsubsection{5.4.1\hspace{0.2cm}优化级别}

优化器的所有不同行为,都可以通过编译选项传递的特定标志进行配置。要了解所有这些方法挺花时间的,并且需要了解大量关于编译器、处理器和内存内部工作原理的知识。若只想要在大多数情况下运行良好的最佳可能场景,能做什么呢?可以找到一个通用的解决方案——优化级说明符。

大多数编译器提供了从0到3的四个基本优化级别,用-O<level>选项来指定。-O0表示不进行优化,通常这是编译器的默认级别。另一方面,-O2认为是完全优化,生成高度优化的代码,但以最慢的编译时间为代价。

有一个介于两者之间的-O1级别,这(取决于需要)可能是一个很好的折衷方案——其支持合理数量的优化机制,而不会大幅降低编译速度。

可以使用-O3,这是完全优化,就像-O2一样,但使用更积极的方法实现子程序内联和循环向量化。

还有一些优化变量会优化生成文件的大小(不一定是速度),-Os。有一个超级激进的优化,-Ofast,不严格遵守C++标准的-O3优化。最明显的区别是-fast-math和-ffinite-math标志的使用,若程序是关于精确计算的(大多数都是这样),最好望不使用这个参数。

CMake了解不是所有的编译器都是相同的。因此,其通过为编译器提供一些默认标志来标准化开发人员的体验,存储在所用语言(CXX用于C++)和构建配置(DEBUG或RELEASE)的系统范围(不是特定于目标的)变量中:

\begin{itemize}
\item 
CMAKE\_CXX\_FLAGS\_DEBUG 等于 -g

\item 
CMAKE\_CXX\_FLAGS\_RELEASE 等于 -O3 -DNDEBUG
\end{itemize}

Debug配置不支持任何优化,发布配置直接针对O3。可以使用set()指令直接修改,或者添加一个目标编译选项,覆盖默认行为。另外两个标志(-g, -DNDEBUG)与调试相关——将在为调试器提供信息一节中讨论。

诸如CMAKE\_<LANG>\_FLAGS\_<CONFIG>这样的变量是全局变量——适用于所有目标。建议通过属性和target\_compile\_options()等指令配置目标,而不是依赖全局变量。通过这种方式,可以更细粒度的控制目标。

通过使用-O<level>选择优化级别,间接地设置了一长串标志,每个标志控制特定的优化行为。然后,可以通过添加更多标志来微调优化:

\begin{itemize}
\item 
使用-f选项启用它们: -finline-functions.
	
\item 
使用-fno选项禁用它们: -fno-inline-functions.
\end{itemize}

其中一些标志需要花点时间了解,因为它们会影响程序的工作方式和调试方式。

\subsubsubsection{5.4.2\hspace{0.2cm}函数内联}

代码中,可以通过在类声明块中定义函数或显式使用inline关键字来提示编译器内联一些函数:

\begin{lstlisting}[style=styleCXX]
struct X {
	void im_inlined(){ cout << "hi\n"; };
	void me_too();
};
inline void X::me_too() { cout << "bye\n"; };
\end{lstlisting} 

由编译器决定函数是否内联。若启用了内联,并且函数只在一个地方使用(或者是在几个地方使用的相对较小的函数),那么很可能会触发内联。

这是一种非常有趣的优化技术。其工作原理是,从有问题的函数中提取代码,并将其放在调用函数的所有位置,替换原始调用并节省宝贵的CPU周期。

考虑以下类:

\begin{lstlisting}[style=styleCXX]
int main() {
	X x;
	x.im_inlined();
	x.me_too();
	return 0;
}
\end{lstlisting} 

若没有内联,代码将在main()中执行,直到方法调用。然后,为im\_inline()创建一个新函数,在单独的作用域中执行,并返回main()。me\_too()也是同样的情况。

当内联时,编译器将替换调用:

\begin{lstlisting}[style=styleCXX]
int main() {
	X x;
	cout << "hi\n";
	cout << "bye\n";
	return 0;
}
\end{lstlisting} 

这不是一种精确的表示,因为内联发生在汇编或机器码级别(而不是源码级别)。编译器这样做是为了节省时间,不需要创建和删除一个新的调用作用域,也不需要查找要执行(和返回)的下一条指令的地址,而且可以更好地缓存附近的指令。

当然,内联有一些重要的副作用;若函数多次使用,则必须将其复制到所有位置(意味着更大的文件大小和更多的内存使用)。现在,这可能不像过去那么重要了,但仍是相关的,因为不断开发的软件需要运行在没有太多RAM的低端设备上。

除此之外,当调试代码时,会对我们产生严重影响。内联代码不再是其最初编写的行号,因此不容易(有时甚至不可能)跟踪。这正是,放置在内联函数中的断点从未命中的原因(尽管代码仍然以某种方式执行)。为了避免这个问题,需要为调试版本禁用内联(代价是不能测试与发布版本完全相同)。

可以通过为目标指定-O0级别,或者直接在负责的标志后面执行:

\begin{itemize}
\item 
-finline-functions-called-once: 只有GCC可用

\item 
-finline-functions: Clang和GCC都可用

\item 
-finline-hint-functions: 只有Clang可用

\item 
-finline-functions-called-once: 只有GCC可用
\end{itemize}

可以使用-fno-inline-...,显式禁用内联。有关详细信息,请参阅编译器的特定版本的文档。

\subsubsubsection{5.4.3\hspace{0.2cm}循环展开}

循环展开是一种优化技术,方法是将循环转换为一组语句,以达到相同的效果。可以用程序的大小来换取执行速度,将减少或消除控制循环指针算术或循环结束测试的指令。

考虑下面的例子:

\begin{lstlisting}[style=styleCXX]
void func() {
	for(int i = 0; i < 3; i++)
	cout << "hello\n";
}
\end{lstlisting}

前面的代码将转换成这样:

\begin{lstlisting}[style=styleCXX]
void func() {
	cout << "hello\n";
	cout << "hello\n";
	cout << "hello\n";
}
\end{lstlisting}

结果相同,但是不再需要分配i变量,增加它的值,或者将它与值3进行三次比较。若在程序的生命周期中调用func()的次数足够多,那么展开即使是这样一个短小的函数也会产生显著的差异。

然而,重要的是要了解两个限制因素。循环展开只有在编译器知道或能够有效估计迭代量的情况下才会有用。其次,循环展开可能会对现代CPU产生不良影响,因为增加的代码大小可能会消耗有效的缓存。

每个编译器提供的该标志版本略有不同:

\begin{itemize}
\item 
-floop-unroll: GCC

\item 
-funroll-loops: Clang
\end{itemize}

若有疑问,请测试此标志是否影响程序,并显式启用或禁用它。注意,在GCC上是用-O3会隐式使用-floop-unroll-and-jam标志。

\subsubsubsection{5.4.4\hspace{0.2cm}循环向量化}

单指令多数据(SIMD)是20世纪60年代早期为实现并行而发展起来的机制之一。就像其名称一样,可以同时对多条信息执行相同的操作。实践中意味着什么?考虑下面的例子:

\begin{lstlisting}[style=styleCXX]
int a[128];
int b[128];
// initialize b
for (i = 0; i<128; i++)
	a[i] = b[i] + 5;
\end{lstlisting}

前面的代码将循环128次,但若有一个功能强大的CPU,可以通过同时计算数组中的两个或多个元素来更快地执行代码。这是可行的,因为连续的元素之间没有依赖关系,数组之间的数据也没有重叠。智能编译器可以将前面的循环转换为类似的东西(发生在汇编级别):

\begin{lstlisting}[style=styleCXX]
for (i = 0; i<32; i+=4) {
	a[ i ] = b[ i ] + 5;
	a[i+1] = b[i+1] + 5;
	a[i+2] = b[i+2] + 5;
	a[i+3] = b[i+3] + 5;
}
\end{lstlisting}

GCC将在-O3处启用这种循环的自动向量化,Clang默认启用它。这两个编译器提供了不同的标志来启用/禁用向量化:

\begin{itemize}
\item 
-ftree-vectorize -ftree-slp-vectorize在GCC中启用向量化

\item 
-fno-vectorize -fno-slp-vectorize在Clang中禁用向量化
\end{itemize}

向量化的性能来自于利用CPU供应商提供的特殊指令,而不是简单地用展开版本替换原来的循环形式。因此,通过手动操作不可能达到相同的性能水平(而且,这不是非常干净的代码)。

优化器的作用对于在运行时增强程序的性能非常重要。通过有效地运用它的策略,将得到更大的回报。效率不仅在编码完成后很重要,而且在软件开发过程中也很重要。若编译时间很长,可以通过更好地管理过程来进行改进。




