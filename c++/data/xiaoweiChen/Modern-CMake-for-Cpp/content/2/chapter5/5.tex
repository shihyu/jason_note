
作为开发者和构建工程师,还需要考虑编译的其他方面——完成编译所需的时间,以及在构建解决方案的过程中发现和修复错误的难易程度。

\subsubsubsection{5.5.1\hspace{0.2cm}减少编译时间}

每天(或每小时)需要多次重新编译的繁忙项目中,尽可能快地编译是至关重要的。这不仅会影响代码编译-测试循环的紧凑程度,还会影响注意力和工作流程。

由于有独立的翻译单元,C++已经非常擅长管理编译时间。CMake将只负责重新编译受最近更改影响的源代码。若需要进一步改进,可以使用一些技术——头文件预编译和统一构建。

\hspace*{\fill} \\ %插入空行
\noindent
\textbf{预编译头文件}

头文件(.h)在实际编译开始之前由预处理器包含在翻译单元中,所以每次.cpp实现文件更改时都必须重新编译。若多个翻译文件使用相同的共享头文件,那么每次包含它时都必须编译它。这是一种浪费,但在很长一段时间里情况就是这样。

从3.16版本开始,CMake提供了一个命令来启用头预编译。

这允许编译器将头文件与实现文件分开处理,从而加快编译速度。以下是相应的指令:

\begin{lstlisting}[style=styleCMake]
target_precompile_headers(<target>
	<INTERFACE|PUBLIC|PRIVATE> [header1...]
	[<INTERFACE|PUBLIC|PRIVATE> [header2...] ...])
\end{lstlisting}

添加的头文件列表存储在PRECOMPILE\_HEADERS目标属性中。第4章已经介绍,可以使用PUBLIC或INTERFACE关键字,通过传播的属性与依赖的目标共享,但对于通过install()指令导出的目标不行。其他项目不应该强制使用预编译的头文件。

\begin{tcolorbox}[colback=blue!5!white,colframe=blue!75!black,title=重要的Note]
若需要内部预编译头文件,并且仍然想安装-导出目标文件,\$<BUILD\_INTERFACE:...>的生成器表达式将防止头文件出现在使用中,但仍然会添加到使用export()指令从构建树导出的目标中。
\end{tcolorbox}

CMake将把所有头文件的名称放在一个cmake\_pch.h|xx文件中,然后该文件将预编译为一个特定于编译器的二进制文件,扩展名为.pch、.gch或.pchi。

可以这样使用:

\begin{lstlisting}[style=styleCMake]
# chapter05/06-precompile/CMakeLists.txt

add_executable(precompiled hello.cpp)
target_precompile_headers(precompiled PRIVATE <iostream>)
\end{lstlisting}

\begin{lstlisting}[style=styleCXX]
// chapter05/06-precompile/hello.cpp

int main() {
	std::cout << "hello world" << std::endl;
}
\end{lstlisting} 

main.cpp文件中,不需要包含cmake\_pch.h或任何其他头文件——将由cmake使用特定于编译器的命令行选项强制包含。

前面的例子中,使用了一个内置头文件。这里,可以很容易地用类或函数定义添加自己的头文件:

\begin{itemize}
\item 
header.h作为相对于当前源目录的文件,将包含在一个绝对路径中。

\item 
{}[["header.h"]]根据编译器的实现进行解释,通常在INCLUDE\_DIRECTORIES变量中找到。使用target\_include\_directories()来配置。
\end{itemize}

一些在线引用会阻止预编译不属于标准库的头文件,比如<iostream>,或者完全使用预编译的头文件。这是因为更改列表或编辑自定义头文件,将需要重新编译目标中的所有翻译单元。使用CMake不需要担心太多,特别是在正确地组织了项目的情况下(具有相对较小的目标,专注于单个领域)。每个目标都有一个单独的预编译头文件,其限制了头文件更改的影响。

另一方面,若头文件相当稳定,那么在另一个目标中重用预编译的头文件是个好主意。CMake为此提供了一个方便的指令:

\begin{lstlisting}[style=styleCMake]
target_precompile_headers(<target> REUSE_FROM
	<other_target>)
\end{lstlisting}

这将设置重用头文件的目标的PRECOMPILE\_HEADERS\_REUSE\_FROM属性,并在这些目标之间创建一个依赖关系。通过使用此方法,目标不再能够指定自己的预编译头文件。此外,所有编译选项、编译标志和编译定义必须在目标之间匹配。请注意需求,特别是使用双括号格式([["header.h"]])使用头文件。两个目标都需要设置头文件包含路径,以确保编译器能够找到这些头文件。

\hspace*{\fill} \\ %插入空行
\noindent
\textbf{统一构建}

CMake 3.16还引入了另一个编译时间优化特性——统一构建或巨型构建。统一构建用\#include指令组合多个实现源文件(毕竟,编译器不知道包含的是头文件还是实现)。其中,有些是真的有用的,而另一些是潜在的有害的。

从最明显的一个开始——当CMake创建统一构建文件时,需要避免在不同的翻译单元重新编译头文件:

\begin{lstlisting}[style=styleCXX]
#include "source_a.cpp"
#include "source_b.cpp"
\end{lstlisting}

当这两个源文件都包含\#include "header.h"时,由于有头文件包含守卫,相应的头文件只会解析一次(假设没有忘记添加这些守卫)。这没有预编译头文件那么优雅,但这是一个选项。

这种类型构建的第二个好处是,优化器现在可以在更大的规模上执行,并在所有绑定的源之间优化过程间调用,这类似于链接时优化。

然而,这些好处是有代价的。当减少了目标文件的数量和处理步骤时,也增加了处理更大文件所需的内存量。此外,减少了可并行工作的数量。编译器实际上并不擅长多线程编译——构建系统通常会启动许多编译任务,在不同的线程上同时执行所有文件。当把所有的文件聚集在一起时,这会使编译过程变得更加困难,因为CMake现在会在我们创建多少个大型构建之间调度并行构建。

统一构建中,还需要考虑一些C++语义含义,这些可能不太容易捕捉到——跨文件隐藏符号的匿名命名空间现在的作用域是组。静态全局变量、函数和宏定义也会发生同样的情况。这可能导致名称冲突,或执行错误的函数重载。

重新编译时,大型构建是不可取的,因为会编译比所需的多得多的文件。当代码要以尽可能快的速度编译所有文件时,其工作效率最高。在Qt Creator上进行的测试表明,性能提升幅度在20\% ~ 50\%之间(取决于使用的编译器)。

要启用统一构建,有两个选项:

\begin{itemize}
\item 
将CMAKE\_UNITY\_BUILD变量设置为true —— 在此之后,定义的每个目标上都会初始化UNITY\_BUILD属性。

\item 
在每个应该使用统一构建的目标上手动设置UNITY\_BUILD为true。
\end{itemize}

第二个选项是通过以下函数实现的:

\begin{lstlisting}[style=styleCMake]
set_target_properties(<target1> <target2> ...
					PROPERTIES UNITY_BUILD true)
\end{lstlisting}

通常,CMake将创建包含8个源文件的构建,由目标的UNITY\_BUILD\_BATCH\_SIZE属性指定(从CMAKE\_UNITY\_BUILD\_BATCH\_SIZE变量创建目标时复制)。这里,可以更改目标属性或默认变量。

从3.18版本开始,可以决定显式地定义如何将文件与命名组捆绑在一起。为此,改变目标的UNITY\_BUILD\_MODE属性为GROUP(默认总是BATCH)。然后,需要将源文件分配给组,通过设置他们的UNITY\_GROUP属性为设置的名称:

\begin{lstlisting}[style=styleCMake]
set_property(SOURCE <src1> <src2>...
			PROPERTY UNITY_GROUP "GroupA")
\end{lstlisting}

然后,CMake将忽略UNITY\_BUILD\_BATCH\_SIZE,并将组中的所有文件添加到单个统一构建中。

CMake的文档建议默认情况下不要为公共项目启用统一构建,建议应用程序的最终用户能够通过提供DCMAKE\_UNITY\_BUILD命令行参数来决定他们是否需要统一构建。而且,若因为代码的编写方式而引起问题,应该显式地将目标的统一构建属性设置为false。但是,没有什么可以阻止为内部使用的代码启用该特性,例如:公司内部或私人项目中。

\hspace*{\fill} \\ %插入空行
\noindent
\textbf{不支持C++20的模块}

若密切关注C++标准发布,就应该了解C++20中引入的新特性——模块。这是一个重大的游戏规则改变者,其允许使用头文件时避免许多麻烦,减少构建时间,并允许更干净、更紧凑的代码,更容易浏览和推理。

本质上,不需要创建单独的头文件和实现文件,而是创建一个带有模块声明的文件:

\begin{lstlisting}[style=styleCXX]
export module hello_world;
import <iostream>;
export void hello() {
	std::cout << "Hello world!\n";
}
\end{lstlisting}

然后,可以通过导入使用:

\begin{lstlisting}[style=styleCXX]
import hello_world;
int main() {
	hello();
}
\end{lstlisting}

注意,这里不再依赖于预处理器;模块有自己的关键字——import、export和module。作为编写和构建C++解决方案的新方法,主流编译器的最新版本已经可以执行支持模块的所有必要任务。我希望在本章开始的时候,CMake中已经提供了对模块的一些早期支持。但这还没有发生,有可能在你买这本书的时候(或不久之后)就有了。

Kitware开发人员已经创建(并在3.20中发布)了一个新的实验性特性,以支持对Ninja生成器的C++20模块依赖项扫描。目前,其仅针对编译器开发者,以便他们可以在开发依赖项扫描工具时进行测试。

当这个备受期待的特性完成并在稳定版本中可用时,我建议深入研究它。希望它能够简化和加快编译速度,这是目前可用的方法都无法媲美的。

\subsubsubsection{5.5.2\hspace{0.2cm}查找错误}

作为开发者,花了很多时间来寻找bug。这是一个可悲的事实。发现错误并解决它们通常会让我们感到恼火,特别是当这需要花费很长时间的时候。若是盲目查找,没有工具帮助我们,那就更困难了。这就是为什么我们应该小心地设置环境,使这个过程尽可能简单。为此,需要使用target\_compile\_options()配置编译器。那么,哪些编译选项可以帮助我们呢?

\hspace*{\fill} \\ %插入空行
\noindent
\textbf{错误和警告的配置}

软件开发中有很多压力很大的事情——在半夜修复关键的bug,在大型系统中处理高可见度、代价高昂的故障,处理烦人的编译错误,特别是那些难以理解或难以修复的错误。当为了简化工作和减少失败的机会而研究一个主题时,会发现很多关于如何配置编译器警告的建议。

一个很好的建议就是为所有构建启用-Werror标志作为默认值。这个标志的作用非常简单——所有警告都视为错误,除非解决了所有警告,否则代码不会编译。虽然这看起来像是一个好主意,但这从来都不是一个好主意。

警告不是错误是有原因的,这要由你来决定怎么做。开发者有忽视警告的自由,特别是当对解决方案进行试验和创建原型时,通常是一件好事。

另一方面,若有一段完美的、没有任何警告的、完美的代码,未来的修改破坏这种状态就太可惜了。启用警告视为错误,并持续使用有什么坏处呢?似乎没有。至少在编译器升级之前是这样。新版本的编译器倾向于严格限制已弃用的功能,或者只是更好地建议需要改进的地方。当不把所有的警告都视为错误时,没问题,但当启用了,会发现有一天在没有更改代码时,构建开始崩溃,或者更令人沮丧的是,需要快速修复一个与新警告完全无关的问题。那么,什么时候应该启用呢?

最简单的答案是,当在编写一个公共库时。然后,真的希望避免因为代码是在更严格的环境中编译而发出抱怨,抱怨代码不规范。若决定启用它,请确保跟上了编译器的新版本,及其引入的警告。

否则,让警告成为警告,并专注于错误。若觉得内部需要学究气,请使用-Wpedantic标志。这是一个有趣的选项——启用了严格的ISO C和ISO C++要求的所有警告。请注意,不能使用此标志检查代码是否符合标准——其只会查找需要诊断消息的非ISO实践。

更宽容和接地气的开发者将满足-Wall和可选的-Wextra,这些是认为是非常有用和有意义的警告。当有空闲时间时,应该在代码中修复它们。

还有许多其他的警告标志,根据项目的类型,这些警告标志可能是有用的。我建议您阅读所选编译器的手册,看看相应的警告具体代表了什么。

\hspace*{\fill} \\ %插入空行
\noindent
\textbf{调试构建}

有时候,编译会崩溃。这通常发生在我们试图重构一堆代码或清理构建系统时。有时,问题很容易解决,但随后会出现更复杂的问题,需要深入了解配置步骤。我们已经知道如何打印更详细的CMake输出,但是如何分析每个阶段实际发生的事情呢?

\hspace*{\fill} \\ %插入空行
\noindent
\textbf{各个阶段的调试}

有一个-save-temps标志可以传递给编译器(GCC和Clang都有),将强制每个阶段的输出存储在一个文件而不是内存中:

\begin{lstlisting}[style=styleCMake]
# chapter05/07-debug/CMakeLists.txt

add_executable(debug hello.cpp)
target_compile_options(debug PRIVATE -save-temps=obj)
\end{lstlisting}

上面的代码段通常会产生两个文件:

\begin{itemize}
\item 
<build-tree>/CMakeFiles/<target>.dir/<source>.ii: 存储预处理阶段的输出,用注释解释源代码的每个部分来自哪里:

\begin{lstlisting}[style=styleCXX]
# 1 "/root/examples/chapter05/06-debug/hello.cpp"
# 1 "<built-in>"
# 1 "<command-line>"
# 1 "/usr/include/stdc-predef.h" 1 3 4
# / / / ... removed for brevity ... / / /
# 252 "/usr/include/x86_64-linuxgnu/c++/9/bits/c++config.h" 3
namespace std
{
	typedef long unsigned int size_t;
	typedef long int ptrdiff_t;
	typedef decltype(nullptr) nullptr_t;
}
...
\end{lstlisting}
	
\item 
<build-tree>/CMakeFiles/<target>.dir/<source>.s: 语言分析阶段的输出,为汇编阶段做好准备:

\begin{lstlisting}[style=styleCXX]
	.file "hello.cpp"
	.text
	.section .rodata
	.type _ZStL19piecewise_construct, @object
	.size _ZStL19piecewise_construct, 1
_ZStL19piecewise_construct:
	.zero 1
	.local _ZStL8__ioinit
	.comm _ZStL8__ioinit,1,1
.LC0:
	.string "hello world"
	.text
	.globl main
	.type main, @function
main:
( ... )
\end{lstlisting}

\end{itemize}

根据问题的种类,可以发现实际问题是什么。预处理器的输出对于发现错误很有用,比如不正确的include路径(提供了错误的库版本)和导致不正确的\#ifdef计算的错误。

语言分析的输出,对于特定处理器和解决关键优化问题很有用。

\hspace*{\fill} \\ %插入空行
\noindent
\textbf{头文件的调试问题}

错误包含的文件是一个非常难调试的问题。这是我在公司的第一份工作,将整个代码库从一个构建系统移植到另一个。若发现自己处于一个需要精确理解哪些路径用来包含请求的头文件的位置,可以使用-H:

\begin{lstlisting}[style=styleCMake]
# chapter05/07-debug/CMakeLists.txt

add_executable(debug hello.cpp)
target_compile_options(debug PRIVATE -H)
\end{lstlisting}

打印出来的结果看起来类似这样:

\begin{tcblisting}{commandshell={}}
[ 25%] Building CXX object
CMakeFiles/inclusion.dir/hello.cpp.o
. /usr/include/c++/9/iostream
.. /usr/include/x86_64-linux-gnu/c++/9/bits/c++config.h
... /usr/include/x86_64-linux-gnu/c++/9/bits/os_defines.h
.... /usr/include/features.h
-- removed for brevity --
.. /usr/include/c++/9/ostream
\end{tcblisting}

目标文件名之后,输出中的每一行都包含到头文件的路径。行首的一个点表示顶层包含(\#include指令在hello.cpp中)。两个圆点表示该文件包含在<iostream>中。每一个进一步的点表示另一个嵌套级别。

输出的最后,可能还会发现对代码的改进建议:

\begin{tcblisting}{commandshell={}}
Multiple include guards may be useful for:
/usr/include/c++/9/clocale
/usr/include/c++/9/cstdio
/usr/include/c++/9/cstdlib
\end{tcblisting}

不需要修复标准库,但可能会看到自己的一些头文件。有可能需要改正它们。

\hspace*{\fill} \\ %插入空行
\noindent
\textbf{为调试器提供信息}

机器码是用二进制格式编码的指令和数据的神秘代码,因为CPU并不关心程序的目标是什么或者所有指令的意义是什么。唯一的要求是代码的正确性。编译器将把上述所有信息转换为CPU指令的数字标识符、一些初始化内存的数据和数千个内存地址。换句话说,最终的二进制文件不需要包含实际的源代码、变量名、函数签名或程序员关心的任何其他细节。这是编译器的默认输出- raw和dry。

这样做主要是为了节省空间和执行时没有太多的开销。巧合的是,我们也(在某种程度上)保护程序不受逆向工程的影响。是的,可以在没有源代码的情况下理解每个CPU指令的作用(例如,将这个整数复制到那个寄存器),但即使是最基本的程序也包含了太多这样的东西,以至于很难全局性的考虑。

若是一个特别有动力的人,可以使用一个叫做反汇编的工具,有了大量的知识(和一点运气),就能理解可能发生的事情。这种方法不是很实用,因为反汇编的代码没有原始符号,所以理清代码的位置非常困难和耗时。

相反,可以要求编译器将源代码存储在生成的二进制文件中,以及包含已编译代码和原始代码之间引用的映射。然后,可以将调试器与正在运行的程序挂钩,并查看在任何给定时刻正在执行哪一行源程序。这在我们编写代码时是必不可少的,比如:编写新功能或纠正错误。

这两个用例是两个配置的原因:Debug和Release。CMake会在默认情况下向编译器提供一些标志来管理这个过程,首先将它们存储在全局变量中:

\begin{itemize}
\item 
CMAKE\_CXX\_FLAGS\_DEBUG 包含 -g

\item 
CMAKE\_CXX\_FLAGS\_RELEASE 包含 -DNDEBUG
\end{itemize}

-g标志仅仅表示添加调试信息,以操作系统的本机格式提供——stabs、COFF、XCOFF或DWARF。这些格式可以用诸如gdb(GNU调试器)这样的调试器访问,这对于像CLion这样的IDE已经足够好了(底层使用gdb)。其他情况下,请参考所提供的调试器的手册,并检查所选编译器的相应标志是什么。

对于RELEASE配置,CMake将添加-DNDEBUG标志。这是一个预处理器定义,表明不是调试构建。启用此选项时,一些面向调试的宏可能无法工作。其中之一是assert,在<assert.h>头文件中可用。若决定在生产代码中使用断言,它们将完全不起作用:

\begin{lstlisting}[style=styleCXX]
int main(void)
{
	bool my_boolean = false;
	assert(my_boolean);
	std::cout << "This shouldn't run. \n";
	return 0;
}
\end{lstlisting}

assert(my\_boolean)调用在Release配置中不会有任何作用,但在Debug配置会正常工作。若正在练习断言式编程,但仍然需要在发布版本中使用assert(),该怎么办?要么改变CMake提供的默认值(从CMAKE\_CXX\_FLAGS\_RELEASE中移除NDEBUG),要么通过在包含头文件之前取消宏的定义:

\begin{lstlisting}[style=styleCXX]
#undef NDEBUG
#include <assert.h>
\end{lstlisting}

更多信息请参考assert说明文档: \url{https://en.cppreference.com/w/c/error/assert}
