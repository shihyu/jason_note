当使用现代的、支持良好的项目时,管理依赖关系并不复杂。大多数情况下,可以简单地依赖于系统中可用的库,若不可用则返回到FetchContent。若依赖关系相对较小且易于构建,那么没什么问题。

对于一些非常大的库(比如Qt),从源代码构建需要花费大量的时间。要在这些情况下提供自动依赖项解析,必须求助于提供与用户环境相匹配的编译版本库的包管理器。Apt或Conan等外部工具不在本书的范围内,因为它们要么太依赖于系统,要么太复杂。

好消息是只要提供了这样做的明确指示,大多数用户知道如何安装项目可能需要的依赖项。本章中,了解了如何使用CMake的查找模块和绑定在库中的配置文件,并检测安装在系统中的包。

还了解了一个不支持CMake的库,而是与.pc文件一起发布,我们将依赖于PkgConfig工具和与CMake绑定的FindPkgConfig查找模块。当使用上述方法找到库时,CMake将自动创建构建目标,这非常优雅。还讨论了依赖Git,及其子模块和克隆整个存储库。当其他方法不可行或难以实现时,这种方法就有用。

最后,探讨了ExternalProject模块及其功能和限制。研究了FetchContent如何扩展ExternalProject模块,两个模块间的异同,以及为什么FetchContent更好用。

现在,可以在项目中使用常规库了。另一种类型的依赖我们应该了解——测试框架。每个严肃的项目都需要进行正确性测试,CMake是一个很好的工具,可以使这个过程自动化。我们将在下一部分学习如何进行自动化。






























