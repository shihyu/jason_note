用CMake语言编写有点棘手,第一次阅读CMake文件时,可能会有这样的印象:其中的语言非常简单。接下来的工作通常是在没有完全理解代码如何工作的情况下,尝试引入更改并对代码进行试验。开发者通常很忙,并且过分热衷于用很少的投资解决与构建相关的问题。所以倾向于做出基于直觉的改变,希望能奏效,这种解决技术问题的方法称为“巫毒编程”。

CMake语言看起来很简单:在完成了添加、修复或修改或添加一行代码后,意识到有些东西无法工作,花在调试上的时间通常比花在实际研究主题上的时间要长。不过,本章将介绍在实践中使用CMake语言所需的绝大多数关键性知识。

本章中,不仅将学习CMake语言的构建模块——注释、指令、变量和控制结构——而且还将给出必要的背景知识,并在一个干净而现代的CMake示例中进行尝试。CMake将开发者置于一个独特的位置:一方面,扮演构建工程师的角色——需要理解编译器、平台以及两者之间的所有复杂之处;另一方面,你是一个开发人员——需要编写生成构建系统的代码。编写好的代码是困难的,需要同时从多个维度进行思考——它应该能够工作并易于阅读,但也应该易于分析、扩展和维护。

最后,将介绍CMake中一些最有用和最常用的指令。不经常使用的指令将放在附录部分(这将包括字符串、列表和文件操作命令的完整参考指南)。

本章中,我们将讨论以下主题:

\begin{itemize}
\item 
基本语法

\item 
变量

\item 
列表

\item 
控制结构

\item 
实用指令
\end{itemize}