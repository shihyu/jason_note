本章结束了我们通过CMake的漫长旅程。现在您完全理解了CMake的目标是解决什么问题,以及自动化这些解决方案需要哪些步骤。

前三章中,探索了所有的基础知识:什么是CMake,用户如何利用它来激活原始源代码,CMake的关键组件是什么,以及不同的项目文件有什么目的。我们解释了CMake的语法:注释、命令调用、参数、变量和控制结构。了解模块和子项目是如何工作的,正确的项目结构是什么,以及如何使用各种平台和工具链。

本书的第二部分教我们如何使用CMake进行构建:如何使用目标、定制命令、构建类型和生成器表达式。我们深入研究了编译的技术细节,以及预处理器和优化器的配置,讨论了链接并介绍了不同的库类型。然后,研究了CMake如何使用FetchContent和ExternalProject模块帮助管理项目的依赖性,还研究了Git子模块作为可能的替代方案,研究了如何使用find\_package()和FindPkgConfig查找已安装的包。若这些还不够,可以编写自己的查找模块。

最后一部分告诉我们如何进行自动化测试、分析、文档编制、安装和打包。我们研究了CTest和测试框架:Catch2、GoogleTest和GoogleMock。报告也包括在内。第9章,程序分析工具,了解了不同的分析工具:格式化器和静态检查器(Clang-Tidy、Cppcheck等),并解释了如何从Valgrind套件中添加Memcheck内存分析器,这里简要描述了如何使用Doxygen生成文档,以及如何使文档看起来更美观。最后,演示了如何在系统上安装项目,创建可重用的CMake包,以及配置和使用CPack生成二进制包。

最后一章使用了所有这些知识来展示一个完全专业的项目。

\textit{恭喜你阅读完了这本书。我们介绍了开发、测试和打包高质量C++软件所需的所有内容。从这里取得进步的最好方法是将所学到的应用到实践中,从而创建更多优秀的软件。祝诸君好运!}

R.















