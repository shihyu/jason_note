若没有像CMake这样的工具,以跨平台的方式编写安装脚本是一项非常复杂的任务。虽然仍然需要一些工作来设置,但这是一个更加精简的过程,与本书中迄今为止使用的所有其他概念和技术紧密相连。

首先,我们学习了如何从项目中导出CMake目标,以便可以在其他项目中使用,而无需安装。然后,学习了如何安装已经为此目的配置的项目。

然后,开始从最重要的主题开始探索安装的基础:安装CMake目标。我们现在知道了CMake如何处理各种工件类型的不同目录,以及如何处理有些特殊的公共头文件。为了在较低级别上管理这些安装步骤,讨论了install()指令的各种模式,包括安装文件、程序和目录,以及在安装期间调用脚本。

解释了如何编写安装步骤之后,了解了CMake的可重用包。学习了如何使项目中的目标可重定位,以便包可以安装在用户想要的地方。然后,着重于使用find\_package()形成一个可以让其他项目使用的包,这需要准备目标导出文件、配置文件和版本文件。

认识到不同的用户可能需要包的不同部分,了解了如何在安装组件中对构件和操作进行分组,以及与CMake包的组件之间的区别。最后,介绍了CPack并学习了如何准备基本的二进制包,这些包可用于以预编译的形式分发软件。

要完全掌握安装和打包的所有细节和复杂性还有很长的路要走,但本章为我们提供了处理最常见场景的坚实基础,并有信心进一步探索它们。

下一章中,将通过创建一个连贯的、专业的项目,将所学到的一切付诸实践。








