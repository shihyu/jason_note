\begin{flushright}
\zihao{0} 前言
\end{flushright}

高性能編程的藝術仍在歸來途中。我還是編程菜鳥時,那時候的開發者必須知道每一個數據位的去向(有時的確要這樣——用面板上的開關)。現在,計算機已經有能力完成日常工作,就沒必要在意有些細節。不過,還是有一些領域沒有足夠的計算能力。其實,大多數開發者都可以寫出高效的代碼,這不是一件壞事。在不受性能限制的前提下,這樣開發者就可以更專注於如何使代碼變的更好。

本書首先闡明瞭,為什麼越來越多的開發者需要關注性能和效率。這將為整本書定下基調,因為其定義了我們在後續章節中所使用的方法論。關於性能的知識最終必須來自於測試,並且每個與性能相關的聲明都必須有數據支撐。

其中,有五個要素決定了程序的性能。

首先,我們深入研究所有性能的底層基礎——用於計算的硬件(沒有交換機——那些日子已經一去不復返了)。從單個部件——處理器和內存——到多處理器計算系統。在此過程中,我們會瞭解了內存模型、數據共享的成本,還有無鎖編程。

高性能編程的第二個要素是對編程語言的使用,這一點正是本書針對C++(其他語言也有低效率)進行的說明。而後的是第三個要素,即是幫助編譯器提高程序性能的技能。

第四個要素是設計。如果設計沒有將性能作為其明確目標,那麼幾乎不可能在之後再為程序添加良好的性能。然而,因為這是一個高級概念,它彙集了之前所學到的所有知識,所以我們最後再來瞭解性能設計。

高性能編程的最後,也是第五個要素就是——正在閱讀本書的你,你的知識和技能水平將最終決定結果。為了幫助讀者進行學習,這本書提供了許多例子,可以用於動手探索和自學。學習是一項終身的運動,切勿在看完本書後停止學習。

\hspace*{\fill} \\ %插入空行
\noindent\textbf{適讀人群}

這本書是為有經驗的開發人員編寫,從事對性能至關重要的項目,並希望學習不同的技術來提高代碼的性能。算法交易、遊戲、生物信息學、基因組學或流體動力學社區的開發者,都可以從這本書中學習各種技術,並將其應用到他們的工作領域中。

雖然本書使用的是C++,但本書的概念可以轉移或應用到其他編譯語言,如C、Java、Rust、Go等。

\hspace*{\fill} \\ %插入空行
\textbf{本書內容}

\textit{第1章,性能和併發性}。討論了重視程序性能的原因,特別是性能好的應用為什麼不會憑空出現。為了實現最佳性能,甚至是提升性能,理解影響性能的不同因素和程序特定行為的原因(無論是快執行還是慢執行)。

\textit{第2章,性能測試}。本章內容都是關於性能測試的。性能有時並不直觀,所有涉及效率的決策,從設計選擇到具體優化策略,都應該以可靠的數據為指導。本章描述了不同類型的性能評估方式,解釋了它們的不同之處,以及應該在什麼時候使用哪種方式,並瞭解如何在不同的情況下正確地評估性能。

\textit{第3章,CPU架構、資源和性能}。研究硬件,以及如何高效地使用硬件資源,從而達到最佳性能。本章將瞭解CPU資源和能力,以及使用方法。瞭解CPU資源沒有得到充分利用的原因,以及如何解決這些問題。

\textit{第4章,內存架構與性能}。瞭解現代內存架構,以及其缺點,以及避免或規避這些缺點的方法。對於許多程序來說,性能完全依賴於開發者是否可以利用硬件特性來提高內存性能。

\textit{第5章,線程、內存和併發}。繼續學習內存系統,及其對性能的影響,但現在會擴展到多核系統和多線程領域。事實證明,內存已經是性能的“關鍵”,當添加了併發時,內存管理可能會很棘手。雖然硬件帶來的物理限制無法克服,但大多數程序的性能其實還未觸及這些限制,而且對於資深的開發者來說,代碼效率還有很大的提高空間。所以,本章為讀者們提供了必要的知識和工具。

\textit{第6章,併發性和性能}。瞭解如何為線程安全的程序開發高性能併發算法和數據結構。一方面,為了充分利用併發性,必須從更高的角度看待問題和解決方案策略:數據組織、工作分區,甚至是解決方案的定義都會對程序性能產生重大影響。另一方面,性能受到底層的極大影響,比如:緩存中數據的分佈,即使是最好的設計也可能因糟糕的實現,達不到預期的性能。

\textit{第7章,用於併發的數據結構}。解釋併發程序中數據結構的本質,以及在多線程上下文中使用數據結構,比如:定義常用的數據結構(如“堆棧”和“隊列”等)。

\textit{第8章,C++中的併發}。介紹了C++17和C++20標準中最近添加的併發特性。雖然,現在談論使用這些特性獲得最佳性能的方式還為時過早,但我們可以描述其作用,以及編譯器當前的支持情況。

\textit{第9章,高性能C++}。重點從硬件資源的使用,轉移到特定編程語言的應用。雖然,我們學到的所有東西都可以直接應用到任何語言中,但本章將討論C++的特性。讀者將瞭解C++語言的哪些特性可能會導致性能問題,以及如何規避這些問題。

\textit{第10章,C++中的編譯器優化}。本章還會討論編譯器優化的問題,以及開發者如何幫助編譯器生成更高效的代碼。

\textit{第11章,未定義的行為和性能}。這裡有兩個重點:一方面,解釋了開發者在試圖最大化代碼性能時,經常忽略的未定義行為的危險性。另一方面,解釋瞭如何利用未定義的行為來提高性能,以及如何正確地控制和記錄這些情況。總的來說,與“任何事情都可能發生”相比,這一章提供了一些常見方式來理解未定義行為的問題。

\textit{第12章,為性能而設計}。回顧本書中所學到的所有與性能相關的因素和特性,並總結所獲得的知識,瞭解如何在開發新軟件系統或重新架構現有軟件系統時,如何做出的設計決策。

\hspace*{\fill} \\ %插入空行
\textbf{編譯環境}

除了特定於C++效率的章節,不依賴於任何C++知識。所有的例子都是用C++編寫(但是有關硬件性能、高效數據結構和性能設計的部分適用於任何編程語言)。要看懂這些示例,至少需要具備中等程度的C++知識。

\begin{table}[H]
	\begin{tabular}{|l|l|}
		\hline
		C++ compiler(GCC, Clang, Visual Studio等)                                                                                                                  & 操作系統                                                             \\ \hline
		LLVM版本高於或等於12.x                                                                                                                  & \begin{tabular}[c]{@{}l@{}}Windows, macOS或Linux\end{tabular}                                                             \\ \hline
		\begin{tabular}[c]{@{}l@{}} Profiler(VTune, Perf, GoogleProf等)\end{tabular} &  \\ \hline
		Benchmark Library(GoogleBench)                                                                                                                                  &                                                                                  \\ \hline
	\end{tabular}
\end{table}

每一章都會提到編譯和執行示例所需的軟件(如果有的話)。大多數情況下,現代C++編譯器都可以與示例一起使用,除了第8章(C++中的併發),該章要求支持C++20,從而可以展示\textbf{協程}的相關示例。

\textbf{如果你正在使用這本書的數字版本,我們建議自己輸入代碼或通過GitHub庫訪問代碼(鏈接在下一節中提供)。這樣做將幫助您避免與複製和粘貼代碼相關的任何潛在錯誤}

\hspace*{\fill} \\ %插入空行
\textbf{下載示例}

可以從GitHub網站\url{https://	github.com/PacktPublishing/The-Art-of-Writing-Efficient-Programs}下載本書的示例代碼文件。如果代碼有更新,會在現有的GitHub庫中更新。

我們還有其他的代碼包,還有豐富的書籍和視頻目錄,都在\url{https://github.com/PacktPublishing/}。去看看吧!

\hspace*{\fill} \\ %插入空行
\textbf{聯繫方式}

我們歡迎讀者的反饋。

\textbf{反饋}:如果你對這本書的任何方面有疑問,需要在你的信息的主題中提到書名,並給我們發郵件到\url{customercare@packtpub.com}。

\textbf{勘誤}:儘管我們謹慎地確保內容的準確性,但錯誤還是會發生。如果您在本書中發現了錯誤,請向我們報告,我們將不勝感激。請訪問\url{www.packtpub.com/support/errata},選擇相應書籍,點擊勘誤表提交表單鏈接,並輸入詳細信息。

\textbf{盜版}:如果您在互聯網上發現任何形式的非法拷貝,非常感謝您提供地址或網站名稱。請通過\url{copyright@packt.com}與我們聯繫,並提供材料鏈接。

\textbf{如果對成為書籍作者感興趣}:如果你對某主題有專長,又想寫一本書或為之撰稿,請訪問\url{authors.packtpub.com}。

\hspace*{\fill} \\ %插入空行
\textbf{歡迎評論}

請留下評論。當您閱讀並使用了本書,為什麼不在購買網站上留下評論呢?其他讀者可以看到您的評論,並根據您的意見來做出購買決定。我們在Packt可以瞭解您對我們產品的看法,作者也可以看到您對他們撰寫書籍的反饋。謝謝你!

想要了解Packt的更多信息,請訪問\url{packt.com}。










