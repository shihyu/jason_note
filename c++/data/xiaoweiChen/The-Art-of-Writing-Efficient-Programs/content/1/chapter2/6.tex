這一章中,已經瞭解了整本書中最重要的概念,若不參照具體的衡量標準,談論甚至思考性能都是沒有意義的。剩下的大部分是動手環節,我們介紹了幾種測量性能的方法。從整個程序開始,然後深入到每一行代碼。

一個大型的高性能項目中,可以看到本章所學到的每一種工具和方法會多次使用。粗略測量——對整個程序或大部分程序進行基準測試和分析——指向需要進一步研究的代碼區域。隨後會進行更多輪的基準測試或收集更詳細的數據。最終,確定需要優化的代碼,然後問題就變成了,如何才能更快地完成這項工作?此時,可以使用微型基準測試或小型基準測試來測試優化代碼。甚至可能會發現,您對這段代碼的理解並不如自己所想的那麼透徹,並且需要對其性能進行更詳細的分析。同時,不要忘記可以對微基準測試結果進行分析!

最終,將得到性能關鍵代碼的新版本,在小型基準測試中看起來是沒問題的。但是,不要做任何假設!現在必須通過測試整個程序的性能,來確定所做的優化或增強是否有效。有時,這些測量將確認您對問題的理解,並驗證解決方案。有時,會發現問題並不像想象的那樣。優化本身雖然有益,但並沒有對整個程序產生預期的效果(甚至會使事情變得更糟)。當有了一個新的數據點,可以比較新舊解決方案的數據,並在比較二者差異中尋找答案。

高性能程序的開發和優化從來不是線性的、循序漸進的過程。相反,它有許多迭代,從高級概述到低級數據分析,然後重複。這個過程中,直覺起著作用,只要確保測試和確認期望相符即可。因為涉及到性能時,沒有什麼是真正的顯而易見。

下一章,將看到我們之前遇到的問題的解決方案,刪除不必要的代碼使程序變慢。為了做到這一點,我們必須瞭解如何有效地使用CPU以獲得最大的性能,下一章將專門來討論這個問題。