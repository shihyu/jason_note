無論是編寫新性能程序,還是優化現有的程序,在這之前需要了解代碼當前的性能。衡量成功的標準是讓程序的表現提高多少。這兩種看法都說明瞭性能指標的存在,而且指標是可測量和可量化的。不過,沒有一個定義能滿足所有情況。要量化性能時,需要衡量的指標是什麼,取決於具體問題。

但衡量標準不是簡單地定義目標和確認成功。性能優化的每一步,無論是現有代碼還是新代碼,都應該有指標進行引導。

性能的第一條規則:\textit{永遠不要猜測性能}。本章的第一部分是讓所有人認同這條規則。為了打破對直覺的信任,我們必須學習使用如何使用測試性能的工具。

本章將討論以下內容:

\begin{itemize}
\item 為什麼性能測試是必要的
\item 為什麼所有與性能相關的決策都必須由測試和數據驅動
\item 如何測試實際程序的性能
\item 什麼是基準測試、數據分析和微基準測試,以及如何使用它們來衡量性能
\end{itemize}

