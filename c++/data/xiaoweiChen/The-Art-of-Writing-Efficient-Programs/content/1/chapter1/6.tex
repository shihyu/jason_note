這一導論性章節中,我們討論了為什麼現代計算機的計算能力飛速發展,人們對軟件性能和效率的興趣卻在上升。因此,為了理解限制性能的因素,以及如何克服它們,所以需要回到計算的基本要素上來。理解計算機和程序是如何在底層上工作的,瞭解硬件並有效地使用,理解併發性,理解C++語言特性和編譯器優化,以及其對性能的影響。

這種底層知識必然是詳細和具體的,但當瞭解處理器或編譯器的具體情況後,我們也會瞭解得出這些結論的過程。在更深層次上,本書更是關於如何學習的方法論。

我們進一步瞭解到,如果不定義衡量性能的標準,性能的概念就沒有意義。需要根據特定的指標來評估性能,這意味著關於性能的工作都由數據和指標驅動。下一章,我們來瞭解一下如何對性能進行測試。