如何讓程序獲得“高性能”?是“效率”。首先,這並不總是正確的(儘管它經常正確);其次,這裡迴避了一個問題,因為下一個問題就是:如何使得程序有“效率”?為了寫出高效或高性能的程序,需要學習什麼呢?讓我們製作一個列表,來看看需要哪些技能和知識:

\begin{itemize}
\item 選擇正確的算法
\item 有效利用CPU資源
\item 高效的使用內存
\item 避免不必要的計算
\item 有效地使用併發和多線程
\item 有效地使用編程語言,避免效率低下
\item 測試性能和分析結果
\end{itemize}

實現高性能的最重要因素是選擇一個好的算法,我們不能通過優化實現來“修復”算法的缺點。對於算法好壞的討論,超出了本書的範疇。當然,算法是具體問題的解法,這裡必須進行深入研究,從而找出解決問題的最優解法。

另一方面,實現高性能的方法和技術在很大程度上與問題無關。當然,這些技術和方法依賴於性能指標,例如:實時系統的優化是高度特化領域的問題。本書中,我們主要關注高性能計算上的性能指標,儘可能快地進行計算。

為了成功地完成這項任務,必須學會盡可能多地使用可用的計算資源。這個目標有空間和時間的兩部分組成:空間方面,使用更多的晶體管,處理器擁有巨大的數量的晶體管。處理器也正在變得更大,甚至更快。增加的面積提供了新的計算能力。時間方面,應該使用盡可能多的資源。如果計算資源空閒,那麼這些資源對我們來說是沒有用的,所以我們的目標是避免這種情況。與此同時,忙碌有時也會沒有回報,所以要避免做任何不需要的事情。我們要從哪裡下手解決問題呢?有很多方法可以讓程序避免執行不需要的計算。

本書中,我們將從單處理器開始,並瞭解如何有效地使用計算資源。然後,擴展視圖,不僅包括處理器,還包括內存。當然,也會考慮同時使用多個處理器的情況。

有效地使用硬件只是高性能程序的特性之一,高效地完成可以避免的工作對我們沒有任何幫助。高效工作的關鍵是有效地使用編程語言,我們的例子中使用的是C++(我們對硬件的大部分了解可以應用到其他語言,但是有些語言優化是C++特有的)。此外,編譯器位於我們編寫的語言和使用的硬件之間,因此必須學習如何使用編譯器來生成高效的代碼。

最後,量化剛剛列出的目標成功的方法就是進行測試。比如:使用了多少CPU資源?花了多少時間等待內存?增加一個線程是否可以獲得更好的性能?等等。獲得良好的量化性能數據並不容易,這需要對測試工具有更細節的理解,而分析結果往往會更難。

可以從這本書中學習以上的技能。我們還會來學習硬件架構,以及隱藏在一些編程語言特性背後的東西,以及如何像編譯器那樣看代碼。技能固然重要,但更重要的是理解為什麼會以這種方式運行。計算可能硬件經常發生變化,語言也在不斷髮展,開發者也可以為編譯器發明瞭新的優化算法。因此,這些領域特定知識的保質期不會太久。現在,可以先了解特定處理器或編譯器的最佳使用方法,再瞭解獲得這些知識的方法,從而可以重複這個過程,並進行更深入的學習。
































