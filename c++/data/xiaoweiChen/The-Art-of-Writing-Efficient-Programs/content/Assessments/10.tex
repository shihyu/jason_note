\begin{enumerate}
\item 
最重要的約束是,程序的結果(或者更嚴格地說,可觀察到的行為)必須不變。這裡的門檻很高,只有當能夠證明結果對所有可能的輸入都是正確的,編譯器才允許進行優化。其次考慮的是實用性,編譯器必須在編譯時間和優化代碼的效率之間做出權衡。在啟用了最高優化的情況下,證明某些代碼轉換不會破壞程序的代價可能會很大。 

\item 
除了明顯的效果(消除函數調用)外,內聯使編譯器能夠分析更大的代碼段。如果沒有內聯,編譯器通常會假設在函數體中“任何事情都是可能的”,編譯器可以看到對函數的調用是否產生任何可觀察的行為,比如I/O。內聯只在一定程度上是有益的。當內聯過度時,會增加機器碼的大小。此外,編譯器很難分析非常長的代碼段(片段越長,優化器處理所需的內存和時間就越多)。編譯器具有判斷某個函數是否值得內聯的方法。

\item 
編譯器不會進行優化,這通常是因為不能保證這個優化是正確的。編譯器對如何使用程序的瞭解與開發者不同,編譯器會假定輸入的組合是有效的。另一個常見的原因是,預期的優化不會都那麼有效。在這一點上,編譯器可能是正確的,若結果表明開發者是正確的,那可以在源代碼中進行強制優化。

\item
內聯的主要好處不是消除了函數調用的成本,它允許編譯器看到函數內部發生了什麼,允許對函數調用前後的代碼進行分析。將一個更大的代碼片段優化為單個基本塊時,孤立地考慮每一段代碼時,可以將不可能實現的優化變為可能。

\end{enumerate}