\begin{enumerate}
\item 
許多領域中,問題的規模的增長速度與可用的計算資源一樣快,甚至更快。隨著計算變得越來越普遍,沉重的工作負載可能需要在功率有限的處理器上執行。

\item 
大約在15年前,單核處理能力基本上停止了增長,處理器設計和製造的進步很大程度上轉化為更多的處理核和大量的專用計算單元。這些資源不能自動使用,所以需要理解其工作方式。

\item 
效率是指在更多的時間內使用更多的可用計算資源,並且不做不必要的工作。性能指的是滿足特定的指標,而這些指標取決於計劃要解決的問題。

\item
不同的環境中,性能的定義可能完全不同。超級計算機中,計算的原始速度可能是最重要的,但在交互系統中,只要系統比與它交互的人更快,處理速度可能就沒那麼重要。

\item
性能必須加以衡量,定量測試結果和分析是對成功和失敗都很重要。
	
\end{enumerate}