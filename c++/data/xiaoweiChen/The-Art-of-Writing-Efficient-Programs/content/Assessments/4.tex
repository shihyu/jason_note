\begin{enumerate}
\item 
現代CPU甚至比最好的內存都要快,訪問內存中隨機位置的延遲是幾納秒,這足以讓CPU執行數十次操作。即使在流訪問中,內存的總帶寬也不足以以相同的速度為CPU提供數據,以執行計算。

\item 
內存系統包括CPU和主存之間的緩存層次結構,因此影響速度的第一因素是數據集的大小,這最終決定了數據是否適合緩存。對於給定的大小,內存訪問模式是關鍵,若硬件可以預測下一次訪問,就可以通過在請求數據之前開始將數據傳輸到緩存來隱藏延遲。

\item 
低效的內存訪問是顯式的性能數據文件或計時器輸出,對於具有良好數據封裝的模塊化代碼來說也是如此。在計時分析沒有進行統計和分析的地方,緩存有效性可能會在整個代碼中顯示低效訪問的數據。

\item
使用較少內存的優化可能會提高內存性能,因為更多的數據適合於緩存。但對大量數據的順序訪問可能比對少量數據的隨機訪問要快,較小的數據適合L1緩存,或者適合L2緩存。直接針對內存性能的優化通常採用數據結構優化的形式,主要目的是避免隨機訪問和間接內存訪問。除此之外,還必須改變算法,將內存訪問模式改為更友好的緩存模式。

\end{enumerate}