\begin{enumerate}
\item 
現代的CPU有多個計算單元,其中許多可以同時運行。儘可能多地使用CPU計算能力,是使程序效率最大化的方法。

\item 
任何兩個可以在同一時間完成的計算,只需要兩個計算中較長的時間,另一個看上去不花費任何時間。許多程序中,可以用現成的計算來代替一些將來要完成的計算。這種權衡通常是現在比以後做更多的計算,但只要計算不需要額外的時間,就可以提高整體性能,因為這些計算是與其他工作是並行的關係。

\item 
這種情況稱為數據依賴。解決方法是使用流水線,不依賴於任何未知數據的部分計算,將與程序順序中在其之前的代碼並行執行。

\item
條件分支使計算具有不確定性,這就阻礙了CPU對它們進行流水線處理。CPU試圖預測將要執行的代碼,以便維護流水線。每當這樣的預測失敗時,流水必須刷新,所有預測錯誤的指令結果都將丟棄。

\item
根據CPU的分支預測執行的代碼可能需要,進行預測性評估。在投機執行的環境中,不能撤消的操作都不能完全提交:CPU不能覆蓋內存,不能做任何I/O操作,不能發出中斷,也不能報告錯誤。CPU有必要的硬件來暫停這些操作,直到預測執行的代碼確認為真正的代碼,或者不是。後一種情況下,投機執行的所有結果都會丟棄,對可見性沒有影響。

\item 
一個分支預測良好的程序,通常只對性能的影響很小。因此,兩種主要的解決方案是:重寫代碼,使條件變得更可預測,或者更改計算方式,使用有條件訪問的數據。後者稱為無分支計算。
	
\end{enumerate}