本章中,我們瞭解了併發程序的基本構件的性能。對共享數據的所有訪問都必須進行保護或同步,但是在實現同步時,有很多選擇。雖然互斥鎖是最常用和最簡單的方法,但我們已經瞭解了其他幾個性能更好的選項:自旋鎖及其變體,以及無鎖同步。

高效併發程序的關鍵是使盡可能多的數據位於線程本地,並儘量減少對共享數據的操作。特定於每個問題的需求通常不能完全消除此類操作,因此本章主要討論如何提高併發數據訪問的效率。

我們研究瞭如何跨多個線程計數或累積,嘗試了使用和不使用鎖的方法。通過理解數據依賴關係,發現發佈協議可以使用線程安全的智能指針實現,並適用於不同的應用程序。

現在,我們已經做好了充分的準備,將本章中幾個構建塊以更復雜的線程安全數據結構的形式放在一起。下一章中,將瞭解如何使用這些技術為併發程序設計數據結構。