上一章中,我們瞭解了影響併發程序性能的基本因素。現在是時候將這些知識用於實際,並學習如何為線程安全的程序開發高性能併發算法和數據結構。

一方面,要充分利用併發性,必須以高級視角看待問題和解決方案策略:選擇數據組織方式、工作分區,有時甚至是解決方案的定義都會對程序性能產生重大影響。另一方面,性能受到底層影響極大,比如緩存中數據的排列,即使是最好的設計也可能被糟糕的實現所破壞。這些底層的細節常常難以分析,難以用代碼表達,並且需要非常仔細的編碼。我們不期望這些代碼散落的到處都是,因此必須對它們進行封裝,所以需要考慮封裝這種複雜性的最佳方式。

本章將討論以下內容:

\begin{itemize}
\item 高效的併發性
\item 鎖的使用、鎖的缺陷,以及無鎖編程
\item 線程安全的計數器和累加器
\item 線程安全的智能指針
\end{itemize}





