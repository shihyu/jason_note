使用併發性來提高性能非常簡單,第一種方法是為併發線程和進程提供足夠的工作,使它們始終處於忙碌狀態;第二個是減少共享數據的使用,併發訪問共享變量的開銷非常大。剩下的只是如何實現的問題。

但實現往往相當殘酷,而且當期望的性能增益更大,並且當硬件變得更強大時,難度就會增加。每個從事併發工作的開發者都聽說過Amdahl法則,但並不是每個人都完全理解它的含義。

法則本身很簡單:對於一個具有並行(可擴展)部分和單線程部分的程序,最大可能的加速\textit{s}如下所示:

\begin{center}
$ s = \dfrac{s_0}{s_0(1-p)+p} $
\end{center}

這裡,計算是程序並行部分的加速比,也是程序並行部分的分數。現在考慮一下在大型多處理器系統上運行程序的情況:如果有256個處理器,並且能夠充分利用它們,除了運行時間的1/256,程序的總加速會限制為128,加速比削減了一半。換句話說,如果只有1/256的程序是單線程的,或者是在鎖下執行的,那麼不管如何優化程序的其餘部分,在這個256個處理器的系統的加速比永遠不會超過50\%。

這就是為什麼在開發併發程序時,設計、實現和優化的重點應該是使單線程計算併發化,並減少程序訪問共享數據所花費的時間。

第一個目標,從算法的選擇開始使計算並行化,但是許多設計決策會影響結果,所以應該更多地進行了解。第二種方法是降低數據共享的成本,延續了上一章的主題,當所有線程都在等待訪問某個共享變量或鎖(它本身也是一個共享變量)時,程序實際上是單線程的,只有當前有訪問權限的線程在運行,這就是為什麼全局鎖和全局共享數據對性能不利的原因。但是,即使是多個線程之間共享的數據,如果併發訪問這些線程,也會限制這些線程的性能。

數據共享的需求基本上是由問題本身導致的,特定問題的數據共享量可能受到算法、數據結構選擇和其他設計決策以及實現的影響。有些數據共享是實現的產物,或數據結構選擇的結果,但其他共享數據則是問題本身。如果需要計算滿足某個屬性的數據元素,比如只有一個計數,所有線程必須將其更新為共享變量。然而,實際發生了多少共享,以及對總體程序加速的影響,在很大程度上取決於具體實現。

本章中,我們將追尋兩條線索:首先,考慮到一些不可避免的數據共享,將研究如何使這個過程更有效。然後,考慮設計和實現技術,這些可以用來減少數據共享的需求或減少等待訪問該數據的時間。從第一個問題開始把,如何進行高效的數據共享。
































