C++11是第一個添加線程存在的標準,為併發環境中記錄C++程序的行為奠定了基礎,並在標準庫中提供了一些有用的功能。除了這些功能之外,基本的同步原語和線程也是有用的。後續版本對這些特性進行了改進。

C++17以並行STL的形式帶來了重大的進步。當然,性能由實現決定。只要數據足夠多,觀察到的性能就會更好,即使是在搜索和分區等難以並行化的算法上。然而,如果數據序列太短,並行算法的性能並不是很好。

C++20增加了協程支持,我們已經從理論上和一些基本的示例中瞭解了無棧協程的工作方式。然而,現在談論使用C++20協程的性能和最佳實踐還為時過早。

本章結束了本書對併發的探索。接下來,我們繼續學習C++語言本身的使用如何影響程序的性能。