開發者經常談論一種語言是否有效。特別是C++,它的開發有著明確的效率目標。與此同時,它在某些圈子裡認為C++是低效的。這是怎麼回事呢?

效率在不同的情況下或對不同的人有不同的含義。例如:

\begin{itemize}
\item
C++的設計遵循\textbf{零開銷}原則:除了少數的例外,不會為任何不使用的特性支付運行成本。從這個意義上說,C++可以認為是高效的語言。

\item
必須為所使用的語言特性付出一些代價,若將它們轉化為一些運行時的話。C++的優點是不需要任何運行時代碼,來完成可以在編譯期間完成的工作(儘管編譯器的實現和標準庫的效率各不相同)。高效的語言不會為執行請求的代碼增加任何開銷,C++在這裡做的還是相當不錯的,我們將在下面討論一個主要的限制。

\item
如果上述情況屬實,那麼C++是如何被持這種觀點的人貼上“效率低下”的標籤的呢?現在來看另一個視角定義的效率:用這種語言編寫高效的代碼有多容易?或者,做一件看起來很自然,但實際上是一種非常低效的解決問題的方法有多容易?與之前提到的問題密切相關,C++能夠非常高效地完成開發者要求它做的事情。但是,要在語言中準確地表達想要的東西並不容易,而且編寫代碼的方式有時會強制約束和要求開發者,這些約束和要求有些有運行時成本。
	
\end{itemize}

從語言設計者的角度來看,最後一個問題不是語言的低效:要求機器做X和Y,就需要時間來做X和Y,沒有做超出要求的事情。但從開發者的角度來看,如果開發者只想做X而不關心Y,那麼這就是一種效率低下的語言。 

本章的目標是幫助讀者編寫代碼,並清楚地表達希望機器做去什麼,這樣做的目的有兩個。若認為代碼主要讀者是編譯器,通過精確地描述想要的事情和了解編譯器可以自由修改的內容,可以給編譯器生成更高效代碼的自由。但是對於代碼的讀者來說也是一樣的,他們只能推斷出作者在代碼中表達的內容,而不能推斷出想要表達的內容。如果代碼的某些方面發生了變化,那麼優化代碼是否安全?這種行為是有意為之,還是錯誤的實現,並且可以去更改嗎?這再次提醒我們,編程更多的是一種與同伴交流的方式,而不僅僅是與機器交流的方式。

我們將從明顯低效的簡單代碼示例開始。但這種情況,也會出現在該語言資深開發者的代碼中。














