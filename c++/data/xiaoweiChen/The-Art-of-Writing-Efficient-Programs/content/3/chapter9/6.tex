本章中,我們從語言的角度討論了C++效率的兩個主要方面中的第一個:避免低效的語言結構,這可以歸結為只做必要的工作。我們研究過的許多優化技術都與前面的內容相吻合,比如:訪問內存的效率和避免併發程序中的錯誤共享。

每個開發者面臨的一個大難題是,在編寫高效代碼之前應該投入多少工作,以及應該留給優化的工作多少時間。首先,高性能始於設計階段,是開發高性能軟件的重中之重。 

除此之外,還應該區分過早的優化和不必要的悲觀。創建臨時變量以避免混疊是不明智的,除非性能測試數據顯示,正在優化的函數對總體執行時間有很大的貢獻(或者提高了可讀性,這是另一回事)。在分析器報告之前,按值傳遞大型數據只會使運行效率下降,應該從一開始就避免這樣做。 

兩者之間的界限並不明確,所以必須權衡幾個因素。需要考慮修改對程序的影響,會使代碼更難閱讀、更復雜,還是更難測試?通常,不希望為了性能而冒險製造更多的Bug,除非測試報告說必須這樣做。另一方面,有時可讀性更強或更簡單的代碼也是高效的代碼,因此不能認為優化是過早的。 

C++效率的第二個方面是幫助編譯器生成更高效的代碼。我們將在下一章討論這個問題。