本書的最後一章回顧了我們所瞭解到的關於性能和決定性能的因素的所有知識,然後使用這些知識來提出高性能軟件系統的設計指南。提供了一些關於設計接口、數據組織、組件和模塊的建議,並描述了在有一個性能可以測試的實現之前,如何根據良好的測量結果做出設計決策的方法。

必須再次強調,為性能而設計不會自動產生良好的性能,但允許實現具有高性能。另一種選擇是一種對性能不利的設計,並固化決策,限制並阻止有效的代碼和數據結構。

這本書是一個旅程:從學習單個硬件組件的性能開始,然後研究它們之間的相互作用,以及它們如何影響對編程語言的使用。這條路最終將我們引向了性能設計的理念。這是本書的最後一章,但不是各位讀者旅程的最後一程:現在可以將知識應用於有待解決的實際問題,想想就令人興奮。