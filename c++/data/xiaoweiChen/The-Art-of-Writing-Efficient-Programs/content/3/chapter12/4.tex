
有許多書籍和文章介紹了API設計的最佳實踐,通常關注可用性、清晰度和靈活性。常見的指導方針,如“使接口清晰且易於正確使用”和“使接口難以誤用”,並不直接處理性能問題,但也不會干擾促進良好性能和效率的實踐。前一節中,我們討論了在為性能設計接口時應該記住的兩條重要準則。本節中,我們將探討一些針對性能的更具體的指導方針。許多高性能程序依賴於併發執行,因此首先解決併發設計是有意義的。

\subsubsubsection{12.4.1\hspace{0.2cm}併發的API設計}

設計併發組件及其接口時,最重要的規則是提供明確的線程安全的保證。注意,“明確”並不意味著“健壯”。事實上,為了獲得最佳性能,在底層接口上提供較弱的保證通常性能更好。STL選擇的方法是一個很好的例子,所有可能改變對象狀態的方法都提供了一個弱保證:只要有一個線程在使用容器,程序就是定義良好的。 

如果想要更強的保證,可以在應用程序級別使用鎖。一個更好的方式是創建自己的鎖,為接口提供強有力的保證。有時,這些類只是鎖的裝飾器,將裝飾對象的每個成員函數封裝在鎖中。常見的是,一個鎖必須保護多個操作。

為什麼?因為在操作完成“一半”之後,讓客戶端看到特定的數據結構是沒有意義的。這將我們引向一個更普遍的情況:作為一條規則,線程安全的接口也應該是事務性的(原子的)。組件(類、服務器、數據庫等)的狀態在進行API調用之前和調用之後都應該有效,接口保證了所有不變量都應該得到維護。很有可能的情況是,在執行請求的成員函數(用於類)期間,對象經歷了一個或多個無效的狀態,其不維護指定的不變量。該接口應該使其他線程不能觀察到處於這種無效狀態的對象。讓我們用一個例子來說明。

回想一下上一節的索引樹。如果想讓這棵樹是線程安全的(這是提供強保證的簡稱),應該讓插入新元素是安全的,即使在多個線程同時調用時也是安全的:

\begin{lstlisting}[style=styleCXX]
template <typename T> class index_tree {
	public:
	void insert(const T& t) {
		std::lock_guard guard(m_);
		data_.push_back(t);
		idx_.insert(&(data_[data_.size() - 1]));
	}
	private:
	std::set<T*, compare_ptr<T>> idx_;
	std::vector<T> data_;
	std::mutex m_;
};

\end{lstlisting}

當然,其他方法也必須受到保護。顯然,我們不想分別鎖定\texttt{push\_back()}和\texttt{insert()}。客戶端將如何處理一個對象,客戶如何處理數據存儲中有新元素但不在索引中的對象?接口甚至沒有定義這個新元素是否在容器中,使用迭代器遍歷索引也不行。但若使用\texttt{find()}查找數據存儲,那麼應該是可以的。這種不一致性說明,索引樹容器的不變量是在插入前和後維護的。因此,其他線程不能看到這樣一個定義不明確的狀態,這一點非常重要。我們通過確保接口同時是線程安全的和事務性的來實現這一點。同時調用多個成員函數是安全的,一些線程會阻塞並等待其他線程完成。每個成員函數將對象從一個定義良好的狀態移動到另一個定義良好的狀態(換句話說,它執行一個事務,比如添加一個新元素)。這兩個因素的組合使得對象的使用是安全的。

如果需要一個反例(設計併發接口時不要做什麼),回想一下第7章中關於\texttt{std::queue}的討論。從隊列中刪除元素的接口不是事務性的:\texttt{front()}返回前端元素但不刪除它,而\texttt{pop()}刪除前端元素但不返回任何東西。如果隊列為空,這兩種方法都會產生未定義行為。單獨鎖定這些方法對我們沒有好處,所以線程安全的API必須使用我們在第7章(併發的數據結構)中考慮過的,構造事務性操作的方法,並用鎖來保護。

現在轉向效率,如果作為容器構建塊的各個對象都進行自我鎖定,那麼效率肯定很低。想象一下,若\texttt{std::deque<T>::push\_back()}本身用一個鎖保護,將使\texttt{deque}線程安全(當然,假設其他相關方法也進行鎖定)。但這對我們沒有好處,因為仍然需要用鎖來保護整個事務。它所做的只是浪費一些時間來獲取和釋放一個我們不需要的鎖。

另外,請記住並不是所有的數據都是併發訪問的。在將共享狀態的數量最小化的程序中,大多數工作都是在特定於線程的數據(對象和獨佔一個線程的其他數據)上完成的,對共享數據的更新相對較少。獨佔線程的對象不應該使用鎖或其他同步操作。

現在似乎有了一個矛盾:一方面,應該用線程安全的事務接口來設計類和其他組件。另一方面,因為可能正在構建進行鎖操作的高級組件,所以不應該用鎖或其他同步機制來增加接口的負擔。 

解決這一矛盾的方法一般是兩種都做:提供非鎖定接口(可以用作高級組件的構建塊)和提供線程安全接口(有意義的地方)。通常,後者是通過使用鎖保護來裝飾非鎖定接口來實現的。當然,必須在合理範圍內進行。首先,非事務性接口都是專為單線程或構建高級接口而存在的,不需要鎖定。其次,在特定的設計中,有一些組件和接口是在上下文中使用,也許數據結構是專門為每個線程單獨完成的工作而設計的。同樣,也沒有理由增加併發性的開銷。根據設計,有些組件可能僅用於併發,並且是頂級組件——應該具有線程安全的事務接口。這仍然保留了許多類和其他組件,可能以兩種方式使用,並且需要鎖定和非鎖定不同的版本。

根本上說,有兩種方法可以做到這一點。第一種是設計一個可以在請求時使用鎖的組件,例如:

\begin{lstlisting}[style=styleCXX]
template <typename T> class index_tree {
	public:
	explicit index_tree(bool lock) : lock_(lock) {}
	void insert(const T& t) {
		optional_lock_guard guard(lock_ ? &m_ : nullptr);
		…
	}
	private:
	…
	std::mutex m_;
	const bool lock_;
};
\end{lstlisting}

為此,需要一個條件\texttt{lock\_guard}。可以構建一個使用\texttt{std::optional}或\texttt{std::unique\_ptr},但這種方式不好看,效率還低。編寫類似\texttt{std::lock\_guard}的RAII類就要容易得多:

\begin{lstlisting}[style=styleCXX]
template <typename L> class optional_lock_guard {
	L* lock_;
	public:
	explicit optional_lock_guard(L* lock) : lock_(lock) {
		if (lock_) lock_->lock();
	}
	~optional_lock_guard() {
		if (lock_) lock_->unlock();
	}
	optional_lock_guard(const optional_lock_guard&) = delete;
	// Handle other copy/move operations.
};
\end{lstlisting}

除了不可複製之外,\texttt{std::lock\_guard}還是不可移動的。可以遵循相同的設計或使類可移動。對於類,通常可以在編譯時而不是運行時處理鎖定條件。這種方法使用具有鎖定策略的設計:

\begin{lstlisting}[style=styleCXX]
template <typename T, typename LP> class index_tree : private 
LP {
	public:
	void insert(const T& t) {
		std::lock_guard<LP> guard(*this);
		…
	}
};
\end{lstlisting}

至少應該有兩個版本的鎖定策略LP:

\begin{lstlisting}[style=styleCXX]
struct locking_policy {
	std::mutex m_;
	void lock() { m_.lock(); }
	void unlock() { m_.unlock(); }
};
struct non_locking_policy {
	void lock() {}
	void unlock() {}
};
\end{lstlisting}

現在可以創建具有弱或強線程安全保護的\texttt{index\_tree}對象:

\begin{lstlisting}[style=styleCXX]
index_tree<int, locking_policy> strong_ts_tree;
index_tree<int, non_locking_policy> weak_ts_tree;
\end{lstlisting}

當然,這種編譯時方法適用於類,但可能不適用於其他類型的組件和接口。例如,在與遠程服務器通信時,可能希望在運行時瞭解當前會話是共享的,還是獨佔的。

第二個選項在前面討論過,鎖的裝飾器。在這個版本中,原來的類(\texttt{index\_tree})只提供了弱的線程安全保證。這個封裝類提供了強保證:

\begin{lstlisting}[style=styleCXX]
template <typename T> class index_tree_ts :
private index_tree<T> 
{
	public:
	using index_tree<T>::index_tree;
	void insert(const T& t) {
		std::lock_guard guard(m_);
		index_tree<T>::insert(t);
	}
	private:
	std::mutex m_;
};
\end{lstlisting}

注意,雖然封裝通常優於繼承,但這裡繼承的優點是可以避免複製修飾類的所有構造函數。 

同樣的方法也可以應用於其他API,使用顯式的參數來控制鎖和裝飾器。使用哪一種主要取決於設計細節——它們有各自的優點和缺點。注意,即使鎖的開銷與特定API調用所做的工作相比微不足道,也要避免使用不必要的鎖。特別是,這種鎖添加了檢查死鎖的代碼。

所有線程安全的接口都應該是事務性的準則,與設計異常安全(或更普遍地說,錯誤安全的接口)的最佳實踐之間有很多重疊。後者更復雜,因為不僅要保證調用接口前後的有效狀態,而且在檢測到錯誤後,系統仍要保持良好定義的狀態。 

從性能的角度來看,錯誤處理本質上是開銷,並不期望錯誤會頻繁出現(否則,它們就不是真正的錯誤,而是必須處理的情況)。幸運的是,編寫錯誤安全代碼的最佳實踐(比如使用RAII對象進行清理)也非常有效,很少會帶來明顯的開銷。儘管如此,一些錯誤條件是很難檢測出來的,正如在第11章中看到的那樣。

本節中,瞭解了一些設計高效併發API的指導原則:

\begin{itemize}
\item 
用於併發使用的接口應該是事務性的。

\item 
接口應該提供最小的必要線程安全保證(對於不打算併發使用的接口提供弱保證)。

\item 
對於既用作客戶端可見API,又用作創建自己的、更復雜的事務,需要提供鎖的高級組件構建塊接口,通常有兩個版本:一個具有強線程安全保證,另一個具有弱(或鎖定和非鎖定)線程安全保證。這可以通過條件鎖定或使用裝飾器來實現。
\end{itemize}

這些指導原則與設計健壯且清晰的API的其他最佳實踐大體一致。因此,為了獲得更好的性能,很少需要做出設計上的權衡。 

現在讓我們將併發問題拋在腦後,轉而討論性能設計的其他領域。

\subsubsubsection{12.4.2\hspace{0.2cm}複製和發送數據}

這個討論將在第9章中討論的問題進行總結,當時討論了不必要的複製,通常複製都需要發送或接收一些數據。這是一個非常籠統的概念,除了同樣普遍的“注意數據傳輸的成本”外,無法提供任何普遍適用的指導方針。對於一些常見的接口類型,可以對此進行一些說明。 

已經討論了C++中複製內存的開銷,以及對接口的考慮,並在第9章中討論了實現。對於設計可以強調通用的指導原則,擁有定義良好的數據所有權和生命週期管理。它出現在性能上下文中的原因是過度複製通常是所有權混亂導致的,這是一種解決數據在使用時消失的方法,因為複雜系統的許多部分的生命週期不好理解。 

在分佈式程序、客戶端-服務器應用程序,或者在帶寬限制很重要的組件之間的接口中,都需要管理一些問題。在這樣的情況下,通常會使用數據壓縮。用CPU時間換取帶寬,因為壓縮和解壓縮數據會消耗處理時間,但傳輸速度會更快。是否壓縮特定通道中的數據不能在設計時做出決定,因為已知信息不夠,無法做出權衡。因此,在設計系統時考慮壓縮的可能性就很重要,這對於設計可轉換為壓縮格式的數據結構的接口很重要。如果設計要求壓縮整個數據集將其傳輸,然後將其轉換回已解壓縮的格式,那麼用於處理數據的接口不會改變,但內存需求會增加,因為在某個時刻內存中存儲了已壓縮和未壓縮的狀態。另一種是在內部存儲壓縮數據的結構,在設計接口時需要事先進行考慮。 

舉個例子,假設使用簡單的結構來存儲三維位置和屬性:

\begin{lstlisting}[style=styleCXX]
struct point {
	double x, y, z;
	int color;
	… maybe more data …
};
\end{lstlisting}

流行的準則是應該避免getter和setter方法,只訪問相應的數據成員。這裡,不建議這樣做:

\begin{lstlisting}[style=styleCXX]
class point {
	double x, y, z;
	int color;
	public:
	double get_x() const { return x; }
	void set_x(double x_in) { x = x_in; } // Same for y etc
};
\end{lstlisting}

將這些對象存儲在一個\texttt{point}集合中:

\begin{lstlisting}[style=styleCXX]
class point_collection {
	point& operator[](size_t i);
};
\end{lstlisting}

這種設計在一段時間內還不錯,但隨著需求不斷變化,現在必須存儲和傳輸數百萬個點。很難想象如何用這個接口引入內部壓縮,索引操作符返回一個對對象的引用,該對象必須有三個可直接訪問的\texttt{double}數據成員。如果有getter和setter,可能已經能夠將這個點實現為集合內的一個壓縮點集的代理了:

\begin{lstlisting}[style=styleCXX]
class point {
	point_collection& coll_;
	size_t point_id_;
	public:
	double get_x() const { return coll_[point_id_]; }
	…
};
\end{lstlisting}

集合存儲壓縮數據,並可以動態地解壓其中的部分數據,可以通過\texttt{point\_id\_}訪問相應的點。

當然,更有利於壓縮的接口會要求我們順序地遍歷整個集合。我們剛剛重新回顧了指導方針,該指導方針要求儘可能少地透露關於集合內部工作的信息。對壓縮的關注提供了一個特殊的點。如果考慮數據壓縮的可能性,或者存儲和傳輸的替代數據的表示,還必須考慮限制對該數據的訪問。也許可以想出一種算法,不用隨機訪問數據就能完成所有需要的計算?如果通過設計限制訪問,那麼就能保留了壓縮數據的可能性(或以其他方式利用有限的訪問模式)。

當然,還有其他類型的接口,它們都有與傳輸大量數據的運行時、內存和存儲空間成本相關。進行性能設計時,考慮到這些成本可能對性能至關重要,並嘗試限制接口,以實現內部數據表示的最大自由化。當然,都應該在合理的範圍內進行,比如手寫的配置文件不太可能成為性能瓶頸(無論哪種格式,通常讀要比寫快)。 

我們已經談到了數據佈局的問題,因為它會影響接口設計。現在讓我們直接關注數據組織對性能的影響。


























