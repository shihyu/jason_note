本章有兩個重點。一方面,解釋了開發者在試圖最大化代碼性能時經常忽略的未定義行為的危險。另一方面,解釋瞭如何利用未定義行為來提高性能,以及如何正確地指定和記錄這種情況。與通常的“都可能發生”相比,這一章提供了一種有點不同尋常,但更相關的方式來理解未定義行為的問題。

本章將討論以下內容:

\begin{itemize}
\item 
理解未定義行為及其存在的原因

\item 
理解關於未定義行為的真相和傳說

\item 
如何利用未定義行為

\item 
內存帶寬和延遲

\item 
學習未定義行為和效率之間的聯繫,以及如何利用它
\end{itemize}

將瞭解在(別人的)代碼中遇到未定義行為時,如何識別並瞭解未定義的行為如何與性能相關。本章還描述如何,通過有意地允許、記錄,並在其周圍設置保護措施,從而使用未定義行為。