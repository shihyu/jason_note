
编程语言中,向量一词有多种解释,想要编写高性能和可扩展的代码时,理解语言或编译器的解释很重要。DPC++编译器围绕这样的思想构建:源码中的向量是工作项的本地函数,编译器跨工作项的隐式向量值,可能映射到硬件中的SIMD指令。当我们想要编写显式映射到SIMD指令集时,应该查看供应商的文档和对SYCL和DPC++的扩展。使用多个工作项(例如ND-Range)编写内核,并依赖编译器跨工作项向量化应该是大多数应用程序的方式,这样做利用了SPMD强大的抽象功能,它提供了一个简单的编程模型,提供了跨设备和架构的可扩展性。\par

本章描述了vec接口,想要对类似的数据类型进行操作(例如,一个具有多个颜色通道的像素)时,提供了便利。本章还简要介绍了硬件中的SIMD指令,以便在第15和16章中进行更详细的讨论。\par

\newpage