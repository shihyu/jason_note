在本书的前一部分,前端开发中,我们首先介绍了Clang的内部原理,Clang是LLVM为C族编程语言设计的官方前端。我们讨论了各种示例,包括技能和知识,它们可以处理与源代码紧密耦合的各种问题。

本书的这一部分,我们将使用\textbf{LLVM IR}——用于编译器优化和代码生成(目标无关)的\textbf{中间表示(IR)}。与Clang的\textbf{抽象语法树(AST)}相比,LLVM IR通过封装相应的执行细节来实现更强大的程序分析和转换,从而提供了不同级别的抽象。除了设计LLVM IR之外,围绕这种IR格式还有一个成熟的生态系统,提供了无数的资源,比如库、工具和算法实现。我们将讨论有关LLVM IR的各种主题,包括最常见的LLVM Pass开发,使用和编写程序分析器,以及使用LLVM IR API的实践技巧。此外,我们还将回顾更高级的技能,如\textbf{程序引导优化(PGO)}和杀毒器的开发。

本章中,我们将讨论为新\textbf{PassManager}编写转换\textbf{Pass}和程序分析。LLVM Pass是整个项目中最基本、最关键的概念之一。它允许开发人员将程序处理逻辑封装到一个模块化单元中,该单元可以根据情况由\textbf{PassManager}自由地与其他Pass组合。在Pass基础设施的设计方面,LLVM实际上已经对PassManager和AnalysisManager进行了全面检查,以提高它们的运行时性能和优化质量。新PassManager使用了一个完全不同的接口来封装通行证。然而,新接口并不向后兼容旧接口,这就不能在新PassManager中运行旧Pass,反之亦然。更糟糕的是,尽管现在LLVM和Clang默认都启用了这个新接口,但没有太多的在线学习资源讨论这个新接口。本章的内容将填补这一空白,并提供LLVM中这个关键系统的最新指南。

本章中,我们将讨论以下内容:

\begin{itemize}
\item 为新PassManager写一个LLVM Pass
\item 使用新AnalysisManager
\item 新PassManager中的设备
\end{itemize}

有了从本章学到的知识,读者们应该能够编写一个LLVM Pass,使用新的Pass基础结构来转换,甚至优化代码。还可以利用LLVM程序分析框架提供的分析数据进一步提高Pass的质量。











