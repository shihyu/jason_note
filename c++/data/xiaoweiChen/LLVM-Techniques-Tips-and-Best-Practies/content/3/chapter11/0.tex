前一章中,我们学习了\textbf{低层虚拟机(LLVM)的中间表示(IR)}(LLVM中与目标无关的中间表示)的基础知识,以及如何使用C++\textbf{应用程序编程接口(API)}检查和操作它,这些是在LLVM中进行程序分析和转换的核心技术。除了这些功能集,LLVM还提供了许多配套工具,以提高编译器开发人员使用LLVM IR时的生产效率。我们将在本章中讨论这些工具。

编译器是一种复杂的软件。它不仅需要处理数千种不同的情况——包括不同类型的输入程序和各种各样的目标体系结构——而且编译器的正确性也是一个重要的话题:编译后的代码需要具有与原始代码相同的行为。LLVM虽然是一个大型编译器框架(可能是最大的编译器框架之一),但也不能例外。

为了解决这些复杂性问题,LLVM提供了一系列小工具来改善开发体验。本章中,我们将展示如何使用这些工具。本文介绍的工具,可以帮助诊断正在开发的LLVM代码中出现的问题。这包括更高效的调试、错误处理和分析能力,例如:其中一个工具可以收集关键组件的统计数字——比如:特定Pass处理的基本块的数量——并自动生成报告。另一个例子是LLVM自己的错误处理框架,可以尽可能多地防止未处理的错误(一种常见的编程错误)。

以下是我们将在本章中介绍的内容:

\begin{itemize}
\item 打印诊断信息
\item 收集统计信息
\item 添加时间测量
\item LLVM中的错误处理工具
\item 了解\texttt{Expected}和\texttt{ErrorOr}类
\end{itemize}

在这些实用程序的帮助下,我们将有更好的时间调试和诊断LLVM代码,从而使开发者能够专注于想要用LLVM实现的核心逻辑。












