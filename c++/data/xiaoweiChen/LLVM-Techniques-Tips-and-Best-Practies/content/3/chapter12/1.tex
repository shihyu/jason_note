本章中,将使用多个子项目。其中之一——Compiler-RT——需要通过修改CMake配置来包含在你的构建中。请打开\texttt{CMakeCache.tx}t文件,并将\texttt{compiler-rt}字符串添加到\texttt{LLVM\_ENABLE\_PROJECTS}变量中。下面是一个例子:

\begin{lstlisting}[style=styleCMake]
//Semicolon-separated list of projects to build…
LLVM_ENABLE_PROJECTS:STRING="clang;compiler-rt"
\end{lstlisting}

编辑该文件后,用任何构建目标重新启动构建,CMake会重新配置。

当一切就绪,就可以构建本章所需的组件了。下面是一个命令示例:

\begin{tcblisting}{commandshell={}}
$ ninja clang compiler-rt opt llvm-profdata
\end{tcblisting}

这将构建我们都熟悉的\texttt{clang}工具和一系列Compiler-RT库。

可以在GitHub库中找到本章的示例代码: \url{https://github.com/PacktPublishing/LLVM-Techniques-Tips-and-Best-Practices-Clang-and-Middle-End-Libraries/tree/main/Chapter12}.





















