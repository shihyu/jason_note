\begin{flushright}
	\zihao{0} 前言
\end{flushright}

大多数程序员的开发流程都有编译器——或者某种形式的编译技术。现代编译器不仅将高级编程语言转换为低层机器码,而且在优化所编译速度、大小甚至内存占用方面也扮演着关键性角色。要有这些特性,构建可使用的编译器就是一项具有挑战性的任务。

LLVM是一个编译器优化和代码生成的框架,提供了构建块,大大减少了开发人员创建高质量优化编译器和编程语言工具的工作。Clang是该公司最著名的产品之一,是C族语言编译器,可构建数千个广泛使用的软件,包括谷歌Chrome浏览器和iOS应用程序。LLVM也用于许多不同的编程语言的编译器中,比如Apple的Swift。毫不夸张地说,可以用LLVM创建一种新的编程语言,就可以成为时下最热门的话题之一。

LLVM提供了广泛的特性,有成百个库和成千个API,包括优化关键功能到实际实用程序当中。本书中,为LLVM中两个最重要的子系统——Clang和中端库提供了一个完整的开发指南。首先介绍几个组件和开发最佳实践,可以积累一些LLVM的开发经验,再了解如何使用Clang进行开发。更具体地说,我们将专注于增强和定制Clang中的功能。本书的最后一部分,将了解有关LLVM IR开发的关键知识。这包括如何用最新的语法编写LLVM Pass,以及如何处理不同的IR结构。还会有几个实用例程,可以极大地提高LLVM开发的生产率。注意,在本书中不假设任何特定的LLVM版本——试图保持最新,并囊括来自LLVM源代码的最新特性。

本书每一章都提供了一些代码片段和项目示例。可以从本书的GitHub存储库中下载它们,并尝试进行修改。

\hspace*{\fill} \\ %插入空行
\noindent\textbf{适读人群}

本书适用于所有具有LLVM工作经验的软件工程师,本书会提供了简明的开发指南和参考。如果你是一名学术研究者,这本书将助你在短时间内学习有用的LLVM技能,使你能够快速构建项目原型。编程语言爱好者也会发现这本书中的内容,在LLVM的帮助下构建一种新的编程语言也十分有趣。


\hspace*{\fill} \\ %插入空行
\textbf{本书内容}

\textit{第1章,使用有限的资源构建LLVM}。简要介绍了LLVM,并介绍如何在不过多消耗CPU、内存资源和磁盘空间的情况下构建LLVM。这为以后的开发周期和更流畅的体验奠平了道路。

\textit{第2章,探索LLVM的构建系统}。介绍如何编写用于源码树内和树外LLVM开发的CMake构建脚本,将了解到如何利用LLVM的自定义构建系统来编写更具表现力和健壮性的构建脚本。

\textit{第3章,LLVM LIT测试}。介绍了如何使用LLVM的LIT基础架构运行测试。本章不仅可以更好地理解LLVM源码树中测试的工作原理,还能将这种直观的、可扩展的测试结构集成到其他项目中。

\textit{第4章,TableGen开发}。介绍了如何编写TableGen——一种由LLVM发明的\textbf{特殊领域特定语言(DSL)}。这里使用TableGen作为处理结构化数据的工具,可以了解在LLVM之外如何使用TableGen。

\textit{第5章,探索Clang架构}。标志着关于Clang主题的开始。本章为概述了Clang,特别是编译流程,并介绍了各个组件在Clang的编译流程中的作用。

\textit{第6章,扩展预处理器}。展示了Clang的预处理器架构,以及如何不需要修改LLVM源代码树中的任何代码,开发一个插件来扩展其功能。

\textit{第7章,处理AST}。介绍了如何在Clang中使用\textbf{抽象语法树(AST)}进行开发。包括了解使用AST的内存表示的主题,以及创建插件的教程,该插件将自定义的AST处理逻辑插入到编译流中。

\textit{第8章,使用编译器标志和工具链}。涵盖了向Clang添加自定义编译器标志和工具链的步骤,以及如何让Clang支持新特性或新平台。

\textit{第9章,使用PassManager和AnalysisManager}。标志着对LLVM中端库讨论的开始。本章的重点是编写一个LLVM Pass——使用PassManager语法,以及如何通过AnalysisManager访问程序分析数据。

\textit{第10章,处理LLVM IR}。这是一个很大的章节,包含了关于LLVM IR的各种核心知识,包括LLVM IR的内存表示结构和使用不同的IR单元(如函数、指令和循环)等技能。

\textit{第11章,准备相关的工具}。介绍了一些实用程序,在使用LLVM IR时,用以提高工作效率(如有较好的调试经验)。

\textit{第12章,学习LLVM IR表达式}。展示表达式如何在LLVM IR上进行工作。它涵盖了两个主要的用例:消毒器和\textbf{数据引导优化(PGO)}。对于前者,将了解如何创建自定义消毒器。对于后者,将了解如何在LLVM Pass中利用PGO数据。

\hspace*{\fill} \\ %插入空行
\textbf{编译环境}

这本书旨在向您介绍LLVM的最新特性,因此我们鼓励您在12.0版本之后使用LLVM,或使用开发分支(即主分支)。

假设您正在Linux或Unix系统(包括macOS)上工作。本书中的工具和示例命令大多是在命令行界面中运行的,您可以自由地使用任何代码编辑器或IDE来编写代码。

\begin{table}[H]
	\begin{tabular}{|l|l|}
		\hline
		书中涉及的软件/硬件                                                                                                                  & 操作系统                                                             \\ \hline
		LLVM版本高于或等于12.x                                                                                                                  & \begin{tabular}[c]{@{}l@{}}macOS或Linux(任意衍生版)\end{tabular}                                                             \\ \hline
		\begin{tabular}[c]{@{}l@{}} GCC 5.1或更高版本, 或者是支持C++14的任意编译器\end{tabular} &  \\ \hline
		CMake 3.13.4或更高版本                                                                                                                                  &                                                                                  \\ \hline
		Python 3.x                                                                                                                                           &                                                                                  \\ \hline
		Ninja Build或 GNU Make                                                                                                                                    &                                                                                  \\ \hline
		Graphviz                                                                                                                                    &                                                                                  \\ \hline
	\end{tabular}
\end{table}

在\textit{第1章,使用有限的资源构建LLVM}中,将详细介绍如何从源代码构建LLVM。

\textbf{如果你正在使用这本书的数字版本,我们建议自己输入代码或通过GitHub库访问代码(链接在下一节中提供)。这样做将帮助您避免与复制和粘贴代码相关的任何潜在错误。}

\hspace*{\fill} \\ %插入空行
\textbf{下载示例}

您可以从GitHub网站\url{https://github.com/PacktPublishing/LLVM-Techniques-Tips-and-Best-Practices-Clang-and-Middle-End-Libraries}下载本书的示例代码文件。如果代码有更新,它将在现有的GitHub库中更新。

我们还有其他的代码包,还有丰富的书籍和视频目录,都在\url{https://github.com/PacktPublishing/}。去看看吧!

\hspace*{\fill} \\ %插入空行
\textbf{下载彩图}

我们还提供了一个PDF文件,其中包含了本书中使用的屏幕截图/图表的彩色图像。可以在这里下载:\url{https://static.packt-cdn.com/downloads/9781838824952\_ColorImages.pdf}。

\hspace*{\fill} \\ %插入空行
\textbf{联系方式}

我们欢迎读者的反馈。

\textbf{反馈}:如果你对这本书的任何方面有疑问,需要在你的信息的主题中提到书名,并给我们发邮件到\url{customercare@packtpub.com}。

\textbf{勘误}:尽管我们谨慎地确保内容的准确性,但错误还是会发生。如果您在本书中发现了错误,请向我们报告,我们将不胜感激。请访问\url{www.packtpub.com/support/errata},选择相应书籍,点击勘误表提交表单链接,并输入详细信息。

\textbf{盗版}:如果您在互联网上发现任何形式的非法拷贝,非常感谢您提供地址或网站名称。请通过\url{copyright@packt.com}与我们联系,并提供材料链接。

\textbf{如果对成为书籍作者感兴趣}:如果你对某主题有专长,又想写一本书或为之撰稿,请访问\url{authors.packtpub.com}。

\hspace*{\fill} \\ %插入空行
\textbf{欢迎评论}

请留下评论。当您阅读并使用了本书,为什么不在购买网站上留下评论呢?其他读者可以看到您的评论,并根据您的意见来做出购买决定。我们在Packt可以了解您对我们产品的看法,作者也可以看到您对他们撰写书籍的反馈。谢谢你!

想要了解Packt的更多信息,请访问\url{packt.com}。

\newpage












