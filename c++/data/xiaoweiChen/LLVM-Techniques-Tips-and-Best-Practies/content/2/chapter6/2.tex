当使用源码文件时,最基本的问题之一是编译器前端如何能够在文件中定位字符串片段。一方面,良好地打印格式消息(例如,编译错误和警告消息)是一项至关重要的工作,其中必须显示准确的行号和列号。另一方面,前端可能需要同时管理多个文件,并以一种有效的方式访问内存中的内容。在Clang中,这些问题主要由两个类处理:\texttt{SourceLocation}和\texttt{SourceManager}。我们将简要介绍它们,并展示如何在实践中进行使用。

\subsubsubsection{6.2.1\hspace{0.2cm}SourceLocation}

\texttt{SourceLocation}类用于表示代码在文件中的位置。在实现时,使用行号和列号可能是最直观的方式。然而,在现实场景中,事情可能会变得复杂,例如:在内部不能天真地使用一对数字作为源代码位置的内存表示。主要原因是,Clang的代码库中广泛使用了\texttt{SourceLocation}实例,并且基本上贯穿了整个前端编译管道。因此,重要的是使用一种简洁的方式来存储信息,而不是两个32位整数(这可能还不够,因为还想知道具体的文件!),这很容易使Clang的运行时占用过多的内存。

Clang通过使用优雅设计,将\texttt{SourceLocation}作为一个大数据缓冲区的\textbf{指针}(或句柄)来解决这个问题,该缓冲区存储了所有真正的源代码内容,例如:\texttt{SourceLocation}只使用一个无符号整型,这也意味着其实例可以复制——这可以带来一些性能上的好处。由于\texttt{SourceLocation}是指针,所以只有当它与刚才提到的数据缓冲区一起使用才有意义,数据缓冲区是由本文中的第二个主要角色\texttt{SourceManager}管理的。

\begin{tcolorbox}[colback=blue!5!white,colframe=blue!75!black, fonttitle=\bfseries,title=其他实用工具]
\hspace*{0.7cm}\texttt{SourceRange}是一对\texttt{SourceLocation}对象,表示源代码范围的开始和结束。\texttt{FullSourceLocation}将普通的\texttt{SourceLocation}类和它关联的\texttt{SourceManager}类封装到一个类中,这样只需要一个\texttt{FullSourceLocation}实例即可,而不是两个对象(一个\texttt{SourceLocation}对象和一个\texttt{SourceManager}对象)。
\end{tcolorbox}

\hspace*{\fill} \\ %插入空行
\noindent
\textbf{可复制的}

除非有很好的理由,否则在编写C++的正常情况下,应该避免通过对象的值传递对象(例如,作为函数调用的参数)。因为涉及到对底层数据成员的复制,所以应该通过指针或引用来传递。但是,如果仔细设计,类型实例可以来回复制,而不需要做很多工作,例如:给没有成员变量或成员变量很少的类,加上一个默认的复制构造函数。如果实例可以复制,那么可以通过值来进行传递。

\subsubsubsection{6.2.2\hspace{0.2cm}SourceManager}

\texttt{SourceManager}类管理存储在内存中的所有源文件,并提供访问它们的接口。还提供API来处理源代码位置,通过刚刚介绍的\texttt{SourceLocation}实例,例如:要从\texttt{SourceLocation}实例获取行号和列号,运行以下代码:

\begin{lstlisting}[style=styleCXX]
void foo(SourceManager &SM, SourceLocation SLoc) {
	auto Line = SM.getSpellingLineNumber(SLoc),
	Column = SM.getSpellingColumnNumber(SLoc);
	…
}
\end{lstlisting}

前面代码片段中的\texttt{Line}和\texttt{Column}变量,分别是\texttt{SLoc}所指向源码位置的行号和列号。

想知道为什么在前面的代码片段中使用术语\texttt{spellingLineNumber},而不是使用\texttt{LineNumber}吗?在宏展开(或任何在预处理过程中发生的展开)的情况下,Clang会在展开之前和之后跟踪宏内容的\texttt{SourceLocation}实例。拼写位置表示最初编写源代码的位置,而非宏扩展后的位置。

也可以使用以下API创建一个新的与拼写相关的扩展:

\begin{lstlisting}[style=styleCXX]
SourceLocation NewSLoc = SM.createExpansionLoc(
	SpellingLoc, // The original macro spelling location
	ExpansionStart, // Start of the location where macro is
	//expanded
	ExpansionEnd, // End of the location where macro is
	// expanded
	Len // Length of the content you want to expand
);
\end{lstlisting}

返回的\texttt{NewSLoc}可以通过\texttt{SourceManager}查询与拼写扩展关联的位置信息。

在后面的章节中,这些重要的概念和API将帮助您处理源代码位置——特别是在处理预处理程序时。下一节将介绍Clang中预处理器和词法分析器开发的一些背景知识,这些知识对于开发自定义预处理器插件和回调非常有用。













