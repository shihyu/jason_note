

本章需要构建\texttt{clang}可执行文件。如果没有,可以使用以下命令构建:

\begin{tcblisting}{commandshell={}}
$ ninja clang
\end{tcblisting}

另外,可以使用下面的命令行标志来打印AST的文本表示:

\begin{tcblisting}{commandshell={}}
$ clang -Xclang -ast-dump foo.c
\end{tcblisting}

例如,\texttt{foo.c}包含以下内容:

\begin{lstlisting}[style=styleCXX]
int foo(int c) { return c + 1; }
\end{lstlisting}

使用\texttt{-Xclang -ast-dump}命令行标志,我们可以为\texttt{foo.c}输出AST:

\begin{tcblisting}{commandshell={}}
TranslationUnitDecl 0x560f3929f5a8 <<invalid sloc>> <invalid sloc>
|…
`-FunctionDecl 0x560f392e1350 <foo.c:2:1, col:30> col:5 foo
'int (int)'
  |-ParmVarDecl 0x560f392e1280 <col:9, col:13> col:13 used c 'int'
\end{tcblisting}
\begin{tcblisting}{commandshell={}}
  `-CompoundStmt 0x560f392e14c8 <col:16, col:30>
    `-ReturnStmt 0x560f392e14b8 <col:17, col:28>
      `-BinaryOperator 0x560f392e1498 <col:24, col:28> 'int' '+'
        |-ImplicitCastExpr 0x560f392e1480 <col:24> 'int' <LValueToRValue>
        | `-DeclRefExpr 0x560f392e1440 <col:24> 'int' lvalue ParmVar 
        0x560f392e1280 'c' 'int'
        `-IntegerLiteral 0x560f392e1460 <col:28> 'int' 1
\end{tcblisting}

这个标志对于找出用什么C++类来表示代码非常有用。例如,正式的函数参数/参数由\texttt{ParmVarDecl}类表示,在前面的代码中突出显示了这个类。

本章的代码示例连接: \url{https://github.com/PacktPublishing/LLVM-Techniques-Tips-and-Best-Practices-Clangand-Middle-End-Libraries/tree/main/Chapter07}.





