正如本章开头所说,如果用默认的(CMake)配置来构建LLVM,可以通过CMake构建整个项目,整个过程可能需要几个小时才能完成:

\begin{tcblisting}{commandshell={}}
$ cmake ../llvm
$ make all
\end{tcblisting}

可以通过使用一些工具和改变环境来避免这么久的耗时。本节中,将介绍一些指导原则,以帮助您选择正确的工具和配置,从而加快构建时间,以及改善内存占用。

\subsubsubsection{1.2.1\hspace{0.2cm}使用Ninja替换GNU Make}

第一个改进是使用Ninja(\url{https://ninja-build.org})而不是GNU Make,后者是CMake在主要Linux/Unix平台上生成的默认构建系统。

以下是在系统上设置Ninja构建的步骤:

\begin{enumerate}

\item 在Ubuntu上,可以通过以下命令来安装Ninja:
\begin{tcblisting}{commandshell={}}
$ git clone https://github.com/llvm/llvm-project
\end{tcblisting}
Ninja is also available in most Linux distributions.
	
\item 然后,为LLVM构建调用CMake时,需要额外添加一个参数:
\begin{tcblisting}{commandshell={}}
$ cmake -G "Ninja" ../llvm
\end{tcblisting}
	
\item 最后,使用下面的命令进行构建:
\begin{tcblisting}{commandshell={}}
$ ninja all
\end{tcblisting}

\end{enumerate}

Ninja在大型代码库(如LLVM)上的运行速度明显快于GNU Make。Ninja的快速运行速度的秘密是,虽然像Makefile这样的大多数构建脚本都是手工编写的,但Ninja的构建脚本的语法是build.ninja,更类似于汇编代码(它不应该由开发人员编辑,而是由其他高级构建系统(如CMake)生成)。Ninja使用类似于程序集的构建脚本,这使得它能够在底层进行优化,并消除许多冗余,比如:在调用构建时,解析速度会变慢。Ninja在构建目标之间产生更好的依赖关系方面也能很好的完成。

Ninja会根据其并行度做出决定,以智能的方式绝对最大程度并行执行多个作业。如果想显式地分配工作线程的数量,GNU Make使用的命令行选项仍然可以在这里使用:

\begin{tcblisting}{commandshell={}}
$ ninja -j8 all
\end{tcblisting}

接下来,让我们看看如何避免使用BFD链接器。

\subsubsubsection{1.2.2\hspace{0.2cm}避免使用BFD连接器}

第二个改进是使用BFD链接器以外的链接器,这是在大多数Linux系统中使用的默认链接器。BFD连接器是Unix/Linux系统上最成熟的连接器,但在速度或内存消耗方面并没有得到优化。这将造成性能瓶颈,特别是对于像LLVM这样的大型项目。因为与编译阶段不同,链接阶段很难进行文件级的并行化。更不用说,在建立LLVM时,BFD链路的峰值内存消耗通常需要20GB左右,这对内存较少的计算机造成了负担。幸运的是,目前至少有两个连接器提供了良好的单线程性能和较低的内存消耗:GNU gold连接器和LLVM自己的连接器LLD。

gold链接器最初是由谷歌开发的,并捐赠给GNU的binutils。在现代Linux发行版中,默认情况下应该将其放在binutils包中。LLD是LLVM的子项目之一,具有更快的连接速度和实验性的并行连接技术。一些Linux发行版(例如较新的Ubuntu版本)已经在包存储库中包含了LLD。您也可以从LLVM的官方网站下载预构建版本。

要使用gold链接器或LLD来构建LLVM,添加一个额外的CMake参数,使用你想要使用的链接器的名称。

对于gold链接器,可以使用以下命令:

\begin{tcblisting}{commandshell={}}
$ cmake -G "Ninja" -DLLVM_USE_LINKER=gold ../llvm
\end{tcblisting}

类似地,对于LLD,使用如下命令:

\begin{tcblisting}{commandshell={}}
$ cmake -G "Ninja" -DLLVM_USE_LINKER=lld ../llvm
\end{tcblisting}

\begin{tcolorbox}[colback=blue!5!white,colframe=blue!75!black, title=限制用于链接的并行线程的数量]
\hspace*{0.7cm}限制用于链接的并行线程的数量是减少(峰值)内存消耗的另一种方法。你可以在cmake过程中,通过指定\texttt{LLVM\_PARALLEL\_LINK\_JOBS=<N>}变量来实现,其中\texttt{N}是期望的工作线程数。
\end{tcolorbox}

因此,只要使用不同的工具,构建时间就可以显著减少。下一节中,我们将通过调整LLVM的CMake参数来提高构建速度。




