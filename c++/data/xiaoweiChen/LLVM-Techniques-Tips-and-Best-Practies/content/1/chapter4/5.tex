本章中,介绍了TableGen,是一种用于表示结构化数据的强大DSL。本章已经展示了它在解决各种任务时的通用性(最初是为编译器开发而创建的)。通过在TableGen中编写甜甜圈食谱,我们了解了它的核心语法。接着,了解了关于开发自定义TableGen后端,并了解了如何使用\texttt{C++} API与从源输入解析的内存中的TableGen指令直接交互,从而让读者能自行创建一个完整和独立的TableGen工具链来实现自己自定义逻辑。学习如何掌握TableGen不仅可以帮助开发与llvm相关的项目,还可以为任意项目中解决结构化数据问题提供更多选择。

本节标志着第一部分的结束——介绍LLVM项目中常用的组件。从下一章开始,我们将进入LLVM的核心编译管道。我们将讨论的第一个主题就是Clang——LLVM的C族编程语言官方前端。