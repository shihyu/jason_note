
\textbf{FileCheck}是LLVM的高级模式检查器,与Unix/Linux系统中的grep类似,使用基于行的上下文,从而提供了更强大而简单的语法。此外,可以将\texttt{FileCheck}指令放在测试目标旁,可以让测试用例自包含,使测试更容易理解。

虽然基本的\texttt{FileCheck}语法很容易上手,但\texttt{FileCheck}还有许多其他功能,它们才能真正展示\texttt{FileCheck}的强大功能,并极大地改善了开发者的测试体验——例如创建更简洁的测试脚本和解析更复杂的程序输出,本节将向您展示其中的一些技巧。

\subsubsubsection{3.3.1\hspace{0.2cm}准备示例项目}

首先需要构建\texttt{FileCheck}命令行工具。与前一节类似,在LLVM树中构建一个\texttt{check-XXX}(伪)目标是最简单的方法:

\begin{tcblisting}{commandshell={}}
$ ninja check-llvm-support
\end{tcblisting}

在本节中,我们将使用一个假想的命令行工具\texttt{js-obfuscator},这是一个JavaScript 混淆工具。\textbf{混淆}是一种常用的技术,用于隐藏知识产权或加强安全保护。例如,可以在以下JavaScript代码中使用JavaScript混淆器:

\begin{lstlisting}[style=styleJavaScript]
const onLoginPOST = (req, resp) => {
	if(req.name == 'admin')
		resp.send('OK');
	else
		resp.sendError(403);
}
myReset.post('/console', onLoginPOST);
\end{lstlisting}

将转换成以下代码:

\begin{lstlisting}[style=styleJavaScript]
const t = "nikfmnsdzaO";
const aaa = (a, b) => {
	if(a.z[0] == t[9] && a.z[1] == t[7] &&…)
		b.f0(t[10] + t[2].toUpperCase());
	else
		b.f1(0x193);
}
G.f4(YYY, aaa);
\end{lstlisting}

这个工具将尽量使原始脚本让人看不懂。测试部分面临的挑战是,在验证其正确性的同时仍然为随机性保留足够的空间。简单地说,\texttt{js-obfuscator}只有4条混淆规则:

\begin{enumerate}
\item 只混淆局部变量名,包括形参。形式参数名以\textit{<小写单词><参数索引号>}格式进行混淆。局部变量名混淆成小写字母和大写字母的组合。

\item 如果用箭头语法来声明函数——例如,\texttt{foo = (arg1, arg2) => \{…\}}——箭头和左花括号\texttt{(=> \{})需要放在下一行。
	
\item 用不同表示形式的相同值替换文字数,例如:将87替换为0x57或87.000。

\item 当使用\texttt{-\,-shuffle-funcs}选项时,会改变顶层函数的声明/出现顺序。

\end{enumerate}

最后,下面的JavaScript代码是使用\texttt{js-obfuscator}的一个示例:

\begin{lstlisting}[style=styleJavaScript]
const square = x => x * x;
const cube = x => x * x * x;
const my_func1 = (input1, input2, input3) => {
	// TODO: Check if the arrow and curly brace are in the second
	// line
	// TODO: Check if local variable and parameter names are
	// obfuscated
	let intermediate = square(input3);
	let output = input1 + intermediate - input2;
	return output;
}
const my_func2 = (factor1, factor2) => {
	// TODO: Check if local variable and parameter names are
	// obfuscated
	let term2 = cube(factor1);
	// TODO: Check if literal numbers are obfuscated
	return my_func1(94, term2, factor2);
}
console.log(my_func2(1,2));
\end{lstlisting}

\subsubsubsection{3.3.2\hspace{0.2cm}书写FileCheck指令}

下面的步骤将填充前面代码中出现的\texttt{TODO}注释:

\begin{enumerate}
\item 根据行号,第一个任务是检查局部变量和参数是否正确地混淆。根据规范,形参有特殊的重命名规则(即\textit{<小写单词><参数索引号>}),所以使用普通的\texttt{CHECK}指令和FileCheck自己的正则表达式是这里最合适的解决方案:

\begin{lstlisting}[style=styleJavaScript]
// CHECK: my_func1 = ({{[a-z]+0}}, {{[a-z]+1}},
// {{[a-z]+2}})
const my_func1 = (input1, input2, input3) => {
…
\end{lstlisting}

FileCheck使用正则表达式的子集来进行模式匹配,使用\texttt{\{\{…\}\}}或\texttt{[[…]]}。

\item 这段代码看起来非常简单。当进行了混淆,代码也需要正确的语义。因此,除了检查格式之外,对参数的后续引用也需要重构,这就是要使用FileCheck的模式绑定的原因:

\begin{lstlisting}[style=styleJavaScript]
// CHECK: my_func1 = ([[A0:[a-z]+0]],
// [[A1:[a-z]+1]], [[A2:[a-z]+2]])
const my_func1 = (input1, input2, input3) => {
	// CHECK: square([[A2]])
	let intermediate = square(input3);
…
\end{lstlisting}

该代码使用\texttt{[[…]]}语法将形参的模式绑定为名称为\texttt{A0~A2},其中绑定变量名和模式用冒号分开:\texttt{[[<绑定变量>:<模式>]]}。使用相同的\texttt{[[…]]}语法绑定变量的引用位置,不过没有了模式部分。

\begin{tcolorbox}[colback=blue!5!white,colframe=blue!75!black, fonttitle=\bfseries,title=Note]
\hspace*{0.7cm}一个绑定变量可以有多个定义点,其参考点将读取最后的定义值。
\end{tcolorbox}

\item 不要忘记第二条规则——函数头的箭头和左花括号需要放在第二行。要实现“后一行”的概念,可以使用\texttt{CHECK-NEXT}指令:

\begin{lstlisting}[style=styleJavaScript]
// CHECK: my_func1 = ([[A0:[a-z]+0]],
// [[A1:[a-z]+1]], [[A2:[a-z]+2]])
const my_func1 = (input1, input2, input3) => {
	// CHECK-NEXT: => {
\end{lstlisting}

与原来的\texttt{CHECK}指令相比,\texttt{CHECK-NEXT}指令不仅检查模式是否存在,还确保模式在前一个指令的行后面。

\item 接下来,在\texttt{my\_func1}中检查所有的局部变量和形参:

\begin{lstlisting}[style=styleJavaScript]
// CHECK: my_func1 = ([[A0:[a-z]+0]],
// [[A1:[a-z]+1]], [[A2:[a-z]+2]])
const my_func1 = (input1, input2, input3) => {
	// CHECK: let [[IM:[a-zA-Z]+]] = square([[A2]]);
	let intermediate = square(input3);
	// CHECK: let [[OUT:[a-zA-Z]+]] =
	// CHECK-SAME: [[A0]] + [[IM]] - [[A1]];
	let output = input1 + intermediate - input2;
	// CHECK: return [[OUT]];
	return output;
}
\end{lstlisting}

正如前面代码所示,\texttt{CHECK-SAME}指令用于在同一行中匹配后续的模式。这背后的原理是\texttt{FileCheck}期望不同的\texttt{CHECK}指令在不同的行中进行匹配。那么,假设这段代码的一部分是这样的:

\begin{lstlisting}[style=styleJavaScript]
// CHECK: let [[OUT:[a-zA-Z]+]] =
// CHECK: [[A0]] + [[IM]] - [[A1]];
\end{lstlisting}

它将只匹配跨越两行或更多的代码,如下所示:

\begin{lstlisting}[style=styleJavaScript]
let BGHr =
	r0 + jkF + r1;
\end{lstlisting}

否则将抛出一个错误。如果想避免编写超长的检查语句行,可以使测试脚本更加简洁,并且可读性更强,这个指令就非常有用。

\item 进入\texttt{my\_func2},现在检查文字数字是否正确混淆。这里的检查语句旨在接受除原始数字之外的任何实例/模式。因此,这里使用\texttt{CHECK-NOT}指令就足够了:

\begin{lstlisting}[style=styleJavaScript]
…
// CHECK: return my_func1(
// CHECK-NOT: 94
return my_func1(94,
				term2, factor2);
\end{lstlisting}

\begin{tcolorbox}[colback=blue!5!white,colframe=blue!75!black, fonttitle=\bfseries,title=Note]
\hspace*{0.7cm}因为\texttt{CHECK-NOT}在返回\texttt{my\_func1(94},所以第一个\texttt{CHECK}指令是必需的。这里,\texttt{CHECK-NOT}将给出假阴性结果,而没有\texttt{CHECK}指令将光标移动到正确的行。
\end{tcolorbox}

此外,\texttt{CHECK-NOT}在表示\texttt{不<特定模式>…但<正确模式>}时,与\texttt{CHECK-SAME}一起使用会非常好用。

例如,如果混淆规则声明所有的文字数字都需要混淆成十六进制数,那可以用下面的代码来表达\textit{“不想看到94…但是想看到0x5E或0x5e”}的断言:

\begin{lstlisting}[style=styleJavaScript]
…
// CHECK: return my_func1
// CHECK-NOT: 94,
// CHECK-SAME: {{0x5[eE]}}
return my_func1(94,
				term2, factor2);
\end{lstlisting}

\item 现在,只需要验证一个混淆规则:\texttt{js-obfuscator}工具提供了命令行选项\texttt{-\,-shuffle-funcs},可以打乱所有顶级函数。这时,即使它们已经打乱了,也还需要检查顶级函数是否保持一定的顺序。在JavaScript中,函数在调用时解析。所以,\texttt{cube}、\texttt{square}、\texttt{my\_func1}和\texttt{my\_func2}可以有任意的顺序,只要放在\texttt{console.log(…)}语句之前。使用\texttt{CHECK-DAG}指令表达这种灵活性非常好用。

相邻的\texttt{CHECK-DAG}指令将以任意顺序匹配文本。例如,假设有以下指令:

\begin{lstlisting}[style=styleJavaScript]
// CHECK-DAG: 123
// CHECK-DAG: 456
\end{lstlisting}

这些指令将匹配以下内容:

\begin{lstlisting}[style=styleJavaScript]
123
456
\end{lstlisting}

还将匹配以下内容:

\begin{lstlisting}[style=styleJavaScript]
456
123
\end{lstlisting}

然而,这种排序自由在\texttt{CHECK}或\texttt{CHECK-NOT}指令中都不存在。例如,假设有这些指令:

\begin{lstlisting}[style=styleJavaScript]
// CHECK-DAG: 123
// CHECK-DAG: 456
// CHECK: 789
// CHECK-DAG: abc
// CHECK-DAG: def
\end{lstlisting}

这些指令将匹配以下文本:

\begin{lstlisting}[style=styleJavaScript]
456
123
789
def
abc
\end{lstlisting}

但是,不匹配以下文本:

\begin{lstlisting}[style=styleJavaScript]
456
789
123
def
abc
\end{lstlisting}

\item 回到我们的例子,可以使用下面的代码来检查混淆规则:

\begin{lstlisting}[style=styleJavaScript]
…
// CHECK-DAG: const square =
// CHECK-DAG: const cube =
// CHECK-DAG: const my_func1 =
// CHECK-DAG: const my_func2 =
// CHECK: console.log
console.log(my_func2(1,2));
\end{lstlisting}

但是,只有在向工具提供相应的命令行选项时,才会发生函数变换。不过,\texttt{FileCheck}提供了一种将多个不同检查套件集成到单个文件中的方法,其中每个套件可以自定义运行方式,并将检查与其他套件分离。

\item \texttt{FileCheck}中检查前缀的思想非常简单:可以创建一个独立运行的检查套件。前面提到的所有指令(\texttt{CHECK-NOT}和\texttt{CHECK-SAME})中,每个套件将用相应字符串替换,包括\texttt{CHECK}本身,以区别于同一文件中的其他套件。例如,可以创建一个带有\texttt{YOL}O前缀的套件,这样这个例子(部分)现在看起来如下所示:

\begin{lstlisting}[style=styleJavaScript]
// YOLO: my_func2 = ([[A0:[a-z]+0]], [[A1:[a-z]+1]])
const my_func2 = (factor1, factor2) => {
…
// YOLO-NOT: return my_func1(94,
// YOLO-SAME: return my_func1({{0x5[eE]}},
return my_func1(94,
				term2, factor2);
…
\end{lstlisting}

要使用自定义前缀,需要在\texttt{-\,-check-prefix}选项中指定。这里,像这样使用\texttt{FileCheck}命令:

\begin{tcblisting}{commandshell={}}
$ cat test.out.js | FileCheck --check-prefix=YOLO test.js
\end{tcblisting}

\item 最后,回到我们的例子。最后一个混淆规则可以通过\texttt{CHECK-DAG}指令使用另一个前缀来解决:

\begin{lstlisting}[style=styleJavaScript]
…
// CHECK-SHUFFLE-DAG: const square =
// CHECK-SHUFFLE-DAG: const cube =
// CHECK-SHUFFLE-DAG: const my_func1 =
// CHECK-SHUFFLE-DAG: const my_func2 =
// CHECK-SHUFFLE: console.log
console.log(my_func2(1,2));
\end{lstlisting}

\end{enumerate}

这必须与默认的检查套件相结合。本节中提到的所有检查可以在两个独立的命令中运行,如下所示:

\begin{tcblisting}{commandshell={}}
# Running the default check suite
$ js-obfuscator test.js | FileCheck test.js
# Running check suite for the function shuffling option
$ js-obfuscator --shuffle-funcs test.js | \
    FileCheck --check-prefix=CHECK-SHUFFLE test.js
\end{tcblisting}

本节中,我们通过示例项目展示了一些高级的\texttt{FileCheck}技能。这些技能提供了不同的方法来编写验证模式,并使LIT测试脚本更简洁。

目前为止,我们一直在讨论测试方法,以及在类Shell环境(即以\texttt{ShTest} LIT格式)中运行测试。下一节中,我们将介绍替代LIT的框架——\texttt{llvm-test-suite}项目的\textit{TestSuite}框架和测试格式——提供了一种与LIT\textit{不同的}的测试方法。







