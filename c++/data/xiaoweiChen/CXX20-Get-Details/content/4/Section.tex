恭喜你!当你阅读这些文字时,已经掌握了具有挑战性和令人激动的C++20标准。C++20是一个C++标准,其他C++标准一样具有影响力,例如:C++98和C++11。由于C++11,C++社区使用了以下C++标准的名称。

\begin{itemize}
\item 
旧C++: C++98和C++03

\item 
现代C++ : C++11, C++14和C++17

\item 
<占位符>: C++20
\end{itemize}
	
我不确定将来C++20会用什么名字。我只确定C++20开创了新的C++领域,特别是四大特性改变了我们使用C++的方式。

\begin{itemize}
\item 
概念:彻底改变了我们思考和编写泛型代码的方式。从而,可以第一次在语义类别(如数字或顺序)中对我们的程序进行推理。

\item 
模块:模块是软件组件的起点,有助于克服遗留头文件和宏的不足。

\item 
范围:范围库用功能思维扩展了标准模板库。算法可以直接在容器上操作,可以延迟计算,也可以进行组合。

\item 
协程:使得异步编程成为C++中的一等公民。协程转换等待中的阻塞函数调用,在模拟、服务器或用户界面等事件驱动系统中非常有用。
\end{itemize}

C++20只是一个起点。C++23中,还有很多工作要做,以充分集成和利用C++中四大特性的潜力。这里,我来说一些对C++未来的想法。

\begin{itemize}
\item 
标准模板库是由\href{https://en.wikipedia.org/wiki/Alexander_Stepanov}{Alexander Stephanov}设计的。不过,C++20中缺少对概念的集成。

\item 
可以期待一个模块化的标准模板库,并希望C++有一个打包系统。

\item 
许多函数式编程中已知的算法在范围库中仍然没有,未来的C++标准应该改进范围算法和标准容器之间的互动。

\item 
还没有真正“好用的”协程,目前只有一个用来构建协程强大框架,而协程库很有可能在C++23中出现。
\end{itemize}

在关于C++23及以后标准的章节中,我给出了关于C++不久的将来的更多细节。简而言之:C++具有光明的未来。






