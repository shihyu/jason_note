两个指针可以指向同一个内存单元,它们互为别名。在LLVM模型中没有输入内存,这使得优化器很难决定两个指针是否为别名。如果编译器能够证明两个指针没有别名,那么就有进行更多优化的可能。下一节中,在实现这种方法之前,我们将更仔细地研究这个问题,并研究添加的元数据将如何发挥作用。\par

\hspace*{\fill} \par %插入空行
\textbf{理解添加元数据的需求}

为了演示这个问题,让我们看看下面的函数:\par

\begin{lstlisting}[caption={}]
void doSomething(int *p, float *q) {
	*p = 42;
	*q = 3.1425;
}
\end{lstlisting}

优化器不能确定p和q指针是否指向相同的内存单元。优化过程中,这是一个重要的分析,称为别名分析。如果p和q指向同一个存储单元,那么它们就是别名。如果优化器能够证明两个指针从来没有别名,这将提供优化机会,例如:在soSomething()函数中,可以在不改变结果的情况下重新排序存储。\par

一种类型的变量是否可以是另一种类型的变量的别名,这取决于语言的定义。请注意,语言也可能包含打破基于类型的别名假设的表达式——例如,不相关类型之间的类型转换。\par

LLVM开发人员选择的解决方案是添加元数据来加载和存储指令。元数据有两个用途:\par

\begin{itemize}
\item 首先,根据可以别名其他类型的类型层次结构定义类型层次结构
\item 其次,描述了加载或存储指令中的内存访问
\end{itemize}

让我们看看C中的类型层次结构。每种类型的层次结构都以一个根节点开始,根节点可以\textbf{命名},也可以\textbf{匿名}。LLVM假设具有相同名称的根节点描述相同类型的层次结构。您可以在同一个LLVM模块中使用不同的类型层次结构,并且LLVM做了一个安全的假设,即这些类型可以别名。在根节点之下,有用于标量类型的节点。聚合类型的节点不附加到根节点,但它们引用标量类型和其他聚合类型。Clang为C定义了如下层次结构:\par

\begin{itemize}
\item 根节点称为Simple C/C++ TBAA
\item 根节点下面是用于字符类型的节点。这是C语言中的一种特殊类型,因为所有指针都可以转换为char类型的指针。
\item 在char节点下面是用于其他标量类型的节点和用于所有指针的类型。
\end{itemize}

聚合类型定义为成员类型和偏移量的序列。\par

这些元数据定义用于加载和存储指令的访问标记。访问标记由三部分组成:基类型、访问类型和偏移量。根据基类型的不同,访问标签描述内存访问有两种可能的方式:\par

\begin{enumerate}
\item 如果基类型是聚合类型,则访问标记描述结构成员的内存访问,具有访问类型并位于给定偏移量。
\item 如果基类型是标量类型,则访问类型必须与基类型相同,并且偏移量必须为0。
\end{enumerate}

有了这些定义,我们现在可以在访问标记上定义一个关系,该关系用于计算两个指针是否可以别名。元组(基类型,偏移量)的直接父类由基类型和偏移量决定:\par

\begin{itemize}
\item 如果基类型是标量类型并且偏移量为0,则直接父类型是(父类型,0),父类型是类型层次结构中定义的父节点的类型。如果偏移量不为0,则直接父节点未定义。
\item 基类型是一个聚合类型,然后tuple(基类型,偏移量)的直接父类型是tuple(新类型,新偏移量),而新类型是偏移量处成员的类型。新偏移量会根据新开始调整的新类型的起始位置。
\end{itemize}

这个关系的传递闭包是父关系。双内存访问类型——例如,(基类型1,访问类型1,偏移量1)和(基类型2,访问类型2,偏移量2)——可以别名为(基类型1,偏移量1)和(基类型2,偏移量2),反之亦然。这其实与父关系相关。\par

用一个例子来说明:\par

\begin{lstlisting}[caption={}]
struct Point { float x, y; }
void func(struct Point *p, float *x, int *i, char *c) {
	p->x = 0; p->y = 0; *x = 0.0; *i = 0; *c = 0; 
}
\end{lstlisting}

使用前面对标量类型的内存访问标记定义,参数i的访问标记是(int, int, 0),参数c的访问标记是(char, char, 0)。在类型层次结构中,int类型的节点的父节点是char节点,因此(int, 0)的直接父节点是(char, 0),两个指针都可以别名。参数x和参数c也是一样的,但是参数x和i不相关,因此它们不会互为别名。struct Point的y成员的访问是(Point、float、4),4是结构中y成员的偏移量。(Point, 4)的直接父对象是(float, 0),因此对p->y和x的访问可能会别名,而且出于同样的原因,参数c也是如此。\par

要创建元数据,我们使用llvm::MDBuilder类,在llvm/IR/MDBuilder.h头文件中声明。数据本身存储在llvm::MDNode和llvm::MDString类的实例中。使用构建器类可以让我们避开构造的内部细节。\par

通过调用createTBAARoot()方法创建根节点,该方法将类型层次结构的名称作为参数并返回根节点。可以使用createAnonymousTBAARoot()方法可以创建一个匿名的根节点。\par

使用createTBAAScalarTypeNode()方法将标量类型添加到层次结构中,该方法以类型的名称和父节点作为参数。为聚合类型添加类型节点稍微复杂一些,createTBAAStructTypeNode()方法以类型名称和字段列表作为参数。指定为std::pair<llvm::mdnode*, uint64\underline{~}t> instance。第一个元素表示成员的类型,第二个元素表示结构类型中的偏移量。\par

使用createTBAAStructTagNode()方法创建访问标记,该方法接受基类型、访问类型和偏移量作为参数。\par

最后,元数据必须附加到加载或存储指令。指令类有一个setMetadata()方法,用于添加各种元数据。第一个参数必须是llvm::LLVMContext::MD\underline{~}tbaa,第二个参数必须是访问标记。\par

有了这些知识,我们将在下一节中为tinylang添加基于类型的别名分析(TBAA)的元数据。\par

\hspace*{\fill} \par %插入空行
\textbf{向tinylang添加TBAA元数据}

为了支持TBAA,需要添加了一个新的CGTBAA类,该类负责生成元数据节点。我们让它成为CGModule类的成员,称它为TBAA。每个加载和存储指令都可能注释,为此我们在CGModule类中也放置了一个新函数。该函数试图创建标记访问信息。如果成功,则将元数据附加到指令。这种设计还允许我们在不需要元数据时关闭元数据生成——例如,在关闭了优化的构建中:\par

\begin{lstlisting}[caption={}]
void CGModule::decorateInst(llvm::Instruction *Inst,
							TypeDenoter *TyDe) {
	if (auto *N = TBAA.getAccessTagInfo(TyDe))
		Inst->setMetadata(llvm::LLVMContext::MD_tbaa, N);
}
\end{lstlisting}

我们将新CGTBAA类的声明放入include/tinylang/CodeGen/CGTBAA.h头文件中,并将定义放入lib/CodeGen/CGTBAA.cpp文件中。除了抽象语法树(AST)定义之外,头文件还需要包含定义元数据节点和构建器的文件:\par

\begin{lstlisting}[caption={}]
#include "tinylang/AST/AST.h"
#include "llvm/IR/MDBuilder.h"
#include "llvm/IR/Metadata.h"
\end{lstlisting}

CGTBAA类需要存储一些数据成员,看看如何一步步地做到这些:\par

\begin{enumerate}
\item 首先,需要缓存类型层次结构的根:
\begin{lstlisting}[caption={}]
class CGTBAA {
	llvm::MDNode *Root;
\end{lstlisting}

\item 为了构造元数据节点,需要MDBuilder类的实例:
\begin{lstlisting}[caption={}]
 	llvm::MDBuilder MDHelper;
\end{lstlisting}

\item 最后,存储为类型生成的元数据以供重用:
\begin{lstlisting}[caption={}]
	llvm::DenseMap<TypeDenoter *, llvm::MDNode *> 
		MetadataCache;
// …
};
\end{lstlisting}
\end{enumerate}

在定义了构造所需的变量之后,现在添加创建元数据所需的方法:\par


\begin{enumerate}
\item 构造函数初始化数据成员:
\begin{lstlisting}[caption={}]
CGTBAA::CGTBAA(llvm::LLVMContext &Ctx)
	  : MDHelper(llvm::MDBuilder(Ctx)), Root(nullptr) {}
\end{lstlisting}

\item 惰性地实例化类型层次结构的根,将其命名为Simple tinylang TBAA:
\begin{lstlisting}[caption={}]
llvm::MDNode *CGTBAA::getRoot() {
	if (!Root)
		Root = MDHelper.createTBAARoot("Simple tinylang 
										TBAA");
	return Root;
}
\end{lstlisting}

\item 对于标量类型,在MDBuilder类的帮助下根据类型的名称创建元数据节点。新的元数据节点存储在缓存中:
\begin{lstlisting}[caption={}]
llvm::MDNode *
CGTBAA::createScalarTypeNode(TypeDeclaration *Ty,
							 StringRef Name,
							 llvm::MDNode *Parent) {
	llvm::MDNode *N =
		MDHelper.createTBAAScalarTypeNode(Name, Parent);
	return MetadataCache[Ty] = N;
}
\end{lstlisting}

\item 为记录创建元数据的方法更为复杂,必须枚举记录的所有字段:
\begin{lstlisting}[caption={}]
llvm::MDNode *CGTBAA::createStructTypeNode(
	TypeDeclaration *Ty, StringRef Name,
	llvm::ArrayRef<std::pair<llvm::MDNode *, 
		uint64_t>>
		Fields) {
	llvm::MDNode *N =
		MDHelper.createTBAAStructTypeNode(Name, Fields);
	return MetadataCache[Ty] = N;
}
\end{lstlisting}

\item 要返回tinylang类型的元数据,需要创建类型层次结构。由于tinylang的类型系统非常有限,可以使用一种简单的方法。每个标量类型映射到附加到根节点的唯一类型,并将所有指针映射到单个类型。结构化类型然后引用这些节点。如果不能映射类型,则返回nullptr:
\begin{lstlisting}[caption={}]
llvm::MDNode *CGTBAA::getTypeInfo(TypeDeclaration *Ty) {
	if (llvm::MDNode *N = MetadataCache[Ty])
		return N;
	
	if (auto *Pervasive =
			llvm::dyn_cast<PervasiveTypeDeclaration>(Ty)) {
		StringRef Name = Pervasive->getName();
		return createScalarTypeNode(Pervasive, Name, 
			getRoot());
	}
	if (auto *Pointer =
			llvm::dyn_cast<PointerTypeDeclaration>(Ty)) {
		StringRef Name = "any pointer";
		return createScalarTypeNode(Pointer, Name, 
			getRoot());
	}
	if (auto *Record =
			llvm::dyn_cast<RecordTypeDeclaration>(Ty)) {
		llvm::SmallVector<std::pair<llvm::MDNode *, 
			uint64_t>,
					4>
			Fields;
		auto *Rec =
			llvm::cast<llvm::StructType>(
				CGM.convertType(Record));
		const llvm::StructLayout *Layout =
			CGM.getModule()->getDataLayout()
				.getStructLayout(Rec);
			
		unsigned Idx = 0;
		for (const auto &F : Record->getFields()) {
			uint64_t Offset = Layout->getElementOffset(Idx);
			Fields.emplace_back(getTypeInfo(F.getType()), 
				Offset);
			++Idx;
		}
		StringRef Name = CGM.mangleName(Record);
		return createStructTypeNode(Record, Name, Fields);
	}
	return nullptr;
}
\end{lstlisting}

\item 获取元数据的一般方法是getAccessTagInfo()。因为只需要查找指针类型,所以我们检查它。否则,返回nullptr:
\begin{lstlisting}[caption={}]
llvm::MDNode *CGTBAA::getAccessTagInfo(TypeDenoter *TyDe) 
{
	if (auto *Pointer = llvm::dyn_cast<PointerType>(TyDe)) 
	{
		return getTypeInfo(Pointer->getTyDen());
	}
	return nullptr;
}
\end{lstlisting}
\end{enumerate}

要启用TBAA元数据的生成,只需要将元数据附加到生成的加载和存储指令。例如,在CGProced\allowbreak ure::writeVariable()中,对全局变量进行存储,使用存储指令:\par

\begin{lstlisting}[caption={}]
Builder.CreateStore(Val, CGM.getGlobal(D));
\end{lstlisting}

为了修饰指令,需要将前面的行替换为以下行:\par

\begin{lstlisting}[caption={}]
auto *Inst = Builder.CreateStore(Val,
								 CGM.getGlobal(Decl));
CGM.decorateInst(Inst, V->getTypeDenoter());
\end{lstlisting}

有了这些更改,就完成了TBAA元数据的生成。\par

下一节中,我们将讨论一个非常类似的主题:对元数据的生成进行调试。\par








