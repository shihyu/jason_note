
未来是属于新的Pass管理器,专为旧Pass管理器开发一个新的Pass没有什么意义。然而,在正在进行的转换阶段,如果Pass可以与两个Pass管理器一起工作,就很有用,因为LLVM中的大多数Pass已经这样做了。\par

旧Pass管理器需要从某些基类派生的Pass,例如:Pass函数必须派生自FunctionPass基类。还有更多的差异,Pass管理器运行的方法名为runOnFunction(),还必须提供Pass的ID。在这里遵循的策略是创建一个单独的类,可以在旧Pass管理器中使用,并以一种可以在两个Pass管理器中使用的功能的方式重构源码。\par

我们使用Pass插件作为基础。在include/CountIR.h头文件中,我们添加了一个新的类定义,如下所示:\par

\begin{enumerate}
\item 这个新类需要从FunctionPass类派生,所以包含了一个额外的头文件来获取类定义:
\begin{lstlisting}[caption={}]
#include "llvm/Pass.h"
\end{lstlisting}
	
\item 我们将这个新类命名为CountIRLegacyPass。类需要一个ID作为内部LLVM机制,用它初始化父类:
\begin{lstlisting}[caption={}]
class CountIRLegacyPass : public llvm::FunctionPass {
public:
	static char ID;
	CountIRLegacyPass() : llvm::FunctionPass(ID) {}
\end{lstlisting}
		
\item 为了实现Pass功能,必须重写两个函数。每个LLVM IR函数都会调用runOnFunction()方法,并实现计数功能。getAnalysisUsage()方法用来声明所有的分析结果都保存了:
\begin{lstlisting}[caption={}]
	bool runOnFunction(llvm::Function &F) override;
	void getAnalysisUsage(llvm::AnalysisUsage &AU) const 
	override;
};
\end{lstlisting}
	
\item 对头文件的更改完成后,可以增强lib/CountIR.cpp文件中的实现。为了重用计数功能,我们将源代码移动到一个新函数中:
\begin{lstlisting}[caption={}]
void runCounting(Function &F) {
	for (BasicBlock &BB : F) {
		++NumOfBB;
		for (Instruction &I : BB) {
			(void)I;
			++NumOfInst;
		}
	}
}
\end{lstlisting}
	
\item 为了使用新函数,新Pass管理器的方法需要更新:
\begin{lstlisting}[caption={}]
PreservedAnalyses
CountIRPass::run(Function &F, FunctionAnalysisManager 
&AM) {
	runCounting(F);
	return PreservedAnalyses::all();
}
\end{lstlisting}
	
\item 以同样的方式,为旧Pass管理器实现方法。使用false返回值,表示IR没有改变:
\begin{lstlisting}[caption={}]
bool CountIRLegacyPass::runOnFunction(Function &F) {
	runCounting(F);
	return false;
}
\end{lstlisting}
	
\item 为了保留现有的分析结果,必须以以下方式实现getAnalysisUsage()方法。这类似于新Pass管理器中的PreservedAnalyses::all()返回值。如果不实现这个方法,那么默认情况下所有的分析结果都会丢弃:
\begin{lstlisting}[caption={}]
void CountIRLegacyPass::getAnalysisUsage(
AnalysisUsage &AU) const {
	AU.setPreservesAll();
}
\end{lstlisting}
	
\item ID字段可以用任意值初始化,因为LLVM使用该字段的地址。公共值是0,所以可以直接使用:
\begin{lstlisting}[caption={}]
char CountIRLegacyPass::ID = 0;
\end{lstlisting}
	
\item 现在只有Pass注册没有了。要注册新Pass,需要提供RegisterPass<>模板的静态实例。第一个参数是调用新Pass的命令行选项的名称。第二个参数是Pass的名称,是在使用-help选项时为用户提供信息使用。
\begin{lstlisting}[caption={}]
static RegisterPass<CountIRLegacyPass>
X("countir", "CountIR Pass");
\end{lstlisting}
	
\item 这些更改足以让我们在旧Pass管理器和新Pass管理器下使用新的Pass。要测试添加的内容,请回到构建文件夹并编译Pass:
\begin{tcolorbox}[colback=white,colframe=black]
\$ ninja
\end{tcolorbox}
	
\item 为了在旧Pass管理器中加载插件,需要使用\verb|--|load选项。新Pass是可以通过\verb|--|countir选项调用:
\begin{tcolorbox}[colback=white,colframe=black]
\$ opt \verb|--|load-pass-plugin=lib/CountIR.so \verb|--|countir \verb|–-|stats$\setminus$ \\
\hspace*{0.5cm}\verb|--|disable-output demo.ll
\end{tcolorbox}
	
\begin{tcolorbox}[colback=blue!5!white,colframe=blue!75!black, title=Tip]
还请在前一节的命令行中检查,使用新Pass管理器调用我们的Pass仍然可以正常工作!
\end{tcolorbox}
	
\end{enumerate}

能够使用llvm提供的工具运行新Pass非常香,但最终,我们希望在我们的编译器中运行它。在下一节中,我们将探讨如何设置优化和如何自定义流水。\par






