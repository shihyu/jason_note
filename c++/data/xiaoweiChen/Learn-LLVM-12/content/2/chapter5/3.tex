为了从AST生成SSA形式的IR编码,我们使用了一种称为\textbf{AST编码}的方法。基本思想是,对于每个基本块,我们存储这个基本块中局部变量的当前值。\par

虽然很简单,但仍然需要几个步骤。我们将首先介绍所需的数据结构,然后实现对基本块局部值的读写。然后,将处理在几个基本块中使用的值,并通过优化所创建的phi指令得出结论。\par

\hspace*{\fill} \par %插入空行
\textbf{定义保存值的数据结构}

使用struct BasicBlockDef来保存单个块的信息:\par

\begin{lstlisting}[caption={}]
struct BasicBlockDef {
llvm::DenseMap<Decl *, llvm::TrackingVH<llvm::Value>> Defs;
// ...
};
\end{lstlisting}

LLVM类LLVM::Value表示一个SSA形式的值,Value类的作用类似于计算结果上的标签。它通常通过IR指令创建一次,然后使用。在各种优化过程中可能会有更改。例如,如果优化器检测到值\%1和\%2总是相同的,那么可以将\%2的使用替换为\%1,这会改变标签,但不会改变计算。要注意这些变化,不能直接使用Value类。相反,我们需要一个值句柄,有不同功能的值句柄。要跟踪替换,要使用llvm::TrackingVH<>类。因此,Defs成员将AST的声明(变量或形式参数)映射到其当前值。现在我们需要为每个基本块存储这些信息:\par

\begin{lstlisting}[caption={}]
llvm::DenseMap<llvm::BasicBlock *, BasicBlockDef>
	CurrentDef;
\end{lstlisting}

有了这个数据结构,就能够处理局部值了。\par

\hspace*{\fill} \par %插入空行
\textbf{读写基本块的局部值}

为了将局部变量的当前值存储在一个基本块中,只需在映射中创建一个条目:\par

\begin{lstlisting}[caption={}]
void writeLocalVariable(llvm::BasicBlock *BB, Decl *Decl,
						llvm::Value *Val) {
	CurrentDef[BB].Defs[Decl] = Val;
}
\end{lstlisting}

变量值的查找稍微复杂一点,因为值可能不在基本块中。这种情况下,需要使用可能的递归搜索将搜索扩展到前面几个基本块:\par

\begin{lstlisting}[caption={}]
llvm::Value *
readLocalVariable(llvm::BasicBlock *BB, Decl *Decl) {
	auto Val = CurrentDef[BB].Defs.find(Decl);
	if (Val != CurrentDef[BB].Defs.end())
		return Val->second;
	return readLocalVariableRecursive(BB, Decl);
}
\end{lstlisting}

真正的工作是搜索前几个基本块,这将在下一节中实现。\par

\hspace*{\fill} \par %插入空行
\textbf{搜索前一个块以查找值}

如果我们正在查看的当前基本块只有一个前块,那么就在那里搜索变量的值。如果基本块有几个前块,那么需要搜索所有这些块中的值,并将结果合并。为了说明这种情况,可以查看上一节中带有WHILE语句条件的基本块。\par

这个基本块有两个前块—一个来自WHILE循环之前的语句,另一个来自WHILE循环主体的分支。条件中使用的变量应该有一个初始值,并且很可能在循环体中更改。所以,需要收集这些定义并从中创建一条phi指令。由WHILE语句创建的基本块包含一个循环。\par

因为我们递归地搜索前一个块,所以必须打破这个循环。为此,需要使用一个简单的技巧——插入一条空的phi指令,并将其记录为变量的当前值。如果在搜索中再次看到这个基本块,那么就会发现这个变量有值,我们会使用它,并且搜索到此为止。在我们收集了所有的值之后,必须对phi指令进行更新。\par

我们仍将面临一个问题。查找时,并不是所有基本块的前块都是已知的。这是怎么可能呢?看看WHILE语句的基本块的创建。首先生成循环条件的IR,但从body的末端返回到包含条件的基本块的分支只能在生成body的IR之后添加,因为这个基本块之前是不知道的。如果需要在条件中读取一个变量的值,就会卡住,因为不是所有的前一个变量都是已知的。\par

要解决这个问题,就必须再多做一点:\par

\begin{itemize}
\item 首先,给基本块附加一个标志。
	
\item 然后,如果知道基本块的所有前块,就定义一个基本块为“密封的”。基本块没有密封,我们需要查找这个基本块中尚未定义的变量值,然后插入一个空的phi指令。
	
\item 我们还需要记住这个标识。该块之后密封时,需要用实际值更新指令。IncompletePhis的map记录了需要更新的phi指令,Sealed标志表明基本块是否密封:
\begin{lstlisting}[caption={}]
llvm::DenseMap<llvm::PHINode *, Decl *>
	IncompletePhis;
unsigned Sealed : 1;
\end{lstlisting}

\item 然后,该方法的实现如下:
\begin{lstlisting}[caption={}]
llvm::Value *readLocalVariableRecursive(
								llvm::BasicBlock *BB,
								Decl *Decl) {
	llvm::Value *Val = nullptr;
	if (!CurrentDef[BB].Sealed) {
		llvm::PHINode *Phi = addEmptyPhi(BB, Decl);
		CurrentDef[BB].IncompletePhis[Phi] = Decl;
		Val = Phi;
	} else if (auto *PredBB = BB
							->getSinglePredecessor()) {
		Val = readLocalVariable(PredBB, Decl);
	} else {
		llvm::PHINode *Phi = addEmptyPhi(BB, Decl);
		Val = Phi;
		writeLocalVariable(BB, Decl, Val);
		addPhiOperands(BB, Decl, Phi);
	}
	writeLocalVariable(BB, Decl, Val);
	return Val;
}
\end{lstlisting}

\item addEmptyPhi()方法在基本块的开头插入一条空phi指令:
\begin{lstlisting}[caption={}]
llvm::PHINode *addEmptyPhi(llvm::BasicBlock *BB, Decl
*Decl) {
	return BB->empty()
		? llvm::PHINode::Create(mapType(Decl), 0,
			"", BB)
		: llvm::PHINode::Create(mapType(Decl), 0,
			"", &BB->front());
}
\end{lstlisting}

\item 为了将缺失的操作数加到phi指令中,我们首先搜索基本块的所有前任,然后将操作数对值和基本块加到phi指令中。然后,尝试优化指令:
\begin{lstlisting}[caption={}]
void addPhiOperands(llvm::BasicBlock *BB, Decl *Decl,
					llvm::PHINode *Phi) {
	for (auto I = llvm::pred_begin(BB),
			  E = llvm::pred_end(BB);
		 I != E; ++I) {
		Phi->addIncoming(readLocalVariable(*I, Decl), *I);
	}
	optimizePhi(Phi);
}
\end{lstlisting}

\end{itemize}

这种算法会产生不必要的phi指令。下一节将尝试实现一种优化方法。\par

\hspace*{\fill} \par %插入空行
\textbf{优化生成的phi指令}

我们如何优化phi指令,为什么要这样做?虽然SSA形式对许多优化是有利的,但算法通常无法理解phi指令,所以会阻碍优化。因此,我们生成的phi指令是越少越好:\par

\begin{itemize}
\item 如果指令只有一个操作数或所有操作数都有相同的值,则用这个值替换指令。如果指令没有操作数,则用特殊值Undef替换该指令。只有当指令有两个或多个不同的操作数时,才需要保留指令:
\begin{lstlisting}[caption={}]
void optimizePhi(llvm::PHINode *Phi) {
	llvm::Value *Same = nullptr;
	for (llvm::Value *V : Phi->incoming_values()) {
		if (V == Same || V == Phi)
			continue;
		if (Same && V != Same)
			return;
		Same = V;
	}
	if (Same == nullptr)
		Same = llvm::UndefValue::get(Phi->getType());
\end{lstlisting}

\item 删除一个phi指令可能会导致其他phi指令有优化的机会。我们搜索值在其他phi指令中的用法后,尝试优化这些指令:
\begin{lstlisting}[caption={}]
	llvm::SmallVector<llvm::PHINode *, 8> CandidatePhis;
	for (llvm::Use &U : Phi->uses()) {
		if (auto *P =
					llvm::dyn_cast<llvm::PHINode>(U.getUser()))
			CandidatePhis.push_back(P);
	}
	Phi->replaceAllUsesWith(Same);
	Phi->eraseFromParent();
	for (auto *P : CandidatePhis)
		optimizePhi(P);
}
\end{lstlisting}

\end{itemize}

如果需要,该算法还可以进一步改进。可以选择并记住两个不同的值,而不是迭代每个phi指令的值列表。优化函数中,可以检查这两个值是否仍然在phi指令的列表中。如果是,那么就知道没有什么需要优化的了。但即使没有这个优化,这个算法也运行得很快,所以我们现在不打算实现它。\par

我们快实现完了。只有密封基本块的操作还没有实现,我们将在下一节中继续实现。\par

\hspace*{\fill} \par %插入空行
\textbf{密封块}

只要我们知道一个块的所有前块,就可以密封该块。如果源语言只包含结构化语句,比如tinylang,那么很容易确定块可以被密封的位置。再次查看为WHILE语句生成的基本块。包含条件的基本块可以密封后,从分支体的末端添加,因为这是最后一个缺失的前块。要封住一个块,只需将缺少的操作数添加到不完整的phi指令中,并设置标志:\par

\begin{lstlisting}[caption={}]
void sealBlock(llvm::BasicBlock *BB) {
	for (auto PhiDecl : CurrentDef[BB].IncompletePhis) {
		addPhiOperands(BB, PhiDecl.second, PhiDecl.first);
	}
	CurrentDef[BB].IncompletePhis.clear();
	CurrentDef[BB].Sealed = true;
}
\end{lstlisting}

有了这些方法,就可以为表达式生成IR代码了。\par

\hspace*{\fill} \par %插入空行
\textbf{为表达式创建IR代码}

一般来说,你可以按照第3章编译器的结构来翻译表达式,唯一有难度的部分是如何访问变量。上一节介绍了局部变量,但还有其他类型的变量。让我们简单讨论一下需要做什么:\par

\begin{itemize}
\item 对于过程的局部变量,使用上一节中的readLocalVariable()和writeLocalVariable()。
\item 对于封闭过程中的局部变量,需要指向封闭过程框架的指针。后面的部分将对此进行处理。
\item 对于全局变量,生成加载和存储指令。
\item 对于形式参数,必须区分按值传递和按引用传递(tinylang中的VAR参数)。按值传递的参数被视为局部变量,按引用传递的参数将视为全局变量。
\end{itemize}

把所有这些放在一起,我们可以得到下面的代码,从而可以读取变量或形式参数:\par

\begin{lstlisting}[caption={}]
llvm::Value *CGProcedure::readVariable(llvm::BasicBlock
										*BB,
										Decl *D) {
	if (auto *V = llvm::dyn_cast<VariableDeclaration>(D)) {
		if (V->getEnclosingDecl() == Proc)
			return readLocalVariable(BB, D);
		else if (V->getEnclosingDecl() ==
				CGM.getModuleDeclaration()) {
			return Builder.CreateLoad(mapType(D),
									  CGM.getGlobal(D));
		} else
			llvm::report_fatal_error(
				"Nested procedures not yet supported");
	} else if (auto *FP =
					llvm::dyn_cast<FormalParameterDeclaration>(
					D)) {
		if (FP->isVar()) {
			return Builder.CreateLoad(
				mapType(FP)->getPointerElementType(),
				FormalParams[FP]);
		} else
		return readLocalVariable(BB, D);
	} else
		llvm::report_fatal_error("Unsupported declaration");
}
\end{lstlisting}

写入变量或形式参数是对称的,只需要将读的方法与写的方法交换,并使用store指令,而不是load指令。\par

接下来,在为函数生成IR代码时应用这些函数,我们将在接下来实现这些函数。\par

\hspace*{\fill} \par %插入空行
\textbf{生成函数的IR代码}

大多数IR代码将存在于函数中。IR代码中的函数类似于C中的函数,它指定了名称、参数类型、返回值和其他属性。要在不同的编译单元中调用函数,需要声明函数。这类似于C语言中的原型。如果向函数中添加基本块,那么就可以定义函数。我们将在下一节中完成所有这些工作,首先讨论关于符号名的可见性。\par

\hspace*{\fill} \par %插入空行
\textbf{使用连接和名称修饰的方式控制可见性}

函数(以及全局变量)有一个附加的链接样式。使用链接样式,我们定义了符号名的可见性,以及如果多个符号具有相同的名称应该发生什么。最基本的链接样式是私有和外部可访问的。具有私有链接的符号仅在当前编译单元中可见,而具有外部链接的符号则是全局可用的。\par

对于没有适当模块概念的语言,如C,这足够了。对于模块,我们需要做更多的工作。假设有一个名为Square的模块提供了一个Root()函数,而Cube模块也提供了一个Root()函数。如果函数是私有的,没有问题。该函数获取名称Root和私有链接。如果导出函数,情况就不同了,这样就可以在其他模块中调用它。仅使用函数名是不够的,因为函数名不唯一。\par

解决方案是调整名称,使其具有全局唯一性,这叫做命名修饰。如何做到这一点取决于语言的需求和特征。我们的例子中,基本思想是使用模块名和函数名的组合来创建一个全局唯一名。使用Square.Root作为名称看起来是一个简单的解决方案,但可能会导致汇编程序的问题,因为点可能有特殊的含义。我们不需要在名称组件之间使用分隔符,而是可以用名称组件的长度作为前缀:6Square4Root,从而获得类似的效果。这不是LLVM的合法标识符,但可以通过在整个名称前加上\underline{~}t (t代表tinylang): \underline{~}t6Square4Root来解决这个问题。通过这种方式,我们可以为导出的符号,并创建唯一的名称:\par

\begin{lstlisting}[caption={}]
std::string CGModule::mangleName(Decl *D) {
	std::string Mangled;
	llvm::SmallString<16> Tmp;
	while (D) {
		llvm::StringRef Name = D->getName();
		Tmp.clear();
		Tmp.append(llvm::itostr(Name.size()));
		Tmp.append(Name);
		Mangled.insert(0, Tmp.c_str());
		D = D->getEnclosingDecl();
	}
	Mangled.insert(0, "_t");
	return Mangled;
}
\end{lstlisting}

如果你的源语言支持类型重载,那么需要用类型名来扩展这个方案。例如,为了区分C++函数\textit{int root(int)}和\textit{double root(double)},形参的类型和返回值需要添加到函数名中。\par

您还需要考虑生成的名称的长度,因为一些链接器对长度进行了限制。在C++中使用嵌套的名称空间和类,修饰后的名称可能会相当长。所以,C++定义了一个压缩方案来避免一遍又一遍地重复命名组件。\par

接下来,看看如何处理参数类型。\par

\hspace*{\fill} \par %插入空行
\textbf{将类型从AST描述转换为LLVM类型}

函数的参数也需要考虑。首先,需要将源语言的类型映射到LLVM类型。因为tinylang目前只有两种类型:\par

\begin{lstlisting}[caption={}]
llvm::Type *convertType(TypeDeclaration *Ty) {
	if (Ty->getName() == "INTEGER")
		return Int64Ty;
	if (Ty->getName() == "BOOLEAN")
		return Int1Ty;
	llvm::report_fatal_error("Unsupported type");
}
\end{lstlisting}

Int64Ty、Int1Ty和后来的void都是类成员,它们持有LLVM类型i64、i1和void的类型表示。\par

对于通过引用传递的形式参数是不够的,该参数的LLVM类型为指针。我们将函数推广并考虑形式参数:\par

\begin{lstlisting}[caption={}]
llvm::Type *mapType(Decl *Decl) {
	if (auto *FP = llvm::
	dyn_cast<FormalParameterDeclaration>(
			Decl)) {
		llvm::Type *Ty = convertType(FP->getType());
		if (FP->isVar())
			Ty = Ty->getPointerTo();
		return Ty;
	}
	if (auto *V = llvm::dyn_cast<VariableDeclaration>(Decl))
		return convertType(V->getType());
	return convertType(llvm::cast<TypeDeclaration>(Decl));
}
\end{lstlisting}

有了这些助手函数,接下就来创建LLVM IR函数。\par

\hspace*{\fill} \par %插入空行
\textbf{创建LLVM IR函数}

要在LLVM IR中计算函数,需要一个函数类型,这类似于C中的原型。创建函数类型需要映射类型,然后调用工厂方法来创建函数类型:\par

\begin{lstlisting}[caption={}]
llvm::FunctionType *createFunctionType(
	ProcedureDeclaration *Proc) {
	llvm::Type *ResultTy = VoidTy;
	if (Proc->getRetType()) {
		ResultTy = mapType(Proc->getRetType());
	}
	auto FormalParams = Proc->getFormalParams();
	llvm::SmallVector<llvm::Type *, 8> ParamTypes;
	for (auto FP : FormalParams) {
		llvm::Type *Ty = mapType(FP);
		ParamTypes.push_back(Ty);
	}
	return llvm::FunctionType::get(ResultTy, ParamTypes,
									/* IsVarArgs */ false);
}
\end{lstlisting}

根据函数类型,我们还创建了LLVM函数,将函数类型与链接和修饰后的名称关联起来:\par

\begin{lstlisting}[caption={}]
llvm::Function *
createFunction(ProcedureDeclaration *Proc,
			   llvm::FunctionType *FTy) {
	llvm::Function *Fn = llvm::Function::Create(
		Fty, llvm::GlobalValue::ExternalLinkage,
		mangleName(Proc), getModule());
\end{lstlisting}

getModule()方法返回当前的LLVM模块,稍后我们将对其进行设置。\par

创建这个函数后,可以向它添加更多的信息。首先,可以给出参数的名称。这使得IR更具可读性。其次,可以向函数和参数添加属性,以指定一些特征。例如,可以对通过引用传递的参数进行这样的操作。\par

从LLVM层来看,这些参数是指针。但是从源语言设计来看,这些都是非常受限的指针。类似于C++中的引用,需要为VAR参数指定一个变量。因此,通过设计,我们知道这个指针永不为空,而且总是可解引用的,这意味着可以冒着保护故障的风险读取指向的值。所以,这个指针不能传递。特别地,没有指针的副本比函数调用的时间周期长。因此,我们说指针没有捕获。\par

AttributeBuilder类用于为形参构建属性集。要获得参数类型的存储大小,可以简单地查询数据结构:\par

\begin{lstlisting}[caption={}]
	size_t Idx = 0;
	for (auto I = Fn->arg_begin(), E = Fn->arg_end(); I != E;
	++I, ++Idx) {
		llvm::Argument *Arg = I;
		FormalParameterDeclaration *FP =
		Proc->getFormalParams()[Idx];
		if (FP->isVar()) {
			llvm::AttrBuilder Attr;
			llvm::TypeSize Sz =
				CGM.getModule()
					->getDataLayout().getTypeStoreSize(
						CGM.convertType(FP->getType()));
			Attr.addDereferenceableAttr(Sz);
			Attr.addAttribute(llvm::Attribute::NoCapture);
			Arg->addAttrs(Attr);
		}
	Arg->setName(FP->getName());
	}
	return Fn;
}
\end{lstlisting}

现在我们已经创建了IR功能。下一节中,我们将函数体的基本块添加到函数中。\par

\hspace*{\fill} \par %插入空行
\textbf{生成函数体}

我们几乎完成了函数的IR代码生成!我们只需要将各个部分组合在一起就可以生成一个函数,包括函数体:\par

\begin{itemize}
\item 给定tinylang中的过程声明,首先创建函数类型和函数:
\begin{lstlisting}[caption={}]
void run(ProcedureDeclaration *Proc) {
	this->Proc = Proc;
	Fty = createFunctionType(Proc);
	Fn = createFunction(Proc, Fty);
\end{lstlisting}

\item 接下来,创建函数的第一个基本块,并将其设置为当前块:
\begin{lstlisting}[caption={}]
	llvm::BasicBlock *BB = llvm::BasicBlock::Create(
		CGM.getLLVMCtx(), "entry", Fn);
	setCurr(BB);
\end{lstlisting}

\item 然后遍历所有形式参数。为了正确处理VAR参数,需要初始化FormalParams成员(在readVariable()中使用)。与局部变量不同,形式参数在第一个基本块中有一个值,所以这些值是已知的:
\begin{lstlisting}[caption={}]
	size_t Idx = 0;
	auto &Defs = CurrentDef[BB];
	for (auto I = Fn->arg_begin(), E = Fn->arg_end(); I !=
		 E; ++I, ++Idx) {
		llvm::Argument *Arg = I;
		FormalParameterDeclaration *FP = Proc->
			getParams()[Idx];
		FormalParams[FP] = Arg;
		Defs.Defs.insert(
			std::pair<Decl *, llvm::Value *>(FP, Arg));
	}
\end{lstlisting}

\item 按照这个设定,可以调用emit()方法来开始为语句生成IR代码:
\begin{lstlisting}[caption={}]
	auto Block = Proc->getStmts();
	emit(Proc->getStmts());
\end{lstlisting}

\item 生成IR代码后的最后一个块可能还没有密封,因此现在调用sealBlock()。tinylang中的过程可能有隐式的return,所以还要检查最后一个基本块是否有合适的终止符,如果没有,就添加一个:
\begin{lstlisting}[caption={}]
	sealBlock(Curr);
	if (!Curr->getTerminator()) {
		Builder.CreateRetVoid();
	}
}
\end{lstlisting}

\end{itemize}

这就完成了函数IR代码的生成。我们仍然需要创建LLVM模块,它会将所有IR代码保存在一起。\par



