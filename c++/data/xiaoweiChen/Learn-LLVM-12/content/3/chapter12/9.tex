目标描述中指令的定义允许构造解码器表,解码器表用于将目标文件反汇编成文本汇编程序。解码器表和解码器函数是由llvm-tblgen工具生成的。除了生成的代码外,我们只需要提供注册和初始化M88kDisassembler类的代码,以及一些解码寄存器和操作数的帮助函数。\par

我们将实现放在Disassembler/M88kDisassembler.cpp文件中。M88kDisassembler类的get\allowbreak Instruction()方法执行解码工作。它接受一个字节数组作为输入,并将下一条指令解码到MCInst类的实例中。类声明如下:\par

\begin{lstlisting}[caption={}]
using DecodeStatus = MCDisassembler::DecodeStatus;

namespace {
	
class M88kDisassembler : public MCDisassembler {
public:
	M88kDisassembler(const MCSubtargetInfo &STI, MCContext &Ctx)
		: MCDisassembler(STI, Ctx) {}
	~M88kDisassembler() override = default;
	
	DecodeStatus getInstruction(MCInst &instr, uint64_t &Size,
								ArrayRef<uint8_t> Bytes,
								uint64_t Address,
								raw_ostream &CStream) const
														override;
};
}
\end{lstlisting}

生成的类以非限定的方式引用DecodeStatus枚举,因此必须使名称可见。\par

为了初始化反汇编器,我们定义了一个工厂函数,只是实例化一个新对象:\par

\begin{lstlisting}[caption={}]
static MCDisassembler *
createM88kDisassembler(const Target &T,
						const MCSubtargetInfo &STI,
						MCContext &Ctx) {
	return new M88kDisassembler(STI, Ctx);
}
\end{lstlisting}

LLVMInitializeM88kDisassembler()函数中,将工厂函数注册到目标注册表中:\par

\begin{lstlisting}[caption={}]
extern "C" LLVM_EXTERNAL_VISIBILITY void
LLVMInitializeM88kDisassembler() {
	TargetRegistry::RegisterMCDisassembler(
		getTheM88kTarget(), createM88kDisassembler);
}
\end{lstlisting}

当LLVM核心库初始化时,会调用initializealldisassemers()函数或InitializeNativeTarget\allowbreak Disassembler()函数。\par

生成的解码器函数需要助手函数来解码寄存器和操作数,这些元素的编码通常涉及目标描述中没有表示的特殊情况,例如:两个指令之间的距离总是偶数,所以最小的位可以忽略,因为它总是零。\par

要解码寄存器,必须定义DecodeGPRRegisterClass()函数。32个寄存器用0到31之间的数字进行编码,我们可以使用静态GPRDecoderTable表来映射编码和生成的寄存器枚举之间的关系:\par

\begin{lstlisting}[caption={}]
static const uint16_t GPRDecoderTable[] = {
	M88k::R0, M88k::R1, M88k::R2, M88k::R3,
	M88k::R4, M88k::R5, M88k::R6, M88k::R7,
	M88k::R8, M88k::R9, M88k::R10, M88k::R11,
	M88k::R12, M88k::R13, M88k::R14, M88k::R15,
	M88k::R16, M88k::R17, M88k::R18, M88k::R19,
	M88k::R20, M88k::R21, M88k::R22, M88k::R23,
	M88k::R24, M88k::R25, M88k::R26, M88k::R27,
	M88k::R28, M88k::R29, M88k::R30, M88k::R31,
};

static DecodeStatus
DecodeGPRRegisterClass(MCInst &Inst, uint64_t RegNo,
						uint64_t Address,
						const void *Decoder) {
	if (RegNo > 31)
		return MCDisassembler::Fail;
		
	unsigned Register = GPRDecoderTable[RegNo];
	Inst.addOperand(MCOperand::createReg(Register));
	return MCDisassembler::Success;
}
\end{lstlisting}

所有其他所需的解码器函数遵循与DecodeGPRRegisterClass()函数相同的模式:\par

\begin{enumerate}
\item 检查要解码的值是否符合所需的大小限制。如果不是,则返回MCDisassembler::Fail。

\item 解码该值并将其添加到MCInst实例中。

\item 返回MCDisassembler::Success表示成功。

\end{enumerate}

然后,需要可以包括生成的解码器表和函数:\par

\begin{lstlisting}[caption={}]
#include "M88kGenDisassemblerTables.inc"
\end{lstlisting}

最后,可以定义getInstruction()方法。这个方法有两个结果值,解码的指令和指令的大小。如果字节数组太小,则必须将大小设置为0。这很重要,因为即使解码失败,size参数也会将指针推进到下一个内存位置。\par

在M88k架构的情况下,方法很简单,因为所有指令都是4个字节长。因此,从数组中提取4个字节后,可以调用生成的解码器函数:\par

\begin{lstlisting}[caption={}]
DecodeStatus M88kDisassembler::getInstruction(
		MCInst &MI, uint64_t &Size, ArrayRef<uint8_t> Bytes,
		uint64_t Address, raw_ostream &CS) const {
	if (Bytes.size() < 4) {
		Size = 0;
		return MCDisassembler::Fail;
	}
	Size = 4;
	
	uint32_t Inst = 0;
	for (uint32_t I = 0; I < Size; ++I)
		Inst = (Inst << 8) | Bytes[I];
	return decodeInstruction(DecoderTableM88k32, MI, Inst,
							 Address, this, STI);
}
\end{lstlisting}

这就完成了反汇编程序的实现。\par

在实现了所有类之后,只需要设置构建系统来获取新的目标后端,我们将在下一节中添加它。\par





























