您的目标CPU可能有LLVM还不支持的机器指令。例如,使用MIPS架构的制造商经常向核心MIPS指令集添加特殊指令。RISC-V指令集的规范明确允许制造商添加新的指令,或者要添加一个全新的后端,然后必须添加CPU指令。在下一节中,我们将为LLVM后端添加对单个新机器指令的汇编器支持。\par

\hspace*{\fill} \par %插入空行
\textbf{向汇编程序和代码生成添加新指令}

新的机器指令通常与特定的CPU特性绑定在一起。只有当用户使用\verb|--|mattr=选项选择llc时,新指令才能识别。\par

作为一个例子,我们将添加一个新的机器指令到MIPS后端。新的机器指令首先将两个输入寄存器的值平方\$2和\$3,然后将两个平方和的和赋给输出寄存器\$1:\par

\begin{tcolorbox}[colback=white,colframe=black]
sqsumu \$1, \$2, \$3
\end{tcolorbox}

该指令的名称是sqsumu,由平方和求和运算派生而来。名称中的最后一个u表示该指令适用于无符号整数。\par

我们首先添加的CPU特性称为sqsum。这将允许使用\verb|--|mattr=+sqsum选项调用llc,以启用对新指令进行识别。\par

我们将添加的大部分代码都在描述MIPS后端的TableGen文件中。所有文件都位于llvm/lib/\allowbreak Target/Mips文件夹中。顶层文件为Mips.td,查看该文件并找到定义各种特性的部分。这里添加了我们新特性的定义:\par

\begin{tcolorbox}[colback=white,colframe=black]
def FeatureSQSum \\
\hspace*{1cm}: SubtargetFeature<"sqsum", "HasSQSum", "true", \\
\hspace*{6cm}"Use square-sum instruction">;
\end{tcolorbox}

SubtargetFeature类接受四个模板参数。第一个是sqsum,它是特性的名称,用于命令行。第二个参数HasSQSum是表示此特性的Subtarget类中的属性名称。下一个参数是默认值和特性的描述,用于在命令行上提供帮助。TableGen为MipsSubtarget类生成基类,其在MipsSubtarget.h文件中定义。这个文件中,类的私有部分添加了新属性,所有其他属性都在这里定义:\par

\begin{lstlisting}[caption={}]
// Has square-sum instruction.
bool HasSQSum = false;
\end{lstlisting}

在public部分,还使用了一个方法来检索属性的值:\par

\begin{lstlisting}[caption={}]
bool hasSQSum() const { return HasSQSum; }
\end{lstlisting}

通过这些那方法,我们已经能够在命令行上设置sqsum特性,尽管目前还没有效果。\par

要将新指令绑定到sqsum特性,需要定义一个谓词来指示是否选择了该特性。我们将其添加到MipsInstrInfo.td文件中,或者在定义所有其他谓词的部分,亦或者简单地放在结尾:\par

\begin{tcolorbox}[colback=white,colframe=black]
def HasSQSum : Predicate<"Subtarget->hasSQSum()">, \\
\hspace*{6cm}AssemblerPredicate<(all\underline{~}of FeatureSQSum)>;
\end{tcolorbox}

谓词使用前面定义的hasSQSum()方法。此外,AssemblerPredicate模板指定为汇编程序生成源代码时所使用的条件,只是简单地引用前面定义的特性。\par

我们还需要更新调度模型。MIPS目标同时使用行程和机器指令调度程序,对于路线模型,为MipsSchedule中的每条指令定义一个instritclass.td记录文件。在此文件中定义所有路线的部分中添加以下行:\par

\begin{tcolorbox}[colback=white,colframe=black]
def II\underline{~}SQSUMU : InstrItinClass;
\end{tcolorbox}

我们还需要提供说明费用的细节。通常,可以在CPU的文档中找到这些信息。对于我们的指令,乐观地假设它在ALU中只需要一个循环。该信息能添加到的MipsGenericItineraries定义中:\par

\begin{tcolorbox}[colback=white,colframe=black]
InstrItinData<II\underline{~}SQSUMU, [InstrStage<1, [ALU]>]>
\end{tcolorbox}

这样,对基于路线的调度模型的更新就完成了。MIPS目标还在MipsScheduleGeneric.td文件中,定义了基于MipsScheduleGeneric中的机器指令调度器模型的通用调度模型。因为这是一个包含所有指令的完整模型,所以还需要添加指令add。因为它是基于乘法的,所以我们简单地扩展了MULT和MULTu指令的定义:\par

\begin{tcolorbox}[colback=white,colframe=black]
def : InstRW<[GenericWriteMul], (instrs MULT, MULTu, SQSUMu)>;
\end{tcolorbox}

MIPS目标还在MipsScheduleP5600.td中定义了P5600 CPU的调度模型。这个目标显然不支持我们的新指令,所以把它添加到不支持的特性列表中:\par

\begin{tcolorbox}[colback=white,colframe=black]
list<Predicate> UnsupportedFeatures = [HasSQSum, HasMips3, … 
\end{tcolorbox}

现在,我们准备在Mips64InstrInfo.td文件的末尾添加新指令。TableGen的定义总是简洁的,因此需要对其进行剖析。该定义使用了来自MIPS目标描述的一些预定义类,我们的新指令是一个算术指令。通过设计,它适合于ArithLogicR类。第一个参数sqsumu指定指令的汇编助记符,下一个参数GPR64Opnd表示指令使用64位寄存器作为操作数,下面的1个参数表示操作数是可交换的。最后,给出了路线指令。添加\underline{~}FM类用于指定指令的二进制编码。对于真正的指令,必须根据文档选择参数。然后跟随ISA\underline{~}MIPS64谓词,该谓词指示指令对哪个指令集是有效的。最后,我们的SQSUM声明,只有在启用我们的特性时,指令才有效。完整的定义如下:\par

\begin{tcolorbox}[colback=white,colframe=black]
def SQSUMu : ArithLogicR<"sqsumu", GPR64Opnd, 1, II\underline{~}SQSUMU>, \\
\hspace*{4cm}ADD\underline{~}FM<0x1c, 0x28>, ISA\underline{~}MIPS64, SQSUM
\end{tcolorbox}

如果您的目标只是支持新指令,那么这个定义就够了,所以一定要完成的定义。通过添加选择DAG模式,代码生成器可以使用该指令。该指令使用两个操作数寄存器\$rs和\$rt以及目标寄存器\$rd,这三个寄存器都由ADD\underline{~}FM二进制格式类定义。理论上,要匹配的模式很简单:使用乘数运算符将每个寄存器的值平方,然后使用add运算符将两个乘积相加,并将它们赋给目标寄存器\$rd。这个模式变得有点复杂,因为使用MIPS指令集,乘法的结果存储在特殊的寄存器对中。为了便于使用,结果必须移到通用寄存器中。在操作的合法化过程中,通用mul操作符被MIPS特定的MipsMult操作所取代,用于乘法运算和MipsMFLO运算,以将结果的下一部分移动到通用寄存器中。在编写模式时,必须考虑到这一点,模式如下所示:\par

\begin{tcolorbox}[colback=white,colframe=black]
\{
\hspace*{0.5cm}let Pattern = [(set GPR64Opnd:\$rd, \\
\hspace*{4cm}(add (MipsMFLO (MipsMult \\
\hspace*{4.5cm}GPR64Opnd:\$rs, \\
\\
\hspace*{4.5cm}GPR64Opnd:\$rs)), \\
\hspace*{5.5cm}(MipsMFLO (MipsMult \\
\hspace*{6cm}GPR64Opnd:\$rt, \\
\\
\hspace*{6cm}GPR64Opnd:\$rt))) \\
\hspace*{4.5cm})];\\
\}
\end{tcolorbox}

正如在带选择DAG部分的指令选择中所描述的,如果该模式与当前DAG节点匹配,则选择我们的新指令。由于SQSUM谓词,这只在激活SQSUM特性时发生。我们用一个测试来检查它!\par

\hspace*{\fill} \par %插入空行
\textbf{测试新指令}

如果扩展了LLVM,那么最好使用自动化测试来验证它。如果想将您的扩展贡献给LLVM项目,那么就需要良好的测试。\par

在像上一节一样添加一个新的机器指令之后,我们必须进行两个检查:\par

\begin{itemize}
\item 首先,必须验证指令编码是否正确。
\item 其次,必须确保代码生成按照预期工作。
\end{itemize}

LLVM项目使用LIT(LLVM Integrated Tester)作为测试工具。基本上,测试用例是一个包含输入、要运行的命令和应该执行的检查的文件。添加新测试就像将一个新文件复制到测试目录中一样简单。为了验证新指令的编码,使用llvm-mc工具。除了其他任务外,该工具还可以显示指令的编码。对于临时检查,可以执行以下命令显示编码指令:\par

\begin{tcolorbox}[colback=white,colframe=black]
\$ echo "sqsumu $\setminus$\$1,$\setminus$\$2,$\setminus$\$3" | $\setminus$ \\
\hspace*{0.5cm}llvm-mc \verb|--|triple=mips64-linux-gnu -mattr=+sqsum $\setminus$ \\
\hspace*{2.5cm}--show-encoding
\end{tcolorbox}

这已经显示了要在自动化测试用例中运行的部分输入和命令。要验证结果,可以使用FileCheck工具,llvm-mc的输出通过管道传输到这个工具中。另外,FileCheck读取测试用例文件。测试用例文件包含用CHECK:关keyword标记的行,其后是预期的输出。FileCheck尝试将这些行与传入它的数据进行匹配。如果没有找到匹配,则显示一个错误。将包含测试用例文件sqsumu.s放到llvm/test/MC/Mips目录中:\par

\begin{tcolorbox}[colback=white,colframe=black]
\# RUN: llvm-mc \%s -triple=mips64-linux-gnu -mattr=+sqsum $\setminus$ \\
\# RUN: \verb|--|show-encoding | FileCheck \%s \\
\# CHECK: sqsumu \$1, \$2, \$3 \# encoding: [0x70,0x43,0x08,0x28] \\
\\
\hspace*{1cm}sqsumu \$1, \$2, \$3
\end{tcolorbox}

如果位于llvm/test/Mips/MC文件夹中,那么可以使用以下命令运行测试,并在最后报告成功:\par

\begin{tcolorbox}[colback=white,colframe=black]
\$ llvm-lit sqsumu.s \\
\verb|--| Testing: 1 tests, 1 workers \verb|--| \\
PASS: LLVM :: MC/Mips/sqsumu.s (1 of 1) \\
Testing Time: 0.11s \\
\hspace*{0.5cm}Passed: 1
\end{tcolorbox}

LIT工具解释RUN:line,用当前文件名替换\%s。FileCheck工具读取文件,并解析CHECK:line,尝试匹配来自流水的输入。这是一种非常有效的测试方法。\par

如果位于构建目录中,则可以使用以下命令调用LLVM测试:\par

要为代码生成构建一个测试用例,需要遵循相同的策略。sqsum.ll文件包含计算斜边平方的LLVM IR代码:\par

\begin{tcolorbox}[colback=white,colframe=black]
define i64 @hyposquare(i64 \%a, i64 \%b) \{ \\
\hspace*{0.5cm}\%asq = mul i64 \%a, \%a \\
\hspace*{0.5cm}\%bsq = mul i64 \%b, \%b \\
\hspace*{0.5cm}\%res = add i64 \%asq, \%bsq \\
\hspace*{0.5cm}ret i64 \%res \\
\}
\end{tcolorbox}

要查看生成的汇编代码,可以使用llc工具:\par

\begin{tcolorbox}[colback=white,colframe=black]
\$ llc –mtriple=mips64-linux-gnu –mattr=+sqsum < sqsum.ll
\end{tcolorbox}

确信在输出中看到了新的sqsum指令。如果删除-mattr=+sqsum选项,还请检查该指令是否未生成。\par

有了这些,就可以构建测试用例了。这一次,使用两个RUN:line:一行检查新指令是否生成,另一行检查是否没有生成。我们可以用一个测试用例文件做到这两点,因为可以告诉FileCheck工具寻找与“CHECK:”不同的标签。将测试文件sqsum.ll和以下内容放置到llvm/test/CodeGen/Mips文件夹:\par

\begin{tcolorbox}[colback=white,colframe=black]
; RUN: llc -mtriple=mips64-linux-gnu -mattr=+sqsum < \%s |$\setminus$ \\
; RUN: FileCheck -check-prefix=SQSUM \%s \\
; RUN: llc -mtriple=mips64-linux-gnu < \%s |$\setminus$ \\
; RUN: FileCheck --check-prefix=NOSQSUM \%s \\
\\
define i64 @hyposquare(i64 \%a, i64 \%b) \{ \\
; SQSUM-LABEL: hyposquare: \\
; SQSUM: sqsumu \$2, \$4, \$5 \\
; NOSQSUM-LABEL: hyposquare: \\
; NOSQSUM: dmult \$5, \$5 \\
; NOSQSUM: mflo \$1 \\
; NOSQSUM: dmult \$4, \$4 \\
; NOSQSUM: mflo \$2 \\
; NOSQSUM: addu \$2, \$2, \$1 \\
\hspace*{0.5cm}\%asq = mul i64 \%a, \%a \\
\hspace*{0.5cm}\%bsq = mul i64 \%b, \%b \\
\hspace*{0.5cm}\%res = add i64 \%asq, \%bsq \\
\hspace*{0.5cm}ret i64 \%res \\
\}
\end{tcolorbox}

与其他测试一样,可以使用以下命令在文件夹中单独运行测试:\par

\begin{tcolorbox}[colback=white,colframe=black]
\$ llvm-lit squm.ll
\end{tcolorbox}

或者,可以使用以下命令从构建目录运行:

\begin{tcolorbox}[colback=white,colframe=black]
\$  ninja check-llvm-mips-codegen
\end{tcolorbox}

通过这些步骤,可以使用新指令增强LLVM汇编程序,允许指令选择使用这个新指令,并验证编码是否正确,是否按照预期完成代码生成的工作。\par

































