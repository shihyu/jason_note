无论是编写新性能程序,还是优化现有的程序,在这之前需要了解代码当前的性能。衡量成功的标准是让程序的表现提高多少。这两种看法都说明了性能指标的存在,而且指标是可测量和可量化的。不过,没有一个定义能满足所有情况。要量化性能时,需要衡量的指标是什么,取决于具体问题。

但衡量标准不是简单地定义目标和确认成功。性能优化的每一步,无论是现有代码还是新代码,都应该有指标进行引导。

性能的第一条规则:\textit{永远不要猜测性能}。本章的第一部分是让所有人认同这条规则。为了打破对直觉的信任,我们必须学习使用如何使用测试性能的工具。

本章将讨论以下内容:

\begin{itemize}
\item 为什么性能测试是必要的
\item 为什么所有与性能相关的决策都必须由测试和数据驱动
\item 如何测试实际程序的性能
\item 什么是基准测试、数据分析和微基准测试,以及如何使用它们来衡量性能
\end{itemize}

