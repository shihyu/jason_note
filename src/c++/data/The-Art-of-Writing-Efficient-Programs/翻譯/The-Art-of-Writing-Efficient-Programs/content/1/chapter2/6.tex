这一章中,已经了解了整本书中最重要的概念,若不参照具体的衡量标准,谈论甚至思考性能都是没有意义的。剩下的大部分是动手环节,我们介绍了几种测量性能的方法。从整个程序开始,然后深入到每一行代码。

一个大型的高性能项目中,可以看到本章所学到的每一种工具和方法会多次使用。粗略测量——对整个程序或大部分程序进行基准测试和分析——指向需要进一步研究的代码区域。随后会进行更多轮的基准测试或收集更详细的数据。最终,确定需要优化的代码,然后问题就变成了,如何才能更快地完成这项工作?此时,可以使用微型基准测试或小型基准测试来测试优化代码。甚至可能会发现,您对这段代码的理解并不如自己所想的那么透彻,并且需要对其性能进行更详细的分析。同时,不要忘记可以对微基准测试结果进行分析!

最终,将得到性能关键代码的新版本,在小型基准测试中看起来是没问题的。但是,不要做任何假设!现在必须通过测试整个程序的性能,来确定所做的优化或增强是否有效。有时,这些测量将确认您对问题的理解,并验证解决方案。有时,会发现问题并不像想象的那样。优化本身虽然有益,但并没有对整个程序产生预期的效果(甚至会使事情变得更糟)。当有了一个新的数据点,可以比较新旧解决方案的数据,并在比较二者差异中寻找答案。

高性能程序的开发和优化从来不是线性的、循序渐进的过程。相反,它有许多迭代,从高级概述到低级数据分析,然后重复。这个过程中,直觉起着作用,只要确保测试和确认期望相符即可。因为涉及到性能时,没有什么是真正的显而易见。

下一章,将看到我们之前遇到的问题的解决方案,删除不必要的代码使程序变慢。为了做到这一点,我们必须了解如何有效地使用CPU以获得最大的性能,下一章将专门来讨论这个问题。