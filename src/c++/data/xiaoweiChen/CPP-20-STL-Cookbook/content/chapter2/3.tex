
结构化绑定可以很容易地将值解压缩到单独的变量中,从而提高代码的可读性。

使用结构化绑定,可以直接将成员值分配给变量:

\begin{lstlisting}[style=styleCXX]
things_pair<int,int> { 47, 9 };
auto [this, that] = things_pair;
cout << format("{} {}\n", this, that);
\end{lstlisting}

输出为:

\begin{tcblisting}{commandshell={}}
47 9
\end{tcblisting}

\subsubsection{How to do it…}

\begin{itemize}
\item 
结构化绑定使用pair、tuple、array和struct。C++20起,还会包括位域。下面的例子使用了C数组:

\begin{lstlisting}[style=styleCXX]
int nums[] { 1, 2, 3, 4, 5 };
auto [ a, b, c, d, e ] = nums;
cout << format("{} {} {} {} {}\n", a, b, c, d, e);
\end{lstlisting}

输出为:

\begin{tcblisting}{commandshell={}}
1 2 3 4 5
\end{tcblisting}

因为结构化绑定使用自动类型推断,所以类型必须是auto。各个变量的名称都在方括号内,[a, b, c, d, e]。

这个例子中,int型c数组nums包含五个值。使用结构化绑定将这五个值赋给变量(a、b、c、d和e)。

\item 
这也适用于STL数组对象:

\begin{lstlisting}[style=styleCXX]
array<int,5> nums { 1, 2, 3, 4, 5 };
auto [ a, b, c, d, e ] = nums;
cout << format("{} {} {} {} {}\n", a, b, c, d, e);
\end{lstlisting}

输出为:

\begin{tcblisting}{commandshell={}}
1 2 3 4 5
\end{tcblisting}

\item 
或者将它与tuple一起使用:

\begin{lstlisting}[style=styleCXX]
tuple<int, double, string> nums{ 1, 2.7, "three" };
auto [ a, b, c ] = nums;
cout << format("{} {} {}\n", a, b, c);
\end{lstlisting}

输出为:

\begin{tcblisting}{commandshell={}}
1 2.7 three
\end{tcblisting}

\item 
将它与结构体一起使用时,将按照定义的顺序接受变量:

\begin{lstlisting}[style=styleCXX]
struct Things { int i{}; double d{}; string s{}; };
Things nums{ 1, 2.7, "three" };
auto [ a, b, c ] = nums;
cout << format("{} {} {}\n", a, b, c);
\end{lstlisting}

输出为:

\begin{tcblisting}{commandshell={}}
1 2.7 three
\end{tcblisting}

\item 
可以使用带有结构化绑定的引用,可以修改绑定容器中的值,同时避免数据复制:

\begin{lstlisting}[style=styleCXX]
array<int,5> nums { 1, 2, 3, 4, 5 };
auto& [ a, b, c, d, e ] = nums;
cout << format("{} {}\n", nums[2], c);
c = 47;
cout << format("{} {}\n", nums[2], c);
\end{lstlisting}

输出为:

\begin{tcblisting}{commandshell={}}
3 3
47 47
\end{tcblisting}

因为变量绑定为引用,所以可以给c赋一个值,这也会改变数组中的值(nums[2])。

\item 
可以声明数组const来避免值的改变:

\begin{lstlisting}[style=styleCXX]
const array<int,5> nums { 1, 2, 3, 4, 5 };
auto& [ a, b, c, d, e ] = nums;
c = 47; // this is now an error
\end{lstlisting}

或者可以声明为const绑定来达到同样的效果,同时允许在其他地方更改数组,从而避免复制数据:

\begin{lstlisting}[style=styleCXX]
array<int,5> nums { 1, 2, 3, 4, 5 };
const auto& [ a, b, c, d, e ] = nums;
c = 47; // this is also an error
\end{lstlisting}

\end{itemize}

\subsubsection{How it works…}

结构化绑定使用auto类型推断将结构解压缩到变量中。其独立地确定每个值的类型,并为每个变量分配相应的类型。

\begin{itemize}
\item 
因为结构化绑定使用auto类型推断,所以不能为绑定指定类型,必须使用auto。若使用一个类型进行绑定,会得到相应的错误信息:

\begin{lstlisting}[style=styleCXX]
array<int,5> nums { 1, 2, 3, 4, 5 };
int [ a, b, c, d, e ] = nums;
\end{lstlisting}

输出为:

\begin{tcblisting}{commandshell={}}
error: structured binding declaration cannot have type 'int'
note: type must be cv-qualified 'auto' or reference to
cv-qualified 'auto'
\end{tcblisting}

当我尝试使用int与结构化绑定一起使用,上面就是来自GCC的报错。

\item 
对于函数的返回类型,通常使用结构化绑定:

\begin{lstlisting}[style=styleCXX]
struct div_result {
	long quo;
	long rem;
};

div_result int_div(const long & num, const long & denom)
{
	struct div_result r{};
	r.quo = num / denom;
	r.rem = num % denom;
	return r;
}

int main() {
	auto [quo, rem] = int_div(47, 5);
	cout << format("quotient: {}, remainder {}\n",
		quo, rem);
}
\end{lstlisting}

输出为:

\begin{tcblisting}{commandshell={}}
quotient: 9, remainder 2
\end{tcblisting}

\item 
因为map容器类为每个元素返回一个pair,所以使用结构化绑定来检索键/值对就很方便:

\begin{lstlisting}[style=styleCXX]
map<string, uint64_t> inhabitants {
	{ "humans", 7000000000 },
	{ "pokemon", 17863376 },
	{ "klingons", 24246291 },
	{ "cats", 1086881528 }
};

// I like commas
string make_commas(const uint64_t num) {
	string s{ std::to_string(num) };
	for(int l = s.length() - 3; l > 0; l -= 3) {
		s.insert(l, ",");
	}
	return s;
}

int main() {
	for(const auto & [creature, pop] : inhabitants) {
		cout << format("there are {} {}\n",
		make_commas(pop), creature);
	}
}
\end{lstlisting}

输出为:

\begin{tcblisting}{commandshell={}}
there are 1,086,881,528 cats
there are 7,000,000,000 humans
there are 24,246,291 klingons
there are 17,863,376 pokemon
\end{tcblisting}

使用结构化绑定来解包结构可以使代码更清晰、更容易维护。
\end{itemize}












