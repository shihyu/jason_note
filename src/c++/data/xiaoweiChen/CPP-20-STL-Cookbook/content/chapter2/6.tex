
if constexpr(condition)语句用于根据编译时条件执行代码的地方,条件可以是bool类型的constexpr表达式。

\subsubsection{How to do it…}

试想,有一个模板函数,需要根据模板形参的类型进行不同的操作。

\begin{lstlisting}[style=styleCXX]
template<typename T>
auto value_of(const T v) {
	if constexpr (std::is_pointer_v<T>) {
		return *v; // dereference the pointer
	} else {
		return v; // return the value
	}
}

int main() {
	int x{47};
	int* y{&x};
	cout << format("value is {}\n", value_of(x)); // value
	cout << format("value is {}\n", value_of(y)); // pointer
	return 0;
}
\end{lstlisting}

输出为:

\begin{tcblisting}{commandshell={}}
value is 47
value is 47
\end{tcblisting}

模板形参T的类型在编译时可用,constexpr if语句可以让代码轻松区分指针和值。

\subsubsection{How it works…}

constexpr if语句的工作方式与普通if语句类似,只不过其是在编译时求值的。运行时代码将不包含来自constexpr if语句的任何分支。考虑上面的分支语句:

\begin{lstlisting}[style=styleCXX]
if constexpr (std::is_pointer_v<T>) {
	return *v; // dereference the pointer
} else {
	return v; // return the value
}
\end{lstlisting}

is\_pointer\_v<T>会测试模板参数,该参数在运行时不可用。当模板形参<T>可用时,constexpr关键字会告诉编译器这个if语句需要在编译时求值。

这将使元编程更加容易,if constexpr语句可在C++17及更高版本中使用。












