
头文件在C语言出现之初就存在了。最初,主要用于文本替换宏和翻译单元之间的外部符号链接。随着模板的引入,C++利用头文件来放置实现的代码。因为模板需要重新编译以适应相应的特化,所以多年来我们一直在这样使用头文件。随着STL多年的发展,这些头文件的体积也在不断增长。目前这种情况已经难以处理,并且可扩展性越来越差。

头文件通常包含比模板更多的内容,通常包含系统所需的配置宏和其他符号。随着头文件数量的增加,符号冲突的机率也在增加。考虑到使用更多的宏时,问题就更大了,它们不受命名空间的限制,也不受其他形式的类型安全限制。

C++20中使用了模块解决了这个问题。

\subsubsection{How to do it…}

读者们可能习惯于创建这样的头文件:

\begin{lstlisting}[style=styleCXX]
#ifndef BW_MATH
#define BW_MATH
namespace bw {
	template<typename T>
	T add(T lhs, T rhs) {
		return lhs + rhs;
	}
}
#endif // BW_MATH
\end{lstlisting}

这个极简的例子说明了模块解决的几个问题。BW\_MATH符号用作包含守卫,唯一目的是防止头文件多次包含,但其符号贯穿整个翻译单元。当在源文件中包含这个头文件时,看起来像这样:

\begin{lstlisting}[style=styleCXX]
#include "bw-math.h"
#include <format>
#include <string>
#include <iostream>
\end{lstlisting}

现在BW\_MATH符号可用于包含的其他头文件,以及其他头文件包含的头文件等。这就会增大冲突的机率,并且编译器不能检查这些冲突。他们是宏,在编译器有看到它们前,就已经使用预处理器翻译了。

现在打开头文件看看,即对模板函数进行了解:

\begin{lstlisting}[style=styleCXX]
template<typename T>
T add(T lhs, T rhs) {
	return lhs + rhs;
}
\end{lstlisting}

因为是模板,每次使用add()时,编译器需要进行特化。模板函数每次调用时,都需要解析和特化。这就是为什么模板实现要放在头文件中的原因,源代码必须在编译时可见。随着STL的发展和壮大,现在已经有许多大型模板类和函数,这就成为了一个可扩展性的问题。

模块解决了这些问题,以及其他问题。

作为一个模块,bw-math.h变成了bw-math.ixx(MSVC的命名约定),内容如下:

\begin{lstlisting}[style=styleCXX]
export module bw_math;
export template<typename T>
T add(T lhs, T rhs) {
	return lhs + rhs;
}
\end{lstlisting}

注意,导出的符号是模块名bw\_math和函数名add()。这可使命名空间保持干净。

这种用法也更简洁。在module-test.cpp中使用时,可以这样:

\begin{lstlisting}[style=styleCXX]
import bw_math;
import std.core;

int main() {
	double f = add(1.23, 4.56);
	int i = add(7, 42);
	string s = add<string>("one ", "two");
	
	cout <<
		"double: " << f << "\n" <<
		"int: " << i << "\n" <<
		"string: " << s << "\n";
}
\end{lstlisting}

import声明用在可能使用\#include预处理器指令的地方。会从模块中导入符号表进行链接。

示例的输出如下所示:

\begin{tcblisting}{commandshell={}}
$ ./module-test
double: 5.79
int: 49
string: one two
\end{tcblisting}

模块版本的工作原理与头文件中相同,只是更干净、更高效。

\begin{tcolorbox}[colback=webgreen!5!white,colframe=webgreen!75!black,title=Note]
编译后的模块包含单独的元数据文件(module-name.ifc是MSVC中的命名约定),描述了模块接口,允许模块支持模板。元数据包含编译器创建模板特化所需的所有信息。
\end{tcolorbox}

\subsubsection{How it works…}

导入和导出声明是模块实现的核心,再来看看bw-math.ixx:

\begin{lstlisting}[style=styleCXX]
export module bw_math;
export template<typename T>
T add(T lhs, T rhs) {
	return lhs + rhs;
}
\end{lstlisting}

请注意这两个导出声明。第一个函数使用export module bw\_math导出模块本身,这将翻译单元声明为一个模块。每个模块文件的顶部,以及其他语句之前必须有一个模块声明。第二个导出使函数add()对模块使用者可用。

若模块需要\#include指令,或者其他全局代码段,需要一个简单的模块声明:

\begin{lstlisting}[style=styleCXX]
module;
#define SOME_MACRO 42
#include <stdlib.h>
export module bw_math;
...
\end{lstlisting}

module;声明在文件顶部的一行中单独引入了一个全局模块,只有预处理器指令可以出现在全局模块片段中。之后必须立即声明一个标准模块(export module bw\_math;)和其余的模块内容。来看看它是如何工作的:

\begin{itemize}
\item 
导出声明使一个符号对模块使用者可见,即导入模块的代码,符号默认为private。

\begin{lstlisting}[style=styleCXX]
export int a{7}; // visible to consumer
int b{42}; // not visible
\end{lstlisting}

\item 
可以导出一个代码块,像这样:

\begin{lstlisting}[style=styleCXX]
export {
	int a() { return 7; }; // visible
	int b() { return 42; }; // also visible
}
\end{lstlisting}

\item 
导出命名空间:

\begin{lstlisting}[style=styleCXX]
export namespace bw { // all of the bw namespace is
	visible
	template<typename T>
	T add(T lhs, T rhs) { // visible as bw::add()
		return lhs + rhs;
	}
}
\end{lstlisting}

\item 
或者,可以从命名空间中导出单独的符号:

\begin{lstlisting}[style=styleCXX]
namespace bw { // all of the bw namespace is visible
	export template<typename T>
	T add(T lhs, T rhs) { // visible as bw::add()
		return lhs + rhs;
	}
}
\end{lstlisting}

\item 
import声明在调用代码中导入一个模块:

\begin{lstlisting}[style=styleCXX]
import bw_math;
int main() {
	double f = bw::add(1.23, 4.56);
	int i = bw::add(7, 42);
	string s = bw::add<string>("one ", "two");
}
\end{lstlisting}

\item 
甚至可以导入一个模块,并将其导出传递给调用代码:

\begin{lstlisting}[style=styleCXX]
export module bw_math;
export import std.core;
\end{lstlisting}

export关键字必须在import关键字之前。

std.core模块现在可以在调用代码中使用:

\begin{lstlisting}[style=styleCXX]
import bw_math;
using std::cout, std::string, std::format;

int main() {
	double f = bw::add(1.23, 4.56);
	int i = bw::add(7, 42);
	string s = bw::add<string>("one ", "two");
	
	cout <<
		format("double {} \n", f) <<
		format("int {} \n", i) <<
		format("string {} \n", s);
}
\end{lstlisting}

\end{itemize}

模块是头文件的简化、直接的替代品,很多人都期待着模块的广泛使用。这样就可以大大减少了对头文件的依赖。

\begin{tcolorbox}[colback=webgreen!5!white,colframe=webgreen!75!black,title=Note]
撰写本文时,模块的唯一完整实现是在MSVC的预览版中。对于其他编译器,模块扩展名(.ixx)可能不同。此外,使用合并的std.core模块(MSVC版本中将STL作为模块实现的一部分),其他编译器可能不使用这个约定。在完全兼容的实现发布时,可能会发生变化。
\end{tcolorbox}

示例文件中,包含了基于格式的print()函数的模块版本,这适用于MSVC的当前预览版本。当其他系统支持模块规范,当前的代码可能需要一些修改才能在其他系统上工作。
