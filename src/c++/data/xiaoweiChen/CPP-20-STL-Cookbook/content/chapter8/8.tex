
shared\_ptr类是一个智能指针,拥有自己的托管对象,并维护一个使用计数器来跟踪副本。此示例探索了shared\_ptr的使用方式,以在共享指针副本的同时管理内存。

\begin{tcolorbox}[colback=webgreen!5!white,colframe=webgreen!75!black,title=Note]
有关智能指针的更多详细信息,请参阅本章前面关于使用std::unique\_ptr管理已分配内存的介绍。
\end{tcolorbox}

\subsubsection{How to do it…}

在这个示例中,用一个演示类检查std::shared\_ptr,当调用它的构造函数和析构函数时进行打印输出:

\begin{itemize}
\item 
首先,创建一个简单的演示类:

\begin{lstlisting}[style=styleCXX]
struct Thing {
	string_view thname{ "unk" };
	Thing() {
		cout << format("default ctor: {}\n", thname);
	}
	Thing(const string_view& n) : thname(n) {
		cout << format("param ctor: {}\n", thname);
	}
	~Thing() {
		cout << format("dtor: {}\n", thname);
	}
};
\end{lstlisting}

该类有一个默认构造函数、一个参数化构造函数和一个析构函数。每一个都有打印语句,显示调用了哪些函数。

\item 
shared\_ptr类的工作原理与其他智能指针非常相似,也可以用new操作符或make\_shared()函数来构造:

\begin{lstlisting}[style=styleCXX]
int main() {
	shared_ptr<Thing> p1{ new Thing("Thing 1") };
	auto p2 = make_shared<Thing>("Thing 2");
	cout << "end of main()\n";
}
\end{lstlisting}

输出为:

\begin{tcblisting}{commandshell={}}
param ctor: Thing 1
param ctor: Thing 2
end of main()
dtor: Thing 2
dtor: Thing 1
\end{tcblisting}

推荐使用make\_shared()函数,它会管理构造过程,并且不太容易出错。

与其他智能指针一样,当指针超出作用域时,托管对象将销毁,其内存将返回到堆中。

\item 
下面是一个检查shared\_ptr对象的使用计数的函数:

\begin{lstlisting}[style=styleCXX]
void check_thing_ptr(const shared_ptr<Thing>& p) {
	if(p) cout << format("{} use count: {}\n",
	p->thname, p.use_count());
	else cout << "invalid pointer\n";
}
\end{lstlisting}

thname是Thing类的成员,因此可以通过带有p->成员解引用操作符的指针访问它。use\_count()函数是shared\_ptr类的成员,因此,可以使用p.object成员操作符访问它。

让我们使用指针来对其进行调用:

\begin{lstlisting}[style=styleCXX]
check_thing_ptr(p1);
check_thing_ptr(p2);
\end{lstlisting}

输出为:

\begin{tcblisting}{commandshell={}}
Thing 1 use count: 1
Thing 2 use count: 1
\end{tcblisting}


\item 
当复制指针时,使用计数会增加,但不会构造新的对象:

\begin{lstlisting}[style=styleCXX]
cout << "make 4 copies of p1:\n";
auto pa = p1;
auto pb = p1;
auto pc = p1;
auto pd = p1;
check_thing_ptr(p1);
\end{lstlisting}

输出为:

\begin{tcblisting}{commandshell={}}
make 4 copies of p1:
Thing 1 use count: 5
\end{tcblisting}

\item 
当我们检查其他副本时,会得到相同的结果:

\begin{lstlisting}[style=styleCXX]
check_thing_ptr(pa);
check_thing_ptr(pb);
check_thing_ptr(pc);
check_thing_ptr(pd);
\end{lstlisting}

输出为:

\begin{tcblisting}{commandshell={}}
Thing 1 use count: 5
Thing 1 use count: 5
Thing 1 use count: 5
Thing 1 use count: 5
\end{tcblisting}

每个指针都会报告相同的使用计数。

\item 
当副本超出作用域时,将进行销毁,并使计数递减:

\begin{lstlisting}[style=styleCXX]
{ // new scope
	cout << "make 4 copies of p1:\n";
	auto pa = p1;
	auto pb = p1;
	auto pc = p1;
	auto pd = p1;
	check_thing_ptr(p1);
} // end of scope
check_thing_ptr(p1);
\end{lstlisting}

输出为:

\begin{tcblisting}{commandshell={}}
make 4 copies of p1:
Thing 1 use count: 5
Thing 1 use count: 1
\end{tcblisting}

\item 
销毁副本将减少使用计数,但不会销毁管理对象。当最终副本超出作用域且使用计数为零时,对象将销毁:

\begin{lstlisting}[style=styleCXX]
{
	cout << "make 4 copies of p1:\n";
	auto pa = p1;
	auto pb = p1;
	auto pc = p1;
	auto pd = p1;
	check_thing_ptr(p1);
	pb.reset();
	p1.reset();
	check_thing_ptr(pd);
} // end of scope
\end{lstlisting}

输出为:

\begin{tcblisting}{commandshell={}}
make 4 copies of p1:
Thing 1 use count: 5
Thing 1 use count: 3
dtor: Thing 1
\end{tcblisting}

销毁pb(一个副本)和p1(原始)会留下三个指针副本(pa、bc和pd),因此管理对象仍然存在。

其余三个指针副本将在创建它们的作用域的末尾销毁。然后销毁对象,并将其内存返回到堆中。

\end{itemize}

\subsubsection{How it works…}

shared\_ptr类的区别在于它对指向同一个托管对象的多个指针的管理。

shared\_ptr对象的复制构造函数和复制赋值操作符增加一个使用计数器。析构函数将使用计数器减为零,然后销毁托管对象,并将其内存返还给堆。

shared\_ptr类同时管理托管对象和堆分配的控制块。控制块包含使用计数器以及其他管家对象,控制块与管理对象一起在副本之间进行管理和共享。所以原始shared\_ptr对象将控制权转让给其副本,以便其他shared\_ptr可以管理对象及其内存。





























