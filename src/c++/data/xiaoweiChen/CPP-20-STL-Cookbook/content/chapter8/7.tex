
智能指针是管理已分配堆内存的优秀工具。

堆内存由C函数malloc()和free()在最低层进行管理。malloc()从堆中分配一块内存,free()将内存返还给堆。这些函数不执行初始化,也不调用构造函数或析构函数。若未能通过调用free()将已分配的内存返回给堆,则该行为是未定义的,通常会导致内存泄漏和安全漏洞。

C++提供了new和delete操作符来分配和释放堆内存,用以取代malloc()和free()。new和delete操作符调用对象构造函数和析构函数,但仍然不管理内存。若使用new分配内存,而未能使用delete释放它,还是会泄漏内存。

C++14中引入的智能指针遵循资源获取即初始化(Resource Acquisition Is Initialization, RAII)习惯用法,当为一个对象分配内存时,将调用该对象的构造函数。当调用对象的析构函数时,内存会自动返回到堆中。

例如,当使用make\_unique()创建一个新的智能指针时:

\begin{lstlisting}[style=styleCXX]
{ // beginning of scope
	auto p = make_unique<Thing>(); // memory alloc’d,
	// ctor called
	process_thing(p); // p is unique_ptr<Thing>
} // end of scope, dtor called, memory freed
\end{lstlisting}

make\_unique()为Thing对象分配内存,调用Thing默认构造函数,构造一个unique\_ptr<Thing>对象,并返回unique\_ptr。当p超出作用域时,调用Thing析构函数,内存自动返回到堆中。

除了内存管理之外,智能指针的工作原理非常类似于基本指针:

\begin{lstlisting}[style=styleCXX]
auto x = *p; // *p derefs the pointer, returns Thing object
auto y = p->thname; // p-> derefs the pointer, returns member
\end{lstlisting}

unique\_ptr是一个智能指针,只允许一个指针实例。它可以移动,但不能复制。来仔细了解一下,如何使用unique\_ptr。

\subsubsection{How to do it…}

在这个示例中,用一个演示类检查std::unique\_ptr,当使用构造函数和析构函数时输出:

\begin{itemize}
\item 
首先,创建一个简单的演示类:

\begin{lstlisting}[style=styleCXX]
struct Thing {
	string_view thname{ "unk" };
	Thing() {
		cout << format("default ctor: {}\n", thname);
	}
	Thing(const string_view& n) : thname(n) {
		cout << format("param ctor: {}\n", thname);
	}
	~Thing() {
		cout << format("dtor: {}\n", thname);
	}
};
\end{lstlisting}

该类有一个默认构造函数、一个参数化构造函数和一个析构函数。每一个都有打印语句,告诉我们调用了什么。

\item 
当只构造一个unique\_ptr时,不会分配内存或构造一个托管对象:

\begin{lstlisting}[style=styleCXX]
int main() {
	unique_ptr<Thing> p1;
	cout << "end of main()\n";
}
\end{lstlisting}

输出为:

\begin{tcblisting}{commandshell={}}
end of main()
\end{tcblisting}


\item 
当使用new操作符时,会分配内存并构造一个Thing对象:

\begin{lstlisting}[style=styleCXX]
int main() {
	unique_ptr<Thing> p1{ new Thing };
	cout << "end of main()\n";
}
\end{lstlisting}

输出为:

\begin{tcblisting}{commandshell={}}
default ctor: unk
end of main()
dtor: unk
\end{tcblisting}

new操作符通过调用默认构造函数构造Thing对象。unique\_ptr<Thing>析构函数在智能指针到达其作用域的末尾时,调用Thing析构函数。

Thing默认构造函数不初始化thname字符串,会保留其默认值“unk”。

\item 
也可以使用make\_unique():

\begin{lstlisting}[style=styleCXX]
int main() {
	auto p1 = make_unique<Thing>();
	cout << "end of main()\n";
}
\end{lstlisting}

输出为:

\begin{tcblisting}{commandshell={}}
default ctor: unk
end of main()
dtor: unk
\end{tcblisting}

make\_unique()工厂函数负责内存分配并返回unique\_ptr对象,这是构造unique\_ptr的推荐方法。

\item 
传递给make\_unique()的参数都将用于构造目标对象:

\begin{lstlisting}[style=styleCXX]
int main() {
	auto p1 = make_unique<Thing>("Thing 1") };
	cout << "end of main()\n";
}
\end{lstlisting}

输出为:

\begin{tcblisting}{commandshell={}}
param ctor: Thing 1
end of main()
dtor: Thing 1
\end{tcblisting}

参数化构造函数将值赋给thname,因此Thing对象现在是“Thing 1”。

\item 
创建一个函数,其有一个unique\_ptr<Thing>参数:

\begin{lstlisting}[style=styleCXX]
void process_thing(unique_ptr<Thing> p) {
	if(p) cout << format("processing: {}\n",
		p->thname);
	else cout << "invalid pointer\n";
}
\end{lstlisting}

若尝试传递一个unique\_ptr给这个函数,会得到一个编译器错误:

\begin{lstlisting}[style=styleCXX]
process_thing(p1);
\end{lstlisting}

编译器错误为:

\begin{tcblisting}{commandshell={}}
error: use of deleted function...
\end{tcblisting}

这是因为函数调用试图复制unique\_ptr对象,但删除了unique\_ptr复制构造函数以防止复制。解决方案是让函数接受一个const\&引用参数:

\begin{lstlisting}[style=styleCXX]
void process_thing(const unique_ptr<Thing>& p) {
	if(p) cout << format("processing: {}\n",
		p->thname);
	else cout << "invalid pointer\n";
}
\end{lstlisting}

输出为:

\begin{tcblisting}{commandshell={}}
param ctor: Thing 1
processing: Thing 1
end of main()
dtor: Thing 1
\end{tcblisting}

\item 
可以用一个临时对象调用process\_thing(),其会在函数作用域的末尾立即销毁:

\begin{lstlisting}[style=styleCXX]
int main() {
	auto p1{ make_unique<Thing>("Thing 1") };
	process_thing(p1);
	process_thing(make_unique<Thing>("Thing 2"));
	cout << "end of main()\n";
}
\end{lstlisting}

输出为:

\begin{tcblisting}{commandshell={}}
param ctor: Thing 1
processing: Thing 1
param ctor: Thing 2
processing: Thing 2
dtor: Thing 2
end of main()
dtor: Thing 1
\end{tcblisting}

\end{itemize}

\subsubsection{How it works…}

智能指针只是一个对象,在拥有和管理另一个对象的资源时,可以提供指针接口。

unique\_ptr类由其删除的复制构造函数和复制赋值操作符来区分,这可以防止智能指针的复制。

不能复制unique\_ptr:

\begin{lstlisting}[style=styleCXX]
auto p2 = p1;
\end{lstlisting}

编译器错误:

\begin{tcblisting}{commandshell={}}
error: use of deleted function...
\end{tcblisting}

但可以移动unique\_ptr:

\begin{lstlisting}[style=styleCXX]
auto p2 = std::move(p1);
process_thing(p1);
process_thing(p2);
\end{lstlisting}

移动后,p1无效,p2为“Thing 1”。

输出为:

\begin{tcblisting}{commandshell={}}
invalid pointer
processing: Thing 1
end of main()
dtor: Thing 1
\end{tcblisting}

unique\_ptr接口有一个重置指针的方法:

\begin{lstlisting}[style=styleCXX]
p1.reset(); // pointer is now invalid
process_thing(p1);
\end{lstlisting}

输出为:

\begin{tcblisting}{commandshell={}}
dtor: Thing 1
invalid pointer
\end{tcblisting}

reset()方法也可以将管理对象替换为另一个相同类型的对象:

\begin{lstlisting}[style=styleCXX]
p1.reset(new Thing("Thing 3"));
process_thing(p1);
\end{lstlisting}

输出为:

\begin{tcblisting}{commandshell={}}
param ctor: Thing 3
dtor: Thing 1
processing: Thing 3
\end{tcblisting}




















