
严格地说,std::weak\_ptr不是一个智能指针。相反,它是一个与shared\_ptr合作运行的观察者。weak\_ptr本身不保存指针。

某些情况下,shared\_ptr对象可能会创建悬空指针或数据竞争,这可能导致内存泄漏或其他问题。解决方案是使用具有shared\_ptr的weak\_ptr对象。

\subsubsection{How to do it…}

这个示例中,使用std::shared\_ptr来检查std::weak\_ptr的使用。创建一个演示类,当调用它的构造函数和析构函数时进行打印。

\begin{itemize}
\item 
展示将要在shared\_ptr和unique\_ptr中创建的类型:

\begin{lstlisting}[style=styleCXX]
struct Thing {
	string_view thname{ "unk" };
	Thing() {
		cout << format("default ctor: {}\n", thname);
	}
	Thing(const string_view& n) : thname(n) {
		cout << format("param ctor: {}\n", thname);
	}
	~Thing() {
		cout << format("dtor: {}\n", thname);
	}
};
\end{lstlisting}

该类有一个默认构造函数、一个参数化构造函数和一个析构函数。每一个都有一个简单的打印语句告诉我们调用了什么。

\item 
还需要一个函数来检查weak\_ptr对象:

\begin{lstlisting}[style=styleCXX]
void get_weak_thing(const weak_ptr<Thing>& p) {
	if(auto sp = p.lock()) cout <<
		format("{}: count {}\n", sp->thname,
			p.use_count());
	else cout << "no shared object\n";
}
\end{lstlisting}

weak\_ptr本身不作为指针操作,其需要使用shared\_ptr.lock()函数返回一个shared\_ptr对象,然后可以使用该对象访问托管对象。

\item 
因为weak\_ptr需要一个相关的shared\_ptr,将通过创建一个shared\_ptr<Thing>对象来启动main()。当创建一个weak\_ptr对象而不分配shared\_ptr时,会对过期(expired)标志进行设置:

\begin{lstlisting}[style=styleCXX]
int main() {
	auto thing1 = make_shared<Thing>("Thing 1");
	weak_ptr<Thing> wp1;
	cout << format("expired: {}\n", wp1.expired());
	get_weak_thing(wp1);
}
\end{lstlisting}

输出为:

\begin{tcblisting}{commandshell={}}
param ctor: Thing 1
expired: true
no shared object
\end{tcblisting}

make\_shared()函数分配内存并构造一个Thing对象。

weak\_ptr<Thing>声明构造了一个weak\_ptr对象,但没有分配shared\_ptr。因此,当检查过期标志时,其为真,表明并没有相关的shared\_ptr。

因为没有shared\_ptr可用,所以get\_weak\_thing()函数不能获得锁。

\item 
将shared\_ptr分配给weak\_ptr时,可以使用weak\_ptr来访问托管对象:

\begin{lstlisting}[style=styleCXX]
wp1 = thing1;
get_weak_thing(wp1);
\end{lstlisting}

输出为:

\begin{tcblisting}{commandshell={}}
Thing 1: count 2
\end{tcblisting}

get\_weak\_thing()函数现在可以获取锁并访问托管对象。lock()方法返回一个shared\_ptr,use\_count()反映了现在有第二个shared\_ptr管理Thing对象的情况。

新shared\_ptr在get\_weak\_thing()作用域的末尾销毁。

\item 
weak\_ptr类有一个构造函数,其接受shared\_ptr进行一步构造:

\begin{lstlisting}[style=styleCXX]
weak_ptr<Thing> wp2(thing1);
get_weak_thing(wp2);
\end{lstlisting}

输出为:

\begin{tcblisting}{commandshell={}}
Thing 1: count 2
\end{tcblisting}

use\_count()再次为2,之前的shared\_ptr在其封闭的get\_weak\_thing()作用域结束时销毁。

\item 
重置shared\_ptr时,其关联的weak\_ptr对象将过期:

\begin{lstlisting}[style=styleCXX]
thing1.reset();
get_weak_thing(wp1);
get_weak_thing(wp2);
\end{lstlisting}

输出为:

\begin{tcblisting}{commandshell={}}
dtor: Thing 1
no shared object
no shared object
\end{tcblisting}

reset()后,使用计数为零,管理对象销毁,内存释放。
\end{itemize}


\subsubsection{How it works…}

weak\_ptr对象是一个观察者,其对shared\_ptr对象属于非所有引用。weak\_ptr会观察shared\_ptr,这样就知道托管对象什么时候可用,什么时候不可用。这允许在您不需要知道托管对象是否处于活动状态的情况下,使用shared\_ptr。

weak\_ptr类有一个use\_count()函数,返回shared\_ptr的使用计数,若管理对象已删除,则返回0:

\begin{lstlisting}[style=styleCXX]
long use_count() const noexcept;
\end{lstlisting}

weak\_ptr还有一个expired()函数,会报告托管对象是否已删除:

\begin{lstlisting}[style=styleCXX]
bool expired() const noexcept;
\end{lstlisting}

lock()函数是访问shared\_ptr的首选方法,其检查expired()以查看管理对象是否可用。若可用,将返回一个新的shared\_ptr,该shared\_ptr与管理对象共享所有权。否则,会返回一个空的shared\_ptr。这些会作为一个原子操作完成:

\begin{lstlisting}[style=styleCXX]
std::shared_ptr<T> lock() const noexcept;
\end{lstlisting}


\subsubsection{There's more…}

weak\_ptr的一个重要用例是有可能循环引用shared\_ptr对象。例如,考虑两个相互链接的类的情况(可能在一个层次结构中):

\begin{lstlisting}[style=styleCXX]
struct circB;
struct circA {
	shared_ptr<circB> p;
	~circA() { cout << "dtor A\n"; }
};
struct circB {
	shared_ptr<circA> p;
	~circB() { cout << "dtor B\n"; }
};
\end{lstlisting}

在析构函数中有print语句,因此可以看到对象何时销毁。现在,可以通过shared\_ptr创建两个相互指向的对象:

\begin{lstlisting}[style=styleCXX]
int main() {
	auto a{ make_shared<circA>() };
	auto b{ make_shared<circB>() };
	a->p = b;
	b->p = a;
	cout << "end of main()\n";
}
\end{lstlisting}

当运行这个函数时,会发现析构函数从未调用:

\begin{tcblisting}{commandshell={}}
end of main()
\end{tcblisting}

因为对象维护着相互引用的共享指针,所以使用计数永远不会达到零,并且托管对象永远不会销毁。

可以通过更改其中一个类来使用weak\_ptr来解决这个问题:

\begin{lstlisting}[style=styleCXX]
struct circB {
	weak_ptr<circA> p;
	~circB() { cout << "dtor B\n"; }
};
\end{lstlisting}

main()中的代码保持不变,我们会得到这样的输出:

\begin{tcblisting}{commandshell={}}
end of main()
dtor A
dtor B
\end{tcblisting}

通过将一个shared\_ptr更改为weak\_ptr,解决了循环引用,并且现在在对象作用域的末尾,可以正确地销毁对象了。








