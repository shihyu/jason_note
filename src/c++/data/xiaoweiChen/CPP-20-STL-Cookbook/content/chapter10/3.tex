
文件系统库包括用于操作path对象内容的函数。本节中,我们将了解其中的一些工具。

\subsubsection{How to do it…}

这个示例中,我们检查了一些操作path对象内容的函数:

\begin{itemize}
\item 
从命名空间指令和格式化特化开始。本章的每个示例中都会这样做:

\begin{lstlisting}[style=styleCXX]
namespace fs = std::filesystem;
template<>
struct std::formatter<fs::path>:
std::formatter<std::string> {
	template<typename FormatContext>
	auto format(const fs::path& p, FormatContext& ctx) {
		return format_to(ctx.out(), "{}", p.string());
	}
};
\end{lstlisting}

\item 
可以使用current\_path()函数获取当前的工作目录,该函数返回一个path对象:

\begin{lstlisting}[style=styleCXX]
cout << format("current_path: {}\n", fs::current_path());
\end{lstlisting}

输出为:

\begin{tcblisting}{commandshell={}}
current_path: /home/billw/chap10
\end{tcblisting}

\item 
absolute()函数的作用是:从相对路径返回绝对路径:

\begin{lstlisting}[style=styleCXX]
cout << format("absolute(p): {}\n", fs::absolute(p));
\end{lstlisting}

输出为:

\begin{tcblisting}{commandshell={}}
absolute(p): /home/billw/chap10/testdir/foo.txt
\end{tcblisting}

absolute()也会解除对符号链接的引用。

\item 
/=操作符将一个字符串追加到路径字符串的末尾,并返回一个新的路径对象:

\begin{lstlisting}[style=styleCXX]
cout << format("append: {}\n",
	fs::path{ "testdir" } /= "foo.txt");
\end{lstlisting}

输出为:

\begin{tcblisting}{commandshell={}}
append: testdir/foo.txt
\end{tcblisting}

\item 
canonical()函数返回完整的规范目录路径:

\begin{lstlisting}[style=styleCXX]
cout << format("canonical: {}\n",
	fs::canonical(fs::path{ "." } /= "testdir"));
\end{lstlisting}

输出为:

\begin{tcblisting}{commandshell={}}
canonical: /home/billw/chap
\end{tcblisting}

\item 
equivalent()函数测试两个相对路径,看是否解析到相同的文件系统示例:

\begin{lstlisting}[style=styleCXX]
cout << format("equivalent: {}\n",
	fs::equivalent("testdir/foo.txt",
		"testdir/../testdir/foo.txt"));
\end{lstlisting}

输出为:

\begin{tcblisting}{commandshell={}}
equivalent: true
\end{tcblisting}

\item 
文件系统库包含了filesystem\_error类用于异常处理:

\begin{lstlisting}[style=styleCXX]
try {
	fs::path p{ fp };
	cout << format("p: {}\n", p);
	...
	cout << format("equivalent: {}\n",
	fs::equivalent("testdir/foo.txt",
	"testdir/../testdir/foo.txt"));
} catch (const fs::filesystem_error& e) {
	cout << format("{}\n", e.what());
	cout << format("path1: {}\n", e.path1());
	cout << format("path2: {}\n", e.path2());
}
\end{lstlisting}

filesystem\_error类包括用于显示错误消息和获取错误涉及的path方法。

若在equivalent()中引入一个错误,可以看到fileystem\_error:

\begin{lstlisting}[style=styleCXX]
cout << format("equivalent: {}\n",
	fs::equivalent("testdir/foo.txt/x",
		"testdir/../testdir/foo.txt/y"));
\end{lstlisting}

输出为:

\begin{tcblisting}{commandshell={}}
filesystem error: cannot check file equivalence: No
such file or directory [testdir/foo.txt/x] [testdir/../
testdir/foo.txt/y]
path1: testdir/foo.txt/x
path2: testdir/../testdir/foo.txt/y
\end{tcblisting}

这是使用GCC在Debian上的输出。

filesystem\_error类通过其path1()和path2()方法提供了更多的细节,这些方法可以返回path对象。

\item 
也可以将std::error\_code用于文件系统函数:

\begin{lstlisting}[style=styleCXX]
fs::path p{ fp };
std::error_code e;
cout << format("canonical: {}\n",
	fs::canonical(p /= "foo", e));
cout << format("error: {}\n", e.message());
\end{lstlisting}

输出为:

\begin{tcblisting}{commandshell={}}
canonical:
error: Not a directory
\end{tcblisting}

\item 
即使Windows使用了不同的文件系统,这段代码仍会按照预期工作,使用Windows文件命名约定:

\begin{tcblisting}{commandshell={}}
p: testdir/foo.txt
current_path: C:\Users\billw\chap10
absolute(p): C:\Users\billw\chap10\testdir\foo.txt
concatenate: testdirfoo.txt
append: testdir\foo.txt
canonical: C:\Users\billw\chap10\testdir
equivalent: true
\end{tcblisting}
\end{itemize}

\subsubsection{How it works…}

这些函数大多数接受一个path对象,一个可选的std::error\_code对象,并返回一个path对象:

\begin{lstlisting}[style=styleCXX]
path absolute(const path& p);
path absolute(const path& p, std::error_code& ec);
\end{lstlisting}

equivalent()函数接受两个path对象,并返回bool类型:

\begin{lstlisting}[style=styleCXX]
bool equivalent( const path& p1, const path& p2 );
bool equivalent( const path& p1, const path& p2,
	std::error_code& ec );
\end{lstlisting}

path类具有连接和追加操作符。这两个算子都是破坏性的,会修改运算符左边的path:

\begin{lstlisting}[style=styleCXX]
p1 += source; // concatenate
p1 /= source; // append
\end{lstlisting}

对于右边,这些操作符接受一个路径对象、一个字符串、一个string\_view、一个C-string或一对迭代器。

连接操作符将操作符右侧的字符串,添加到p1路径字符串的末尾。

append操作符添加分隔符(例如,/或\verb|\|),后面跟着从操作符右侧到路径字符串p1末尾的字符串。







