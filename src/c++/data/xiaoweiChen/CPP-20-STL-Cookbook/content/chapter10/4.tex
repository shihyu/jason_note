

文件系统库提供了一个directory\_entry类,其中包含关于给定路径的目录相关信息,可以使用它来创建有用的目录列表。

\subsubsection{How to do it…}

在这个示例中,使用directory\_entry类中的信息创建目录列表:

\begin{itemize}
\item 
从显示路径对象的命名空间别名和格式化特化开始:

\begin{lstlisting}[style=styleCXX]
namespace fs = std::filesystem;
template<>
struct std::formatter<fs::path>:
std::formatter<std::string> {
	template<typename FormatContext>
	auto format(const fs::path& p, FormatContext& ctx) {
		return format_to(ctx.out(), "{}", p.string());
	}
};
\end{lstlisting}

\item 
directory\_iterator类可以方便地列出目录:

\begin{lstlisting}[style=styleCXX]
int main() {
	constexpr const char* fn{ "." };
	const fs::path fp{fn};
	for(const auto& de : fs::directory_iterator{fp}) {
		cout << format("{} ", de.path().filename());
	}
	cout << '\n';
}
\end{lstlisting}

输出为:

\begin{tcblisting}{commandshell={}}
chrono Makefile include chrono.cpp working formatter
testdir formatter.cpp working.cpp
\end{tcblisting}

\item 
可以添加命令行选项来实现这个功能,比如Unix的ls:

\begin{lstlisting}[style=styleCXX]
int main(const int argc, const char** argv) {
	fs::path fp{ argc > 1 ? argv[1] : "." };
	if(!fs::exists(fp)) {
		const auto cmdname {
			fs::path{argv[0]}.filename() };
		cout << format("{}: {} does not exist\n",
			cmdname, fp);
		return 1;
	}
	if(is_directory(fp)) {
		for(const auto& de :
		fs::directory_iterator{fp}) {
			cout << format("{} ",
			de.path().filename());
		}
	} else {
		cout << format("{} ", fp.filename());
	}
	cout << '\n';
}
\end{lstlisting}

若有命令行参数,可以用它来创建一个path对象。否则,需要使用"."表示当前目录。

我们使用if\_exists()检查路径是否存在。若不存在,则打印错误消息并退出。错误信息包括来自argv[0]的cmdname。

接下来,我们检查is\_directory()。若有一个目录,可对每个条目循环使用directory\_iterator。directory\_iterator遍历directory\_entry对象。de.path().filename()会从每个directory\_entry对象获取路径和文件名。

输出为:

\begin{tcblisting}{commandshell={}}
$ ./working
chrono Makefile include chrono.cpp working formatter
testdir formatter.cpp working.cpp
$ ./working working.cpp
working.cpp
$ ./working foo.bar
working: foo.bar does not exist
\end{tcblisting}

\item 
若要对输出进行排序,可以将directory\_entry对象存储在可排序容器中。

文件的顶部,为fs::directory\_entry创建一个别名:

\begin{lstlisting}[style=styleCXX]
using de = fs::directory_entry;
\end{lstlisting}

在main()的顶部,声明了一个de的vector:

\begin{lstlisting}[style=styleCXX]
vector<de> entries{};
\end{lstlisting}

在is\_directory()块中,我们加载vector对它排序,并进行显示:

\begin{lstlisting}[style=styleCXX]
if(is_directory(fp)) {
	for(const auto& de : fs::directory_iterator{fp}) {
		entries.emplace_back(de);
	}
	std::sort(entries.begin(), entries.end());
	for(const auto& e : entries) {
		cout << format("{} ", e.path().filename());
	}
} else { ...
\end{lstlisting}

现在输出已排序vector:

\begin{tcblisting}{commandshell={}}
Makefile chrono chrono.cpp formatter formatter.cpp
include testdir working working.cpp
\end{tcblisting}

注意,Makefile首先排序,显然是无序的。这是因为大写字母在小写字母之前按ASCII顺序排序。

\item 
若想要一个不区分大小写的排序,需要一个忽略大小写的比较函数。首先,需要一个函数返回小写字母的字符串:

\begin{lstlisting}[style=styleCXX]
string strlower(string s) {
	auto char_lower = [](const char& c) -> char {
		if(c >= 'A' && c <= 'Z') return c + ('a' - 'A');
		else return c;
	};
	std::transform(s.begin(), s.end(), s.begin(),
		char_lower);
	return s;
}
\end{lstlisting}

现在,需要一个函数来比较两个directory\_entry对象,使用strlower():

\begin{lstlisting}[style=styleCXX]
bool dircmp_lc(const de& lhs, const de& rhs) {
	const auto lhstr{ lhs.path().string() };
	const auto rhstr{ rhs.path().string() };
	return strlower(lhstr) < strlower(rhstr);
}
\end{lstlisting}

现在可以在排序中使用dircmp\_lc():

\begin{lstlisting}[style=styleCXX]
std::sort(entries.begin(), entries.end(), dircmp_lc);
\end{lstlisting}

输出现在排序,并忽略大小写:

\begin{tcblisting}{commandshell={}}
chrono chrono.cpp formatter formatter.cpp include
Makefile testdir working working.cpp
\end{tcblisting}

\item 
现在,有了一个简单的目录清单程序。

从文件系统库中可以获得更多信息。来创建一个print\_dir()函数来收集更多信息,并将其格式化为Unix的ls风格:

\begin{lstlisting}[style=styleCXX]
void print_dir(const de& dir) {
	using fs::perms;
	const auto fpath{ dir.path() };
	const auto fstat{ dir.symlink_status() };
	const auto fperm{ fstat.permissions() };
	const uintmax_t fsize{
		is_regular_file(fstat) ? file_size(fpath) : 0 };
	const auto fn{ fpath.filename() };
	
	string suffix{};
	if(is_directory(fstat)) suffix = "/";
	else if((fperm & perms::owner_exec) != perms::none) {
		suffix = "*";
	}
	cout << format("{}{}\n", fn, suffix);
}
\end{lstlisting}

print\_dir()函数接受一个directory\_entry参数。然后,从directory\_entry对象中检索一些有用的对象:

\begin{itemize}
\item 
dir.path()返回一个path对象

\item 
dir.symlink\_status()返回一个file\_status对象,无符号链接。

\item 
fstat.permissions()返回一个perms对象.

\item 
ssize是文件的大小,fn是文件名字符串。使用的时候,再仔细地研究吧。
\end{itemize}

Unix ls使用文件名后面的尾随字符来表示目录或可执行文件。使用is\_directory()测试fstat对象,以查看文件是否是目录,并在文件名后面添加/。同样,可以用fperm对象测试文件是否可执行。

在sort()之后的for循环中调用print\_dir():

\begin{lstlisting}[style=styleCXX]
std::sort(entries.begin(), entries.end(), dircmp_lc);
for(const auto& e : entries) {
	print_dir(e);
}
\end{lstlisting}

现在输出是这样的:

\begin{tcblisting}{commandshell={}}
chrono*
chrono.cpp
formatter*
formatter.cpp
include*
Makefile
testdir/
working*
working.cpp
\end{tcblisting}

\item 
注意include*,这实际上是一个符号链接。可以通过链接来正确地标记它,以获得目标路径:

\begin{lstlisting}[style=styleCXX]
string suffix{};
if(is_symlink(fstat)) {
	suffix = " -> ";
	suffix += fs::read_symlink(fpath).string();
}
else if(is_directory(fstat)) suffix = "/";
else if((fperm & perms::owner_exec) != perms::none)
suffix = "*";
\end{lstlisting}

read\_symlink()函数的作用是:返回一个路径对象。我们获取返回路径对象的string()表示形式,并将其添加到输出的尾部:

\begin{tcblisting}{commandshell={}}
chrono*
chrono.cpp
formatter*
formatter.cpp
include -> /Users/billw/include
Makefile
testdir/
working*
working.cpp
\end{tcblisting}

\item 
Unix ls命令还包含一串字符来表示文件的权限位。它看起来像这样:drwxr-xr-x。

第一个字符表示文件的类型,例如:d表示目录,l表示符号链接,-表示普通文件。

type\_char()函数返回相应的字符:

\begin{lstlisting}[style=styleCXX]
char type_char(const fs::file_status& fstat) {
		 if(is_symlink(fstat)) return 'l';
	else if(is_directory(fstat)) return 'd';
	else if(is_character_file(fstat)) return 'c';
	else if(is_block_file(fstat)) return 'b';
	else if(is_fifo(fstat)) return 'p';
	else if(is_socket(fstat)) return 's';
	else if(is_other(fstat)) return 'o';
	else if(is_regular_file(fstat)) return '-';
	return '?';
}
\end{lstlisting}

剩下的字符串是三元组。每个三元组以rwx的形式包含读、写和执行权限位的位置。若一个位没有设置,其字符将使用-替换。三组权限分别对应三个三元组:owner、group和other。

\begin{lstlisting}[style=styleCXX]
string rwx(const fs::perms& p) {
	using fs::perms;
	auto bit2char = [&p](perms bit, char c) {
		return (p & bit) == perms::none ? '-' : c;
	};
	return { bit2char(perms::owner_read, 'r'),
		bit2char(perms::owner_write, 'w'),
		bit2char(perms::owner_exec, 'x'),
		bit2char(perms::group_read, 'r'),
		bit2char(perms::group_write, 'w'),
		bit2char(perms::group_exec, 'x'),
		bit2char(perms::others_read, 'r'),
		bit2char(perms::others_write, 'w'),
		bit2char(perms::others_exec, 'x') };
}
\end{lstlisting}

perms对象表示POSIX权限位图,但它不一定以位的形式实现。每个条目必须与perms::none值进行比较。我们的函数满足了这个要求。

我们将这个定义添加到print\_dir()函数的顶部:

\begin{lstlisting}[style=styleCXX]
const auto permstr{ type_char(fstat) + rwx(fperm) };
\end{lstlisting}

更新format()字符串:

\begin{lstlisting}[style=styleCXX]
cout << format("{} {}{}\n", permstr, fn, suffix);
\end{lstlisting}

得到这样的输出:

\begin{tcblisting}{commandshell={}}
-rwxr-xr-x chrono*
-rw-r--r-- chrono.cpp
-rwxr-xr-x formatter*
-rw-r--r-- formatter.cpp
lrwxr-xr-x include -> /Users/billw/include
-rw-r--r-- Makefile
drwxr-xr-x testdir/
-rwxr-xr-x working*
-rw-r--r-- working.cpp
\end{tcblisting}

\item 
现在,让我们添加一定长度的字符串。fsize值来自file\_size()函数,该函数返回std::uintmax\_t类型,这表示目标系统上的最大自然整数大小。uintmax\_t并不总是与size\_t相同,并不总是可以转换。值得注意的是,uintmax\_t在Windows上是32位,而size\_t是64位:

\begin{lstlisting}[style=styleCXX]
string size_string(const uintmax_t fsize) {
	constexpr const uintmax_t kilo{ 1024 };
	constexpr const uintmax_t mega{ kilo * kilo };
	constexpr const uintmax_t giga{ mega * kilo };
	string s;
	if(fsize >= giga ) return
		format("{}{}", (fsize + giga / 2) / giga, 'G');
	else if (fsize >= mega) return
		format("{}{}", (fsize + mega / 2) / mega, 'M');
	else if (fsize >= kilo) return
		format("{}{}", (fsize + kilo / 2) / kilo, 'K');
	else return format("{}B", fsize);
}
\end{lstlisting}

这个函数中,我选择使用1024作为1K,因为这似乎是Linux和BSD Unix上的默认值。在生产环境中,这可能是一个命令行选项。

在main()中更新format()字符串:

\begin{lstlisting}[style=styleCXX]
cout << format("{} {:>6} {}{}\n",
	permstr, size_string(fsize), fn, suffix);
\end{lstlisting}

现在,可以得到这样的输出:

\begin{tcblisting}{commandshell={}}
-rwxr-xr-x 284K chrono*
-rw-r--r--   2K chrono.cpp
-rwxr-xr-x 178K formatter*
-rw-r--r-- 906B formatter.cpp
lrwxr-xr-x   0B include -> /Users/billw/include
-rw-r--r-- 642B Makefile
drwxr-xr-x   0B testdir/
-rwxr-xr-x 197K working*
-rw-r--r--   5K working.cpp
\end{tcblisting}

\begin{tcolorbox}[colback=webgreen!5!white,colframe=webgreen!75!black,title=Note]
该程序是为POSIX系统设计的,例如Linux和macOS。可以在Windows系统上运行,但Windows权限系统与POSIX系统不同。所以,在Windows上,权限位总是显示完全设置。
\end{tcolorbox}

\end{itemize}

\subsubsection{How it works…}

文件系统库通过其directory\_entry和相关类携带了一组丰富的信息。在这个示例中使用的主要类包括:

\begin{itemize}
\item 
path类根据目标系统的规则表示文件系统路径。path对象由字符串或另一个路径构造,不需要表示现有的路径,甚至不需要表示可能的路径。path字符串可解析为多个部分,包括根名称、根目录和可选的一系列文件名和目录分隔符。

\item 
directory\_entry类包含一个path对象作为成员,还可以存储其他属性,包括硬链接计数、状态、符号链接、文件大小和最后写入时间。

\item 
file\_status类携带有关文件类型和权限的信息。perms对象可以是file\_status的成员,表示文件的权限结构。
\end{itemize}

有两个函数用于从file\_status中检索perms对象。status()函数和symlink\_status()函数都返回一个perms对象,区别在于如何处理符号链接。status()函数将跟随符号链接并返回目标文件中的perms。symlink\_status()将返回符号链接本身的perms。

\subsubsection{There's more…}

我原本打算在目录列表中包括每个文件的最后写入时间。

directory\_entry类有一个成员函数last\_write\_time(),它返回一个file\_time\_type对象,表示最后一次写入文件的时间戳。

但在编写本文时,可用的实现缺乏将file\_time\_type对象转换为标准chrono::sys\_time的可移植方法。

现在,这里有一个可以在GCC上工作的解决方案:

\begin{lstlisting}[style=styleCXX]
string time_string(const fs::directory_entry& dir) {
	using std::chrono::file_clock;
	auto file_time{ dir.last_write_time() };
	return format("{:%F %T}",
		file_clock::to_sys(dir.last_write_time()));
}
\end{lstlisting}

建议用户代码使用std::chrono::clock\_cast,而非file::clock::to\_sys来转换时钟之间的时间点。目前的可用实现中,都没有用于此目的的std::chrono::clock\_cast特化。

使用这个time\_string()函数,可以将时间添加到print\_dir():

\begin{lstlisting}[style=styleCXX]
const string timestr{ time_string(dir) };
\end{lstlisting}

然后,可以改变format()字符串:

\begin{lstlisting}[style=styleCXX]
cout << format("{} {:>6} {} {}{}\n",
	permstr, sizestr, timestr, fn, suffix);
\end{lstlisting}

并得到这样的输出:

\begin{tcblisting}{commandshell={}}
-rwxr-xr-x 248K 2022-03-09 09:39:49 chrono*
-rw-r--r--   2K 2022-03-09 09:33:56 chrono.cpp
-rwxr-xr-x 178K 2022-03-09 09:39:49 formatter*
-rw-r--r-- 906B 2022-03-09 09:33:56 formatter.cpp
lrwxrwxrwx   0B 2022-02-04 11:39:53 include -> /home/billw/
include
-rw-r--r-- 642B 2022-03-09 14:08:37 Makefile
drwxr-xr-x   0B 2022-03-09 10:38:39 testdir/
-rwxr-xr-x 197K 2022-03-12 17:13:46 working*
-rw-r--r--   5K 2022-03-12 17:13:40 working.cpp
\end{tcblisting}

这适用于Debian的GCC-11,但不要指望它很容易的就能在其他系统上工作。










