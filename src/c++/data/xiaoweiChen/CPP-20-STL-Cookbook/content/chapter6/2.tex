
复制算法通常用于从容器复制到容器,若与迭代器一起工作,则要灵活得多。


\subsubsection{How to do it…}

这个示例中,将使用std::copy和std::copy\_n进行实验,以便更好地理解其工作原理:

\begin{itemize}
\item 
从一个打印容器的函数开始:

\begin{lstlisting}[style=styleCXX]
void printc(auto& c, string_view s = "") {
	if(s.size()) cout << format("{}: ", s);
	for(auto e : c) cout << format("[{}] ", e);
	cout << '\n';
}
\end{lstlisting}

\item 
在main()中,定义了一个vector,并使用printc()将其打印出来:

\begin{lstlisting}[style=styleCXX]
int main() {
	vector<string> v1
	{ "alpha", "beta", "gamma", "delta",
		"epsilon" };
	printc(v1);
}
\end{lstlisting}

会得到这样的输出:

\begin{tcblisting}{commandshell={}}
v1: [alpha] [beta] [gamma] [delta] [epsilon]
\end{tcblisting}

\item 
现在,创建第二个vector,使其有足够空间复制第一个vector:

\begin{lstlisting}[style=styleCXX]
vector<string> v2(v1.size());
\end{lstlisting}

\item 
使用std::copy()算法将v1复制到v2:

\begin{lstlisting}[style=styleCXX]
std::copy(v1.begin(), v1.end(), v2.begin());
printc(v2);
\end{lstlisting}

std::copy()算法接受两个用于复制源范围的迭代器和一个用于复制目标范围的迭代器,给定v1的begin()和end()迭代器来复制整个vector。v2的begin()迭代器作为复制的目标。

现在的输出是:

\begin{tcblisting}{commandshell={}}
v1: [alpha] [beta] [gamma] [delta] [epsilon]
v2: [alpha] [beta] [gamma] [delta] [epsilon]
\end{tcblisting}

\item 
copy()算法不为目标分配空间。因此,v2必须已经有用于复制的空间。或者,使用back\_inserter()迭代器适配器将元素插入到vector的后面:

\begin{lstlisting}[style=styleCXX]
vector<string> v2{};
std::copy(v1.begin(), v1.end(), back_inserter(v2))
\end{lstlisting}

\item 
也可以使用ranges::copy()算法来复制整个范围。容器对象作为一个范围,因此可以使用v1作为源。这里,仍然使用目标容器的迭代器:

\begin{lstlisting}[style=styleCXX]
vector<string> v2(v1.size());
ranges::copy(v1, v2.begin());
\end{lstlisting}

这也适用于back\_inserter():

\begin{lstlisting}[style=styleCXX]
vector<string> v2{};
ranges::copy(v1, back_inserter(v2));
\end{lstlisting}

输出为:

\begin{tcblisting}{commandshell={}}
v2: [alpha] [beta] [gamma] [delta] [epsilon]
\end{tcblisting}

\item 
可以使用copy\_n()复制一定数量的元素:

\begin{lstlisting}[style=styleCXX]
vector<string> v3{};
std::copy_n(v1.begin(), 3, back_inserter(v3));
printc(v3, "v3");
\end{lstlisting}

第二个参数中,copy\_n()算法是要复制的元素数量的计数。输出结果为:

\begin{tcblisting}{commandshell={}}
v3: [alpha] [beta] [gamma]
\end{tcblisting}

\item 
还有一个copy\_if()算法,其使用布尔谓词函数来确定要复制哪些元素:

\begin{lstlisting}[style=styleCXX]
vector<string> v4{};
std::copy_if(v1.begin(), v1.end(), back_inserter(v4),
	[](string& s){ return s.size() > 4; });
printc(v4, "v4");
\end{lstlisting}

还有一个copy\_if()的范围版本:

\begin{lstlisting}[style=styleCXX]
vector<string> v4{};
ranges::copy_if(v1, back_inserter(v4),
	[](string& s){ return s.size() > 4; });
printc(v4, "v4");
\end{lstlisting}

输出只包含长度超过4个字符的字符串:

\begin{tcblisting}{commandshell={}}
v4: [alpha] [gamma] [delta] [epsilon]
\end{tcblisting}

注意,beta值被排除在外。

\item 
可以使用这些算法中的任何一个来复制到或从任何序列,包括流迭代器:

\begin{lstlisting}[style=styleCXX]
ostream_iterator<string> out_it(cout, " ");
ranges::copy(v1, out_it)
cout << '\n';
\end{lstlisting}

输出为:

\begin{tcblisting}{commandshell={}}
alpha beta gamma delta epsilon
\end{tcblisting}
\end{itemize}

\subsubsection{How it works…}

std::copy()算法非常简单,等价的函数为:

\begin{lstlisting}[style=styleCXX]
template<typename Input_it, typename Output_it>
Output_it bw_copy(Input_it begin_it, Input_it end_it,
				  Output_it dest_it) {
	while (begin_it != end_it) {
		*dest_it++ = *begin_it++;
	}
	return dest_it;
}
\end{lstlisting}

copy()函数使用目标迭代器的赋值操作符,从输入迭代器复制到输出迭代器,直到到达输入范围的末尾。

这个算法还有一个版本叫做std::move(),其移动元素而不是复制:

\begin{lstlisting}[style=styleCXX]
std::move(v1.begin(), v1.end(), v2.begin());
printc(v1, "after move: v1");
printc(v2, "after move: v2");
\end{lstlisting}

这将执行移动而不是复制赋值。移动操作之后,v1中的元素将为空,v1中的元素现在在v2中。输出如下所示:

\begin{tcblisting}{commandshell={}}
after move1: v1: [] [] [] [] []
after move1: v2: [alpha] [beta] [gamma] [delta] [epsilon]
\end{tcblisting}

还有一个move()算法的范围版本,执行相同的操作:

\begin{lstlisting}[style=styleCXX]
ranges::move(v1, v2.begin());
\end{lstlisting}

这些算法的强大之处在于其简单。通过让迭代器管理数据,这些简单、优雅的函数允许在支持所需迭代器的任何STL容器之间无缝复制或移动。








