
std::sample()算法获取值序列的随机样本,并用该样本填充目标容器。其对于分析更大的数据集很有用,其中随机样本可用来代表整个数据。

样本集允许近似一个大数据集的特征,而不需要分析整个数据集。这提供了效率与准确性的交换,在许多情况下是一种公平的交换。

\subsubsection{How to do it…}

这个示例中,我们将使用一个具有标准正态分布的200,000个随机整数数组。我们将对几百个值进行采样,以创建每个值频率的直方图。

\begin{itemize}
\item 
我们将从一个简单的函数开始,从一个double型返回一个四舍五入的int型。标准库缺少这样一个函数,我们稍后会创建它:

\begin{lstlisting}[style=styleCXX]
int iround(const double& d) {
	return static_cast<int>(std::round(d));
}
\end{lstlisting}

标准库提供了几个版本的std::round(),包括一个返回长int的版本。但我们需要一个int,这是一个简单的解决方案,可以避免编译器关于缩小转换的警告,同时隐藏难看的static\_cast。

\item 
main()函数中,可从一些有用的常量开始:

\begin{lstlisting}[style=styleCXX]
int main() {
	constexpr size_t n_data{ 200000 };
	constexpr size_t n_samples{ 500 };
	constexpr int mean{ 0 };
	constexpr size_t dev{ 3 };
	...
}
\end{lstlisting}

我们有n\_data和n\_samples的值,分别用于数据和样本容器的大小。也有均值和dev的值,随机值正态分布的均值和标准差参数。

\item 
现在,设置随机数生成器和分布对象,用于初始化源数据集:

\begin{lstlisting}[style=styleCXX]
std::random_device rd;
std::mt19937 rng(rd());
std::normal_distribution<> dist{ mean, dev };
\end{lstlisting}

random\_device对象提供对硬件随机数生成器的访问。mt19937类是Mersenne Twister随机数算法的实现,这是一种高质量的算法,在我们所使用的这种大小的数据集的大多数系统上都能很好地执行。正态分布类提供了均值附近的随机数分布,并提供了标准差。

\item 
现在可以用n\_data个数的随机int值填充一个数组:

\begin{lstlisting}[style=styleCXX]
array<int, n_data> v{};
for(auto& e : v) e = iround(dist(rng));
\end{lstlisting}

数组容器的大小固定,因此模板参数包含一个size\_t值,表示要分配的元素数量。这里使用for()循环填充数组。

rng对象是硬件随机数生成器,其可传递给normal\_distribution对象的dist(),然后传递给round(),我们的整数舍入函数。

\item 
此时,就有一个包含200,000个数据点的数组。这有很多要分析的,我们将使用sample()算法取500个值的样本:

\begin{lstlisting}[style=styleCXX]
array<int, n_samples> samples{};
sample(data.begin(), data.end(), samples.begin(),
	n_samples, rng);
\end{lstlisting}

我们定义另一个数组对象来保存样本,这个的大小是n\_samples。然后,使用sample()算法用n\_samples随机数据点填充数组。

\item 
我们创建一个直方图来分析样本。map结构是完美的,可以很容易地映射每个值的频率:

\begin{lstlisting}[style=styleCXX]
std::map<int, size_t> hist{};
for (const int i : samples) ++hist[i];
\end{lstlisting}

for()循环从样本容器中获取每个值,并将其用作map中的键。增量表达式++hist[i]计算样本集中每个值出现的次数。

\item 
我们使用C++20的format()函数打印出直方图:

\begin{lstlisting}[style=styleCXX]
constexpr size_t scale{ 3 };
cout << format("{:>3} {:>5} {:<}/{}\n",
	"n", "count", "graph", scale);
for (const auto& [value, count] : hist) {
	cout << format("{:>3} ({:>3}) {}\n",
		value, count, string(count / scale, '*'));
}
\end{lstlisting}

format()说明符看起来像\{:>3\}为一定数量的字符腾出空间。尖括号指定左对齐或右对齐。

string(count, char)构造函数创建一个字符串,其中一个字符重复指定的次数,在本例中是n个星号字符*,其中n是count/scale,即直方图中某个值的频率,除以缩放常数。

输出如下所示:

\begin{tcblisting}{commandshell={}}
$ ./sample
n count graph/3
-9 ( 2)
-7 ( 5) *
-6 ( 9) ***
-5 ( 22) *******
-4 ( 24) ********
-3 ( 46) ***************
-2 ( 54) ******************
-1 ( 59) *******************
 0 ( 73) ************************
 1 ( 66) **********************
 2 ( 44) **************
 3 ( 34) ***********
 4 ( 26) ********
 5 ( 18) ******
 6 ( 9) ***
 7 ( 5) *
 8 ( 3) *
 9 ( 1)
\end{tcblisting}

这是直方图的一个很好的图形表示。第一个数字是值,第二个数字是值的频率,星号是频率的直观表示,其中每个星号代表样本集中出现的规模(3)次。

每次运行代码时输出都会有所不同。
\end{itemize}

\subsubsection{How it works…}

std::sample()函数的作用是:从源容器中的随机位置选择特定数量的元素,并将其复制到目标容器中。

sample()的签名是这样的:

\begin{lstlisting}[style=styleCXX]
OutIter sample(SourceIter, SourceIter, OutIter,
	SampleSize, RandNumGen&&);
\end{lstlisting}

前两个参数是包含完整数据集的容器上的begin()和end()迭代器。第三个参数是用于示例目标的迭代器。第四个参数是样本大小,最后一个参数是随机数生成器函数。

sample()算法采用均匀分布,因此每个数据点的采样概率相同。
























