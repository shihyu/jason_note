
Lambda表达式可以定义和存储供后续使用,也可以作为参数传递,存储在数据结构中,并在不同的上下文中使用不同的参数调用。它们和函数一样灵活,并且具有数据的移动性。

\subsubsection{How to do it…}

从一个简单的程序开始,可用其来测试lambda表达式的各种设置:

\begin{itemize}
\item 
首先定义一个main()函数,并使用它来测试lambda:

\begin{lstlisting}[style=styleCXX]
int main() {
	... // code goes here
}
\end{lstlisting}

\item 
在main()函数内部,将声明两个lambda。lambda基本定义需要一对方括号和花括号:

\begin{lstlisting}[style=styleCXX]
auto one = [](){ return "one"; };
auto two = []{ return "two"; }
\end{lstlisting}

注意,第一个示例one在方括号后包含圆括号,而第二个示例two则没有。空参数括号通常包括在内,但并不总是必需的。返回类型由编译器推断。

\item 
可以用cout或format调用这些函数,或在接受C-string的上下文中使用:

\begin{lstlisting}[style=styleCXX]
cout << one() << '\n';
cout << format("{}\n", two());
\end{lstlisting}

\item 
通常,编译器可以通过自动类型推断确定返回类型。否则,可以使用->操作符指定返回类型:

\begin{lstlisting}[style=styleCXX]
auto one = []() -> const char * { return "one"; };
auto two = []() -> auto { return "two"; };
\end{lstlisting}

lambda使用尾部返回类型语法,由->操作符和类型规范组成。若没有指定返回类型,则认为它是auto。若使用尾部返回类型,则需要参数括号。

\item 
来定义一个lambda来输出其他lambda的值:

\begin{lstlisting}[style=styleCXX]
auto p = [](auto v) { cout << v() << '\n'; };
\end{lstlisting}

p()需要一个lambda(或函数)作为参数v,并在函数体中调用它。

auto类型参数使此lambda成为缩写模板。C++20前,这是创建lambda模板的唯一方法。从C++20开始,可以在捕获括号之后指定模板参数(不带template关键字)。这与模板参数相同:

\begin{lstlisting}[style=styleCXX]
auto p = []<template T>(T v) { cout << v() << '\n'; };
\end{lstlisting}

缩写的auto版本更简单,也更常见。

\item 
现在可以在函数调用中传递一个匿名lambda:

\begin{lstlisting}[style=styleCXX]
p([]{ return "lambda call lambda"; });
\end{lstlisting}

输出结果为:

\begin{tcblisting}{commandshell={}}
lambda call lambda
\end{tcblisting}

\item 
若需要将参数传递给匿名的lambda,可以将其放在lambda表达式后的括号中:

\begin{lstlisting}[style=styleCXX]
<< [](auto l, auto r){ return l + r; }(47, 73)
	<< '\n';
\end{lstlisting}

函数参数47和73传递给函数体后括号中的匿名lambda。

\item 
可以通过将变量包含在方括号中来访问lambda的外部作用域的变量:

\begin{lstlisting}[style=styleCXX]
int num{1};
p([num]{ return num; });
\end{lstlisting}

\item 
或者通过引用来捕获它们:

\begin{lstlisting}[style=styleCXX]
int num{0};
auto inc = [&num]{ num++; };
for (size_t i{0}; i < 5; ++i) {
	inc();
}
cout << num << '\n';
\end{lstlisting}

输出如下所示:

\begin{tcblisting}{commandshell={}}
5
\end{tcblisting}

可以修改捕获的变量。


\item 

也可以定义一个本地捕获变量来维护其状态:

\begin{lstlisting}[style=styleCXX]
auto counter = [n = 0]() mutable { return ++n; };
for (size_t i{0}; i < 5; ++i) {
	cout << format("{}, ", counter());
}
cout << '\n';
\end{lstlisting}

输出为:

\begin{tcblisting}{commandshell={}}
1, 2, 3, 4, 5,
\end{tcblisting}

可变说明符允许lambda修改它的捕获,lambda默认限定为const。

与尾部返回类型一样,任何说明符都需要参数圆括号。

\item 
lambda支持两种默认捕获类型:

\begin{lstlisting}[style=styleCXX]
int a = 47;
int b = 73;
auto l1 = []{ return a + b; };
\end{lstlisting}

若试图编译这段代码,会得到一个错误,其中包括:

\begin{tcblisting}{commandshell={}}
note: the lambda has no capture-default
\end{tcblisting}

一种默认捕获类型用等号表示:

\begin{lstlisting}[style=styleCXX]
auto l1 = [=]{ return a + b; };
\end{lstlisting}

这将捕获lambda作用域中的所有符号,等号通过复制执行捕获。将捕获对象的副本,就像使用赋值操作符复制对象一样。

另一个默认捕获使用\&号进行引用捕获:

\begin{lstlisting}[style=styleCXX]
auto l1 = [&]{ return a + b; };
\end{lstlisting}

这是通过引用进行捕获方式。

默认捕获只在引用时使用,所以并不像看起来那样混乱。我建议在可能的情况下使用显式捕获,这样可读性更高。

\end{itemize}

\subsubsection{How it works…}

lambda表达式的语法如下:

\begin{lstlisting}[style=styleCXX]
// Syntax of the lambda expression
[capture-list] (parameters)
	mutable   (optional)
	constexpr (optional)
	exception attr (optional)
	-> return type (optional)
{ body }
\end{lstlisting}

lambda表达式唯一需要的部分是捕获列表和函数体,函数体可以为空:

\begin{lstlisting}[style=styleCXX]
[]{}
\end{lstlisting}

这是最简单的lambda表达式,什么也没捕捉到,什么也没做。

来了解一下表达式的每一部分。

\noindent
\textbf{捕获列表}

捕获列表指定我们捕获什么(如果有的话)。不能省略,但可以是空的。可以使用[=]通过复制来捕获所有变量,或者使用[\&]通过引用来捕获lambda范围内的所有变量。

可以通过在括号中列出单个变量来捕获:

\begin{lstlisting}[style=styleCXX]
[a, b]{ return a + b; }
\end{lstlisting}

指定的捕获要复制的默认值,可以通过引用操作符来获取:

\begin{lstlisting}[style=styleCXX]
[&a, &b]{ return a + b; }
\end{lstlisting}

当通过引用捕获时,可以修改引用的变量。

\begin{tcolorbox}[colback=webgreen!5!white,colframe=webgreen!75!black,title=Note]
在类内使用lambda时,不能直接捕获对象成员,可以捕获this或*this来解除对类成员的引用。
\end{tcolorbox}

\noindent
\textbf{参数}

与函数一样,形参可在括号中指定:

\begin{lstlisting}[style=styleCXX]
[](int a, int b){ return a + b };
\end{lstlisting}

若没有参数、说明符或尾部返回类型,圆括号是可选的。说明符或尾部返回类型使括号成为必需项:

\begin{lstlisting}[style=styleCXX]
[]() -> int { return 47 + 73 };
\end{lstlisting}

\noindent
\textbf{可变修饰符(可选)}

lambda表达式默认为const限定的,除非指定了可变修饰符。这允许在const上下文中使用,但不能修改复制捕获的变量。例如:

\begin{lstlisting}[style=styleCXX]
[a]{ return ++a; };
\end{lstlisting}

这将导致编译失败,并出现如下错误消息:

\begin{tcblisting}{commandshell={}}
In lambda function:
error: increment of read-only variable 'a'
\end{tcblisting}

使用mutable修饰符,lambda不再是const限定的,捕获的变量可能会更改:

\begin{lstlisting}[style=styleCXX]
[a]() mutable { return ++a; };
\end{lstlisting}


\noindent
\textbf{constexpr说明符(可选)}

可以使用constexpr显式指定您希望将lambda视为常量表达式,其可以在编译时计算。弱lambda满足要求,即使没有说明符,也可以将其视为constexpr。

\noindent
\textbf{异常属性(可选)}

可以使用noexcept说明符表明lambda表达式不抛出任何异常。

\noindent
\textbf{尾部的返回类型(可选)}

默认情况下,lambda返回类型是从return语句中推导出来的,就像是一个wuto返回类型一样。可以选择使用->操作符指定一个尾部返回类型:

\begin{lstlisting}[style=styleCXX]
[](int a, int b) -> long { return a + b; };
\end{lstlisting}

若使用可选说明符或尾部?返回类型,则必须使用参数括号。

\begin{tcolorbox}[colback=webgreen!5!white,colframe=webgreen!75!black,title=Note]
包括GCC在内的一些编译器允许省略空参数括号,即使有说明符或尾随返回类型。这是不正确的!根据规范,当包含参数、说明符和尾部返回类型都是lambda声明器的一部分时,都需要括号。这在C++的未来版本中可能会改变。
\end{tcolorbox}














