
生产者-消费者问题的最简单版本是,一个进程生产数据,另一个进程消费数据,使用一个缓冲区或容器保存数据。这需要生产者和消费者之间的协调来管理缓冲区并防止不必要的副作用。

\subsubsection{How to do it…}

本节中,我们考虑了一个简单的解决方案,使用std::condition\_variable来协调这个过程:

\begin{itemize}
\item 
方便起见,我们从一些命名空间和别名声明开始说起:

\begin{lstlisting}[style=styleCXX]
using namespace std::chrono_literals;
namespace this_thread = std::this_thread;
using guard_t = std::lock_guard<std::mutex>;
using lock_t = std::unique_lock<std::mutex>;
\end{lstlisting}

lock\_guard和unique\_lock的别名,可以更容易的使用这些类型而不会出错。

\item 
有几个常量:

\begin{lstlisting}[style=styleCXX]
constexpr size_t num_items{ 10 };
constexpr auto delay_time{ 200ms };
\end{lstlisting}

可以把它们放在一个地方可以更安全、更容易地试验不同的值。

\item 
可以使用这些全局变量来协调数据存储:

\begin{lstlisting}[style=styleCXX]
std::deque<size_t> q{};
std::mutex mtx{};
std::condition_variable cond{};
bool finished{};
\end{lstlisting}

可以使用deque将数据保存在先进先出(FIFO)队列中。

mutex与condition\_variable一起使用,以协调数据从生产者到消费者的转移。

finished标志表示没有更多数据。

\item 
生产者线程将使用这个函数:

\begin{lstlisting}[style=styleCXX]
void producer() {
	for(size_t i{}; i < num_items; ++i) {
		this_thread::sleep_for(delay_time);
		guard_t x{ mtx };
		q.push_back(i);
		cond.notify_all();
	}
	guard_t x{ mtx };
	finished = true;
	cond.notify_all();
}
\end{lstlisting}

producer()函数循环num\_items迭代,并在每次循环中将一个数字压入deque。

这里,包括使用sleep\_for()来模拟产生每个值的延迟。

conditional\_variable需要一个互斥锁来操作,可以使用lock\_guard(通过guard\_t别名)来获取锁,然后将值推到deque上,然后在conditional\_variable上调用notify\_all()。这将告诉使用者线程有一个可用的新值。

当循环完成时,可以设置finished标志,并通知消费者线程生产者已经完成。

\item 
消费者线程等待来自生产者的每个值,将其显示在控制台上,并等待finished标志:

\begin{lstlisting}[style=styleCXX]
void consumer() {
	while(!finished) {
		lock_t lck{ mtx };
		cond.wait(lck, [] { return !q.empty() ||
			finished; });
		while(!q.empty()) {
			cout << format("Got {} from the queue\n",
				q.front());
			q.pop_front();
		}
	}
}
\end{lstlisting}

wait()方法等待生产者通知,使用lambda作为谓词继续等待,直到deque不为空或设置了finished标志。

当我们获取到一个值时显示它,然后将其从deque中弹出。

\item 
在main()中运行简单的线程对象:

\begin{lstlisting}[style=styleCXX]
int main() {
	thread t1{ producer };
	thread t2{ consumer };
	t1.join();
	t2.join();
	cout << "finished!\n";
}
\end{lstlisting}

输出为:

\begin{tcblisting}{commandshell={}}
Got 0 from the queue
Got 1 from the queue
Got 2 from the queue
Got 3 from the queue
Got 4 from the queue
Got 5 from the queue
Got 6 from the queue
Got 7 from the queue
Got 8 from the queue
Got 9 from the queue
finished!
\end{tcblisting}

注意,每一行之间有200毫秒的延迟,所以生产者和消费者之间的协调正在按照预期的方式工作。

\end{itemize}

\subsubsection{How it works…}

生产者-消费者问题需要在写入和读取缓冲区或容器之间进行协调。在这个例子中,容器是deque<size\_t>:

\begin{lstlisting}[style=styleCXX]
std::deque<size_t> q{};
\end{lstlisting}

当共享变量修改时,conditional\_variable类可以阻塞一个线程或多个线程。然后,通知其他线程该值可用。

condition\_variable需要mutex来执行锁定操作:

\begin{lstlisting}[style=styleCXX]
std::lock_guard x{ mtx };
q.push_back(i);
cond.notify_all();
\end{lstlisting}

std::lock\_guard会获取锁,从而可以将一个值推入到deque中。

condition\_variable上的wait()方法用于阻塞当前线程,直到收到通知:

\begin{lstlisting}[style=styleCXX]
void wait( std::unique_lock<std::mutex>& lock );
void wait( std::unique_lock<std::mutex>& lock,
	Pred stop_waiting );
\end{lstlisting}

wait()的谓词形式相当于:

\begin{lstlisting}[style=styleCXX]
while (!stop_waiting()) {
	wait(lock);
}
\end{lstlisting}

谓词形式用于防止在等待特定条件时出现伪唤醒。我们的例子中,将和lambda一起使用:

\begin{lstlisting}[style=styleCXX]
cond.wait(lck, []{ return !q.empty() || finished; });
\end{lstlisting}

这可以防止消费者在deque有数据或finished标志设置之前醒来。

condition\_variable类有两个通知方法:

\begin{itemize}
\item 
notify\_one()解除一个等待线程的阻塞

\item 
notify\_all()解除所有等待线程的阻塞
\end{itemize}

示例中使用了notify\_all()。因为只有一个使用者线程,所以任何一种通知方法的工作方式都是一样的。

\begin{tcolorbox}[colback=webgreen!5!white,colframe=webgreen!75!black,title=Note]
注意,unique\_lock是唯一在condition\_variable对象上支持wait()的锁。
\end{tcolorbox}






