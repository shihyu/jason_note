這一章,瞭解了主處理器的計算能力,以及如何有效地使用。高性能的關鍵是最大限度地利用所有可用的計算資源,同時計算兩個結果的程序比稍後計算第二個結果的程序快(假設計算能力可用)。CPU有很多用於各種計算的計算單元,大多數在特定時刻都是空閒的,除非程序得到了高度優化。

有效使用CPU指令級並行性的主要限制通常是數據依賴,沒有足夠的工作可以並行完成,以保持CPU繁忙。這個問題的硬件解決方案是流水線,CPU不僅在程序的當前點執行代碼,而且從未來進行一些計算,這些計算有數據依賴項,也能並行執行它們。只要未來可預知,這種方法就能很好地工作。若CPU不能確定這些計算,就不能執行超前的計算。當CPU必須等待決定下一步執行什麼機器指令時,流水線就會停止。為了減少這種情況的頻率,CPU有特殊的硬件,可以預測最有可能的情況,通過採樣條件代碼的路徑,並有根據的推測地執行代碼。因此,程序的性能在很大程度上取決於預測的效果。

我們已經瞭解了使用特殊工具來幫助測試代碼的效率,並識別限制性能的瓶頸。通過測試研究了幾種優化技術,這些技術可以使程序利用更多的CPU資源,減少等待時間,增加計算量,並提高性能。

本章中,我們略過了每個計算都必須做的事情:內存訪問。任何表達式的輸入都駐留在內存中,必須在剩下的計算髮生之前進入寄存器。中間結果可以存儲在寄存器中,但最終必須將某些內容寫回內存中,否則整個代碼不會產生效果。事實證明,內存操作(讀和寫)對性能有顯著影響。在許多程序中,內存操作是阻礙進一步優化的限制因素。下一章將專門研究CPU與內存的交互。







