
收集關於程序性能信息的最簡單的方法就是運行,並測量運行所花費的時間。當然,要進行有效的優化,需要更多的數據。最好知道程序的哪些部分耗時最長,其他代碼可能非常低效,但也只需要很少的時間,因此不會對最終結果有任何影響。

在示例程序中添加計時器後,我們知道了排序需要花費多長時間。簡而言之,這就是\textbf{基準測試}。其餘的工作都是體力活,用計時器檢測代碼,收集信息,並以可讀的格式進行報告。看看有什麼工具可以做到這一點,先從語言本身提供的計時器開始。

\subsubsubsection{2.3.1\hspace{0.2cm}C++的chrono計時器}

C++在chrono庫中有一些工具可以用來收集計時信息,可以測量程序中經過任意兩點之間所需的時間:

%\hspace*{\fill} \\ %插入空行
\noindent
\textbf{example.C}
\begin{lstlisting}[style=styleCXX]
#include <chrono>
using std::chrono::duration_cast;
using std::chrono::milliseconds;
using std::chrono::system_clock;
  …
auto t0 = system_clock::now();
  … do some work …
auto t1 = system_clock::now();
auto delta_t = duration_cast<milliseconds>(t1 – t0);
cout << "Time: " << delta_t.count() << endl;
\end{lstlisting}

應該指出的是,C++計時時鐘測量的是實時時間(通常稱為\textit{掛鐘時間})。通常,這就是想要測量的時間。更詳細的分析通常需要測量CPU時間,即CPU的工作時間,以及CPU空閒的時間。單線程程序中,CPU時間不能大於實際時間。當程序是計算密集型時,這兩個時間理論上相同,這意味著完全使用CPU進行計算了。另一方面,用戶界面程序會把大部分時間都花在等待用戶和閒置CPU上,所以我們希望CPU時間儘可能的短,這樣的程序就是高效的,並且使用盡可能少的CPU資源來滿足用戶的請求。


\subsubsubsection{2.3.2\hspace{0.2cm}高精度計時器}

為了測試CPU時間,我們必須使用特定於操作系統的系統函數,在Linux和其他posix兼容的系統上,可以使用\texttt{clock\_gettime()}來使用硬件的高精度計時器:

%\hspace*{\fill} \\ %插入空行
\noindent
\textbf{clocks.C}
\begin{lstlisting}[style=styleCXX]
timespec t0, t1;
clockid_t clock_id = …; // Specific clock
clock_gettime(clock_id, &t0);
  … do some work …
clock_gettime(clock_id, &t1);
double delta_t = t1.tv_sec – t0.tv_sec +
  1e-9*(t1.tv_nsec – t0.tv_nsec);
\end{lstlisting}

第二個參數的函數返回當前時間,\texttt{tv\_sec}是過去某一時刻到現在的秒數,\texttt{tv\_nsec}是上一整個秒到現在的納秒數。時間的起始是多少並不重要,因為我們總是測量時間間隔。這裡要先減去秒,然後再加納秒。

在前面的代碼中已經使用了幾個硬件計時器,可以通過\texttt{clock\_id}的值進行選擇,這些計時器與我們已經使用過的系統或實時時鐘相同,其ID為\texttt{CLOCK\_REALTIME}。我們感興趣的另外兩個計時器是兩個CPU計時器:\texttt{CLOCK\_PROCESS\_CPUTIME\_ID}是測量當前程序使用CPU時間的計時器,而\texttt{CLOCK\_THREAD\_CPUTIME\_ID}是測量線程調用使用時間的計時器。

對代碼進行基準測試時,使用多個計時器進行測量通常很有幫助。一個單線程程序若進行不間斷的計算時,三個計時器應該返回相同的結果:

%\hspace*{\fill} \\ %插入空行
\noindent
\textbf{clocks.C}
\begin{lstlisting}[style=styleCXX]
double duration(timespec a, timespec b) {
	return a.tv_sec - b.tv_sec + 1e-9*(a.tv_nsec - b.tv_nsec);
}
…
{
	timespec rt0, ct0, tt0;
	clock_gettime(CLOCK_REALTIME, &rt0);
	clock_gettime(CLOCK_PROCESS_CPUTIME_ID, &ct0);
	clock_gettime(CLOCK_THREAD_CPUTIME_ID, &tt0);
	constexpr double X = 1e6;
	double s = 0;
	for (double x = 0; x < X; x += 0.1) s += sin(x);
	timespec rt1, ct1, tt1;
	clock_gettime(CLOCK_REALTIME, &rt1);
	clock_gettime(CLOCK_PROCESS_CPUTIME_ID, &ct1);
	clock_gettime(CLOCK_THREAD_CPUTIME_ID, &tt1);
	cout << "Real time: " << duration(rt1, rt0) << "s, "
	  "CPU time: " << duration(ct1, ct0) << "s, "
	  "Thread time: " << duration(tt1, tt0) << "s" <<
	  endl;
}
\end{lstlisting}

這裡的“CPU密集型的工作”是一種計算,所以三個計時器的時間幾乎相同。時間將取決於計算機的運行速度,結果是這樣的:

\begin{tcblisting}{commandshell={}}
Real time: 0.3717s, CPU time: 0.3716s, Thread time: 0.3716s
\end{tcblisting}

如果CPU時間與實際時間不匹配,則很可能是機器過載(許多其他進程正在爭奪CPU資源),或程序耗盡內存(如果程序使用的內存超過機器上的物理內存,將使用慢得多的磁盤進行數據交換,當程序等待從磁盤調入內存時,CPU無法執行任何工作)。

另外,如果沒有太多的計算,而是等待用戶輸入,或者從網絡接收數據,亦或做一些其他不佔用太多CPU資源的工作,則會看到不同的結果。觀察這種行為的最簡單方法是調用\texttt{sleep()}函數:

\hspace*{\fill} \\ %插入空行
\noindent
\textbf{clocks.C}
\begin{lstlisting}[style=styleCXX]
{
	timespec rt0, ct0, tt0;
	clock_gettime(CLOCK_REALTIME, &rt0);
	clock_gettime(CLOCK_PROCESS_CPUTIME_ID, &ct0);
	clock_gettime(CLOCK_THREAD_CPUTIME_ID, &tt0);
	sleep(1);
	timespec rt1, ct1, tt1;
	clock_gettime(CLOCK_REALTIME, &rt1);
	clock_gettime(CLOCK_PROCESS_CPUTIME_ID, &ct1);
	clock_gettime(CLOCK_THREAD_CPUTIME_ID, &tt1);
	cout << "Real time: " << duration(rt1, rt0) << "s, "
	  "CPU time: " << duration(ct1, ct0) << "s, "
	  "Thread time: " << duration(tt1, tt0) << "s" <<
  	  endl;
}
\end{lstlisting}

可以看到休眠程序幾乎不怎麼使用CPU:

\begin{tcblisting}{commandshell={}}
Real time: 1.000s, CPU time: 3.23e-05s, Thread time: 3.32e-05s
\end{tcblisting}

對於傳輸套接字或讀取文件阻塞的程序,或者正在等待用戶操作的程序,也是這樣幾乎不怎麼使用CPU。

目前為止,還沒有看到兩個CPU計時器之間的區別(除非程序使用線程,否則不會看到區別)。我們可以讓這個需要大量計算的程序,完成同樣的工作,但為其創建一個單獨的線程進行計算:

\hspace*{\fill} \\ %插入空行
\noindent
\textbf{clocks.C}
\begin{lstlisting}[style=styleCXX]
{
	timespec rt0, ct0, tt0;
	clock_gettime(CLOCK_REALTIME, &rt0);
	clock_gettime(CLOCK_PROCESS_CPUTIME_ID, &ct0);
	clock_gettime(CLOCK_THREAD_CPUTIME_ID, &tt0);
	constexpr double X = 1e6;
	double s = 0;
	auto f = std::async(std::launch::async,
	  [&]{ for (double x = 0; x < X; x += 0.1) s += sin(x);
	  });
	f.wait();
	timespec rt1, ct1, tt1;
	clock_gettime(CLOCK_REALTIME, &rt1);
	clock_gettime(CLOCK_PROCESS_CPUTIME_ID, &ct1);
	clock_gettime(CLOCK_THREAD_CPUTIME_ID, &tt1);
	cout << "Real time: " << duration(rt1, rt0) << "s, "
	  "CPU time: " << duration(ct1, ct0) << "s, "
	  "Thread time: " << duration(tt1, tt0) << "s" <<
	  endl;
}
\end{lstlisting}

總的計算量保持不變,並且只有一個線程在執行這項工作,因此我們不期望對實時性或整個進程的CPU時間有任何改進。不過,調用計時器的線程現在是空閒的,它所做的就是等待\texttt{std::async}返回\texttt{future},直到工作完成。這種等待與前面例子中的\texttt{sleep()}非常類似:

\begin{tcblisting}{commandshell={}}
Real time: 0.3774s, CPU time: 0.377s, Thread time: 7.77e-05s
\end{tcblisting}

現在,實時和進程的CPU時間與“計算密集型”示例中的CPU時間相似,但線程特定的CPU時間較低,就像“休眠”示例中的CPU時間一樣。因為程序都在做大量的計算,但是使用計時器的線程,大部分時間都在休眠。

大多數情況下,如果打算使用線程進行計算,我們的目標是更快地進行更多的計算,因此會使用幾個線程,並將不同的工作分配給它們。我們修改一下前面的例子,讓其在主線程上也能進行計算:

\hspace*{\fill} \\ %插入空行
\noindent
\textbf{clocks.C}
\begin{lstlisting}[style=styleCXX]
{
	timespec rt0, ct0, tt0;
	clock_gettime(CLOCK_REALTIME, &rt0);
	clock_gettime(CLOCK_PROCESS_CPUTIME_ID, &ct0);
	clock_gettime(CLOCK_THREAD_CPUTIME_ID, &tt0);
	constexpr double X = 1e6;
	double s1 = 0, s2 = 0;
	auto f = std::async(std::launch::async,
	  [&]{ for (double x = 0; x < X; x += 0.1) s1 += sin(x);
	  });
	for (double x = 0; x < X; x += 0.1) s2 += sin(x);
	f.wait();
	timespec rt1, ct1, tt1;
	clock_gettime(CLOCK_REALTIME, &rt1);
	clock_gettime(CLOCK_PROCESS_CPUTIME_ID, &ct1);
	clock_gettime(CLOCK_THREAD_CPUTIME_ID, &tt1);
	cout << "Real time: " << duration(rt1, rt0) << "s, "
	  "CPU time: " << duration(ct1, ct0) << "s, "
	  "Thread time: " << duration(tt1, tt0) << "s" <<
	  endl;
}
\end{lstlisting}

兩個線程都在進行計算,因此程序的CPU時間是實際時間的兩倍:

\begin{tcblisting}{commandshell={}}
Real time: 0.5327s, CPU time: 1.01s, Thread time: 0.5092s
\end{tcblisting}

這還不錯!我們在0.53秒的實時時間內完成了1秒的計算。理想情況下,這應該是0.5秒,但啟動線程和等待線程會有一定的開銷。另外,兩個線程中的一個可能會花費稍長的時間來完成工作,而另一個線程在某些時候可能空閒。

對程序進行基準測試是一種收集性能數據的方法。通過觀察執行一個函數,或處理一個事件所花費的時間,可以對代碼的性能瞭解很多。在計算密集型的代碼中,可以瞭解程序是否在不間斷地進行計算,或者正在等待什麼。對於多線程程序,可以測量併發的有效性和開銷。不僅限於收集執行時間,還可以輸出相關的任何計數和值,比如:調用函數的次數、排序的平均字符串的長度等,以便進行分析程序的各項指標。

然而,這種靈活性是有代價的。使用基準測試,可以瞭解關於程序性能的問題。不過,這裡只報告了測量數據,如果想知道某個函數需要多長時間,就必須給它添加計時器。問題是,在代碼中到處撒計時器是不行的,這些函數的調用代價相當昂貴,所以使用太多計時器會減慢程序的速度,並嚴重影響性能測試。若具有經驗和良好的編碼規則,可以提前編寫一些代碼,這樣就可以針對其主要部分進行基準測試了。

如果不知道從哪裡開始,該怎麼做?如果接手了一個沒有進行任何基準測試的代碼庫,該怎麼辦?或者,性能瓶頸存在於一大坨代碼中,但其中沒有計時器,該怎麼辦?一種方法是繼續測試代碼,直到有足夠的數據來分析問題。不過,這種暴力的方法很慢,所以需要一些關於性能數據分析的引導。這就是\textbf{分析工具}的作用所在,它可以自動收集程序的性能數據,而非手工檢測,從而便於進行簡單的基準測試。我們將在下一節中學習如何使用性能分析工具。






