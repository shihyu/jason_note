首先,需要一個C++編譯器。本章的示例使用GCC或Clang編譯器,並在Linux上進行編譯。所有的Linux發行版都會將GCC作為常規安裝的一部分,發行版中可能有較新的編譯器版本。Clang編譯器可以通過LLVM項目\url{http://llvm.org/}獲得。Windows上,Microsoft Visual Studio是最常用的編譯器,當然GCC和Clang也可以使用。

其次,需要一個分析工具。本章中,我們將使用Linux的perf性能分析器,其在大多數Linux發行版上都已安裝(或可用於安裝)。文檔的地址:\url{https://perf.wiki.kernel.org/index.php/Main_Page}。

還會演示另一個分析器的使用,來自於谷歌性能工具集(GperfTools)的CPU分析器,地址為:\url{https://github.com/gperftools/gperftools}(同樣,可以通過其源碼進行安裝)。

還有是許多其他可用的分析工具,有免費的,也有商業的。它們以不同的方式展示了相同類型的信息,但會提供許多不同的分析選項進行呈現。通過本章的示例,可以瞭解什麼是分析工具,以及可能存在的限制。有著良好紀律性的開發者,會對使用的工具細節進行詳細的瞭解。

最後,使用微基準測試工具。本章中使用了\url{https://github.com/google/benchmark}谷歌基準庫,需要自己下載和安裝(即使與Linux發行版一起安裝,版本也很可能過時),請按照頁面上的說明進行安裝。

安裝了所有必要的工具後,就可以進行第一次性能測試了。

本章代碼地址: \url{https://github.com/PacktPublishing/The-Art-of-Writing-Efficient-Programs/tree/master/Chapter02}。



