本章中,我們瞭解了內存系統是如何工作的:緩慢的工作,CPU會因內存的低性能而性能變低。但是存儲器間距也包含了潛在解決方案,可以用多個CPU操作來換取一個內存訪問。

瞭解到內存系統非常複雜,並且有層級,所以內存系統沒有純粹的速度。如果內存使用的情況非常糟糕,這可能會嚴重影響程序的性能。同樣,也可以把內存看作是一個機會而不是負擔:從優化內存訪問中獲得的收益可能非常大,以至於超過了開銷。

硬件本身提供了幾種工具來提高內存性能。除此之外,還必須使用內存高效的數據結構。如果還不夠,還要選擇內存高效的算法來提高性能。通常,所有的性能決策都必須由測試指導和支撐。

目前為止,我們所有工作和測試都在單個CPU上進行。幾乎沒有提到現在的每一臺計算機都有多個CPU核,而且經常有多個物理處理器。原因很簡單:我們必須學會有效地使用單CPU,然後才能繼續討論更復雜的多CPU問題。下一章中,我們將注意力轉向併發,並有效地使用多核和多處理器系統。