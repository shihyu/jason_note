動機是學習的關鍵,因此理解為什麼在計算硬件方面有眾多改進的情況下,開發者仍需要努力編寫高性能的代碼?為什麼今天還需要對計算硬件、編程語言和編譯器能力有深刻的理解?本章就來回答這些基本問題。

本章討論了關注性能的原因,特別是“為什麼好性能不能憑空產生”。為了達到最優性能,甚至是足夠的性能,需要了解影響性能的不同因素,以及程序特定行為的原因,無論是快速執行還是緩慢執行。

本章將討論以下內容:

\begin{itemize}
\item 為什麼會有性能問題?
\item 為什麼需要開發者關注性能?
\item 性能的含義是什麼?
\item 如何評估性能
\item 瞭解高性能
\end{itemize}














