指標的概念是性能的基礎,其隱含著可能性和必要性.如果說“有一個指標”,就意味著有一種量化和衡量的方法,而獲得指標值的唯一方法就是測試。

衡量性能的重要性怎麼強調都不為過,性能的第一定律就是\textit{永遠不要去猜測性能}。本書的下一章將專注於性能測試、測試工具、如何使用測試,以及如何分析結果。

不幸的是,很多時候會開發者會對性能進行猜測。還有一些籠統描述語句,如“避免在C++中使用虛函數,它們很慢。”其問題在於描述不準確,並且這裡並沒有說明虛函數相對於非虛函數慢多少。作為給讀者的練習,這裡有幾個答案可供選擇,已經進行量化:

\begin{itemize}
\item 虛函數慢100\%
\item 虛函數慢15\textasciitilde20\%
\item 虛函數對程序沒什麼影響
\item 虛函數快10\textasciitilde20\%
\item 虛函數慢100倍
\end{itemize}

哪個是正確答案呢?如果選擇了這些答案中的任何一個,恭喜你,選擇了正確的答案。沒錯,在特定的環境和特定的上下文中,這些答案都是正確的(要了解原因,需要等到第9章)。

通過接受直覺或猜測性能是不現實的,從而會有落入另一個陷阱的風險。因為我們不猜測性能,所以可以作為藉口編寫低效的代碼,從而“稍後進行優化”。

性能指標不能在後期添加,所以在最初的設計和開發中就應該去考慮。與其他設計目標一樣,性能因素和目標也應在設計階段佔有一席之地。這些早期的目標和永遠不要猜測性能的規則之間存在著矛盾,我們必須找到折衷方案,描述設計階段想要實現的性能目標是一個好辦法。雖然提前知道了不存在最佳優化,但可以確定的是後期優化會很困難,甚至需要重新設計。

開發過程中也會出現同樣的情況,比如:花很長時間優化一個每天只調用一次、只需要一秒鐘的函數是愚蠢的。另一方面,將這些代碼封裝到一個函數中則非常明智的做法。因此,隨著程序的發展,當使用模式發生了變化,則可以以後再進行優化,而無需重寫其餘部分的代碼。

另一種描述“不提前優化規則”的侷限性,並通過“是”來進行限定,但也不要過於悲觀。認識到兩者的差異需要良好的設計實踐/知識,以及加深對編程的理解,從而才能獲得高性能。

那麼,作為一名開發人員/編程者,為了精通開發高性能應用程序技能,需要學習和理解什麼呢?下一節,我們將從這些目標開始,然後進行詳細討論。
















