未定義行為(UB)的概念有些神祕,對不瞭解情況的人來說,它是一種未初始化的警告。Usenet組織\texttt{comp.std.c}警告說:“\textit{當編譯器遇到(一個未定義的結構)時,可以讓惡魔從你的鼻子裡飛出來。}”還有,發射核導彈和閹割你的貓(即使你沒有養貓)等類似言論,都是在類似的背景下提出的。本章要揭開UB的面紗,雖然最終目標是解釋UB與性能之間的關係,並展示如何利用UB,但只有在能夠理性地討論這一概念時才能做到。

首先,在C++(或其他編程語言)的上下文中,UB是什麼?在標準中有使用了未定義行為或格式不良的程序。標準進一步說,如果行為沒有定義,則標準對結果沒有任何要求。相應的情況稱為UB,如下代碼:

\begin{lstlisting}[style=styleCXX]
int f(int k) {
	return k + 10;
}
\end{lstlisting}

標準規定,如果加法導致整數溢出(如果\texttt{k}大於\texttt{INT\_MAX-10}),則上述代碼的結果未定義。

提到UB時,討論傾向於兩個極端,剛剛看到的第一個。誇張的語言可能是對UB危險的善意警告,但也會成為理性解釋的障礙,因為鼻子和貓都不會受到編譯器的攻擊。編譯器最終會根據程序生成一些代碼,然後開發者會運行這些代碼。它不會賦予計算機任何超能力:這個程序做的事情,開發者都可以自主完成,例如:用匯編程序手工編寫一個相同的指令序列。如果開發者都無法執行\textit{發射核導彈}的機器指令,那麼編譯器也無法做到,不論UB是否存在(當然,如果正在為導彈發射控制器編程,那就是另外一個故事了)。當程序的行為未定義時,編譯器根據標準可以生成開發者不期望的代碼。

雖然誇大UB的危險沒什麼意思,但出現了針對UB進行推理的傾向,這也是一種不祥的做法。例如以下代碼:

\begin{lstlisting}[style=styleCXX]
int k = 3;
k = k++ + k;
\end{lstlisting}

雖然C++標準已經逐步收緊了執行這類表達式的規則,但這個特定表達式的結果在C++17中仍沒有定義。許多開發者低估了這種情況的危險性,會覺得“編譯器要麼先求k++,要麼先求k+k”為瞭解釋這裡的錯誤和危險,必須先對標準進行分析。

C++標準有三個容易混淆的行為類別:定義的實現、未指定和未定義。實現必須提供實現定義的行為的確切規範。這不可選,符合標準的實現必須通過語言構造的行為來符合標準的描述。未指定行為與此類似,但實現沒有義務記錄該行為:標準通常提供可能的結果,而實現可以指定自己的結果,而不指定會是哪一個。最後,對於未定義行為,標準沒有對整個程序的行為強加要求。標準沒有說計算表達式\texttt{k++ + k}的幾種備選方法中必須有一種發生(這將是未指定的行為,這不是標準所說的)。標準說整個程序都是病態的,並且對其結果沒有任何限制(在為鼻子感到恐慌和恐懼之前,結果會限制為一些可執行代碼)。

無論編譯器在編譯有UB代碼行的時候做什麼,必須以標準的方式處理代碼的其餘部分。因此(這個論證是這樣的),損壞僅限於特定代碼行可能的結果。正如不誇大危險一樣,理解這種觀點為什麼是錯誤的也很重要。編譯器是基於這樣的假設進行的:程序定義良好,在這種情況下(且僅在這種情況下)需要生成正確的結果。如果違背了這個假設,就沒有什麼先入為主的概念。描述這種情況的方法是,編譯器零容忍UB。回到第一個例子:

\begin{lstlisting}[style=styleCXX]
int f(int k) {
	return k + 10;
}
\end{lstlisting}

由於程序定義不明,\texttt{k}大到足以引起整數溢出,因此編譯器可以假定這種情況永遠不會發生。那真的發生了呢?如果單獨編譯這個函數(在一個單獨的編譯單元中),編譯器會生成一些代碼,為所有\texttt{k <= INT\_MAX-10}生成正確的結果。如果在編譯器和鏈接器中沒有整個程序的轉換,那麼對於更大的\texttt{k},同樣的代碼可能會執行,結果與硬件在這種情況下所做事情一樣。編譯器可以插入對\texttt{k}的檢查,也可能不會(但在某些編譯器選項中可能會)。

如果函數是更大的編譯單元的一部分呢?這就是有趣的地方了。編譯器現在知道\texttt{f()}函數的輸入參數是受限制的,那麼這些信息就可以用於優化了。代碼如下:

\hspace*{\fill} \\ %插入空行
\noindent
\textbf{01\_opt.C}
\begin{lstlisting}[style=styleCXX]
int g(int k) {
	if (k > INT_MAX-5) cout << "Large k" << endl;
	return f(k);
}
\end{lstlisting}

如果\texttt{f()}函數的定義對編譯器可見,編譯器可以推斷出打印輸出從來沒有發生過。如果\texttt{k}足夠大,這個程序可以打印,那麼整個程序就是病態的,標準不要求它打印。如果\texttt{k}的值在某個範圍內,程序將不會打印。無論哪種方式,根據標準,什麼都不打印都是有效結果。請注意,僅僅因為編譯器目前沒有做優化,並不意味著它永遠不會做,這種類型的優化在新的編譯器中可能會更加激進。

那麼第二個例子呢?表達式\texttt{k++ + k}的結果對於\texttt{k}的值都是未定義的。編譯器可以用做什麼呢?同樣,編譯器不需要容忍UB。這個程序能夠保持良好定義的唯一方法是永遠不執行這一行。編譯器可以先這樣假設,然後進行反向推理,從而使這段代碼的函數永遠不會調用。

如果認為編譯器不會做那些事情,那這裡有一個驚喜:

\hspace*{\fill} \\ %插入空行
\noindent
\textbf{02\_infC}
\begin{lstlisting}[style=styleCXX]
int i = 1;
int main() {
	cout << "Before" << endl;
	while (i) {}
	cout << "After" << endl;
}
\end{lstlisting}

這個程序的期望是打印\texttt{Before}並永遠掛起。當用GCC(版本9,優化O3)編譯時,行為與期望一致。當使用Clang(版本13,也是O3)編譯時,會打印Before,然後After,然後立即終止,沒有任何錯誤(不會崩潰,只是退出)。這兩個結果都是有效的,因為遇到無限循環的程序的結果是未定義的(除非滿足某些條件,而這裡沒有適用的條件)。

前面的例子對於理解為什麼有UB具有指導意義。下一節中,將揭開面紗並解釋UB產生的原因。

























