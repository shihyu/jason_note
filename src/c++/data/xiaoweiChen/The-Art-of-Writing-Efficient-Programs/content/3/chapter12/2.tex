優秀的設計是否有助於實現良好的性能,或者確定是否需要犧牲最佳設計實踐來實現最佳性能?這些問題在編程界已經爭論了很久。通常,設計倡導者會認為,若需要在良好的設計和良好的性能之間做選擇,那麼設計者的能力太水。另一方面,黑客(我們使用這個術語的傳統意義,將解決方案組合在一起的開發者,與犯罪方面無關)通常將設計指南視為最佳優化的約束條件。 

這兩種觀點在一定程度上都是正確的。如果把它們視為“全部的真相”,也不對。許多設計實踐在應用於特定軟件系統時,會限制性能,否認這一點就很蠢。另外,許多實現和維護有效代碼的指南也是可靠的設計建議,可以提高性能和設計質量。 

我們對設計和性能之間的關係採取了更微妙的看法。對於特定的系統(開發者最感興趣的是還是正在開發的系統),一些設計指南和實踐確實會導致效率和性能低下。很難想出一個總是與效率相對立的設計規則,但是對於特定的系統,也許在某些特定的環境中,這樣的規則和實踐就很常見。若採用了遵循這些規則的設計,那麼可能會將效率低下的設計嵌入到軟件系統的核心架構中,這將很難通過“優化”進行彌補,很可能就是要重寫程序的關鍵部分。任何忽視或粉飾這個陷阱的潛在嚴重性的開發者,都無法獲得最佳性能。另外,聲稱這是放棄可靠設計實踐的理由的人都是錯誤的,他們的選擇過於二元化。 

如果意識到特定的設計方法可以遵循的實踐,就可以提高設計的清晰度和可維護性,雖然降低了性能,但也很好的設計方法。換句話說,雖然一些好的設計產生了糟糕的性能,但是對於一個給定的軟件系統來說,每一個好的設計都會導致低效,聽起來就很荒謬。所需要做的就是從幾種可能的高質量設計中,選擇一種具有良好性能的設計。 

當然,說起來容易做起來難,但希望這本書能有所幫助。本章的其餘部分,我們將關注問題的兩個方面。首先,當考慮性能時,建議採用什麼設計實踐?其次,當沒有可以運行和測試的程序,但有一個(可能不完整的)設計時,如何評估可能的性能影響?

如果仔細閱讀了最後兩段,會發現性能是一種設計考慮因素,就像在設計中考慮“支持許多用戶”或“在磁盤上存儲TB級數據”等需求一樣,性能目標也是需求的一部分,應該在設計階段明確考慮。這將我們引向設計高性能系統,它是……


































































