上一章中,詳細地探討了可用於確保併發程序正確性的同步。還研究了這些程序的構建塊,線程安全的計數器和指針。

本章中,將繼續研究併發程序的數據結構。本章的目標有兩個:一方面,將瞭解如何設計基本數據結構的線程安全版本。另一方面,給出一些一般原則和觀察,這些對於設計用於併發的數據結構非常重要,對於評估組織方式和存儲數據的最佳方法也非常重要。

本章將討論以下內容:

\begin{itemize}
\item
理解線程安全的數據結構,包括線性容器:堆棧和隊列;和基於節點的容器:鏈表

\item
提高併發性、性能和訪問順序的保證

\item
設計線程安全數據結構的建議

\end{itemize}















