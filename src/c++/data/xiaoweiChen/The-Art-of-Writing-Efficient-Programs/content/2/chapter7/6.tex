設計併發數據結構是困難的,應該利用一切機會進行簡化。應用程序使用特定的數據結構,根據相應的限制,可以讓數據結構更簡單和更快。 

首先,必須做出的決定是代碼的哪些部分需要線程安全,哪部分不需要。最好的解決方案是給每個線程自己的數據,單個線程使用的自己數據,根本不需要考慮線程安全的問題。當這沒辦法使用時,尋找其他特定於應用程序的限制,是否有多個線程修改特定的數據結構?如果只有一個寫線程,實現通常會更簡單。有什麼特定於應用程序的保證可以利用嗎?知道數據結構的最大尺寸嗎?是否需要同時從數據結構中刪除數據以及添加數據,或者可以及時分離這些操作?在一些數據結構不變的情況下,是否存在定義良好的時間段?如果是,則不需要任何同步來讀取。這些和許多其他特定於應用程序的限制,可以用來提高數據結構的性能。 

第二個重要決策是,支持數據結構上的哪些操作?重申最後一段的另一種方法是\textit{實現最小化的必要接口}。實現的接口必須是事務性的,每個操作都必須具有定義良好的數據結構狀態和行為。只有在數據結構處於某種狀態時才有效的操作,不能在併發程序中安全使用,除非使用客戶端將多個操作鎖定組合到一個事務中(在這種情況下,這些組合可能一開始就是一個操作)。

本章還介紹了幾種實現不同類型數據結構的方法,以及評估其性能的方法。最終,只有在實際應用的背景下,使用實際數據才能獲得準確的性能。然而,有用的近似基準測試,可以在開發和評估潛在的替代方案時節省大量時間。 