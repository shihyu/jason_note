\begin{enumerate}
\item 
需要性能測試有兩個主要原因。首先,用來定義目標和描述當前狀態,沒有這樣的衡量指標,就不能說性能是好還是不好,也不能判斷是否達到了目標。其次,測量用於研究各種因素對性能的影響,評估代碼更改和其他優化的結果。

\item 
沒有一種方法可以衡量所有情況下的性能,因為使用方法通常會有太多的影響因素和原因需要分析,而且需要大量的數據來充分描述性能。

\item 
通過代碼的手工工具進行基準測試的優點是,可以收集想要的數據,並且很容易將數據放在上下文中,明確的知道每一行代碼屬於哪個函數或算法的哪個步驟。主要的侷限性在於這種方法的侵掠性:必須知道代碼的哪些部分可以進行檢測,並且能夠這樣做,任何未被數據收集工具覆蓋的代碼區域都不會進行測量。

\item
分析用於收集關於程序中執行時間或其他測試的數據。可以在功能或模塊級別上完成,也可以在更低的級別上完成,只需要一條機器指令。然而,一次收集整個程序的最低詳細級別的數據通常是不現實的,因此程序通常是分階段進行的,從粗粒度到細粒度進行分析。

\item
小規模和微基準測試用於對代碼更改進行快速迭代,並評估它們對性能的影響。還可以用於詳細分析小代碼片段的性能。必須確保微基準測試中的執行上下文與實際程序的上下文儘可能相似。
	
\end{enumerate}