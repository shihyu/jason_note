\begin{enumerate}
\item 
未定義行為是指程序在約定之外執行時所發生的情況,規範規定了有效的輸入是什麼以及結果應該是什麼。當檢測到無效輸入是,這也是約定的一部分。沒有檢測到無效輸入,程序繼續(錯誤)。假設輸入有效,結果未定義,規範沒有說明必須發生什麼。

\item 
C++允許未定義行為有兩個主要原因。首先,有些操作需要硬件支持,或者在不同的硬件上需要執行不同的操作。在某些硬件系統上得到特定的結果可能非常困難,甚至是不可能的。第二個原因是性能:在所有計算體系結構中保證特定結果的代價非常昂貴。

\item 
不,一個未定義的結果並不意味著這個結果一定是錯誤的。合理的結果在未定義的行為下也是允許的,只是不能保證而已。此外,未定義行為會汙染整個程序。將文件中的相同代碼與其他代碼一起編譯,可能會產生意想不到的結果。新版本的編譯器會在未定義行為不會發生的假設下進行更好的優化,所以開發者應該運行殺滅工具,並修復工具報告出來的錯誤。

\item
出於同樣的原因,C++標準保證了性能。如果不增加“正常”情況的開銷,就很難正確處理特殊情況,所以可以選擇根本不處理特殊情況。更可取的方式是在運行時檢測這種情況,這種檢測可能也很昂貴。在這種情況下,驗證輸入應該是可選的。用戶提供了無效輸入,但是沒有運行檢測工具,從而程序的行為沒有定義,因為算法本身假設輸入是有效的,從而導致違背了這個假設。

\end{enumerate}