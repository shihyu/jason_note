\begin{enumerate}
\item 
任何為線程安全而設計的數據結構,都必須有事務接口。如果沒有標準對存在線程的C++程序提供一些保證,就不可能編寫任何可移植的併發C++程序。早在C++11之前就在C++中使用了併發性,但這是由遵循額外標準(如POSIX)的編譯器作者實現的。這種情況的缺點是擴展標準各有不同,在沒有條件編譯和針對每個平臺的特定於操作系統的擴展的情況下,沒有可移植的方法來編寫Linux和Windows的併發程序。類似地,原子操作實現為特定於CPU的擴展。此外,不同編譯器所遵循的標準之間也有一些差異,這有時會導致非常難以發現的錯誤。

\item 
並行算法的使用非常簡單,具有並行版本的算法都可以以執行策略作為第一個參數來調用。若是並行執行策略,則算法將在多個線程上運行。另一方面,為了達到最佳性能,可能需要重新設計程序的某些部分。特別是,當數據序列太短(什麼是短取決於算法和操作數據元素的成本),並行算法就有優勢了。因此,可能有必要重新設計程序,從而操作更大的序列。

\item 
協程是可以暫停正在執行的函數。掛起後,控制權返回給調用者(若這不是第一次掛起,則返回給斷點)。協程可以從代碼中的任何位置恢復,從不同的函數或另一個協程,甚至從另一個線程恢復。

\end{enumerate}