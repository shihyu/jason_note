
std::invoke()只是标准库为实现泛型库提供的有用实用程序的一个例子。接下来,将看看其他一些重要的特性。

\subsubsubsection{11.2.1\hspace{0.2cm}类型特征}

标准库提供了各种类型特征工具,允许计算和修改类型。支持泛型,即代码必须适应实例化类型的功能或对其作出反应。例如:

\begin{lstlisting}[style=styleCXX]
#include <type_traits>
template<typename T>
class C
{
	// ensure that T is not void (ignoring const or volatile):
	static_assert(!std::is_same_v<std::remove_cv_t<T>,void>,
					"invalid instantiation of class C for void type");
public:
	template<typename V>
	void f(V&& v) {
		if constexpr(std::is_reference_v<T>) {
			... // special code if T is a reference type
		}
		if constexpr(std::is_convertible_v<std::decay_t<V>,T>) {
			... // special code if V is convertible to T
		}
		if constexpr(std::has_virtual_destructor_v<V>) {
			... // special code if V has virtual destructor
		}
	}
};
\end{lstlisting}

正如这个例子所示,通过检查某些条件,可以在不同的模板实现之间进行选择。这里,使用编译时if特性,从C++17开始就可用了(请参阅8.5节),但也可以使用std::enable\_if、偏特化或SFINAE来启用或禁用助手模板(请参阅第8章的详细信息)。

必须特别小心地使用类型特征:其行为可能与(菜鸟期)开发者所期望有所不同。例如:

\begin{lstlisting}[style=styleCXX]
std::remove_const_t<int const&> // yields int const&
\end{lstlisting}

这里,因为引用不是const(尽管不能修改),所以调用没有效果,并生成传递类型。

因此,删除引用和const的顺序很重要:

\begin{lstlisting}[style=styleCXX]
std::remove_const_t<std::remove_reference_t<int const&>> // int
std::remove_reference_t<std::remove_const_t<int const&>> // int const
\end{lstlisting}

相反,可以只调用

\begin{lstlisting}[style=styleCXX]
std::decay_t<int const&> // yields int
\end{lstlisting}

但也可以将数组和函数转换为相应的指针类型。有一些情况下,类型特征是有限制的,不满足这些条件会导致未定义行为。

\begin{tcolorbox}[colback=webgreen!5!white,colframe=webgreen!75!black]
\hspace*{0.75cm}有人建议C++17,要求违反类型特征的前提条件必须导致编译时错误。然而,因为一些类型特征有过度约束的条件,例如总是需要完整的类型,这种建议还未完全付诸于行动。
\end{tcolorbox}

例如:

\begin{lstlisting}[style=styleCXX]
make_unsigned_t<int> // unsigned int
make_unsigned_t<int const&> // undefined behavior (hopefully error)
\end{lstlisting}

有时结果可能会令人惊讶。例如:

\begin{lstlisting}[style=styleCXX]
add_rvalue_reference_t<int> // int&&
add_rvalue_reference_t<int const> // int const&&
add_rvalue_reference_t<int const&> // int const& (lvalue-ref remains lvalue-ref)
\end{lstlisting}

这里可能期望add\_rvalue\_reference总是进行右值引用,但是C++的引用折叠规则(参见第15.6.1节)导致左值引用和右值引用的组合产生左值引用。

另一个例子:

\begin{lstlisting}[style=styleCXX]
is_copy_assignable_v<int> // yields true (generally, you can assign an int to an int)
is_assignable_v<int,int> // yields false (can’t call 42 = 42)
\end{lstlisting}

is\_copy\_assignable只是检查是否可以给另一个int赋值(检查lvalues的操作),is\_assignable将值类别(参见附录B)考虑进去(这里检查是否可以给prvalue赋值)。也就是说,第一个表达式等价于

\begin{lstlisting}[style=styleCXX]
is_assignable_v<int&,int&> // yields true
\end{lstlisting}

出于同样的原因:

\begin{lstlisting}[style=styleCXX]
is_swappable_v<int> // yields true (assuming lvalues)
is_swappable_with_v<int&,int&> // yields true (equivalent to the previous check)
is_swappable_with_v<int,int> // yields false (taking value category into account)
\end{lstlisting}

请仔细注意类型特征的准确定义,附录D中详细描述了标准的情况。

\subsubsubsection{11.2.2\hspace{0.2cm}std::addressof()}

std::addressof<>()函数模板生成对象或函数的实际地址。即使对象类型有重载操作符\&,也能工作。尽管后者很少使用,但可能会发生(例如,在智能指针中)。因此,如果需要任意类型对象的地址,建议使用addressof():

\begin{lstlisting}[style=styleCXX]
template<typename T>
void f (T&& x)
{
	auto p = &x; // might fail with overloaded operator &
	auto q = std::addressof(x); // works even with overloaded operator &
	...
}
\end{lstlisting}

\subsubsubsection{11.2.3\hspace{0.2cm}std::declval()}

std::declval<>()函数模板可以用作特定类型的对象引用的占位符。该函数没有定义,因此不能调用(也不创建对象)。因此,只能用于未求值的操作数(decltype和sizeof构造的操作数)。因此,与其尝试创建一个对象,可以假设有一个相应类型的对象。

例如,下面的声明从传递的模板参数T1和T2推导出默认返回类型RT:

\noindent
\textit{basics/maxdefaultdeclval.hpp}
\begin{lstlisting}[style=styleCXX]
#include <utility>
template<typename T1, typename T2,
		 typename RT = std::decay_t<decltype(true ? std::declval<T1>()
												  : std::declval<T2>())>>
RT max (T1 a, T2 b)
{
	return b < a ? a : b;
}
\end{lstlisting}

为了避免必须调用T1和T2的(默认)构造函数才能在表达式中调用三元操作符来初始化RT,这里使用std::declval来“使用”对应类型的对象。不过,这只可能在decltype未求值的上下文中实现。

不要忘记使用std::decay<>类型来确保默认的返回类型不是一个引用,因为std::declval()本身会产生右值引用。否则,像max(1,2)这样的调用将得到一个int\&\&的返回类型。详见19.3.4节。






