概念很像bool类型的constexpr变量模板,但类型没有显式指定:

\begin{lstlisting}[style=styleCXX]
template<typename T> concept LessThanComparable = ... ;
\end{lstlisting}

这里的“…”可以用一个表达式来代替,该表达式使用各种特征来确定,类型T是否确实可以使用小于操作符进行比较。但是概念提供了一个工具可简化这个任务:requires表达式(与上面描述的requires子句不同)。以下是这个概念的完整定义:

\begin{lstlisting}[style=styleCXX]
template<typename T>
concept LessThanComparable = requires(T x, T y) {
	{ x < y } -> bool;
};
\end{lstlisting}

请注意require表达式如何包含一个可选参数列表:这些参数永远不会用参数替换,可以认为是一组“哑变量”,可用来在require表达式体中表达需求。短语表达了这样的需求

\begin{lstlisting}[style=styleCXX]
{ x < y } -> bool;
\end{lstlisting}

这种语法意味着(a)表达式x < y必须在SFINAE意义上有效,(b)表达式的结果必须可转换为bool类型。这种形式的短语中,关键字noexcept可以插入到\texttt{->}标记之前,以表示大括号中的表达式不会抛出异常(即,应用于该表达式的noexcept(…)为true。若不需要这样的约束,短语的隐式转换部分(即\texttt{->}类型)可以省略,若只需要检查表达式的有效性,则可以删除大括号,这样短语就可以简化为表达式。

\begin{lstlisting}[style=styleCXX]
template<typename T>
concept Swappable = requires(T x, T y) {
	swap(x, y);
};
\end{lstlisting}

requires表达式还可以表达对关联类型的需求。考虑前面假设的序列概念:除了要求seq.begin()等表达式的有效性外,还需要相应的序列迭代器类型。可以表示为:

\begin{lstlisting}[style=styleCXX]
template<typename Seq>
concept Sequence = requires(Seq seq) {
	typename Seq::iterator;
	{ seq.begin() } -> Seq::iterator;
	...
};
\end{lstlisting}

typename type;表示类型存在的需求(这称为类型需求)。本例中,必须存在的类型是概念模板参数的成员,但不一定总是这样,可以要求存在一个IteratorFor<Seq>类型,通过require短语实现

\begin{lstlisting}[style=styleCXX]
...
typename IteratorFor<Seq>;
...
\end{lstlisting}

上面的Sequence概念定义展示了,通过逐个列出短语来组合短语。还有第三类需求短语,其只包含调用另一个概念。假设有一个迭代器的概念,希望序列概念不仅要求Seq::iterator是一种类型,而且要求该类型满足iterator概念的约束条件。表达式如下:

\begin{lstlisting}[style=styleCXX]
template<typename Seq>
concept Sequence = requires(Seq seq) {
	typename Seq::iterator;
	requires Iterator<typename Seq::iterator>;
	{ seq.begin() } -> Seq::iterator;
	...
};
\end{lstlisting}

也就是说,可以在requires表达式中添加子句(这种短语视为嵌套需求)。
















