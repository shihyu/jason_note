
特征模板示例已经用于确定模板参数的属性:表示什么类型,应用于该类型值的操作符的结果类型等。这样的特征称为属性特征。

一些特征定义了某些类型应该如何对待,称之为策略特征。这让人想起了前面讨论的策略类的概念(特征和策略之间的区别并不完全清晰),但是策略特征往往是与模板参数相关联的唯一属性(而策略类通常独立于其他模板参数)。

虽然属性特征通常可以作为类型函数实现,但策略特征通常将策略封装在成员函数中。让我们来看一个类型函数,定义了传递只读参数的策略。

\subsubsubsection{19.9.1\hspace{0.2cm}只读参数类型}

C和C++中,函数调用参数默认按值传递。调用方计算的参数值可复制到被调用方控制的位置。对于大型结构来说,这样做的代价会很高,而且对于这样的结构来说,可以通过指向const的引用(或C中指向const的指针)传递参数。对于较小的结构,从性能角度来看,最好的机制取决于所编写代码的架构。大多数情况下,这并不是那么重要,但有时即使是很小的结构也必须谨慎处理。

对于模板,事情变得更加微妙:事先不知道替换的模板参数类型有多大。此外,这个决定不仅取决于大小:小的结构可能会附带一个昂贵的复制构造函数,仍然可以通过引用到const来验证传递只读参数。

这个问题可以通过一个策略特征模板处理,该模板是一个类型函数:该函数将预期的参数类型T映射到最优参数类型T或T const\&。主模板可以对不超过两个指针的类型使用按值传递,对其他类型使用引用const传递:

\begin{lstlisting}[style=styleCXX]
template<typename T>
struct RParam {
	using Type = typename IfThenElseT<sizeof(T)<=2*sizeof(void*),
									T,
									T const&>::Type;
};
\end{lstlisting}

另一方面,对于sizeof返回小值的容器类型,可能会涉及昂贵的复制构造函数,因此可能需要许多特化和偏特化:

\begin{lstlisting}[style=styleCXX]
template<typename T>
struct RParam<Array<T>> {
	using Type = Array<T> const&;
};
\end{lstlisting}

因为这类类型在C++中很常见,所以只使用简单的复制和移动构造函数,作为值类型来标记小类型可能会更安全

\begin{tcolorbox}[colback=webgreen!5!white,colframe=webgreen!75!black]
\hspace*{0.75cm}若对复制或移动构造函数的调用,可以用底层字节的简单复制来替换,则将其称trivial。
\end{tcolorbox}

然后有选择地添加其他类类型(std::is\_trivially\_copy\_constructible和std::is\_trivially\ move\ constructible类型特征是C++标准库的一部分)。

\hspace*{\fill} \\ %插入空行
\noindent
\textit{traits/rparam.hpp}
\begin{lstlisting}[style=styleCXX]
#ifndef RPARAM_HPP
#define RPARAM_HPP

#include "ifthenelse.hpp"
#include <type_traits>

template<typename T>
struct RParam {
	using Type
	= IfThenElse<(sizeof(T) <= 2*sizeof(void*)
					&& std::is_trivially_copy_constructible<T>::value
					&& std::is_trivially_move_constructible<T>::value),
				T,
				T const&>;
};

#endif // RPARAM_HPP
\end{lstlisting}

无论采用哪种方式,策略现在都可以集中在特征模板定义中,外部可以很好地利用它。假设有两个类,其中一个类指定通过值调用更适合只读参数:

\hspace*{\fill} \\ %插入空行
\noindent
\textit{traits/rparamcls.hpp}
\begin{lstlisting}[style=styleCXX]
#include "rparam.hpp"
#include <iostream>

class MyClass1 {
	public:
	MyClass1 () {
	}
	MyClass1 (MyClass1 const&) {
		std::cout << "MyClass1 copy constructor called\n";
	}
};

class MyClass2 {
	public:
	MyClass2 () {
	}
	MyClass2 (MyClass2 const&) {
		std::cout << "MyClass2 copy constructor called\n";
	}
};

// pass MyClass2 objects with RParam<> by value
template<>
class RParam<MyClass2> {
	public:
	using Type = MyClass2;
};
\end{lstlisting}

可以声明使用RParam<>作为只读参数的函数,并调用这些函数:

\hspace*{\fill} \\ %插入空行
\noindent
\textit{traits/rparam1.cpp}
\begin{lstlisting}[style=styleCXX]
#include "rparam.hpp"
#include "rparamcls.hpp"

// function that allows parameter passing by value or by reference
template<typename T1, typename T2>
void foo (typename RParam<T1>::Type p1,
typename RParam<T2>::Type p2)
{
	...
}

int main()
{
	MyClass1 mc1;
	MyClass2 mc2;
	foo<MyClass1,MyClass2>(mc1,mc2);
}
\end{lstlisting}

使用RParam有一些明显的缺点。首先,函数声明明显更混乱。其次,因为模板参数只出现在函数参数的限定符中,所以不能用参数推导调用foo()。因此,调用站点必须明确指定模板参数。

这个选项的一个笨拙的解决方法是使用内联包装器函数模板,提供完美的转发功能(第15.6.3节),但其假定编译器会省略内联函数:

\hspace*{\fill} \\ %插入空行
\noindent
\textit{traits/rparam2.cpp}
\begin{lstlisting}[style=styleCXX]
#include "rparam.hpp"
#include "rparamcls.hpp"

// function that allows parameter passing by value or by reference
template<typename T1, typename T2>
void foo_core (typename RParam<T1>::Type p1,
				typename RParam<T2>::Type p2)
{
	...
}

// wrapper to avoid explicit template parameter passing
template<typename T1, typename T2>
void foo (T1 && p1, T2 && p2)
{
	foo_core<T1,T2>(std::forward<T1>(p1),std::forward<T2>(p2));
}

int main()
{
	MyClass1 mc1;
	MyClass2 mc2;
	foo(mc1,mc2); // same as foo_core<MyClass1,MyClass2>(mc1,mc2)
}
\end{lstlisting}











