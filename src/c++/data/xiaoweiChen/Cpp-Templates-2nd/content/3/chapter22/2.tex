std::function<>类型实际上是C++函数指针的泛化形式,提供了相同的基本操作:

\begin{itemize}
\item 
可以用于调用函数,而调用者不需要知道函数的信息。

\item 
可以复制、移动和赋值。

\item 
可以从另一个函数(带有兼容的签名)初始化或赋值。

\item 
有一个“空”状态,表示没有函数已绑定。
\end{itemize}

与C++函数指针不同的是,std::function<>还可以使用合适的函数操作符存储Lambda或其他函数对象,所有这些操作符都可能具有不同的类型。

本节的剩余部分,将构建自己的通用函数指针类模板——FunctionPtr,需要提供这些核心操作和功能,并用来替代std::function:

\hspace*{\fill} \\ %插入空行
\noindent
\textit{bridge/forupto4.cpp}
\begin{lstlisting}[style=styleCXX]
#include "functionptr.hpp"
#include <vector>
#include <iostream>

void forUpTo(int n, FunctionPtr<void(int)> f)
{
	for (int i = 0; i != n; ++i)
	{
		f(i); // call passed function f for i
	}
}

void printInt(int i)
{
	std::cout << i << ’ ’;
}

int main()
{
	std::vector<int> values;
	
	// insert values from 0 to 4:
	forUpTo(5,
			[&values](int i) {
				values.push_back(i);
			});
		
	// print elements:
	forUpTo(5,
			printInt); // prints 0 1 2 3 4
	std::cout << ’\n’;
}
\end{lstlisting}

FunctionPtr的接口相当简单,提供函数对象的构造、复制、移动、销毁、初始化和赋值,以及底层函数对象的调用。该接口最有趣的部分是,如何在类模板偏特化中进行描述,其将模板参数(一个函数类型)分解为其组件(结果类型和参数类型):

\hspace*{\fill} \\ %插入空行
\noindent
\textit{bridge/functionptr.hpp}
\begin{lstlisting}[style=styleCXX]
// primary template:
template<typename Signature>
class FunctionPtr;

// partial specialization:
template<typename R, typename... Args>
class FunctionPtr<R(Args...)>
{
	private:
	FunctorBridge<R, Args...>* bridge;
	public:
	// constructors:
	FunctionPtr() : bridge(nullptr) {
	}
	FunctionPtr(FunctionPtr const& other); // see functionptr-cpinv.hpp
	FunctionPtr(FunctionPtr& other)
	: FunctionPtr(static_cast<FunctionPtr const&>(other)) {
	}
	FunctionPtr(FunctionPtr&& other) : bridge(other.bridge) {
		other.bridge = nullptr;
	}
	// construction from arbitrary function objects:
	template<typename F> FunctionPtr(F&& f); // see functionptr-init.hpp
	
	// assignment operators:
	FunctionPtr& operator=(FunctionPtr const& other) {
		FunctionPtr tmp(other);
		swap(*this, tmp);
		return *this;
	}
	FunctionPtr& operator=(FunctionPtr&& other) {
		delete bridge;
		bridge = other.bridge;
		other.bridge = nullptr;
		return *this;
	}
	// construction and assignment from arbitrary function objects:
	template<typename F> FunctionPtr& operator=(F&& f) {
		FunctionPtr tmp(std::forward<F>(f));
		swap(*this, tmp);
		return *this;
	}

	// destructor:
	~FunctionPtr() {
	delete bridge;
	}

	friend void swap(FunctionPtr& fp1, FunctionPtr& fp2) {
	std::swap(fp1.bridge, fp2.bridge);
	}
	explicit operator bool() const {
	return bridge != nullptr;
	}

	// invocation:
	R operator()(Args... args) const; // see functionptr-cpinv.hpp
};
\end{lstlisting}

该实现包含非静态成员变量bridge,负责存储和操作存储的函数对象。这个指针的所有权与FunctionPtr对象绑定,因此大多数实现仅去管理这个指针。未实现的函数包含实现中的部分,将在下面的小节中进行描述。






































