
上一节中,我们了解了如何在Clang中为驱动添加自定义标志,并了解了驱动如何将它们转换为前端接受的标志。本节中,我们将讨论工具链——驱动内部的一个重要模块,它能帮助驱动程序适应不同的平台。

记得在本章的第一部分,在图8.1中展示了驱动和工具链之间的关系:驱动程序可以根据目标平台选择合适的工具链,然后利用其基本信息做以下事情:

\begin{enumerate}
\item LExecute生成目标代码所需的正确的汇编器、链接器或任何工具。

\item 向编译器、汇编器或链接器传递特定于平台的标志。
\end{enumerate}

这些信息对于构建源代码至关重要,因为每个平台可能都有自己独特的特征,比如:系统库路径和受支持的汇编/链接器变体。没有它们,甚至无法生成正确的可执行文件或库。

本节希望帮助大家了解如何在将来为定制平台创建Clang工具链。Clang中的工具链框架足够强大,可以适应各种各样的用例,例如:可以创建一个类似于Linux上传统编译器的工具链——包括使用GNU AS进行组装和使用GNU LD进行链接——而不需要对默认库路径或编译器标志进行多次定制。另一方面,可以使用一个怪异的工具链,它甚至不使用Clang来编译源代码,而是使用一个专有的汇编器和带有不常见命令行标志的链接器。本节将尝试使用一个示例来演示最常见的用例,同时又能展示该框架的灵活性。

这一部分的组织如下:将从我们将要从事的项目的概述开始。在此之后,我们将把项目工作负载分解为三个部分——添加定制编译器选项、设置定制汇编器和设置定制链接器——然后将它们放在一起,结束本节。

\begin{tcolorbox}[colback=blue!5!white,colframe=blue!75!black, fonttitle=\bfseries,title=系统需求]	
\hspace*{0.7cm}作为另一个友好的提示,下面的项目只能在Linux系统上工作。请确保已安装OpenSSL。
\end{tcolorbox}

\subsubsubsection{8.4.1\hspace{0.2cm}项目概述}

我们将创建一个名为\textbf{Zipline}的工具链,它使用Clang(前端和后端)来进行普通的编译,但在汇编阶段使用\textbf{Base64}编码生成的汇编代码,并在链接阶段将这些Base64编码的文件打包成\textbf{ZIP文件}(或\texttt{.tarbell}文件)。

\begin{tcolorbox}[colback=blue!5!white,colframe=blue!75!black, fonttitle=\bfseries,title=Base64]	
\hspace*{0.7cm}Base64是一种编码方案,通常用于将二进制文件转换为纯文本。它可以很容易地在不支持二进制格式的上下文中传输(例如,HTTP报头)。还可以将Base64应用于普通文本文件,就像我们的例子一样。
\end{tcolorbox}

这个工具链在生产环境中基本上是无用的。它只是一个演示,模拟开发人员在为定制平台创建新工具链时可能遇到的常见情况。

这个工具链是通过自定义驱动标志\texttt{-ziplin}/\texttt{-\,-zipline}来启用。当提供该标志时,编译器会隐式地将\texttt{my\_include}文件夹添加到你的主目录中,作为搜索头文件路径,例如:上一节中,添加自定义驱动标志,我们的自定义\texttt{-fuse-simple-log}标志将隐式地在输入源代码中包含一个头文件\texttt{simple\_log.h}:

\begin{tcblisting}{commandshell={}}
$ ls
main.cc simple_log.h
$ clang++ -fuse-simple-log -fsyntax-only main.cc
$ # OK
\end{tcblisting}

然而,如果\texttt{simple\_log.h}不在当前目录中,就像前面的代码片段中那样,我们需要通过另一个标志指定它的路径:

\begin{tcblisting}{commandshell={}}
$ ls .
# No simple_log.h in current folder
main.cc
$ clang++ -fuse-simple-log=/path/to/simple_log.h -fsyntax-only
main.cc
$ # OK
\end{tcblisting}

在Zipline的帮助下,可以把\texttt{simple\_log.h}放在\texttt{/home/<user name>/my\_include}中,这样编译器就能找到它了:

\begin{tcblisting}{commandshell={}}
$ ls .
# No simple_log.h in current folder
main.cc
$ ls ~/my_include
simple_log.h
$ clang++ -zipline -fuse-simple-log -fsyntax-only main.cc
$ # OK
\end{tcblisting}

Zipline的第二个特性是,\texttt{clang}可执行文件将源代码编译成由Base64在\texttt{-c}标志下编码的汇编代码,该汇编代码将汇编文件(来自编译器)编译成一个目标文件。下面是命令示例:

\begin{tcblisting}{commandshell={}}
$ clang -zipline -c test.c
$ file test.o
test.o: ASCII text # Not (binary) object file anymore
$ cat test.o
CS50ZXh0CgkuZmlsZQkidGVzdC5jYyIKCS
5nbG9ibAlfWjNmb29pCgkucDJhbGln

bgk0LCAweDkwCgkudHlwZQlfWjNmb29p
LEBmdW5jdGlvbgpfWjNmb29pOgoJLmNm

… # Base64 encoded contents
$
\end{tcblisting}

前面的\texttt{file}命令显示生成的文件\texttt{test.o},从之前使用\texttt{clang}开始,就不再是二进制格式的对象文件。这个文件的内容就是编译器后端生成的汇编代码(Base64编码)版本。

最后,Zipline用一个定制阶段替换了原来的链接阶段,这个定制阶段将前面提到的Base64编码的程序集文件打包并压缩到一个\texttt{.zip}文件中。下面是一个例子:

\begin{tcblisting}{commandshell={}}
$ clang -zipline test.c -o test.zip
$ file test.zip
test.zip: Zip archive, at least v2.0 to extract
$
\end{tcblisting}

如果解压\texttt{test.zip},将发现这些解压的文件是Case64编码的程序集文件。

或者,我们可以使用Linux的\texttt{tar}和\texttt{gzip}工具将它们打包并压缩到Zipline中:

\begin{tcblisting}{commandshell={}}
$ clang -zipline -fuse-ld=tar test.c -o test.tar.gz
$ file test.tar.gz
test.tar.gz: gzip compressed data, from Unix, original size…
$
\end{tcblisting}

通过使用现有的\texttt{-fuse-ld=<链接器名称>}标志,我们可以在自定义链接阶段选择使用\texttt{zip}、\texttt{tar}或\texttt{gzip}。

下一节中,我们将为这个工具链创建框架代码,并展示如何向头文件搜索路径添加额外的文件夹路径。

\subsubsubsection{8.4.2\hspace{0.2cm}创建工具链并添加自定义的包含路径}

本节中,我们将为Zipline工具链创建框架,并展示如何在Zipline的编译阶段添加一个额外的包含文件夹路径——更确切地说,一个额外的\textbf{系统包含路径}。具体步骤如下:

\begin{enumerate}
\item 在添加工具链实现之前,不要忘记使用自定义驱动标志\texttt{-zipline}/\texttt{-\,-zipline}来启用工具链。使用我们在前一节学到的技能,添加自定义驱动标志来完成。在\texttt{clang/include/clang/Dri\\ver/Options.td}里面,我们将增加以下行:

\begin{lstlisting}[style=styleJavaScript]
// zipline toolchain
def zipline : Flag<["-", "--"], "zipline">,
Flags<[NoXarchOption]>;
\end{lstlisting}

同样,\texttt{Flag}说明这是一个布尔标志,而\texttt{NoXarchOption}说明这个标志是驱动唯一的。我们将很快使用这个驱动标志。

\item Clang中的工具链由\texttt{clang::driver::ToolChain}类表示。Clang支持的每个工具链都是从它派生出来的,源文件在\texttt{clang/lib/Driver/ToolChains}文件夹下。我们将在那里创建两个新文件:\texttt{Zipline.h}和\texttt{Zipline.cpp}。

\item 对于\texttt{Zipline.h},先添加以下框架代码:

\begin{lstlisting}[style=styleCXX]
namespace clang {
namespace driver {
namespace toolchains {
struct LLVM_LIBRARY_VISIBILITY ZiplineToolChain
: public Generic_ELF {
	ZiplineToolChain(const Driver &D, const llvm::Triple
	&Triple, const llvm::opt::ArgList &Args)
	: Generic_ELF(D, Triple, Args) {}
	~ZiplineToolChain() override {}
	// Disable the integrated assembler
	bool IsIntegratedAssemblerDefault() const override
	{ return false; }
	bool useIntegratedAs() const override { return false; }
	void
	AddClangSystemIncludeArgs(const llvm::opt::ArgList
	&DriverArgs, llvm::opt::ArgStringList &CC1Args)
	const override;
protected:
	Tool *buildAssembler() const override;
	Tool *buildLinker() const override;
};
} // end namespace toolchains
} // end namespace driver
} // end namespace clang
\end{lstlisting}

我们在这里创建的\texttt{ZiplineToolChain}类,是从\texttt{Generic\_ELF}派生而来,\texttt{Generic\_ELF}是\texttt{ToolChain}的子类,专门用于使用ELF作为其执行格式的系统——包括Linux。除了父类之外,还有三个重要的方法,我们将在本节或后面的章节中实现:\texttt{AddClangSystemIncludeArgs}、\texttt{buildAssembler}和\texttt{buildLinker}。

\item \texttt{buildAssembler}和\texttt{buildLinker方}法生成的\texttt{Tool}实例分别表示要在汇编和链接阶段运行的命令或程序(我们将在下面几节中介绍它们)。现在,我们将实现\texttt{AddClangSystemIncludeArgs}方法。在\texttt{Zipline.cpp}中,我们将添加相应实现:

\begin{lstlisting}[style=styleCXX]
void ZiplineToolChain::AddClangSystemIncludeArgs(
                       const ArgList &DriverArgs,
                       ArgStringList &CC1Args) const {
	using namespace llvm;
	SmallString<16> CustomIncludePath;
	sys::fs::expand_tilde("~/my_include",
	                      CustomIncludePath);
	addSystemInclude(DriverArgs,
	                 CC1Args, CustomIncludePath.c_str());
}
\end{lstlisting}

我们在这里做的唯一一件事是使用\texttt{addSystemInclude}函数与位于主目录中的\texttt{my\_include}文件夹的路径。由于每个用户的主目录不同,我们使用\texttt{sys::fs::expand\_tilde}助手函数将\texttt{~/my\_include}(其中\texttt{~}表示Linux和Unix系统中的主目录)扩展到绝对路径中。另一方面,\texttt{addSystemInclude}函数可以将“\texttt{-internalisystem}”“\texttt{/path/to/my\_include}”标志添加到前端标志的列表中。\texttt{-internal-issystem}标志用于指定系统头文件的文件夹,包括标准库头文件和平台特定的头文件。

\item 最后,当Zipline工具链看到新创建的\texttt{-zipline}/\texttt{-\,-zipline}驱动标志时,需要我们教会驱动使用Zipline工具链。我们需要修改\texttt{clang/lib/Driver/Driver.cpp}中的 \texttt{Driver::getTool\\Chain}方法。\texttt{Driver::getToolChain}方法包含一个巨大的开关盒,用于根据目标操作系统和硬件架构选择不同的工具链(具体的情况,请浏览处理Linux系统的代码)。我们将在这里添加一个额外的分支条件:

\begin{lstlisting}[style=styleCXX]
const ToolChain
&Driver::getToolChain(const ArgList &Args,
const llvm::Triple &Target) const {
	…
	switch (Target.getOS()) {
		case llvm::Triple::Linux:
		…
		  else if (Args.hasArg(options::OPT_zipline))
		    TC = std::make_unique<toolchains::ZiplineToolChain>
		    (*this, Target, Args);
		…
		  break;
		case …
		case …
	}
}
\end{lstlisting}

额外的\texttt{else-if}语句基本上是说,如果目标操作系统是Linux,那么当指定\texttt{-zipline}/\texttt{-\,-zip\\line}时,我们将使用Zipline。

\end{enumerate}

这样,您就添加了Zipline的框架,并成功地告诉驱动程序在给定自定义驱动程序标志时使用Zipline。除此之外,还了解了如何向头文件搜索路径添加额外的系统库文件夹。

下一节中,我们将创建一个自定义汇编阶段,并将其连接到我们创建的工具链中。

\subsubsubsection{8.4.3\hspace{0.2cm}创建自定义汇编阶段}

正如在项目概述中提到的,我们不是在Zipline的汇编阶段将汇编代码转换为目标文件,而是调用一个程序来将Clang生成的汇编文件转换为Base64编码的对应版本。在深入其实现之前,先了解一下工具链中的每个阶段是如何表示的。

上一节中,我们了解到Clang中的工具链是由\texttt{Toolchain}类表示。每个工具链实例负责告知驱动在每个编译阶段运行什么工具——即编译、汇编和链接。这个信息封装在\texttt{clang::driver::Tool}类型中。上一节中\texttt{buildAssembler}和\texttt{buildLinker},它们返回工具类型的对象,分别描述要执行的动作和在汇编和链接阶段运行的工具。本节中,将展示如何实现\texttt{buildAssembler}返回的\texttt{Tool}对象。就让我们开始吧!

\begin{enumerate}
\item 回到\texttt{Zipline.h}。我们在\texttt{clang::driver::tools::zipline}名称空间中添加了一个类\texttt{Assembler}:

\begin{lstlisting}[style=styleCXX]
namespace clang {
namespace driver {
namespace tools {
namespace zipline {
	struct LLVM_LIBRARY_VISIBILITY Assembler : public Tool {
		Assembler(const ToolChain &TC)
		  : Tool("zipeline::toBase64", "toBase64", TC) {}
		bool hasIntegratedCPP() const override { return false;
		}
	void ConstructJob(Compilation &C, const JobAction &JA,
	const InputInfo &Output,
	const InputInfoList &Inputs,
	const llvm::opt::ArgList &TCArgs,
	const char *LinkingOutput) const
	override;
};
} // end namespace zipline
} // end namespace tools

namespace toolchains {
struct LLVM_LIBRARY_VISIBILITY ZiplineToolChain … {
…
};
} // end namespace toolchains
} // end namespace driver
} // end namespace clang
\end{lstlisting}

这里需要注意,因为新创建的\texttt{Assembler}在\texttt{clang::driver::tools::zipline}名称空间中,而我们在上一节创建的\texttt{ZiplineToolChain}位于\texttt{clang::driver::toochains}中。

我们将把调用Base64编码工具的逻辑放在\texttt{Assembler::ConstructJob}方法中。

\item 我们将在\texttt{Zipline.cpp}中,实现\texttt{Assembler::ConstructJob}:

\begin{lstlisting}[style=styleCXX]
void
tools::zipline::Assembler::ConstructJob(Compilation &C,
							const JobAction &JA,
							const InputInfo &Output,
							const InputInfoList &Inputs,
							const ArgList &Args,
							const char *LinkingOutput)
							const {
	ArgStringList CmdArgs;
	const InputInfo &II = Inputs[0];
	
	std::string Exec =
	  Args.MakeArgString(getToolChain().
	    GetProgramPath("openssl"));
	
	// opeenssl base64 arguments
	CmdArgs.push_back("base64");
	CmdArgs.push_back("-in");
	CmdArgs.push_back(II.getFilename());
	CmdArgs.push_back("-out");
	CmdArgs.push_back(Output.getFilename());
	
	C.addCommand(
	  std::make_unique<Command>(
	  		 JA, *this, ResponseFileSupport::None(),
	         Args.MakeArgString(Exec), CmdArgs,
             Inputs, Output));
}
\end{lstlisting}

我们使用OpenSSL来做Base64编码,运行的命令如下:

\begin{tcblisting}{commandshell={}}
$ openssl base64 -in <input file> -out <output file>
\end{tcblisting}

\texttt{ConstructJob}方法的工作是构建\textit{程序调用}来运行前面的命令,是由\texttt{C.addCommand(…)}函数调用\texttt{ConstructJob}实现的。传递给\texttt{addCommand}要调用的\texttt{Command}实例,表示在汇编阶段要运行的具体命令。它包含必要的信息,比如:可执行文件的路径(\texttt{Exec}变量),及其参数(\texttt{CmdArgs}变量)。

对于\texttt{Exec}变量,工具链提供了一个工具,即\texttt{GetProgramPath}函数,用于解析可执行文件的绝对路径。

另一方面,我们为\texttt{openssl}(\texttt{CmdArgs}变量)构建参数的方式,类似于在添加自定义驱动标志时所做的事情:将驱动标志(\texttt{Args}参数)和输入/输出文件信息(\texttt{Output}和\texttt{Inputs}参数)转换为一组新的命令行参数,并存储在\texttt{CmdArgs}中。

\item 最后,通过实现\texttt{ZiplineToolChain::buildAssembler},将这个\texttt{Assembler}类与\texttt{Zipline\\ToolChain}连接起来:

\begin{lstlisting}[style=styleCXX]
Tool *ZiplineToolChain::buildAssembler() const {
	return new tools::zipline::Assembler(*this);
}
\end{lstlisting}

\end{enumerate}

这些是我们需要完成的步骤,该工具实例中需要在Zipline工具链链接阶段运行的命令。

\subsubsubsection{8.4.4\hspace{0.2cm}创建自定义链接阶段}

现在我们已经完成了汇编阶段,是时候进入下一个阶段了——连接阶段。我们将使用与前一节相同的方法,创建一个自定义的工具类来表示链接器。

\begin{enumerate}
\item 在\texttt{Zipline.h}中,创建一个从\texttt{Tool}派生的\texttt{Linker}类:

\begin{lstlisting}[style=styleCXX]
namespace zipline {
struct LLVM_LIBRARY_VISIBILITY Assembler : public Tool {
	…
};

struct LLVM_LIBRARY_VISIBILITY Linker : public Tool {
	Linker(const ToolChain &TC)
	  : Tool("zipeline::zipper", "zipper", TC) {}
	  
	bool hasIntegratedCPP() const override { return false;
}

	bool isLinkJob() const override { return true; }
	
	void ConstructJob(Compilation &C, const JobAction &JA,
						const InputInfo &Output,
						const InputInfoList &Inputs,
						const llvm::opt::ArgList &TCArgs,
						const char *LinkingOutput) const
						override;
private:
	void buildZipArgs(const JobAction&, const InputInfo&,
						const InputInfoList&,
						const llvm::opt::ArgList&,
						llvm::opt::ArgStringList&) const;
						
	void buildTarArgs(const JobAction&,
						const InputInfo&,
						const InputInfoList&,
						const llvm::opt::ArgList&,
						llvm::opt::ArgStringList&) const;
};
} // end namespace zipline
\end{lstlisting}

在这个\texttt{Linker}类中,还需要实现\texttt{ConstructJob}方法,从而来告诉驱动在链接阶段要执行什么。与\texttt{Assembler}不同的是,由于需要同时支持\texttt{zip}和\texttt{tar + gzip}的打包/压缩方案,我们将添加两个额外的方法\texttt{buildZipArgs}和\texttt{buildTarArgs},来处理每个参数的构建。

\item 我们在\texttt{Zipline.cpp}中,将首先关注\texttt{Linker::ConstructJob}的实现:

\begin{lstlisting}[style=styleCXX]
void
tools::zipline::Linker::ConstructJob(Compilation &C,
const JobAction &JA,
const InputInfo &Output,
const InputInfoList &Inputs,
const ArgList &Args,
const char *LinkingOutput) const
{
	ArgStringList CmdArgs;
	std::string Compressor = "zip";
	if (Arg *A = Args.getLastArg(options::OPT_fuse_ld_EQ))
	  Compressor = A->getValue();
	std::string Exec = Args.MakeArgString(
	     getToolChain().GetProgramPath(Compressor.c_str()));
	
	if (Compressor == "zip")
  	  buildZipArgs(JA, Output, Inputs, Args, CmdArgs);
	if (Compressor == "tar" || Compressor == "gzip")
	  buildTarArgs(JA, Output, Inputs, Args, CmdArgs);
	else
	  llvm_unreachable("Unsupported compressor name");
	
	C.addCommand(
	  std::make_unique<Command>(
	    JA, *this, ResponseFileSupport::None(),
	    Args.MakeArgString(Exec),
	    CmdArgs, Inputs, Output));
}
\end{lstlisting}

在这个自定义链接阶段,我们希望使用\texttt{zip}或\texttt{tar}命令(取决于用户指定的\texttt{-fuse-ld}标志)来打包自定义\texttt{Assembler}生成的(Base64编码的)文件。

稍后将解释\texttt{zip}和\texttt{tar}的命令格式。前面的代码中,可以看到这里做的事情与\texttt{Assembler::ConstructJob}类似。\texttt{Exec}变量携带\texttt{zip}或\texttt{tar}可执行文件的绝对路径。\texttt{CmdArgs}变量,由\texttt{buildZipArgs}或\texttt{buildTarArgs}填充,稍后将对此进行解释,它保存了工具的命令行参数(\texttt{zip}或\texttt{tar})。

与\texttt{Assembler::ConstructJob}的最大区别是,要执行的命令可以由用户提供的\texttt{-fuse-ld}指定。因此,我们使用在添加自定义驱动标志部分中学到的技能,来读取驱动程序标志,并设置执行命令。

\item 如果用户决定将文件打包为ZIP文件(这是默认方案,或者可以通过\texttt{-fuse-ld=zip}显式指定),将运行以下命令:

\begin{tcblisting}{commandshell={}}
$ zip <output zip file> <input file 1> <input file 2>…
\end{tcblisting}

因此,我们将构建\texttt{Linker::buildZipArgs}方法,它为前面的命令构造一个参数列表,如下所示:

\begin{lstlisting}[style=styleCXX]
void
tools::zipline::Linker::buildZipArgs(const JobAction &JA,
										const InputInfo &Output,
										const InputInfoList &Inputs,
										const ArgList &Args,
										ArgStringList &CmdArgs) const {
	// output file
	CmdArgs.push_back(Output.getFilename());
	// input files
	AddLinkerInputs(getToolChain(), Inputs, Args, CmdArgs,
	JA);
}
\end{lstlisting}

\texttt{buildZipArgs}的\texttt{CmdArgs}参数将是输出结果的地方。因为链接器可能一次接受多个输入,我们仍然使用相同的方法来获取输出文件名(通过\texttt{Output.getfilename()})。这里使用另一个辅助函数\texttt{AddLinkerInputs},将所有的输入文件名添加到\texttt{CmdArgs}中。

\item 如果用户决定使用\texttt{tar + gzip}的打包方案(使用\texttt{-fuseld=tar}或\texttt{-fuse-ld=gzip}),我们将运行以下命令:

\begin{tcblisting}{commandshell={}}
$ tar -czf <output tar.gz file> <input file 1> <input file 2>…
\end{tcblisting}

因此,我们将构建\texttt{Linker::buildTarArgs}方法,它为前面的命令构造一个参数列表,如下所示:

\begin{lstlisting}[style=styleCXX]
void
tools::zipline::Linker::buildTarArgs(const JobAction &JA,
										const InputInfo &Output,
										const InputInfoList &Inputs,
										const ArgList &Args,
										ArgStringList &CmdArgs)
										const {
	// arguments and output file
	CmdArgs.push_back("-czf");
	CmdArgs.push_back(Output.getFilename());
	// input files
	AddLinkerInputs(getToolChain(), Inputs, Args, CmdArgs,
	 JA);
}
\end{lstlisting}

与\texttt{buildZipArgs}一样,我们通过\texttt{Output.
getFilename()}获取输出文件名,并使用\texttt{AddLinkerInput}将所有的输入文件名添加到\texttt{CmdArgs}中。

\item 最后,连接到\texttt{ZiplineToolChain}:

\begin{lstlisting}[style=styleCXX]
Tool *ZiplineToolChain::buildLinker() const {
	return new tools::zipline::Linker(*this);
}
\end{lstlisting}

\end{enumerate}

这就是我们为Zipline工具链,实现自定义链接阶段的所有步骤。

既然已经为Zipline工具链创建了必要的组件,那么当用户选择这个工具链时,就可以执行我们的定制特性了——对源文件进行编码并将它们打包成一个存档文件。下一节中,我们将了解如何验证这些功能。

\subsubsubsection{8.4.5\hspace{0.2cm}验证自定义工具链}

为了测试本章中实现的功能,我们可以运行项目概述中描述的示例命令,或者可以再次使用\texttt{-\#\#\#}驱动标志转储所有预期的编译器、汇编器和链接器的命令细节。

目前为止,我们已经知道\texttt{-\#\#\#}标志将显示所有被驱动程序翻译的前端标志。实际上,它还会显示计划运行的汇编器和链接器命令。我们使用以下命令:

\begin{tcblisting}{commandshell={}}
$ clang -### -zipline -c test.c
\end{tcblisting}

因为\texttt{-c}标志试图在Clang生成的汇编文件上运行汇编程序,所以在Zipline中的自定义汇编器(即Base64编码器)将触发。因此,将看到类似如下的输出:

\begin{tcblisting}{commandshell={}}
$ clang -### -zipline -c test.c
"/path/to/clang" "-cc1" …
"/usr/bin/openssl" "base64" "-in" "/tmp/test_ae4f5b.s" "-out"
"test.o"
$
\end{tcblisting}

以\texttt{/path/to/clang -cc1}开头的行包含了在前面遇到的前端标志。下面一行是汇编器调用的命令,在我们的例子中是运行openssl来执行Base64编码。

奇怪的\texttt{/tmp/test\_ae4f5b.S}文件名是由驱动创建的临时文件,用于存放编译器生成的汇编代码。

使用相同的技巧,可以验证我们的定制链接器阶段,如下所示:

\begin{tcblisting}{commandshell={}}
$ clang -### -zipline test.c -o test.zip
"/path/to/clang" "-cc1" …
"/usr/bin/openssl" "base64" "-in" "/tmp/test_ae4f5b.s" "-out"
"/tmp/test_ae4f5b.o"
"/usr/bin/zip" "test.zip" "/tmp/test_ae4f5b.o"
$
\end{tcblisting}

由于\texttt{-o}标志在前面的命令中使用,Clang将由\texttt{test.c}构建一个完整的可执行文件,包括汇编器和链接器。因此,由于\texttt{zip}命令是从上一个汇编阶段获取的结果(\texttt{/tmp/test\_ae4f5b.o}文件)。所以,这里可以添加\texttt{加-fuse-ld=tar}标志,从而使用\texttt{zip}对\texttt{tar}命令进行替换。

本节中,展示了如何为Clang的驱动程序创建一个工具链。这是在定制或新平台上支持Clang的关键技能。还了解了Clang的工具链框架非常灵活,可以处理目标平台的各种任务。
































