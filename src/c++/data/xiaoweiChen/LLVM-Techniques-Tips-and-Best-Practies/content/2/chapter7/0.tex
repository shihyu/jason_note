在前一章中,了解了Clang的预处理器如何处理C族语言中的预处理指令。我们还了解了如何编写不同类型的预处理器插件,比如:pragma处理程序,以扩展Clang的功能。当涉及到实现特定领域的逻辑或甚至定制语言特性时,这些技能尤为有用。

本章中,将讨论解析后的原始源代码文件的\textbf{语义感知}表示,即\textbf{抽象语法树(AST)}。AST是一种包含丰富语义信息的格式,其中包括类型、表达式树和符号等。不仅可用作生成LLVM IR的蓝图,供以后的编译阶段使用,而且还是执行静态分析的推荐格式。除此之外,Clang还提供了一个很好的框架,让开发者可以通过一个简单的插件接口在前端管道的中间拦截和操作AST。

在本章中,我们将介绍如何在Clang中处理AST,内存中AST表示的重要API,以及如何编写AST插件来实现自定义逻辑。我们将讨论以下内容:

\begin{itemize}
\item Clang中的AST
\item 编写AST插件
\end{itemize}

在本章结束时,将了解如何在Clang中使用AST,以便在源代码级别对程序进行分析。此外,还将了解如何通过AST插件轻松地将自定义AST处理逻辑注入到Clang中。






































