LIT的核心使用Python编写,所以请确保开发环境中安装有Python 2.7或Python 3.x(Python 3.x更推荐,LLVM正在逐渐弃用Python 2.7)。

此外,还有一些支持实用程序,如FileCheck。不幸的是,要构建这些实用程序,最快的方法是构建check-XXX(伪)目标。例如,可以构建check-llvm-support:

\begin{tcblisting}{commandshell={}}
$ ninja check-llvm-support
\end{tcblisting}

最后一节要求构建llvm-test-suite,这是一个独立于llvm-project的代码库。我们可以克隆它:

\begin{tcblisting}{commandshell={}}
$ git clone https://github.com/llvm/llvm-test-suite
\end{tcblisting}

配置构建最简单的方法是使用CMake的缓存进行配置。例如,要用优化(\texttt{O3})构建测试套件:

\begin{tcblisting}{commandshell={}}
$ mkdir .O3_build
$ cd .O3_build
$ cmake -G Ninja -DCMAKE_C_COMPILER=<desired Clang binary \
  path> -C ../cmake/caches/O3.cmake ../
\end{tcblisting}

然后,进行正常构建:

\begin{tcblisting}{commandshell={}}
$ ninja all
\end{tcblisting}










































