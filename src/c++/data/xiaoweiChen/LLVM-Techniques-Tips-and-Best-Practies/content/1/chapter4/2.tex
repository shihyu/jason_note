
本节简要介绍了重要和常见的TableGen语法,提供了动手编写所需的所有基本知识,在下一节中会提供使用TableGen编写甜甜圈的食谱。

TableGen是一种特定于领域的编程语言,可以对特制的数据布局进行建模。尽管是一种编程语言,但做的事情与传统语言完全不同。\textbf{传统编程语言}通常描述对(输入)数据执行的操作,如何与环境交互,以及它们如何生成结果,而不考虑采用的编程范式(命令式、函数式、事件驱动……)。相比之下,TableGen的行为几乎没有任何描述。

TableGen的描述仅用于静态数据结构。首先,开发人员定义所需数据结构的布局(本质上就是包含许多字段的表)。然后,填充/初始化字段时,需要立即将数据填充到布局中。后一部分可能是TableGen的独特之处:许多编程语言或框架提供了设计特定于领域的数据结构的方法(例如,谷歌的\textbf{Protocol Buffers}),但在这些场景中,数据通常是\textbf{动态}填充的(主要是在使用DSL的代码中)。

\textbf{结构化查询语言(SQL)}在许多方面与TableGen相同:SQL和TableGen(仅)都处理结构化数据,并有定义布局的方法。在SQL中是\texttt{TABLE;}在TableGen中\texttt{class},这将在本节后面介绍。然而,除了精心设计布局之外,SQL还提供了更多的功能,还可以查询(实际上,这就是它名字的由来:\textbf{结构化查询语言})和动态更新数据,这在TableGen中是不可以的。然而,在本章的后面,您将看到TableGen提供了一个很好的框架来灵活地处理和解释这个(TableGen定义的)数据。

现在我们将介绍四种重要的TableGen结构:

\begin{itemize}
\item 布局和记录
\item 叹号操作符
\item 多记录
\item \textbf{有向无环图(DAG)}数据类型
\end{itemize}

\subsubsubsection{4.2.1\hspace{0.2cm}布局和记录}

考虑到TableGen只是描述结构化数据的一种奇特、富有表现力的方式,数据布局和实例化数据都有一个原始的表示。布局通过类语法实现,如下面的代码所示:

\begin{lstlisting}[style=styleCXX]
class Person {
	string Name = "John Smith";
	int Age;
}
\end{lstlisting}

正如上所示,class类似于C和许多其他编程语言中的结构体,只包含一组数据字段。每个字段都有一个类型,可以是任何基本类型(\texttt{int}、\texttt{string}、\texttt{bit}等)或其他用户定义的类类型。字段还可以指定一个默认值,比如\texttt{John Smith}。

了解了布局之后,创建一个实例(对TableGen来说就是一个记录),如下所示:

\begin{lstlisting}[style=styleCXX]
def john_smith : Person;
\end{lstlisting}

这里,\texttt{john\_smith}是一个使用\texttt{Person}作为模板的记录,因此也有两个字段- \texttt{Name}和\texttt{Age}——\texttt{Name}字段用值\texttt{john Smith}填充。这看起来非常简单,但回想一下TableGen应该定义静态数据,并且大多数字段应该用值填充。现在,\texttt{Age}字段还没有初始化。这里可以用括号闭包和语句覆盖它的值,如下所示:

\begin{lstlisting}[style=styleCXX]
def john_smith : Person {
	let Age = 87;
}
\end{lstlisting}

甚至可以为\texttt{john\_smith}记录定义新的字段:

\begin{lstlisting}[style=styleCXX]
def john_smith : Person {
	let Age = 87;
	string Job = "Teacher";
}
\end{lstlisting}

注意,只能覆盖(使用\texttt{let}关键字)声明过的字段。

\subsubsubsection{4.2.1\hspace{0.2cm}叹号操作符}

叹号操作符是一组执行简单任务的函数,例如在TableGen中执行基本的算术或对值进行强制转换。下面是一个简单的把公斤换算成克的例子:

\begin{lstlisting}[style=styleCXX]
class Weight<int kilogram> {
	int Gram = !mul(kilogram, 1000);
}
\end{lstlisting}

常用的操作符包括算术操作符和位操作符(仅举几个例子),其中一些在这里概述:

\begin{itemize}
\ttfamily
\item !add(a, b): 算术加法
\item !sub(a, b): 算术减法
\item !mul(a, b): 算术乘法
\item !and(a, b): 逻辑和
\item !or(a, b): 逻辑或
\item !xor(a, b): 逻辑异或
\end{itemize}

使用条件运算符的情况,下面列出了一些:

\begin{itemize}
\ttfamily
\item !ge(a, b): a大于等于b时返回1,否则返回0
\item !gt(a, b): a大于b时返回1,否则返回0
\item !le(a, b): a小于等于b时返回1,否则返回0
\item !lt(a, b): a小于b时返回1,否则返回0
\item !eq(a, b): a等于b时返回1,否则返回0
\end{itemize}

其他有趣的操作符包括:

\begin{itemize}
\ttfamily

\item !cast<type>(x): 该操作符根据类型参数对x操作数执行类型转换。在类型为数值类型的情况下,例如使用int或bits,这将执行普通的算术类型转换。在一些特殊情况下,假设有以下场景:

如果type是string而x是一个记录,则返回记录的名称。

如果x是一个string,将视为一个记录的名称。TableGen将查找到目前为止的所有记录定义,并对名称为x的记录进行强制转换,并返回与类型参数匹配的类型。

\item !if(pred, then, else): 如果pred为1,该操作符返回then表达式,否则返回else表达式。

\item !cond(cond1 : val1, cond2 : val2, …, condN : valN): 此操作符是!if操作符的增强版本。在返回其关联的val表达式之前,将连续计算cond1…condN,直到其中一个表达式返回1。
\end{itemize}

\begin{tcolorbox}[colback=blue!5!white,colframe=blue!75!black, fonttitle=\bfseries,title=Note]
\hspace*{0.7cm}与运行时求值的函数不同,叹号操作符更像宏,在构建时求值——或者用TableGen的术语来说,当这些语法由TableGen后端处理时进行求值。
\end{tcolorbox}

\subsubsubsection{4.2.3\hspace{0.2cm}多记录}

许多情况下,我们想要同时定义多个记录。例如,下面的代码片段尝试为多辆车创建汽车部件记录:

\begin{lstlisting}[style=styleCXX]
class AutoPart<int quantity> {…}

def car1_fuel_tank : AutoPart<1>;
def car1_engine : AutoPart<1>;
def car1_wheels : AutoPart<4>;
…
def car2_fuel_tank : AutoPart<1>;
def car2_engine : AutoPart<1>;
def car2_wheels : AutoPart<4>;
…
\end{lstlisting}

我们可以通过使用多记录语法进一步简化,如下所示:

\begin{lstlisting}[style=styleCXX]
class AutoPart<int quantity> {…}

multiclass Car<int quantity> {
	def _fuel_tank : AutoPart<quantity>;
	def _engine : AutoPart<quantity>;
	def _wheels : AutoPart<!mul(quantity, 4)>;
	…
}
\end{lstlisting}

创建记录实例时,使用defm语法而不是def,如下所示:

\begin{lstlisting}[style=styleCXX]
defm car1 : Car<1>;
defm car2 : Car<1>;
\end{lstlisting}

因此,仍然会生成名称为\texttt{car1\_fuel\_tank}、\texttt{car1\_engine}、\texttt{car2\_fuel\_tank}的记录。

尽管名称中有\texttt{class},但\texttt{multiclass}与\texttt{class}没有任何关系。\texttt{multiclass}描述的不是记录的布局,而是作为模板来生成记录。在一个\texttt{multiclass}模板中是要创建的预期记录,以及在模板展开之后的记录名称后缀。例如,前面代码段中\texttt{的defm car1: Car<1>}指令最终扩展为三个\texttt{def}指令:

\begin{itemize}
\ttfamily
\item def car1\_fuel\_tank : AutoPart<1>;
\item def car1\_engine : AutoPart<1>;
\item def car1\_wheels : AutoPart<!mul(1, 4)>;
\end{itemize}

正如前面的列表中看到的,会发现在multiclass中的名称后缀(例如,\texttt{\_fuel\_tank})与本例中出现在发现\texttt{defm-car1}之后的名称连接在一起。此外,来自\texttt{multiclass}的\texttt{quantity}模板参数也会实例化到每个扩展的记录中。

简而言之,\texttt{multiclass}会尝试从多个记录实例中提取公共参数,并使一次性创建这些参数成为可能。

\subsubsubsection{4.2.4\hspace{0.2cm}有向无环图(DAG)数据类型}

除了常规的数据类型之外,TableGen还有一个非常独特的第一类类型:\texttt{dag}类型,用于表示DAG实例。要创建一个DAG实例,你可以使用以下语法:

\begin{lstlisting}[style=styleCXX]
(operator operand1, operand2,…, operandN)
\end{lstlisting}

虽然操作符只能是一个记录实例,但操作数\texttt{(operand1…operandN)}可以是任意类型。下面是一个尝试建模算术表达式x * 2 + y + 8 * z的例子:

\begin{lstlisting}[style=styleCXX]
class Variable {…}
class Operator {…}
class Expression<dag expr> {…}

// define variables
def x : Variable;
def y : Variable;
def z : Variable;

// define operators
def mul : Operator;
def plus : Operator;

// define expression
def tmp1 : Expression<(mul x, 2)>;
def tmp2 : Expression<(mul 8, z)>;
def result : Expression<(plus tmp1, tmp2, y)>;
\end{lstlisting}

(可选)可以将操作符和/或每个操作数与标记相关联,如下所示:

\begin{lstlisting}[style=styleCXX]
…
def tmp1 : Expression<(mul:$op x, 2)>;
def tmp2 : Expression<(mul:$op 8, z)>;
def result : Expression<(plus tmp1:$term1, tmp2:$term2,
y:$term3)>;
\end{lstlisting}

标签总是以美元符号\texttt{\$}开头,然后是用户定义的标签名。这些标记提供了每个DAG组件的逻辑描述,在TableGen后端处理DAG时非常有用。

本节中,介绍了TableGen语言的主要组件,并介绍了一些基本的语法。下一节中,将动手使用TableGen编写“美味的甜甜圈”。

















