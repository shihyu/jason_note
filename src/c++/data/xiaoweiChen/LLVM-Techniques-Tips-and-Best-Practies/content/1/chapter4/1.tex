本章主要关注\texttt{utils}文件夹中的一个工具:l\texttt{lvm-tblgen}。使用以下命令对其进行构建:

\begin{tcblisting}{commandshell={}}
$ ninja llvm-tblgen
\end{tcblisting}

\begin{tcolorbox}[colback=blue!5!white,colframe=blue!75!black, fonttitle=\bfseries,title=Note]
\hspace*{0.7cm}如果选择在\textbf{Release模式}下构建\texttt{llvm-tblgen},而不管全局构建类型如何,可以使用第1章中介绍的CMake变量\texttt{LLVM\_OPTIMIZED\_TABLEGEN}。你可能想要更改该设置,因为在本章中使用Debug版本的\texttt{llvm-tblgen}会更好一些。
\end{tcolorbox}

本章中所有的源代码都可以在GitHub库中找到: \url{https://github.com/PacktPublishing/LLVM-Techniques-Tips-andBest-Practices-Clang-and-Middle-End-Libraries/tree/main/
Chapter04}.
















