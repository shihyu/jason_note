在源码树内项目中,因为大多数基础设施已经存在,可以实现编程语言的特性,有利于原型设计。此外,与创建源码树外项目,以及将其链接到LLVM库相比,将整个LLVM源码树拷贝到对应代码库中并不是个好主意。例如,只想使用LLVM的特性创建一个代码重构工具,并想GitHub上开源它,这样的话,其他GitHub上的开发人员需要下载数GB的LLVM源代码,从而才能使用你的小工具,这会让开发者的体验直线下降。

据我所知,至少有两种方式可以用来配置其他项目链接到LLVM源码:

\begin{itemize}
\item 使用\texttt{llvm-config}工具

\item 使用LLVM的CMake模块
	
\end{itemize}

这两种方法可以获取相应的信息,包括头文件和库路径。然而,后者会创建了更简洁和可读的CMake脚本,这对于使用CMake的项目来说比较友好。本节将展示使用LLVM的CMake模块,并将其集成到其他CMake项目的基本步骤。

首先,需要准备一个源码树外(\texttt{C/C++})的CMake项目。在前一节中讨论的核心CMake函数/宏将帮助我们解决这个问题,现在来看下如何使用它们:

\begin{enumerate}
\item 假设项目中已CMakeLists.txt,并需要链接到LLVM库框架:

\begin{lstlisting}[style=styleCMake]
project(MagicCLITool)
set(SOURCE_FILES
	main.cpp)
add_executable(magic-cli
	${SOURCE_FILES})
\end{lstlisting}

不管是在创建生成可执行文件(就像我们在前面的代码块中看到的那样),还是其他工件(如库或甚至LLVM Pass插件),现在最大的问题是如何获取包含路径,以及库路径。

\item 为了解析包含路径和库路径,LLVM提供了CMake包接口,这样直接使用\texttt{find\_package}指令导入各种配置即可:
	
\begin{lstlisting}[style=styleCMake]
project(MagicCLITool)
find_package(LLVM REQUIRED CONFIG)
include_directories(${LLVM_INCLUDE_DIRS})
link_directories(${LLVM_LIBRARY_DIRS})
…
\end{lstlisting}
	
为了让\texttt{find\_package}起作用,需要在项目使用CMake命令时,设置\texttt{LLVM\_DIR}:

\begin{tcblisting}{commandshell={}}
$ cmake -DLLVM_DIR=<LLVM install path>/lib/cmake/llvm …
\end{tcblisting}

确保它指向LLVM安装路径下的lib/cmake/llvm子目录。
	
\item 解析了包含路径和库之后,就可以将主可执行文件链接到LLVM的库了。LLVM的自定义CMake函数(例如,\texttt{add\_llvm\_executable})在这里将非常有用,但需要CMake能够找到这些函数实现。

下面的代码导入了LLVM的CMake模块(更具体地说,是AddLLVM的CMake模块),它包含了前一节中介绍的那些与LLVM相关的函数/宏:

\begin{lstlisting}[style=styleCMake]
find_package(LLVM REQUIRED CONFIG)
…
list(APPEND CMAKE_MODULE_PATH ${LLVM_CMAKE_DIR})
include(AddLLVM)
\end{lstlisting}

\item 下面的代码使用了前一节中介绍的CMake函数,并添加了可执行的构建目标:

\begin{lstlisting}[style=styleCMake]
find_package(LLVM REQUIRED CONFIG)
…
include(AddLLVM)
set(LLVM_LINK_COMPONENTS
	Support
	Analysis)
add_llvm_executable(magic-cli
	main.cpp)
\end{lstlisting}

\item 同理,添加库目标:
\begin{lstlisting}[style=styleCMake]
find_package(LLVM REQUIRED CONFIG)
…
include(AddLLVM)
add_llvm_library(MyMagicLibrary
	lib.cpp
	LINK_COMPONENTS
	Support Analysis)
\end{lstlisting}

\item 最后,添加LLVM Pass插件:
\begin{lstlisting}[style=styleCMake]
find_package(LLVM REQUIRED CONFIG)
…
include(AddLLVM)
add_llvm_pass_plugin(MyMagicPass
	ThePass.cpp)
\end{lstlisting}

\item 实践中,还需要注意\textbf{特定于LLVM的宏定义}和RTTI设置:
\begin{lstlisting}[style=styleCMake]
find_package(LLVM REQUIRED CONFIG)
…
add_definitions(${LLVM_DEFINITIONS})
if(NOT ${LLVM_ENABLE_RTTI})
	# For non-MSVC compilers
	set(CMAKE_CXX_FLAGS "${CMAKE_CXX_FLAGS} -fno-rtti")
endif()
add_llvm_xxx(source.cpp)
\end{lstlisting}

对于RTTI部分更要谨慎,因为在默认情况下,LLVM不支持RTTI,但普通的\texttt{C++}应用支持RTTI。如果代码和LLVM的库之间的RTTI配置不同,编译时会出现错误。

\end{enumerate}

尽管在LLVM的源码树中进行开发很方便,但有时将整个LLVM源代码封装在项目中可能不可行。因此,必须创建树外项目,并将LLVM合成为一个库。本节展示了如何将LLVM集成到基于CMake的其他项目中,并充分利用特性于LLVM的CMake指令。


































