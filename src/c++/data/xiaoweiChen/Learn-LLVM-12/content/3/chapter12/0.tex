LLVM有一个非常灵活的架构。您可以向它添加一个新的后端,后端的核心是目标描述,大部分代码都是从它生成的。但是,还不可能生成一个完整的后端,并且实现调用规则需要进行手动编码。本章中,我们将学习如何添加对老CPU的支持。\par

本章中,我们将学习以下内容:\par

\begin{itemize}
\item 设置一个新的后端时,我们将向您介绍M88k CPU体系架构,并向您展示在哪里可以找到您需要的信息。

\item 将新的体系结构添加到Triple类中,将教会您如何让LLVM了解新的CPU体系架构。

\item 在扩展LLVM中的ELF文件格式定义时,可以向处理ELD对象文件的库和工具中添加对m88k特定重定位的支持。

\item 创建目标描述时,您将使用TableGen语言开发目标描述的所有部分。

\item 实现DAG指令选择类时,您将创建指令选择所需的Pass和支持的类。

\item 生成汇编指令时,会带您了解如何实现汇编打印,生成汇编文本。

\item 生成的机器码时,您将了解必须提供哪些附加类才能使机器码(MC)层将代码写入目标文件。

\item 添加反汇编支持时,您将了解如何实现对反汇编程序的支持。

\item 将这些功能整合在一起时,就可以将新后端的源代码集成到构建系统中。
\end{itemize}

本章结束时,您将了解如何开发一个新的和完整的后端。您将了解后台的不同组成部分,从而更深入地了解LLVM体系结构。\par



















