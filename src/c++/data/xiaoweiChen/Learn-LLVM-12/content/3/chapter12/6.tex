
DAG指令选择器的很大一部分是由llvm-tblgen工具生成的。不过,仍然需要使用生成的代码创建类,并将所有内容放在一起。让我们从初始化过程开始。\par

\hspace*{\fill} \par %插入空行
\textbf{初始化目标机器}

每个后端必须提供至少一个TargetMachine类,通常是LLVMTargetMachine类的子类。M88k\allowbreak TargetMachine类包含代码生成所需的许多细节,还充当其他后端类的工厂,尤其是Subtarget类和TargetPassConfig类。Subtarget类保存代码生成的配置,比如启用了哪些特性。TargetPassConfig类配置后端机器的通行证。M88kTargetMachine类的声明在M88kTargetMachine.h文件中:\par

\begin{lstlisting}[caption={}]
class M88kTargetMachine : public LLVMTargetMachine {
public:
	M88kTargetMachine(/* parameters */);
	~M88kTargetMachine() override;
	const M88kSubtarget *getSubtargetImpl(const Function &)
										  const override;
	const M88kSubtarget *getSubtargetImpl() const = delete;
	TargetPassConfig *createPassConfig(PassManagerBase &PM)
														override;
};
\end{lstlisting}

请注意,每个功能可以有不同的子目标。\par

M88kTargetMachine.cpp文件中的实现非常简单。最有趣的是这个后端机器Pass的设置。这会创建到选择DAG的连接(如果需要,还会创建到全局指令选择的连接)。在类中创建的Pass随后添加到Pass流水中,以便从IR生成目标文件或汇编程序:\par

\begin{lstlisting}[caption={}]
namespace {
class M88kPassConfig : public TargetPassConfig {
public:
	M88kPassConfig(M88kTargetMachine &TM, PassManagerBase
		&PM)
			: TargetPassConfig(TM, PM) {}
		M88kTargetMachine &getM88kTargetMachine() const {
			return getTM<M88kTargetMachine>();
		}
	
		bool addInstSelector() override {
			addPass(createM88kISelDag(getM88kTargetMachine(),
			getOptLevel()));
			return false;
		}
	};
} // namespace

TargetPassConfig *M88kTargetMachine::createPassConfig(
		PassManagerBase &PM) {
	return new M88kPassConfig(*this, PM);
}
\end{lstlisting}

从M88kTargetMachine类返回的SubTarget实现允许访问其他重要类。M88kInstrInfo类返回关于指令的信息,包括寄存器。m88ktargetlower类提供了与调用相关的指令的降低,还允许添加自定义DAG规则。这个类的大部分是由llvm-tblgen工具生成的,而且需要包含生成的头文件。\par

M88kSubTarget.h文件的定义如下:\par

\begin{lstlisting}[caption={}]
#define GET_SUBTARGETINFO_HEADER
#include "M88kGenSubtargetInfo.inc"

namespace llvm {
class M88kSubtarget : public M88kGenSubtargetInfo {
	Triple TargetTriple;
	virtual void anchor();
	
	M88kInstrInfo InstrInfo;
	M88kTargetLowering TLInfo;
	M88kFrameLowering FrameLowering;
	
public:
	M88kSubtarget(const Triple &TT, const std::string &CPU,
				  const std::string &FS,
				  const TargetMachine &TM);
				  
	void ParseSubtargetFeatures(StringRef CPU, StringRef FS);
	
	const TargetFrameLowering *getFrameLowering() const
		override
	{ return &FrameLowering; }
	const M88kInstrInfo *getInstrInfo() const override
	{ return &InstrInfo; }
	const M88kRegisterInfo *getRegisterInfo() const override
	{ return &InstrInfo.getRegisterInfo(); }
	const M88kTargetLowering *getTargetLowering() const
		override
	{ return &TLInfo; }
};
} // end namespace llvm
\end{lstlisting}

接下来,实现选择DAG。\par

\hspace*{\fill} \par %插入空行
\textbf{添加选择DAG的实现}

选择DAG是在同名文件中的M88kDAGtoDAGIsel类中实现的。这里,我们可以从创建目标机器描述中受益:大多数功能都是从这个描述中生成的。第一个实现中,只需要覆盖Select()函数并将其转发给生成的SelectCode函数。某些情况下,可以重写更多函数,例如:如果需要扩展DAG的预处理,或者如果需要添加特殊的内联汇编器规则。\par

因为这个类是一个机器函数的Pass,所以我们也为Pass提供了一个名称。实现的主要部分来自生成的文件,将其包含在类的中间:\par

\begin{lstlisting}[caption={}]
class M88kDAGToDAGISel : public SelectionDAGISel {
	const M88kSubtarget *Subtarget;
public:
	M88kDAGToDAGISel(M88kTargetMachine &TM,
			CodeGenOpt::Level OptLevel)
		: SelectionDAGISel(TM, OptLevel) {}
		
	StringRef getPassName() const override {
		return "M88k DAG->DAG Pattern Instruction Selection";
	}

#include "M88kGenDAGISel.inc"
	void Select(SDNode *Node) override {
		SelectCode(Node);
	}
};
\end{lstlisting}

我们还添加了工厂函数来创建这个文件中的Pass:\par

\begin{lstlisting}[caption={}]
FunctionPass *llvm::createM88kISelDag(M88kTargetMachine &TM,
									CodeGenOpt::Level
									OptLevel) {
	return new M88kDAGToDAGISel(TM, OptLevel);
}
\end{lstlisting}

现在可以实现特定于目标的操作,这些操作不能在目标描述中表示。\par

\hspace*{\fill} \par %插入空行
\textbf{支持针对性的操作}

转向m88ktargetlower类,在M88kISelLowering.h文件中定义。这个类配置指令DAG选择过程,并增强底层对特定于目标的操作。\par

目标描述中,我们定义了新的DAG节点。此文件中还定义了与新类型一起使用的枚举,并继续使用最后一个预定义数字进行编号:\par

\begin{lstlisting}[caption={}]
namespace M88kISD {
enum NodeType : unsigned {
	FIRST_NUMBER = ISD::BUILTIN_OP_END,
	RET_FLAG,
	SET,
};
} // end namespace M88kISD
\end{lstlisting}

类需要为函数调用提供所需的降级方法。为了简单起见,只看返回值。类还可以为需要自定义处理的操作定义LowerOperation()钩子方法。还可以启用自定义DAG组合方法,为此我们定义了PerformDAGCombine()方法:\par

\begin{lstlisting}[caption={}]
class M88kTargetLowering : public TargetLowering {
	const M88kSubtarget &Subtarget;
	
public:
	explicit M88kTargetLowering(const TargetMachine &TM,
								const M88kSubtarget &STI);
	
	SDValue LowerOperation(SDValue Op, SelectionDAG &DAG) const
														override;
	
	SDValue PerformDAGCombine(SDNode *N, DAGCombinerInfo &DCI)
													const override;
	
	SDValue LowerReturn(SDValue Chain, CallingConv::ID CallConv,
			bool IsVarArg,
			const SmallVectorImpl<ISD::OutputArg> &Outs,
			const SmallVectorImpl<SDValue> &OutVals,
			const SDLoc &DL,
			SelectionDAG &DAG) const override;
};
\end{lstlisting}

该类的实现在m88kisellowering.cpp中。首先,我们看看如何降级返回值:\par

\begin{enumerate}
\item 为调用规则生成的函数是需要的,因此需要包含生成的文件:
\begin{lstlisting}[caption={}]
#include "M88kGenCallingConv.inc"
\end{lstlisting}

\item LowerReturn()方法接受许多参数,这些参数都由targetlower超类定义。最重要的是out向量,保存了返回参数的描述,以及outval向量,还保存了返回值的DAG节点:
\begin{lstlisting}[caption={}]
SDValue M88kTargetLowering::LowerReturn(SDValue Chain,
	CallingConv::ID CallConv,
	bool IsVarArg,
	const SmallVectorImpl<ISD::OutputArg>
		&Outs,
	const SmallVectorImpl<SDValue> &OutVals,
	const SDLoc &DL, SelectionDAG &DAG) const {
\end{lstlisting}

\item 我们在CCState类的帮助下分析返回参数,将引用传递给生成的RetCC\underline{~}M88k函数。因此,需要对所有返回参数进行分类:
\begin{lstlisting}[caption={}]
	MachineFunction &MF = DAG.getMachineFunction();
	SmallVector<CCValAssign, 16> RetLocs;
	CCState RetCCInfo(CallConv, IsVarArg, MF, RetLocs,
										*DAG.getContext());
	RetCCInfo.AnalyzeReturn(Outs, RetCC_M88k);
\end{lstlisting}

\item 对于void函数,没有什么可做的,返回就好。请注意,返回的节点类型是RET\underline{~}FLAG。在目标描述中将其定义为新的ret\underline{~}flag节点:
\begin{lstlisting}[caption={}]
	if (RetLocs.empty())
		return DAG.getNode(M88kISD::RET_FLAG, DL,
							MVT::Other, Chain);
\end{lstlisting}

\item 否则,就需要循环遍历返回参数。对于每个返回参数,都有一个CCValAssign类的实例,它会告诉我们如何处理参数:
\begin{lstlisting}[caption={}]
	SDValue Glue;
	SmallVector<SDValue, 4> RetOps;
	RetOps.push_back(Chain);
	for (unsigned I = 0, E = RetLocs.size(); I != E;
			++I) {
		CCValAssign &VA = RetLocs[I];
		SDValue RetValue = OutVals[I];
\end{lstlisting}

\item 这些值可能需要提升。如果有必要,可以添加一个带有所需扩展操作的DAG节点:
\begin{lstlisting}[caption={}]
		switch (VA.getLocInfo()) {
			case CCValAssign::SExt:
				RetValue = DAG.getNode(ISD::SIGN_EXTEND, DL,
										VA.getLocVT(), RetValue);
				break;
			case CCValAssign::ZExt:
				RetValue = DAG.getNode(ISD::ZERO_EXTEND, DL,
										VA.getLocVT(), RetValue);
				break;
			case CCValAssign::AExt:
				RetValue = DAG.getNode(ISD::ANY_EXTEND, DL,
										VA.getLocVT(), RetValue);
				break;
			case CCValAssign::Full:
				break;
			default:
				llvm_unreachable("Unhandled VA.getLocInfo()");
		}
\end{lstlisting}

\item 当值具有正确的类型时,将该值复制到寄存器中以返回该值,并与副本的链接并粘合在一起。这样循环就结束了:
\begin{lstlisting}[caption={}]
		Register Reg = VA.getLocReg();
		Chain = DAG.getCopyToReg(Chain, DL, Reg, RetValue,
								Glue);
		Glue = Chain.getValue(1);
		RetOps.push_back(DAG.getRegister(Reg,
										VA.getLocVT()));
	}
\end{lstlisting}

\item 最后,我们需要更新连接和粘合:
\begin{lstlisting}[caption={}]
	RetOps[0] = Chain;
	if (Glue.getNode())
		RetOps.push_back(Glue);
\end{lstlisting}
	
\item 然后,返回re\underline{~}flag节点,连接降级后的结果:
\begin{lstlisting}[caption={}]
	return DAG.getNode(M88kISD::RET_FLAG, DL,
		MVT::Other,
						RetOps);
}
\end{lstlisting}
	
\end{enumerate}

为了能够调用函数,必须实现LowerFormalArguments()和LowerCall()方法。这两种方法都遵循类似的方法,因此这里没有展示。\par

\hspace*{\fill} \par %插入空行
\textbf{降级配置目标}

降级函数调用和参数的方法必须实现,因为它们依赖于目标。其他操作可能在目标体系结构中得到支持,也可能没有。为了让降级过程了解它,我们在m88ktargetlower类的构造函数中设置了配置:\par

\begin{enumerate}
\item 构造函数将TargetMachine和M88kSubtarget实例作为参数,并用它们初始化了相应的字段:
\begin{lstlisting}[caption={}]
M88kTargetLowering::M88kTargetLowering(
		const TargetMachine &TM, const M88kSubtarget &STI)
	: TargetLowering(TM), Subtarget(STI) {
\end{lstlisting}

\item 首先添加所有的注册类。这里,只定义了通用寄存器,因此它只是一个简单的调用:
\begin{lstlisting}[caption={}]
	addRegisterClass(MVT::i32, &M88k::GPRRegClass);
\end{lstlisting}

\item 添加所有寄存器类之后,计算寄存器的派生属性,例如:由于寄存器是32位宽的,这个函数将64位数据类型标记为需要两个寄存器:
\begin{lstlisting}[caption={}]
	computeRegisterProperties(Subtarget.getRegisterInfo());
\end{lstlisting}

\item 还需要告诉堆栈指针使用了哪个寄存器。在M88k架构中,使用r31寄存器:
\begin{lstlisting}[caption={}]
	setStackPointerRegisterToSaveRestore(M88k::R31);
\end{lstlisting}

\item 还需要定义布尔值是如何表示的。基本上,就是这里使用的值是0和1。其他可能的选项是只查看值的第0位,忽略其他所有位,并将值的所有位设置为0或1:
\begin{lstlisting}[caption={}]
	setBooleanContents(ZeroOrOneBooleanContent);
\end{lstlisting}

\item 对于每个需要特殊处理的操作,必须调用setOperationAction()方法。该方法采用要采用的操作、值类型和操作作为输入。如果操作有效,则使用Legal action值。如果类型应该提升,则使用Promote操作值,如果操作导致库调用,则使用LibCall对值进行操作。\par
如果给定Expand操作值,则指令选择首先尝试将该操作扩展为其他操作。如果不能,则使用库调用。最后,如果使用Custom操作值,我们可以实现自己的操作。本例中,将为具有此操作的节点调用LowerOperation()方法,例如:将CTTZ计数末尾为零的操作设置为Expand操作。这个操作将使用一系列基本的位操作所替代:
\begin{lstlisting}[caption={}]
	setOperationAction(ISD::CTTZ, MVT::i32, Expand);
\end{lstlisting}

\item M88k体系结构具有位字段操作,在目标描述中定义模式并不容易。这里,告诉指令选择,我们想要在或DAG节点上执行额外的匹配:
\begin{lstlisting}[caption={}]
	setTargetDAGCombine(ISD::OR);
}
\end{lstlisting}

\end{enumerate}

根据目标体系结构,在构造函数中设置配置可能要长得多。我们只定义了最小值,忽略值,例如:浮点运算。\par

我们在上面标记了或操作来执行自定义组合。因此,指令选择器在调用生成的指令选择之前调用PerformDAGCombine()方法。这个函数会在指令选择的各个阶段调用,但通常只在操作合法化后执行匹配。常见的实现是查看操作和处理分支函数:\par

\begin{lstlisting}[caption={}]
SDValue M88kTargetLowering::PerformDAGCombine(SDNode *N,
DAGCombinerInfo &DCI) const {
	if (DCI.isBeforeLegalizeOps())
		return SDValue();
	switch (N->getOpcode()) {
		default:
			break;
		case ISD::OR:
			return performORCombine(N, DCI);
	}
	return SDValue();
}
\end{lstlisting}

performcombine()方法中,尝试检查是否可以为or操作生成一个set指令。set指令将从指定的位偏移量开始的连续位数设置为1。这是or操作的一种特殊情况,第二个操作数是匹配这种格式的常量。因为M88k体系结构的or指令只能在16位常量上工作,所以这种匹配是有益的;否则,将合成常量,从而产生两个or指令。该方法使用isShiftedMask()帮助函数来确定常量值是否具有所需的数据。\par

如果第二个操作数是必需形式的常量,则该函数返回一个表示set指令的节点。否则,返回值SDValue()表示没有找到匹配的模式,需要调用生成的DAG模式匹配器:\par

\begin{lstlisting}[caption={}]
SDValue performORCombine(SDNode *N,
		TargetLowering::DAGCombinerInfo &DCI) {
	SelectionDAG &DAG = DCI.DAG;
	uint64_t Width, Offset;
	ConstantSDNode *Mask =
					dyn_cast<ConstantSDNode>(N->getOperand(
						1));
	if (!Mask ||
		!isShiftedMask(Mask->getZExtValue(), Width, Offset))
		return SDValue();
	
	EVT ValTy = N->getValueType(0);
	SDLoc DL(N);
	return DAG.getNode(M88kISD::SET, DL, ValTy,
			N->getOperand(0),
			DAG.getConstant(Width << 5 | Offset, DL,
				MVT::i32));
}
\end{lstlisting}

为了完成整个降低过程的实现,需要实现M88kFrameLowering类。这个类负责处理堆栈帧,这包括生成头部和尾部代码、处理寄存器溢出等等。\par

对于第一个实现,可以只提供空函数。显然,为了实现完整的功能,必须实现这个类,这就完成了指令选择的实现。接下来,看看最后的指令如何产生。\par





























