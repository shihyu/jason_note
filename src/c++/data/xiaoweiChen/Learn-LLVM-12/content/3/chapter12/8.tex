MC层负责以文本或二进制形式发出机器码。大多数功能要么在各种MC类中实现(只需要配置),要么从目标描述生成实现。\par

MC层的初始化在MCTargetDesc/M88kMCTargetDesc.cpp中进行。以下类是在targeregistry单例中注册的:\par

\begin{enumerate}
\item M88kMCAsmInfo: 该类提供基本信息,如代码指针的大小、堆栈增长的方向、注释符号或汇编指令的名称。

\item M88MCInstrInfo: 该类包含指令的信息,例如:指令名称。

\item M88kRegInfo: 该类提供有关寄存器的信息,例如:寄存器的名称,或哪个寄存器是堆栈指针。

\item M88kSubtargetInfo: 这个类保存调度模型的数据,以及解析和设置CPU特性的方法。

\item M88kMCAsmBackend: 这个类提供了助手方法来获取与目标相关的重新定位数据。它还包含对象书写器类的工厂方法。

\item M88kMCInstPrinter:该类包含以文本形式打印指令和操作数的助手方法。如果操作数在目标描述中定义了自定义打印方法,则要在该类中实现该方法。

\item M88kMCCodeEmitter: 该类将指令的编码写入流。
\end{enumerate}

根据后端实现的范围,我们不需要注册和实现所有这些类。如果不支持文本汇编器输出,可以忽略注册MCInstPrinter子类。如果不添加对写入对象文件的支持,可以省略MCAsmBackend和MCCodeEmitter子类。\par
 
文件开始包括生成的部分和提供它的工厂方法:\par

\begin{lstlisting}[caption={}]
#define GET_INSTRINFO_MC_DESC
#include "M88kGenInstrInfo.inc"
#define GET_SUBTARGETINFO_MC_DESC
#include "M88kGenSubtargetInfo.inc"
#define GET_REGINFO_MC_DESC
#include "M88kGenRegisterInfo.inc"

static MCInstrInfo *createM88kMCInstrInfo() {
	MCInstrInfo *X = new MCInstrInfo();
	InitM88kMCInstrInfo(X);
	return X;
}

static MCRegisterInfo *createM88kMCRegisterInfo(
										const Triple &TT) {
	MCRegisterInfo *X = new MCRegisterInfo();
	InitM88kMCRegisterInfo(X, M88k::R1);
	return X;
}

static MCSubtargetInfo *createM88kMCSubtargetInfo(
				const Triple &TT, StringRef CPU, StringRef
					FS) {
	return createM88kMCSubtargetInfoImpl(TT, CPU, FS);
}
\end{lstlisting}

我们还为在其他文件中实现的类提供了一些工厂方法:\par

\begin{lstlisting}[caption={}]
static MCAsmInfo *createM88kMCAsmInfo(
					const MCRegisterInfo &MRI, const Triple &TT,
					const MCTargetOptions &Options) {
	return new M88kMCAsmInfo(TT);
}

static MCInstPrinter *createM88kMCInstPrinter(
					const Triple &T, unsigned SyntaxVariant,
					const MCAsmInfo &MAI, const MCInstrInfo &MII,
					const MCRegisterInfo &MRI) {
return new M88kInstPrinter(MAI, MII, MRI);
}
\end{lstlisting}

为了初始化MC层,只需要将所有的工厂方法注册到TargetRegistry单例中即可:\par

\begin{lstlisting}[caption={}]
extern "C" LLVM_EXTERNAL_VISIBILITY
void LLVMInitializeM88kTargetMC() {
	TargetRegistry::RegisterMCAsmInfo(getTheM88kTarget(),
										createM88kMCAsmInfo);
	TargetRegistry::RegisterMCCodeEmitter(getTheM88kTarget(),
										createM88kMCCodeEmitter);
	TargetRegistry::RegisterMCInstrInfo(getTheM88kTarget(),
										createM88kMCInstrInfo);
	TargetRegistry::RegisterMCRegInfo(getTheM88kTarget(),
										createM88kMCRegisterInfo);
	TargetRegistry::RegisterMCSubtargetInfo(getTheM88kTarget(),
										createM88kMCSubtargetInfo);
	TargetRegistry::RegisterMCAsmBackend(getTheM88kTarget(),
										createM88kMCAsmBackend);
	TargetRegistry::RegisterMCInstPrinter(getTheM88kTarget(),
										createM88kMCInstPrinter);
}
\end{lstlisting}

另外,在MCTargetDesc/M88kTargetDesc.h头文件中,还需要包含生成的源文件的头文件部分:\par

\begin{lstlisting}[caption={}]
#define GET_REGINFO_ENUM
#include "M88kGenRegisterInfo.inc"
#define GET_INSTRINFO_ENUM

#include "M88kGenInstrInfo.inc"
#define GET_SUBTARGETINFO_ENUM
#include "M88kGenSubtargetInfo.inc"
\end{lstlisting}

我们将注册类的源文件全部放在MCTargetDesc目录中。对于第一个实现,只提供这些类的部分就足够了,例如:只要不将内存地址支持添加到目标描述中,就不会生成补丁。M88kMCAsmInfo类可以很快实现,只需要在构造函数中设置一些属性:\par

\begin{lstlisting}[caption={}]
M88kMCAsmInfo::M88kMCAsmInfo(const Triple &TT) {
	CodePointerSize = 4;
	IsLittleEndian = false;
	MinInstAlignment = 4;
	CommentString = "#";
}
\end{lstlisting}

实现了MC层的支持类之后,就可以将机器码生成到文件中了。\par

下一节中,我们实现反汇编所需的类,这是反向的操作:将对象文件转换回汇编文本。\par
























