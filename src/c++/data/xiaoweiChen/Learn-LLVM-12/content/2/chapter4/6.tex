如前一章所示,解析器是从语法派生出来的。让我们回顾一下所有的构造规则:对于每个语法规则,都要创建一个以规则左侧的非终结符命名的方法,以便解析规则的右侧。根据右边的定义,必须做以下事情:\par

\begin{itemize}
\item 对于每个非终结符,调用相应的方法。
\item 处理每个令牌。
\item 对于可选组和可选组或重复组,将检查前瞻性令牌(下一个未使用的令牌),以决定从哪里继续。
\end{itemize}

让我们将这些结构规则应用到以下语法规则中:\par

\begin{tcolorbox}[colback=white,colframe=black]
ifStatement \\
\hspace*{0.5cm}: "IF" expression "THEN" statementSequence \\
\hspace*{1cm}( "ELSE" statementSequence )? "END" ;
\end{tcolorbox}

可以很容易地将其转换为以下C++方法:\par

\begin{lstlisting}[caption={}]
void Parser::parseIfStatement() {
	consume(tok::kw_IF);
	parseExpression();
	consume(tok::kw_THEN);
	parseStatementSequence();
	if (Tok.is(tok::kw_ELSE)) {
		advance();
		parseStatementSequence();
	}
	consume(tok::kw_END);
}
\end{lstlisting}

tinylang的整个语法可以用这种方式转换成C++。不过,必须小心并避免一些陷阱。\par

需要注意的一个问题是左递归规则。如果右边和左边以相同的终端开始,规则是左递归的。表达式的语法中可以找到一个典型的例子:\par

\begin{tcolorbox}[colback=white,colframe=black]
expression : expression "+" term ;
\end{tcolorbox}

如果还不清楚语法,那么下面对C++的翻译应该可以清楚地表明,这将导致无限递归:\par

\begin{lstlisting}[caption={}]
void Parser::parseExpression() {
	parseExpression();
	consume(tok::plus);
	parseTerm();
}
\end{lstlisting}

左递归也可能间接发生,涉及更多规则,这很难发现。这就是为什么存在一种可以检测和消除左递归的算法。\par

每个步骤中,解析器决定如何仅通过使用预先标记继续。如果不能确定,语法就会发生冲突。为了说明这一点,让我们看看C\#中的using语句。与C++类似,using语句可用于在名称空间中使符号可见,例如在using Math;中。还可以为导入的符号定义别名;也就是说,使用\textit{M = Math;}在语法中,可以这样表示:\par

\begin{tcolorbox}[colback=white,colframe=black]
usingStmt : "using" (ident "=")? ident ";"
\end{tcolorbox}

显然,这里有个问题。解析器使用了using关键字之后,就标识了预先标记。但是,这些信息不足以让我们决定是否必须跳过或解析可选组。如果可选组开始的一组标记与可选组后面的一组标记重叠,则总是会出现这种情况。\par

我们用一个替代方案来重写这个规则:\par


\begin{tcolorbox}[colback=white,colframe=black]
usingStmt : "using" ( ident "=" ident | ident ) ";" ;
\end{tcolorbox}

现在,有一个不同的冲突:两个选择都以相同的标记开始。只看预先标记,解析器不能决定哪个选项是正确的。\par

这些冲突非常普遍。因此,需要了解如何处理它们。一种方法是以冲突消失的方式重写语法。在前面的示例中,两个选项都以相同的标记开始。这可以分解,得到以下规则:\par

\begin{tcolorbox}[colback=white,colframe=black]
usingStmt : "using" ident ("=" ident)? ";" ;
\end{tcolorbox}

这种表示没有冲突。然而,还应该注意的是,它的表达性较差。其他两个公式中,很明显哪个标识是别名,哪个标识是名称空间名称。这个无冲突规则中,最左边的标识改变了它的角色。首先,是名称空间名称,但是如果后面跟着等号(=),那么就变成别名。\par

第二种方法是添加一个额外的谓词来区分这两种情况。这个谓词通常称为解释器,它可以使用上下文信息进行决策(例如符号表中的名称查找),但可以查看多个令牌。让我们假设lexer有一个\textit{Token \&peek(int n)}方法,在当前查找令牌之后返回第n个令牌。这里,等号的存在可以作为判定中的附加谓词:\par

\begin{lstlisting}[caption={}]
if (Tok.is(tok::ident) && Lex.peek(0).is(tok::equal)) {
	advance();
	consume(tok::equal);
}
consume(tok::ident);
\end{lstlisting}

现在,让我们加入错误恢复。前一章中,介绍了\textit{恐慌模式}作为一种错误恢复技术。基本思想是跳过标记,直到找到适合继续解析的标记为止。例如,在tinylang中,语句后面跟着分号(:)。\par

如果If语句中存在语法问题,则跳过所有标记,直到找到分号。然后,继续下一个。与其为令牌集的特别定义标记,不如使用系统方法。\par

对于每个非终结符,计算可以在跟随该非终结符的标记集(称为FOLLOW集)。对于非终结符语句,可以跟;、ELSE和END标记。因此,可以在parseStatement()的错误恢复部分中使用这个集合。当然,此方法假设语法错误可以在本地处理。通常来说,这不太可能。因为解析器会跳过标记,所以可能会跳过很多标记,以至于到达输入的末尾。所以,无法进行本地恢复。\par

为了防止无意义的错误消息,需要通知调用方法错误恢复仍未完成。这可以通过bool返回值来完成:true表示错误恢复尚未完成,false表示解析(包括可能的错误恢复)成功。\par

有许多方法可以扩展这个错误恢复方案,一种流行的方法是也使用活动调用者的FOLLOW集合。一个简单的例子,让我们假设parseStatement()由parseStatementSequence()调用,而parseBlock()本身调用parseBlock(),而parseModule()调用parseStatement()。\par

这里,每个对应的非终端都有一个FOLLOW集。如果解析器检测到parseStatement()中的语法错误,那么将跳过令牌,直到令牌至少出现在活动调用者的一个FOLLOW集合中。如果令牌在语句的FOLLOW集合中,则在本地恢复错误,并将一个假值返回给调用者。否则,返回一个真值,这意味着错误恢复必须继续。这个扩展的一个可能的实现策略是,传递std::bitset或std::tuple来表示当前FOLLOW的集合返回给调用者。\par

最后一个问题仍然未解决:如何调用错误恢复?前一章中,使用goto跳转到错误恢复块。这可行,但不是很优雅的解决方案。根据前面的讨论,可以在单独的方法中跳过标记。为此,Clang有一个名为skipUntil()的方法,我们也可以将其用于tinylang。\par

因为下一步是向解析器添加语义操作,所以如果有必要,最好有找个位置完成清理代码(嵌套函数是不错的选择)。C++没有嵌套函数,所以Lambda函数可以用于这项任务。具有错误恢复的parseIfStatement()如下所示:\par

\begin{lstlisting}[caption={}]
bool Parser::parseIfStatement() {
	auto _errorhandler = [this] {
		return SkipUntil(tok::semi, tok::kw_ELSE, tok::kw_END);
	};
	if (consume(tok::kw_IF))
		return _errorhandler();
	if (parseExpression(E))
		return _errorhandler();
	if (consume(tok::kw_THEN))
		return _errorhandler();
	if (parseStatementSequence(IfStmts))
		return _errorhandler();
	if (Tok.is(tok::kw_ELSE)) {
		advance();
		if (parseStatementSequence(ElseStmts))
			return _errorhandler();
	}
	if (expect(tok::kw_END))
		return _errorhandler();
	return false;
}
\end{lstlisting}
























