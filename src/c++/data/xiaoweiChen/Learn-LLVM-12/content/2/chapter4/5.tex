正如前一章中所知,我们需要一个Token类和一个Lexer类。此外,需要一个TokenKind枚举来给每个令牌类一个的数字。拥有一体化的头文件和一个实现文件是无法扩展的,所以需要重构这些内容。TokenKind枚举可以复用,并放置在Basic组件中。Token和Lexer类属于Lexer组件,但是放在不同的头文件和实现文件中。\par

有三种不同类型的标记:关键字、标点符号和表示多值集的标记,例如:CONST关键字,分号分隔符和标识符,它们表示源中的标识符,每个令牌都需要枚举的成员名。关键字和标点符号具有可以用于消息的显示名称。\par

与许多编程语言一样,关键字是标识符的子集。为了将标记分类为关键字,我们需要一个关键字过滤器,它检查所找到的标识符是否确实是关键字。这与C或C++中的行为相同,其中关键字也是标识符的子集。编程语言随着时间的推移而发展,可能会引入新的关键字,例如:K\&R C语言没有使用enum关键字定义的枚举。因此,应该有指示关键字的语言级别的标志。\par

我们收集了几条信息,它们都属于TokenKind枚举的一个成员:枚举成员的标签、标点符号的拼写和关键字的标志。至于调试消息,我们将信息集中存储在一个名为include/tinylang/Basic/Token\allowbreak Kinds.def的文件中,文件的内容如下所示。需要注意的是,关键字的前缀是kw\underline{~}:\par

\begin{lstlisting}[caption={}]
#ifndef TOK
#define TOK(ID)
#endif
#ifndef PUNCTUATOR
#define PUNCTUATOR(ID, SP) TOK(ID)
#endif
#ifndef KEYWORD
#define KEYWORD(ID, FLAG) TOK(kw_ ## ID)
#endif

TOK(unknown)
TOK(eof)
TOK(identifier)
TOK(integer_literal)

PUNCTUATOR(plus, "+")
PUNCTUATOR(minus, "-")
// …

KEYWORD(BEGIN , KEYALL)
KEYWORD(CONST , KEYALL)
// …

#undef KEYWORD
#undef PUNCTUATOR
#undef TOK
\end{lstlisting}

有了这些集中的定义,就很容易在include/tinylang/Basic/TokenKinds.h中创建TokenKind枚举。同样,枚举也可以放在自己的命名空间tok中:\par

\begin{lstlisting}[caption={}]
#ifndef TINYLANG_BASIC_TOKENKINDS_H
#define TINYLANG_BASIC_TOKENKINDS_H

namespace tinylang {
	
namespace tok {
enum TokenKind : unsigned short {
#define TOK(ID) ID,
#include "TokenKinds.def"
	NUM_TOKENS
};
\end{lstlisting}

填充数组使用的模式已经很熟悉了,TOK宏定义为仅返回枚举标签的ID。作为一个有用的加法,我们还将NUM\underline{~}TOKENS定义为枚举的最后一个成员,表示已定义标记的数量:\par

\begin{lstlisting}[caption={}]
		const char *getTokenName(TokenKind Kind);
		const char *getPunctuatorSpelling(TokenKind Kind);
		const char *getKeywordSpelling(TokenKind Kind);
	}
}

#endif
\end{lstlisting}

实现文件lib/Basic/TokenKinds.cpp也使用.def文件来检索名称:\par

\begin{lstlisting}[caption={}]
#include "tinylang/Basic/TokenKinds.h"
#include "llvm/Support/ErrorHandling.h"

using namespace tinylang;

static const char * const TokNames[] = {
#define TOK(ID) #ID,
#define KEYWORD(ID, FLAG) #ID,
#include "tinylang/Basic/TokenKinds.def"
	nullptr
};
\end{lstlisting}

令牌的文本名称派生自其枚举标签的ID,其有两个特点。首先,需要定义两个TOK和KEYWORD宏,因为KEYWORD的默认定义不使用TOK宏。其次,会在数组的末尾添加一个nullptr值,用于计算NUM\underline{~}TOKENS枚举成员:\par

\begin{lstlisting}[caption={}]
const char *tok::getTokenName(TokenKind Kind) {
	return TokNames[Kind];
}
\end{lstlisting}

对于getPunctuatorSpelling()getKeywordSpelling()函数,我们采用了不同的方法。这些函数只返回枚举子集的有意义的值。可以通过switch语句实现,该语句默认返回nullptr值:\par

\begin{lstlisting}[caption={}]
const char *tok::getPunctuatorSpelling(TokenKind Kind) {
	switch (Kind) {
#define PUNCTUATOR(ID, SP) case ID: return SP;
#include "tinylang/Basic/TokenKinds.def"
		default: break;
	}
	return nullptr;
}

const char *tok::getKeywordSpelling(TokenKind Kind) {
	switch (Kind) {
#define KEYWORD(ID, FLAG) case kw_ ## ID: return #ID;
#include "tinylang/Basic/TokenKinds.def"
		default: break;
	}
	return nullptr;
}
\end{lstlisting}

\begin{tcolorbox}[colback=blue!5!white,colframe=blue!75!black,title=Tip]
注意如何定义宏从文件中检索所需的信息片段。
\end{tcolorbox}

前一章中,Token类在与Lexer类相同的头文件中声明。为了使其更加模块化,我们将把Token类放到include/Lexer/Token.h中。前面的例子中,Token存储了一个指针,指向标记的开始、长度和标记的类型:\par

\begin{lstlisting}[caption={}]
class Token {
	friend class Lexer;
	
	const char *Ptr;
	size_t Length;
	tok::TokenKind Kind;
	
public:
	tok::TokenKind getKind() const { return Kind; }
	size_t getLength() const { return Length; }
\end{lstlisting}

SMLoc实例表示源在消息中的位置,它是使用指向令牌的指针创建:\par

\begin{lstlisting}[caption={}]
SMLoc getLocation() const {
	return SMLoc::getFromPointer(Ptr);
}
\end{lstlisting}

getIdentifier()和getLiteralData()允许我们访问标识符和文字数据的标记文本。没有必要访问任何其他令牌类型的文本,因为这是隐式的令牌类型:\par

\begin{lstlisting}[caption={}]
	StringRef getIdentifier() {
		assert(is(tok::identifier) &&
				"Cannot get identfier of non-identifier");
		return StringRef(Ptr, Length);
	}
	StringRef getLiteralData() {
		assert(isOneOf(tok::integer_literal,
					   tok::string_literal) &&
				"Cannot get literal data of non-literal");
		return StringRef(Ptr, Length);
	}
};
\end{lstlisting}

我们在include/Lexer/Lexer.h头文件中声明Lexer类,并将实现放在lib/Lexer/Lexer.cpp文件中。其结构与前一章的calc语言相同。这里,我们必须留意两个细节:\par

\begin{itemize}
\item 首先,有些操作符共享相同的前缀,例如:<和<=。当我们正在查看的当前字符是<时,首先检查下一个字符,然后再决定我们找到了哪个标记。记住,我们要求输入以空字节结束。因此,如果当前字符有效,则可以使用下一个字符:
\begin{lstlisting}[caption={}]
case '<':
	if (*(CurPtr + 1) == '=')
		formTokenWithChars(token, CurPtr + 2, tok::lessequal);
	else
		formTokenWithChars(token, CurPtr + 1, tok::less);
	break;
\end{lstlisting}

\item 另一个细节是,如有更多的关键词,我们该如何处理?一个简单而快速的解决方案是用关键字填充哈希表,这些关键字都存储在tokenkind.def文件中,可在实例化Lexer类时完成。这种方法还可以支持不同级别的语言,因为可以用附加的标志过滤关键字。这里,还不需要这种灵活性。头文件中,关键字过滤器定义如下,使用一个实例llvm::StringMap的哈希表:
\begin{lstlisting}[caption={}]
class KeywordFilter {
	llvm::StringMap<tok::TokenKind> HashTable;
	void addKeyword(StringRef Keyword,
					tok::TokenKind TokenCode);
public:
	void addKeywords();
\end{lstlisting}

getKeyword()方法返回给定字符串的标记类型,如果字符串不代表关键字,则返回默认值:
\begin{lstlisting}[caption={}]
	tok::TokenKind getKeyword(
			StringRef Name,
			tok::TokenKind DefaultTokenCode = tok::unknown) {
		auto Result = HashTable.find(Name);
		if (Result != HashTable.end())
			return Result->second;
		return DefaultTokenCode;
	}
};
\end{lstlisting}

在实现文件中,关键字表填充为:
\begin{lstlisting}[caption={}]
void KeywordFilter::addKeyword(StringRef Keyword,
tok::TokenKind TokenCode)
{
	HashTable.insert(std::make_pair(Keyword, TokenCode));
}

void KeywordFilter::addKeywords() {
#define KEYWORD(NAME, FLAGS)
addKeyword(StringRef(#NAME), tok::kw_##NAME);
#include "tinylang/Basic/TokenKinds.def"
}
\end{lstlisting}

\end{itemize}

使用这些技术,编写一个高效的lexer类并不困难。由于编译速度很重要,许多编译器都使用手写的分析器,Clang就是其中之一。\par


































