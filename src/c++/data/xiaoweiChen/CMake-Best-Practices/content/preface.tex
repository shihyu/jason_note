\begin{flushright}
	\zihao{1} 前言
\end{flushright}

软件世界每天都在发展,CMake也不例外。经过20多年的不断发展,现在CMake是构建C++应用程序的行业标准。虽然CMake具有非常丰富的功能,但其文档中的例子和指南实在是太少了。而这,恰恰就是CMake实践的最佳切入点。

本书不会去解释CMake的细节和特性,而会通过例子来说明在构建软件时,如何使用CMake完成各种任务(而不是覆盖每种边缘情况)。本书的目的是让事情尽可能简单,同时推荐的最佳实践,从而使读者只需要知道自己需要的CMake功能即可。

我们将先解释概念,然后用例子来说明,通过实践进行学习,并且可将这些知识应用到日常工作中。估计本书的读者大多是工程师,所以我们对书的内容进行了相应的调整。写这本书的时候,我们首先扮演的角色是软件工程师,然后才是作者。因此,本书的内容会更实用,而不仅停留在理论层面,从而证明了相应的技术可以在日常工作中使用。

从工程师中来,再回到工程师中去。希望各位能享受这本书。

\hspace*{\fill} \\ %插入空行
\textbf{适宜读者}

这本书针对软件工程师和构建系统维护人员,他们经常使用C或C++,从而可以使用CMake来减轻日常任务的重担。基本的C++和编程知识有助于读者更好地理解书中的例子。

\hspace*{\fill} \\ %插入空行
\textbf{本书内容}

\textit{第1章,启用CMake},介绍CMake,安装CMake和使用CMake尝试进行构建,从而可以了解如何手动安装CMake最新的稳定版本(即使包管理器没有提供)。还介绍了CMake的基本概念,以及为什么CMake是构建系统生成器,而不是构建系统,以及了解CMake如何与现代C++(和C)一起工作。

\textit{第2章,CMake的最佳使用方式},了解如何在GUI和命令行中使用CMake,以及CMake如何与一些常见的IDE和编辑器集成。

\textit{第3章,创建CMake项目},通过项目来构建可执行文件和库,并将两者连接在一起。

\textit{第4章,打包、部署和安装CMake项目},了解如何创建软件项目的可分发版本。可以了解如何添加安装说明和使用CMake和CPack(CMake的打包程序)打包项目。

\textit{第5章,集成第三方库和依赖管理},了解如何将现有的第三方库集成到项目中。还可以了解如何添加已经安装在系统上的库、外部CMake项目和非CMake项目。

\textit{第6章,自动生成文档},探索如何从代码中生成文档,doxygen、dot(graphviz)和plantuml可以作为构建过程的一部分。

\textit{第7章,集成代码质量工具},了解如何将单元测试、代码消毒器、静态代码分析和代码覆盖工具集成到项目中,并展示CMake如何发现和执行测试。

\textit{第8章,执行自定义任务},了解如何将工具集成到构建过程中,将外部程序包放置到自定义目标中,或将它们挂载到构建过程中进行执行。这里将介绍如何使用自定义任务生成文件,以及如何使用其他目标生成的文件。还将了解如何在CMake构建配置期间执行系统命令,以及如何使用CMake脚本模式创建与平台无关的命令。

\textit{第9章,创建可复制的构建环境},了解在各种机器(包括CI/CD管道)之间如何构建可移植的环境,以及如何使用Docker、sysroot和CMake预置来让构建在任何地方都能“开箱即用”。

\textit{第10章,处理大项目和分布式存储库},使用CMake简化了跨多个git库的项目管理。将了解如何创建一个超级构建,该超级构建允许构建特定版本,以及最新的夜间构建。我们将探索超级构建需要什么先决条件,以及如何使用。

\textit{第11章,自动化模糊测试},了解如何在CMake中集成和使用模糊测试工具。

\textit{第12章,跨平台编译和自定义工具链},演示了如何使用跨平台工具链。还将了解如何编写自己的工具链定义,并在CMake中使用不同的工具链。

\textit{第13章,重用CMake代码},了解如何使用CMake模块,以及如何泛化CMake文件。还将了解如何编写可通用的模块,这些模块可以单独提供。

\textit{第14章,优化和维护CMake项目},建议如何获得更快的构建时间,并提供了一些技巧和技巧,以保持CMake项目长时间内的整洁。 

\textit{第15章,迁移到CMake},了解如何在不停止开发的情况下,将现有的大型代码库迁移到CMake的高级策略。

\textit{第16章,对CMake进行贡献},若想进行贡献,去哪里、需要什么,以及基本的贡献指南。还将指引您在哪里找到更多相信的信息或更具体的文献。

\hspace*{\fill} \\ %插入空行
\textbf{编译环境}

将CMake 3.21或更新版本,以及安装的至少支持C++14的现代C++编译器。一些例子可能需要额外的软件来运行,在相关章节中会提到。示例中使用的所有软件都是开源的,可以免费获得。

\begin{table}[H]
	\centering
	\begin{tabular}{|l|l|}
		\hline
		书中涉及的软件/硬件                                                                                                                  & 操作系统                                                             \\ \hline
		CMake 3.13.4或更高版本                                                                                                                                  &                                                                           Linux, Windows或macOS       \\  \hline
		GCC、Clang或MSVC              &                                                                                  Linux, Windows或macOS\\  \hline
		Git  &  Linux, Windows或macOS                                                                                \\ \hline
	\end{tabular}
\end{table}

\textbf{如果您正在使用这本书的数字版本,建议您自己输入代码或从本书的GitHub存储库访问代码(下一节有链接)。这样做将帮助您避免与复制和粘贴代码相关的错误。}

\hspace*{\fill} \\ %插入空行
\textbf{下载示例}

可以从GitHub上下载这本书的示例代码文件\url{https://github.com/PacktPublishing/CMake-Best-Practices}。若代码有更新,会在GitHub存储库中更新。

还可以从丰富的书籍和视频目录中获得其他代码包\url{https://github.com/PacktPublishing/}。

\hspace*{\fill} \\ %插入空行
\textbf{下载彩图}

我们还提供了一个PDF文件,其中有本书中使用的屏幕截图和图表的彩图。下载地址: \url{https://static.packt-cdn.com/downloads/9781803239729_ColorImages.pdf}






















