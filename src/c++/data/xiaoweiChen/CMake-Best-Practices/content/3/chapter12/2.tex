为多个平台构建软件时,最直接的方法是在目标系统上编译。这样做的缺点是每个开发人员都必须有一个目标系统,才能构建软件。若是桌面系统可能还好,若是功能较弱的设备(如嵌入式系统),可能会因为缺乏适当的开发工具或编译软件需要花费很长时间。

因此,从开发人员的角度来看,更方便的方法是使用交叉编译。从而软件工程师在自己的机器上编写代码并构建软件,但生成的二进制文件适用于不同的平台。构建软件的机器通常称为主机平台,而运行软件的平台则称为目标平台。例如,开发人员在运行Linux的x64桌面计算机上编写代码,但生成的二进制文件是针对arm64处理器上的嵌入式Linux的。所以主机平台是x64 Linux,目标平台是arm64 Linux。要交叉编译软件,需要了解以下两件事:

\begin{itemize}
\item 
工具链可以生成正确格式的二进制文件

\item 
提供编译目标系统目的依赖项
\end{itemize}

工具链是一组工具,如编译器、链接器和打包器,用于生成运行在主机系统上为目标系统生成输出的二进制文件。依赖项通常收集在sysroot目录中,sysroot是包含精简版本的根文件系统的目录,所需的库存储在其中。对于交叉编译,这些目录为搜索依赖项的根目录。

一些工具,例如用于构建嵌入式Linux发行版的Yocto Project (YP),可以构建sysroot和工具链,用于开箱即用的交叉编译。这种自动化通常很方便,但当不可用时,手动创建sysroot可以像从目标平台复制或挂载文件系统到主机上的目录一样简单。有时,工具链和sysroot的组合也称为sdk。这些sdk还可能包含用于调试的进一步工具或用于运行跨平台构建的模拟器的定义,CMake使用工具链文件为交叉编译配置构建工具和sysroot。工具链文件是普通的CMake脚本,主要设置一些缓存变量来描述目标平台和工具链的各个组件的位置。工具链文件可以通过设置CMAKE\_TOOLCHAIN\_FILE变量传递给CMake,或使用自CMake 3.21起支持的-{}-toolchain选项:

\begin{tcblisting}{commandshell={}}
cmake -DCMAKE_TOOLCHAIN_FILE=toolchain.cmake -S <SourceDir> -B
  <BuildDir>
cmake --toolchain toolchain.cmake -S <SourceDir> -B <BuildDir>
\end{tcblisting}

两种方式是等价的。若CMAKE\_TOOLCHAIN\_FILE设置为一个环境变量,CMake也会解释这个变量。若使用CMake预设,配置预设可以配置一个带有toolchainFile选项的工具链文件:

\begin{lstlisting}[style=styleCMake]
{
	"name": "arm64-build-debug",
	"generator" : "ninja",
	"displayName": "Arm 64 Debug",
	"toolchainFile": "/path/to/toolchain.cmake",
	"cacheVariables": {
		"CMAKE_BUILD_TYPE": "Debug"
	}
},
\end{lstlisting}

toolchainFile选项支持宏扩展。若工具链文件的路径是一个相对路径,CMake将首先相对于构建目录查找,若在那里没有,将从源目录搜索。由于CMAKE\_TOOLCHAIN\_FILE是一个缓存变量,只需要在CMake的第一次运行时指定,后续的运行将使用缓存的值。

第一次运行时,CMake将执行一些内部检查,以确定工具链支持哪些特性。不管工具链是用工具链文件指定的,还是使用默认的系统工具链。CMake将在第一次运行时输出各种特性和属性的测试结果:

{\footnotesize
\begin{tcblisting}{commandshell={}}
-- The CXX compiler identification is GNU 9.3.0
-- The C compiler identification is GNU 9.3.0
-- Detecting CXX compiler ABI info
-- Detecting CXX compiler ABI info - done
-- Check for working CXX compiler: /usr/bin/arm-linuxgnueabihf-g++- 9 - skipped
-- Detecting CXX compile features
-- Detecting CXX compile features - done
-- Detecting C compiler ABI info
-- Detecting C compiler ABI info - done
-- Check for working C compiler: /usr/bin/arm-linux-gnueabigcc-9 - skipped
-- Detecting C compile features
-- Detecting C compile features - done
\end{tcblisting}
}

特性的检测通常发生在CMakeLists.txt文件中对\texttt{project()}的第一次使用时,但当之后的\texttt{project()}来启用之前禁用的语言,则将触发进一步的检测。若使用\texttt{enable\_language()}来启用CMakeLists.txt文件中的另一种编程语言,也会发生同样的情况。

由于工具链的特性和测试结果可以缓存,因此无法更改已配置构建目录的工具链。CMake可能会检测到工具链已经更改,但缓存变量的替换不完全。因此,在更改工具链之前,最好完全删除构建目录。

\begin{tcolorbox}[colback=blue!5!white,colframe=blue!75!black,title=配置完成后切换工具链]
切换工具链之前,始终要完全清空构建目录。只删除CMakeCache.txt文件是不够的,因为与工具链相关的内容可能会缓存在不同的位置。
\end{tcolorbox}

CMake的工作模式是一个项目应该对所有东西使用相同的工具链。因此,不直接支持使用多个工具链。若这是真的需要的,项目中需要不同工具链的部分必须配置为子构建。

工具链体积应该保持尽可能小,并且与项目完全解耦。理想情况下,可用于不同的项目。工具链文件通常与用于交叉编译的SDK或sysroot捆绑在一起。然而,有时需要手动书写。












