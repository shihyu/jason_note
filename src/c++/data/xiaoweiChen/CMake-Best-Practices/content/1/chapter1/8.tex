使用CMake构建软件时,常见的问题是如何共享环境配置来构建一个项目。通常,人们习惯于具体指定构建工件应该放在哪里,在哪个平台上使用哪个生成器,或者希望CI环境与本地进行相同的配置后进行构建。CMake 3.19在2020年12月发布,所有这些信息可以存储在CMakePresets.json文件中,放在项目的根目录中。此外,每个用户都可以将他们配置添加到CMakeUserPresets.json文件中。基本预设通常置于版本控制之下,但用户预设不会签入版本系统。这两个文件遵循相同的JSON格式,顶层写法如下所示:

\begin{lstlisting}[style=styleCMake]
{
"version": 3,
"cmakeMinimumRequired": {
	"major": 3,
	"minor": 21,
	"patch": 0
},
"configurePresets": [...],
"buildPresets": [...],
"testPresets": [...]
}
\end{lstlisting}

\begin{enumerate}
\item 
第一行“version”:3表示JSON文件的模式版本。CMake 3.21最多支持版本3,但预计新版本将带来该模式的新版本。

\item 
接下来,cmakeMinimumRequired{…}指定使用哪个版本的CMake。尽管这是可选的,但将其放在这里并与CMakeLists.txt文件中指定的版本匹配是一个很好的做法。

\item 
之后,可以使用configurePresets、buildPresets和testPresets添加不同构建阶段的各种预设。顾名思义,configurePresets应用于CMake的构建过程的配置阶段,而其他两个用于构建和测试阶段。构建和测试预设可以继承一个或多个配置预设。若没有指定继承,将应用于前面的所有步骤。
\end{enumerate}

要查看项目中配置了哪些预设值,请运行\texttt{cmake -{}-list-presets}查看可用预设值的列表。要使用预设进行生成,请执行\texttt{cmake -{}-build -{}-preset name}。

要查看JSON模式的完整规范,请参阅\url{https://cmake.org/cmake/help/v3.21/manual/cmake-presets.7.html}。

预设是分享构建的环境一种方式。编写本文的时候,越来越多的IDE和编辑器都在本地添加了对CMake预设的支持,特别是在使用工具链处理交叉编译时。这里,我们只对CMake预设进行概述,这个主题将在第12章中更详细地讨论。









































