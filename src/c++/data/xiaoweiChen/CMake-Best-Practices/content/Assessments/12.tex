

\begin{enumerate}
\item 
工具链文件可以通过-{}-toolchain命令行标志、CMAKE\_TOOLCHAIN\_FILE变量传递,也可以通过CMAKE预置中的toolchainFile选项传递。

\item
通常,交叉编译需要在工具链文件中执行以下操作:

\begin{enumerate}[label=\Alph*]
\item
定义目标系统和架构
 
\item 
提供构建软件所需工具的路径

\item 
设置编译器和链接器的默认标志

\item 
指向sysroot,若使用交叉编译,可以指向暂存目录

\item 
设置CMake的\texttt{find\_}指令的搜索顺序
\end{enumerate}

\item 
暂存目录用CMAKE\_STAGING\_PREFIX变量设置,若sysroot不需要修改,就把构建的工件放在这里。

\item 
模拟器命令以分号分隔的列表形式在CMAKE\_CROSSCOMPILING\_EMULATOR变量中传递。

\item 
项目中调用\texttt{project()}或\texttt{enable\_language()}都会触发对特性的检测。

\item 
编译器检查的配置上下文可以用\texttt{cmake\_push\_check\_state()}保存,也可以用\texttt{cmake\_pop\_check\_state()}恢复到之前的状态。

\item 
若设置了CMAKE\_CROSSCOMPILING,try\_run()调用将只编译测试,而不会运行它,除非设置了模拟器命令。

\item 
构建目录应该完全清除,因为在删除缓存时,编译器检查的临时构件可能无法正确地重新构建。
\end{enumerate}