
版本一致性对于软件项目很重要,在软件世界中没有什么是一成不变的,软件随着时间的推移而发展和变化。这样的更改通常需要提前确认,要么通过对新版本运行一系列测试,要么通过对调用代码本身进行更改。理想情况下,上游代码中的更改不应该对复制现有构建产生影响,直到我们希望它们这样做。如果针对这个组合进行了软件验证和测试,则应该始终使用z.q依赖版本构建项目的x.y版本。这样做的原因是,即使没有API或ABI更改,上游依赖项中的最小更改也可能影响软件的行为。若不保证版本一致性,软件将没有定义良好的行为。因此,找到提供版本一致性的方法很重要。

确保超级构建中的版本一致性取决于超级构建的组织方式。对于从版本控制系统获取的库,这相对容易。克隆项目,但不要原样使用它,而是签出到特定的分支或标签。若没有这样的锚点,则检出到特定的提交,这将为超级构建提供未来的保障。但这可能还不够,标签可能会修改,分支可能会强行推入,历史可能会改写。为了减少这种风险,可能更倾向于Fork项目,并将作为上游库。这样,就可以完全控制上游库的内容,但这种方法带来增加了维护的负担。

这个故事告诉我们,不要盲目地逆流而上,随时关注最近的变化。对于作为存档文件使用的第三方依赖项,始终检查它们的哈希摘要。通过这种方式,将确保在为项目使用期望的版本;若这有变化,则需要手动确认。












































