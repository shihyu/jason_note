打包文档与打包软件及其工件没有什么不同——文档是项目的工件。因此,我们将使用在第4章中学到的知识来打包文档。若还没有阅读第4章,强烈建议在阅读本节之前阅读。

这一节,我们使用第6章的例01,将使在此示例中生成的文档可安装和可打包。让我们回顾 CMakeLists.txt文件,它位于chapter06/ex01\_doxdocgen/文件夹中。使用以下代码,将使HTML和MAN文档可安装:

\begin{lstlisting}[style=styleCMake]
include(GNUInstallDirs)
install(DIRECTORY "${CMAKE_CURRENT_BINARY_DIR}/docs/html/"
	DESTINATION "${CMAKE_INSTALL_DOCDIR}"
	COMPONENT ch6_ex01_html)
install(DIRECTORY "${CMAKE_CURRENT_BINARY_DIR}/docs/man/"
	DESTINATION "${CMAKE_INSTALL_MANDIR}"
	COMPONENT ch6_ex01_man)
\end{lstlisting}

第4章中使用\texttt{install(DIRECTORY…)}来安装文件夹,同时保留其结构。我们通过将docs/html和docs/man安装到GNUInstallDirs模块提供的默认文档和手册页目录中,使生成的文档可安装。另外,若是可安装的,就意味着也是可打包的,因为CMake能够从\texttt{install(…)}生成所需的打包代码。因此,包含CPack模块来为这个示例启用打包。代码如下所示:

\begin{lstlisting}[style=styleCMake]
set(CPACK_PACKAGE_NAME cbp_chapter6_example01)
set(CPACK_PACKAGE_VENDOR "CBP Authors")
set(CPACK_GENERATOR "DEB;RPM;TBZ2")
set(CPACK_DEBIAN_PACKAGE_MAINTAINER "CBP Authors")
include(CPack)
\end{lstlisting}

就这么简单。尝试使用以下命令来构建和打包示例项目:

\begin{tcblisting}{commandshell={}}
cd chapter06/
cmake -S . -B build/
cmake --build build/
cpack --config build/CPackConfig.cmake -B build/pak
\end{tcblisting}

这里,配置并构建了chapter06/的代码,并使用CPack来使用生成的CPackConfig.cmake将项目打包到build/pak文件夹中。为了检查是否一切正常,通过使用以下命令将生成的包的内容提取到/tmp/ch6-ex01目录中:

\begin{tcblisting}{commandshell={}}
dpkg -x build/pak/cbp_chapter6_example01-1.0-Linux.deb
  /tmp/ch6-ex01
export MANPATH=/tmp/ch6-ex01/usr/share/man/
\end{tcblisting}

解压完成后,文档在/tmp/ch6-ex01/usr/share路径下可用。因为我们使用的是非默认路径,所以可以使用MANPATH环境变量让man命令知道文档的路径。首先检查是否可以通过调用man命令访问手册页:

\begin{tcblisting}{commandshell={}}
man chapter6_ex01_calculator
\end{tcblisting}

chapter6\_ex01\_calculator名称由Doxygen从chapter6::ex01::calculator类名称自动推导出来的,应该能够看到我们在前一节中介绍的手册页输出。

我们已经了解了关于生成和打包文档的知识。接下来,将学习如何生成CMake目标的依赖关系图。



