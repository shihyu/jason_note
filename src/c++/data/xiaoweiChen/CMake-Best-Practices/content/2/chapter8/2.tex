CMake具有相当多的功能,因此在构建软件时可以执行许多任务。某些情况下,开发人员需要做一些CMake没有涉及到的事情。常见的例子包括:运行为目标文件做一些预处理或后处理的特殊工具,使用为编译器生成输入的源代码生成器,以及压缩和打包CPack无法处理的构件。构建步骤中必须完成的特殊任务的列表可能非常多。CMake支持三种执行自定义任务的方式:

\begin{itemize}
\item 
通过\texttt{add\_custom\_target}定义执行命令的目标

\item 
通过\texttt{add\_custom\_command}将自定义命令附加到现有的目标,或者通过使目标依赖于由自定义命令生成的文件

\item 
通过\texttt{execute\_process},在配置步骤中执行命令
\end{itemize}

只要有可能,应该在构建步骤中调用外部程序,因为配置步骤对调用难以控制,应该尽可能快地运行。

让我们学习如何定义在构建时执行的任务。