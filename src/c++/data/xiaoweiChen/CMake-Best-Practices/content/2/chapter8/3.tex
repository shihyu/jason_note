


添加自定义任务的通用方法是创建自定义目标,该目标以命令序列的形式执行外部任务。自定义目标的处理方式与其他库或可执行目标相同,不同的是不调用编译器和链接器,它们执行由用户定义的操作。可以使用\texttt{add\_custom\_target}定义自定义目标:

\begin{lstlisting}[style=styleCMake]
add_custom_target(Name [ALL] [command1 [args1...]]
				[COMMAND command2 [args2...] ...]
				[DEPENDS depend depend depend ... ]
				[BYPRODUCTS [files...]]
				[WORKING_DIRECTORY dir]
				[COMMENT comment]
				[JOB_POOL job_pool]
				[VERBATIM] [USES_TERMINAL]
				[COMMAND_EXPAND_LISTS]
				[SOURCES src1 [src2...]])
\end{lstlisting}

\texttt{add\_custom\_target}的核心是通过COMMAND选项传递的命令列表。虽然第一个命令可以不带这个选项,但最好在\texttt{add\_custom\_target}中添加COMMAND选项。默认情况下,定制目标只在显式请求时执行,除非指定了ALL选项。自定义目标总认为是过时的,因此总是运行指定的命令,而不管是否会反复产生相同的结果。使用DEPENDS关键字,可以使定制目标依赖于使用\texttt{add\_custom\_command}或其他目标定义的定制命令的文件和输出。要使自定义目标依赖于另一个目标,可以使用\texttt{add\_dependencies}。若使用自定义目标创建文件,可以在BYPRODUCTS选项下列出这些文件。列出的文件都将使用GENERATED属性标记,CMake使用该属性来确定构建是否过期,并找出需要清理的文件,但使用\texttt{add\_custom\_command}创建文件的任务可能更适合。

通常,命令在当前二进制目录中执行,该目录在CMAKE\_CURRENT\_BINARY\_DIRECTORY缓存变量中。若需要修改,这可以通过WORKING\_DIRECTORY选项来更改。该选项可以是绝对路径,也可以是相对路径(当前二进制目录的相对路径)。

COMMENT选项用于指定在命令运行之前打印的消息,若命令以静默方式运行,那么可以使用这个选项。但并不是所有的生成器都显示这些消息,因此使用它来显示关键信息不太可靠。

VERBATIM标志使所有命令直接传递到平台,而无需进一步转义或由底层Shell替换变量。CMake本身仍然会替换传递给命令或参数的变量。当转义可能出现问题时,建议使用VERBATIM标志。编写自定义任务,使其独立于底层平台也是一种很好的方式。

可能的话,USES\_TERMINAL选项指示CMake提供对终端的命令访问。若使用了Ninja生成器,则在终端作业池中运行。该池中的所有命令都是串行执行的。

使用Ninja生成文件时,可以使用JOB\_POOL选项来控制作业的并发性。它很少使用,也不能与USES\_TERMINAL标志一起使用。但开发者很少需要干预Ninja的工作池,并且处理它也不是一件小事。若想了解更多信息,可以在CMake的JOB\_POOLS属性的官方文档中找到更多信息。

SOURCES属性接受与自定义目标相关联的源文件列表。该属性不影响源文件,但可以使文件在某些IDE中可见。若命令依赖于与项目一起交付的文件,如脚本,则应该在这里添加这些文件。

COMMAND\_EXPAND\_LISTS选项将列表传递给命令前展开列表。有时这是必要的,因为在CMake中,列表只是由分号分隔的字符串,这可能会导致语法错误。当使用COMMAND\_EXPAND\_LISTS选项时,分号将替换为适当的空白字符,具体取决于平台。扩展包括使用\$<JOIN:的生成器表达式生成的列表。

下面是一个自定义目标的例子,使用一个叫做CreateHash的外部程序为另一个目标的输出哈希值:

\begin{lstlisting}[style=styleCMake]
add_executable(SomeExe)
add_custom_target(CreateHash ALL COMMAND Somehasher
	$<TARGET_FILE:SomeExe>)
\end{lstlisting}

这个例子创建了一个名为CreateHash的自定义目标,使用SomeExe目标的二进制文件作为参数调用外部SomeHasher程序。注意,二进制文件是使用\$<TARGET\_FILE:SomeExe>生成器表达式。这有两个目的:消除了用户跟踪目标二进制文件文件名的需要,并在两个目标之间添加了隐式依赖。CMake将识别这些隐式依赖关系,并按正确的顺序执行目标。若生成所需文件的目标还没有构建,CMake将自动构建。也可以使用\$<TARGET\_FILE:生成器直接执行由另一个目标创建的可执行文件。以下生成器表达式会导致目标之间的隐式依赖:

\begin{itemize}
\item 
\$<TARGET\_FILE:target>: 包含目标的主二进制文件的完整路径,例如.exe、.so或.dll。

\item 
\$<TARGET\_LINKER\_FILE: target>: 包含用于链接到目标的文件的完整路径。这通常是库文件本身,在Windows上,它将是与DLL关联的.lib文件。

\item 
\$<TARGET\_SONAME\_FILE: target>: 这包含库文件及其全名,包括SOVERSION属性设置的任何数字,例如.so.3。

\item
\$<TARGET\_PDB\_FILE: target>: 这包含用于调试的生成的程序数据库文件的完整路径。创建自定义目标是在构建时执行外部任务的一种方法。另一种方法是定义自定义命令,可用于向现有目标(包括自定义目标)添加自定义任务。
\end{itemize}

\subsubsubsection{8.3.1\hspace{0.2cm}向现有目标添加自定义任务}

有时,在构建目标时可能需要执行外部任务。CMake中可以使用\texttt{add\_custom\_command}来实现这一点,它有两个签名。一个用于将命令与现有目标挂钩,而另一个用于生成文件。向现有目标添加命令的签名如下所示:

\begin{lstlisting}[style=styleCMake]
add_custom_command(TARGET <target>
					PRE_BUILD | PRE_LINK | POST_BUILD
					COMMAND command1 [ARGS] [args1...]
					[COMMAND command2 [ARGS] [args2...] ...]
					[BYPRODUCTS [files...]]
					[WORKING_DIRECTORY dir]
					[COMMENT comment]
					[VERBATIM] [USES_TERMINAL]
					[COMMAND_EXPAND_LISTS])
\end{lstlisting}

大多数选项的工作原理类似于\texttt{add\_custom\_target}中的选项。TARGET属性可以是当前目录中定义的目标,这是该命令的一个限制。可以在以下时段将命令连接到构建中:

\begin{itemize}
\item 
PRE\_BUILD: 在Visual Studio中,此命令在执行其他构建步骤之前执行。当使用其他生成器时,会在PRE\_LINK命令之前运行。

\item 
PRE\_LINK: 此命令将在编译源代码之后运行,在可执行文件或存档工具链接到静态库之前运行。

\item 
POS\_BUILD: 这将在执行所有其他构建规则后运行该命令。
\end{itemize}

执行自定义步骤最常见的方法是使用POST\_BUILD;其他两个选项很少使用,要么是因为支持有限,要么是因为它们既不能影响链接,也不能影响构建。

向现有目标添加自定义命令相对简单。下面的代码添加了一个命令,在每次编译后生成并存储构建文件的哈希值:

\begin{lstlisting}[style=styleCMake]
add_executable(MyExecutable)

add_custom_command(TARGET MyExecutable
	POST_BUILD
	COMMAND hasher $<TARGET_FILE:ch8_custom_command_example>
		${CMAKE_CURRENT_BINARY_DIR}/MyExecutable.sha256
	COMMENT "Creating hash for MyExecutable"
)
\end{lstlisting}

这个例子中,名为hasher的自定义可执行文件,用来生成MyExecutable目标输出文件的哈希值。

需要在构建之前,执行某些操作的原因是更改文件或生成信息。为此,第二个签名通常是更好的选择。

\subsubsubsection{8.3.2\hspace{0.2cm}使用自定义任务生成文件}

通常,希望自定义任务产生特定的输出文件。这可以通过定义自定义目标,并在目标之间设置必要的依赖项来实现,或者通过与构建步骤挂钩,但PRE\_BUILD不可靠,因为只有Visual Studio生成器正确地支持它。因此,更好的方法是创建一个自定义命令,通过使用add\_custom\_command函数的第二个签名来创建文件:

\begin{lstlisting}[style=styleCMake]
add_custom_command(OUTPUT output1 [output2 ...]
					COMMAND command1 [ARGS] [args1...]
					[COMMAND command2 [ARGS] [args2...] ...]
					[MAIN_DEPENDENCY depend]
					[DEPENDS [depends...]]
					[BYPRODUCTS [files...]]
					[IMPLICIT_DEPENDS <lang1> depend1
									 [<lang2> depend2] ...]
					[WORKING_DIRECTORY dir]
					[COMMENT comment]
					[DEPFILE depfile]
					[JOB_POOL job_pool]
					[VERBATIM] [APPEND] [USES_TERMINAL]
					[COMMAND_EXPAND_LISTS])
\end{lstlisting}

\texttt{add\_custom\_command}的这个签名定义了一个生成OUTPUT中指定的文件的命令。该命令的大多数选项都非常类似于\texttt{add\_custom\_target}和将自定义任务挂钩到构建步骤的签名。DEPENDS选项可用于手动指定文件或目标的依赖项。相比之下,自定义目标的DEPENDS选项只能指向文件。若构建或CMake更新了依赖项,则会再次运行自定义命令。MAIN\_DEPENDENCY选项与此密切相关,指定命令的主要输入文件,工作原理与DEPENDS选项类似,只不过它只接受一个文件。MAIN\_DEPENDENCY主要用来告诉Visual Studio在哪里添加自定义命令。

\begin{tcolorbox}[colback=webgreen!5!white,colframe=webgreen!75!black,title=Note]
若源文件出现在MAIN\_DEPENDENCY中,那么自定义命令将取代所列文件的正常编译,这可能导致链接器报错。
\end{tcolorbox}

其他两个与依赖相关的选项IMPLICIT\_DEPENDS和DEPFILE很少使用,因为它们的支持仅限于Makefile生成器。IMPLICT\_DEPENDS告诉CMake使用C或C++扫描器来检测所列文件的任何编译时依赖项,并从中创建依赖项。另一个选项DEPFILE可以用来指向.d依赖文件,该文件是由Makefile项目生成的.d文件,最初起源于GNU make项目,用起来很强大也很复杂,不应该对大多数项目进行手动管理。下面的例子说明了如何使用自定义命令,在常规目标文件运行前生成一个源文件:

\begin{lstlisting}[style=styleCMake]
add_custom_command(OUTPUT ${CMAKE_CURRENT_BINARY_DIR}/main.cpp
	COMMAND sourceFileGenerator ${CMAKE_CURRENT_SOURCE_DIR}/message.txt
	${CMAKE_CURRENT_BINARY_DIR}/main.cpp
	COMMENT "Creating main.cpp from message.txt"
	DEPENDS message.txt
	VERBATIM
)
add_executable(
	ch8_create_source_file_example
	${CMAKE_CURRENT_BINARY_DIR}/main.cpp
)
\end{lstlisting}

首先,自定义命令将当前二进制指令中的main.cpp文件定义为OUTPUT文件。然后,定义生成该文件的命令——这里使用名为sourceFileGenerator的程序——将消息文件转换为.cpp文件。DEPENDS部分说明,每次message.txt文件更改时都应该重新运行此命令。

之后,为可执行文件创建目标。由于可执行文件引用自定义命令的OUTPUT部分中指定的main.cpp文件,CMake将隐式地在命令和目标之间添加必要的依赖关系。因为适用于所有生成器,所以以这种方式使用自定义命令比使用PRE\_BUILD指令更加可靠和可移植。有时为了创建所需的输出,需要多个命令。若存在产生相同输出的前一个命令,则可以使用APPEND选项将命令链接起来。使用APPEND的自定义命令只需要COMMAND和DEPENDS选项,忽略其他选项。若两个命令产生相同的输出文件,CMake将输出错误,除非指定了APPEND。若命令是可选执行,这很有用。看看下面的例子:

\begin{lstlisting}[style=styleCMake]
add_custom_command(OUTPUT archive.tar.gz
	COMMAND cmake -E tar czf ${CMAKE_CURRENT_BINARY_DIR}/archive.tar.gz
	$<TARGET_FILE:MyTarget>
	COMMENT "Creating Archive for MyTarget"
	VERBATIM
)

add_custom_command(OUTPUT archive.tar.gz
	COMMAND cmake -E tar czf ${CMAKE_CURRENT_BINARY_DIR}/archive.tar.gz
	${CMAKE_CURRENT_SOURCE_DIR}/SomeFile.txt
	APPEND
)
\end{lstlisting}

目标的输出文件MyTarget添加到tar.gz文件中,另一个文件添加到相同的文件中。注意,第一个命令自动依赖于MyTarget,因为其使用命令创建相应的二进制文件,但不会由构建自动执行。第二个自定义命令列出与第一个命令相同的输出文件,但将压缩文件添加为第二个输出。通过指定APPEND,在执行第一个命令时自动执行第二个命令。若缺少APPEND关键字,CMake将输出类似于下面的错误:

\begin{tcblisting}{commandshell={}}
CMake Error at CMakeLists.txt:30 (add_custom_command):
  Attempt to add a custom rule to output
    /create_hash_example/build/hash_example.md5.rule
  which already has a custom rule.
\end{tcblisting}

本例中的定制命令隐式地依赖于MyTarget,但不会自动执行。要执行它们,建议创建一个依赖于输出文件的自定义目标:

\begin{lstlisting}[style=styleCMake]
add_custom_target(create_archive ALL DEPENDS
	${CMAKE_CURRENT_BINARY_DIR}/archive.tar.gz
)
\end{lstlisting}

这里,创建了一个名为create\_archive的自定义目标它作为All构建的一部分执行。因为依赖于自定义命令的输出,所以构建目标将调用自定义命令,依赖于MyTarget。若压缩包不是新的,则构建create\_archive也会触发MyTarget的构建。

\texttt{add\_custom\_command}和\texttt{add\_custom\_target}自定义任务都在CMake的构建步骤中执行,可以在配置时添加任务。我们将在下一节中讨论这个问题。

