进一步讨论之前,先来了解一下模糊测试。模糊测试(Fuzzing)是一种向软件系统提供随机、意外数据的测试方法,以观察系统在特定输入下的行为,模糊测试会报告遇到的意外行为。这使我们能够发现其他测试策略和代码评审遗漏的Bug。发现输入是否会导致安全问题或故障会比较困难,但模糊测试非常有效。当正确使用模糊测试时,可以轻松发现绝大多数关键的安全Bug,如远程代码执行或特权升级。因此,了解模糊测试很重要。

模糊测试既可以手工完成,也可以在软件的帮助下自动完成。第二种方法更有效,可以利用机器算力进行模糊测试。基于语料库的、覆盖引导的模糊工具,是许多软件项目必备的工具。幸运的是,我们有相当不错的工具可以模糊测试C和C++项目。模糊处理可以分为两大类:引导模糊测试和黑箱模糊测试。黑盒模糊测试是一种暴力的测试方法,依赖待测系统(SUT)对测试输入的反应,而向引导模糊测试由覆盖率分析或用户自身自动引导。引导模糊测试可以认为是白盒测试或灰盒测试,因为引导依赖于来自实现的反馈。引导模糊测试是发现软件未知边界的好方法。

为了使模糊测试过程有效,生成的输入不能是随机的。输入必须有效,能够通过模糊测试目标的初始基本检查,这样模糊测试器才能深入研究系统。因此,用户可能需要通过提供一组为SUT提供最大覆盖率的输入来启动模糊测试,这一组输入称为语料库。模糊测试器改变初始语料库数据,生成与原始输入相似但触发不同行为的输入变化。触发意外行为或覆盖之前未发现路径的输入数据,可以通过模糊测试工具保存到语料库中,以扩展语料库数据以供以后使用。模糊测试可能不需要初始语料库数据,这样模糊测试就需要从头开始生成输入数据。本章将采用基于语料库的引导性方式。

深入讨论这个主题之前,让我们先了解一个问题——模糊化不能替代其他测试策略或类型,比如单元测试或系统测试。模糊测试的目的是发现Bug和与预期不符的行为,所以应该使用模糊测试增强现有的测试策略。注意,模糊测试要求系统或单元是可测试的。因此,若还没有合适的测试,建议首先了解传统的测试策略。

好了,我们已经学了很多关于模糊的知识。接下来,将学习如何在C和C++项目中使用两个著名的模糊测试软件AFL++和libFuzzer。




























