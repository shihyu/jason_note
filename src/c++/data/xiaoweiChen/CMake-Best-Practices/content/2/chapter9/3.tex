容器化带来的好处是开发人员可以在一定程度上控制构建环境。容器化构建环境对于设置CI环境也有很大帮助。容器的运行时软件有很多,其中Docker最受欢迎。容器化的深入讨论超出了本书的范围,所以我们使用Docker运行本书的示例。

构建容器包含完整的构建系统,包括CMake和构建特定软件所需的工具和库。通过提供容器定义(Dockerfile)和项目一起提供,或者通过公开可访问的容器注册表提供,任何人都可以使用容器来构建软件。好处是,除了运行容器所需的软件外,开发人员不需要安装其他库或工具。缺点是构建可能需要更长的时间,并且不是所有IDE和工具都支持容器。Visual Studio Code对在容器支持的很好,可参考\url{https://code.visualstudio.com/docs/remote/containers}了解更多详情。

使用构建容器的工作流程如下:

\begin{enumerate}
\item 
定义容器并构建。

\item 
将源代码的副本装入构建容器中。

\item 
运行用于在容器内构建的命令。
\end{enumerate}

构建简单C++应用程序的Dockerfile如下所示:

\begin{lstlisting}[style=styleCMake]
FROM alpine:3.15
RUN apk add --no-cache cmake ninja g++ bash make git
RUN <any command to install additional libraries etc.>
\end{lstlisting}

这将定义一个基于Alpine Linux 3.15的容器,并安装cmake、ninja、bash、g++、make和git。实际容器可能有工具和库安装在里面以方便工作;然而,为了说明如何使用容器构建软件,拥有这样一个小容器就足够了。下面的Docker命令构建容器映像,并用builder\_minimal标记:

\begin{tcblisting}{commandshell={}}
docker build . -t builder_minimal
\end{tcblisting}

当容器是本地的副本,那么源就挂载在容器中,所有的CMake命令都可以在容器中执行。假设用户从源目录执行Docker命令,配置CMake构建项目的命令可能如下所示:

\begin{tcblisting}{commandshell={}}
docker run --user 1000:1000 --rm -v $(pwd):/workspace
  builder_minimal cmake -S /workspace -B /workspace/build
docker run --user 1000:1000 --rm -v $(pwd):/workspace
  builder_minimal cmake --build /workspace/build
\end{tcblisting}

这将启动创建容器并在其中执行CMake命令。使用-v将本地目录作为/workspace挂载到容器中。因为Docker容器通常使用root作为默认用户,所以使用的用户ID和组是通过-{}-user选项传递的。在类Unix操作系统上,这应该与主机的用户ID相匹配,因此创建的文件也可以在容器外部编辑。-{}-rm标志告诉Docker在容器退出后删除该容器。

使用容器的另一种方法是通过向docker run命令传递-ti标志以交互模式运行:

\begin{tcblisting}{commandshell={}}
docker run --user 1000:1000 --rm -ti -v $(pwd):/workspace
  builder_minimal
\end{tcblisting}

这将在容器内启动shell,可以直接使用构建命令,而不需要每次都重新启动容器。

关于编辑器或IDE和构建容器如何协同工作,有几种策略。若IDE本身支持,或者像Visual Studio Code那样通过扩展支持最好了。否则,在容器中装入合适的编辑器,并在容器中执行也是方便开发软件的一种策略。另一种方法是在主机系统上进行编辑并重新配置Docker,不直接调用CMake,而是启动容器时直接执行CMake。

这里展示的是使用容器作为构建环境的最低要求,随着越来越多的IDE支持使用容器构建环境,使用容器将变得容易得多。容器使构建环境在不同机器之间可移植,并且可以帮助确保项目的所有开发人员使用相同的构建环境。将容器定义文件置于版本控制之下也是一个好主意,以便对构建环境的必要更改与代码一起跟踪。

容器是创建隔离构建环境的一种可移植的方法,但若由于某种原因这没办法实现,那么另一种创建可移植构建环境的方法是使用sysroot。



















