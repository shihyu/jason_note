构建软件可能很复杂,特别是涉及到依赖项或特殊工具的情况。在一台机器上编译的程序可能不能在另一台机器上运行,因为缺少一些关键的软件。依靠软件项目文档的正确性来找出所有构建需求,通常是不够的,因此开发者需要花大量的时间梳理各种错误消息,以找出构建失败的原因。

构建或持续集成(CI)环境中,人们避免升级任何东西,因为他们担心每一次更改都可能破坏构建软件的能力。因为担心无法再发布产品,这甚至导致公司拒绝升级正在使用的编译器工具链。创建关于构建环境的健壮和可移植的信息绝对是游戏规则的改变者。通过预置,CMake提供了定义配置项目的通用方法的可能性。当与工具链文件、Docker容器和sysroot结合使用时,创建可以在不同机器上重新创建的构建环境将变得容易得多。

本章中,将学习如何定义CMake预设来配置、构建和测试CMake项目,以及如何定义和使用工具链文件。我们将简要介绍使用容器构建软件,并学习如何使用sysroot工具链文件创建独立的构建环境。本章的主要主题如下:

\begin{itemize}
\item 
使用CMake预设

\item 
使用容器进行构建

\item 
使用sysroot隔离构建环境
\end{itemize}






































