为了理解完美转发的所有规则,来看看这些规则是如何在C++标准中指定的。\par

同样,有以下声明:\par

\begin{lstlisting}[caption={}]
template<typename T>
void f(T&& arg) // arg is universal/forwarding reference
{
	g(std::forward<T>(arg)); // perfectly forward (move() only for passed rvalues)
}
\end{lstlisting}

通常,T只有传递参数的类型:\par

\begin{lstlisting}[caption={}]
MyType v;

f(MyType{}); // T is deduced as MyType, so arg is declared as MyType&&
f(std::move(v)); // T is deduced as MyType, so arg is declared as MyType&&
\end{lstlisting}

然而,对于通用引用传递lvalue,有一个特殊的规则(参见C++标准的[temp.debit.call]节):\par

如果形参类型是对cv-非限定模板形参的右值引用,实参是lvalue,类型“对T的lvalue引用”将用来代替T进行类型推断。\par

\par

这意味着在这种情况下:\par

\begin{itemize}
	\item 如果形参的类型是用\&\&声明的,而不是用\textit{const}或\textit{volatile}声明的
	\item 并且传递的是一个lavlue
	\item 那么将T推导为T\&。
\end{itemize}

例子:\par

\begin{lstlisting}[caption={}]
template<typename T>
void f(T&& arg); // arg is a universal/forwarding reference

MyType v;
const MyType c;

f(v); // T is deduced as MyType&
f(c); // T is deduced as const MyType&
\end{lstlisting}

但已经将\textit{arg}声明为T\&\&。如果T是T\&,那这里C++的引用折叠规则(参见[dcl.]一节。参考]的C++标准)给出了一个答案:\par

\begin{itemize}
	\item Type\& \& 成为 Type\&
	\item Type\& \&\& 成为 Type\&
	\item Type\&\& \& 成为 Type\&
	\item Type\&\& \&\& 成为 Type\&\&
\end{itemize}

这意味着:\par

\begin{lstlisting}[caption={}]
MyType v;
const MyType c;

f(v); // T is deduced as MyType& and arg has this type
f(c); // T is deduced as const MyType& and arg has this type
\end{lstlisting}

现在考虑一下\textit{std::forward<>()}是如何定义的,与\textit{std::move()}相反:\par

\begin{itemize}
	\item \textit{std::move()}总是将类型转换为rvalue引用:
	\begin{lstlisting}[caption={}]
	static_cast<remove_reference_t<T>&&>(t)
	\end{lstlisting}
	它删除引用并转换为相应的rvalue引用类型(删除任何\&并添加\&\&)。
	\item \textit{std::forward<>()}只向传递的类型参数添加rvalue引用:
	\begin{lstlisting}[caption={}]
	static_cast<T&&>(t)
	\end{lstlisting}
	引用折叠规则再次适用:
	\begin{itemize}
		\item[-] 如果类型T是lvalue引用,T\&\&仍然是lvalue引用(\&\&无效)。因此,将\textit{arg}强制转换为lvalue引用,这意味着\textit{arg}没有移动语义。
		\item[-] 但是,如果T是rvalue引用(或者根本不是引用),T\&\&(仍然)是rvalue引用。因此,将\textit{arg}强制转换为rvalue引用,这样就将值类别更改为xvalue,这是\textit{std::move()}的效果。
	\end{itemize}
\end{itemize}

因此:\par

\begin{lstlisting}[caption={}]
template<typename T>
void f(T&& arg) // arg is a universal/forwarding reference
{
	g(std::forward<T>(arg)); // perfectly forward (move() only for passed rvalues)
}

MyType v;
const MyType c;

f(v); // T and arg are MyType&, forward() has no effect in this case
f(c); // T and arg are const MyType&, forward() has no effect in this case
f(MyType{}); // T is MyType, arg is MyType&&, forward() is equivalent to move()
f(std::move(v)); // T is MyType, arg is MyType&&, forward() is equivalent to move()
\end{lstlisting}

注意,字符串字面值是rvalue,因此我们可以推导出的T和\textit{arg}:\par

\begin{lstlisting}[caption={}]
f("hi"); // lvalue passed, so T and arg have type const char(&)[3]
f(std::move("hi")); // xvalue passed, so T is deduced as const char[3]
// and arg has type const char(&&)[3]
\end{lstlisting}

还请记住,对函数的引用总是lvalue,因此,如果对函数的引用传递给通用引用,那么T总是推断为lvalue引用:\par

\begin{lstlisting}[caption={}]
void func(int) {
}
f(func); // lvalue passed to f(), so T and arg have type void(&)(int)
f(std::move(func)); // lvalue passed to f(), so T and arg have type void(&)(int)
\end{lstlisting}

\hspace*{\fill} \par %插入空行
\textbf{10.3.1 通用引用类型的说明}

声明通用/转发引用时,还可以显式指定模板形参的类型,而不是推导。但是,请记住参数声明为T\&\&。因此,有以下行为:\par

\begin{lstlisting}[caption={}]
template<typename T>
void f(T&& arg) // arg is universal/forwarding reference
{
	g(std::forward<T>(arg)); // perfectly forward (move() only for passed rvalues)
}

f<std::string>( ... ); // arg is a raw rvalue reference binding to rvalues only
f<std::string&>( ... ); // arg is an lvalue reference binding to non-const lvalues only
f<const std::string&>( ... ); // arg is a const lvalue reference binding to everything
f<std::string&&>( ... ); // arg is a raw rvalue reference binding to rvalues only
\end{lstlisting}

因此,有了明确的规范,通用引用不再作为通用引用。作为调用者,可以指定获得的引用的具体类型。\par

因此,要传递lvalue(这里仍然需要值),请确保将模板参数指定为lvalue引用。否则,代码将无法编译:\par

\begin{lstlisting}[caption={}]
template<typename T>
void f(T&& arg) // arg is universal/forwarding reference
{
	g(std::forward<T>(arg)); // perfectly forward (move() only for passed rvalues)
}

std::string s;
...
f<std::string>(s); // ERROR: cannot bind rvalue reference to lvalue
f<std::string&>(s); // OK, does not move and forward s
f<std::string>(std::move(s)); // OK, does move and forward s
f<std::string&&>(std::move(s)); // OK, does move and forward s
\end{lstlisting}

最后两个调用是等价的。\par

这些规则同样适用于使用C++20特性,在声明普通函数时使用auto\&\&:\par

\begin{lstlisting}[caption={}]
void f(auto&& arg) {
	g(std::forward<decltype(arg)>(arg));
}
\end{lstlisting}

\hspace*{\fill} \par %插入空行
\textbf{10.3.2 与通用引用冲突的模板参数推断}

推导通用引用模板参数的特殊规则(当传递lvalue时,将类型作为lvalue引用进行推导)可能会导致看似正确的代码出现意外错误。\par

以下代码无法编译时,开发者通常会感到惊讶:\par

\begin{lstlisting}[caption={}]
template<typename T>
void insert(std::vector<T>& vec, T&& elem)
{
	vec.push_back(std::forward<T>(elem));
}

std::vector<std::string> coll;
std::string s;
...
insert(coll, s); // ERROR: no matching function call
\end{lstlisting}

问题是两个参数都可以推导出参数T,但推导出的类型不一样:\par

\begin{itemize}
	\item 使用参数\textit{coll}, T将其推导为std::string。
	\item 但是,根据通用引用的特殊规则,参数\textit{elem}会强制将T推导为std::string\&。
\end{itemize}

因此,编译器会产生歧义错误。\par

有两种方法可以解决这个问题:\par

\begin{itemize}
	\item 可以使用std::remove\_reference<>:
	\begin{lstlisting}[caption={}]
	template<typename T>
	void insert(std::vector<std::remove_reference_t<T>>& vec, T&& elem)
	{
		vec.push_back(std::forward<T>(elem));
	}
	std::vector<std::string> coll;
	std::string s;
	...
	insert(coll, s); // OK, with T deduced as std::string& vec now binds to coll
	\end{lstlisting}
	\item 可以使用两个模板参数:
	\begin{lstlisting}[caption={}]
	template<typename T1, typename T2>
	void insert(std::vector<T1>& vec, T2&& elem)
	{
		vec.push_back(std::forward<T2>(elem));
	}
	\end{lstlisting}
	或者只是:\par
	\begin{lstlisting}[caption={}]
	template<typename Coll, typename T>
	void insert(Coll& coll, T&& elem)
	{
		coll.push_back(std::forward<T>(elem));
	}
	\end{lstlisting}
\end{itemize}

\hspace*{\fill} \par %插入空行
\textbf{10.3.3 泛型类型的纯rvalue引用}

通过导出通用引用的模板形参的特殊规则(当传递左值时,将类型作为lvalue引用),可以约束泛型引用形参仅绑定到rvalue:\par

\begin{lstlisting}[caption={}]
template<typename T>
requires (!std::is_lvalue_reference_v<T>) // bind to rvalues only
void callFoo(T&& arg) {
	foo(std::forward<T>(arg));
}
\end{lstlisting}

C++20之前,必须对类型特征再次使用std::enable\_if<>:\par

\begin{lstlisting}[caption={}]
template<typename T,
	typename
		= typename std::enable_if<!std::is_lvalue_reference<T>::value
			>::type>
void callFoo(T&& arg) {
	foo(std::forward<T>(arg));
}
\end{lstlisting}


