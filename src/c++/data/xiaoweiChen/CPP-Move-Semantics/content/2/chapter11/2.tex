但在泛型代码中,如何在传递返回值的同时,仍然保持其类型和值类别?\par

答案是通用/转发引用,但没有声明为参数。为此,需要auto\&\&。\par

调用时:\par

\begin{lstlisting}[caption={}]
// pass return value of compute() to process():
process(compute(t)); // OK, uses perfect forwarding of returned value
\end{lstlisting}

还可以实现以下功能:\par

\begin{lstlisting}[caption={}]
// pass return value of compute() to process() with some delay:
auto&& ret = compute(t); // initialize a universal reference with the return value
...
process(std::forward<decltype(ret)>(ret)); // OK, uses perfect forwarding of returned value
\end{lstlisting}

或者,使用大括号初始化时:\par

\begin{lstlisting}[caption={}]
// pass return value of compute() to process() with some delay:
auto&& ret{compute(t)}; // initialize a universal reference with the return value
...
process(std::forward<decltype(ret)>(ret)); // OK, uses perfect forwarding of returned value
\end{lstlisting}

请参阅generic/perfectautorefref.cpp以获得所有情况下的完整示例。\par

\hspace*{\fill} \par %插入空行
\textbf{11.2.1 auto\&\&的类型定义}

使用auto\&\&声明时,也声明了一个通用引用。定义一个绑定到所有值类别的引用,该引用的类型保留其初始值的类型和值类别。\par

如果声明\par

\begin{lstlisting}[caption={}]
auto&& ref{ ... }; // ref is a universal/forwarding reference
\end{lstlisting}

类型ref是根据函数模板参数的通用引用类型推导出来的:\par

\begin{lstlisting}[caption={}]
template<typename T>
void callFoo(T&& ref); // ref is a universal/forwarding reference
\end{lstlisting}

根据规则,ref(即auto\&\&或T\&\&类型)的类型是\par

\begin{itemize}
	\item 如果引用lvalue,则为lvalue引用
	\item 如果引用rvalue,则为rvalue引用
\end{itemize}

例如:\par

\begin{lstlisting}[caption={}]
// forward declarations:
std::string retByValue();
std::string& retByRef();
std::string&& retByRefRef();
const std::string& retByConstRef();
const std::string&& retByConstRefRef();

// deduced auto&& types:
std::string s;
auto&& r1{s}; // std::string&
auto&& r2{std::move(s)}; // std::string&&

auto&& r3{retByValue()}; // std::string&&
auto&& r4{retByRef()}; // std::string&
auto&& r5{retByRefRef()}; // std::string&&
auto&& r6{retByConstRef()}; // const std::string&
auto&& r7{retByConstRefRef()}; // const std::string&&
\end{lstlisting}

当使用rvalue(prvalue或xvalue)初始化用auto\&\&声明的引用时,就声明了一个rvalue引用。例如,将引用绑定到标记为\textit{std::move()}的对象或返回的普通值或rvalue引用时,就会出现这种情况。\par

但是,当使用lvalue初始化用auto\&\&声明的引用时,就声明了一个lvalue引用。例如,将引用绑定到已命名的对象或返回lvalue引用的函数的返回值时,就会出现这种情况。\par

因为字符串字面值是lvalue(以字符数组为类型),所以将通用引用绑定到字符串字面值时,也会得到lvalue引用:\par

\begin{lstlisting}[caption={}]
auto&& r8{"hello"}; // const char(&)[6]
\end{lstlisting}

因为对函数的引用总是lvalue,所以将通用引用绑定到函数时,也会得到lvalue引用:\par

\begin{lstlisting}[caption={}]
std::string foo(int); // forward declaration

auto&& r9{foo}; // lvalue of type std::string(&)(int)
\end{lstlisting}

\hspace*{\fill} \par %插入空行
\textbf{11.2.2 完美转发auto\&\&引用}

同样,也可以完美地将传递给通用引用的值作为函数模板形参:\par

\begin{lstlisting}[caption={}]
template<typename T>
void callFoo(T&& ref) {
	foo(std::forward<T>(ref)); // becomes foo(std::move(ref)) for passed rvalues
}
\end{lstlisting}

完全可以转发auto\&\&的通用引用:\par

\begin{lstlisting}[caption={}]
auto&& ref{ ... };

foo(std::forward<decltype(ref)>(ref)); // becomes foo(std::move(ref)) for rvalues
\end{lstlisting}

表达式\textit{std::forward<decltype(ref)>(ref)}是这样实现的:\par

\begin{itemize}
	\item 如果传递的decltype(ref)类型是一个lvalue引用,如果ref是用返回的lvalue引用初始化的,则表达式将ref强制转换为lvalue引用,这意味着ref传递时没有移动语义。
	\item 如果传递的decltype(ref)类型是rvalue引用,如果ref是用返回的普通值或rvalue引用初始化的,则表达式将ref转换为rvalue引用,这是\textit{std::move(ref)}的效果。
\end{itemize}

因此,如果用函数的返回值初始化通用引用:\par

\begin{lstlisting}[caption={}]
auto&& ret{compute(t)}; // initialize a universal reference with the return value
\end{lstlisting}

表达式为\par

\begin{lstlisting}[caption={}]
process(std::forward<decltype(ret)>(ret)); // perfectly forward the return value
\end{lstlisting}

当\textit{compute()}返回rvalue(如临时对象或rvalue引用)时,扩展为\par

\begin{lstlisting}[caption={}]
process(std::move(ret));
\end{lstlisting}
























