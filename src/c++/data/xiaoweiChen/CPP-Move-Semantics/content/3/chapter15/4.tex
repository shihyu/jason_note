原始指针不能从移动语义中获益(它们的地址值总是副本),而智能指针可以从移动语义中获益。\par

\begin{itemize}
	\item 共享指针(std::shared\_ptr<>)支持移动语义,因为移动共享指针比复制共享指针廉价得多。
	\item 唯一指针(std::unique\_ptr<>)甚至只支持移动语义,因为不可能复制unique指针。
\end{itemize}

\hspace*{\fill} \par %插入空行
\textbf{15.4.1 std::shared\_ptr<>的移动语义}

共享指针有共享所有权的概念。多个共享指针可以“拥有”同一个对象,当最后一个所有者销毁(或获得一个新值)时,将调用所拥有对象的“删除器”。\par

例如:\par

\begin{lstlisting}[caption={}]
{
	std::shared_ptr<int> sp1; // init shared pointer that does not own anything
	{
		auto sp2{std::make_shared<int>(42)}; // init shared pointer that owns new int
		...
		sp1 = sp2; // sp1 and sp2 now share ownership
		...
		*sp2 = 77; // modify value via sp2
		...
	} // sp2 destroyed, sp1 is the only owner
	std::cout << *sp1 << '\n'; // use modified value via sp1
} // last owner destroyed so that delete is called
\end{lstlisting}

这个例子中,赋值操作符复制了int对象的所有。之后,两个共享指针都拥有该对象。但请注意,复制所有权是非常昂贵的操作。这是因为计数器必须跟踪所有者的数量:\par

\begin{itemize}
	\item 每次复制一个共享指针时,所有者计数器就递增
	\item 每次销毁共享指针或赋值时,所有者计数器就递减
\end{itemize}

此外,修改计数器的值代价很高,因为修改是原子操作,可以避免多线程处理拥有相同对象的共享指针时出现的问题。\par

因此,通过引用遍历共享指针集合要廉价得多:\par

\begin{lstlisting}[caption={}]
std::vector<std::shared_ptr< ... >> coll;
...
for (auto sp : coll) { // expensive
	...
}
...
for (const auto& sp : coll) { // cheap
	...
}
\end{lstlisting}

对于移动语义,最好是移动共享指针而不是复制它们。例如,不要这样实现:\par

\begin{lstlisting}[caption={}]
std::shared_ptr<int> lastPtr; // init shared pointer that does not own anything
while ( ... ) {
	auto ptr{std::make_shared<int>(getValue())}; // init shared pointer that owns new int
	...
	lastPtr = ptr; // expensive (note: ptr no longer used)
} // ptr destroyed, lastPtr is the only owner
\end{lstlisting}

最好这样做:\par

\begin{lstlisting}[caption={}]
std::shared_ptr<int> lastPtr; // init shared pointer that does not own anything
while ( ... ) {
	auto ptr{std::make_shared<int>(getValue())}; // init shared pointer that owns new int
	...
	lastPtr = std::move(ptr); // cheap
} // ptr destroyed, lastPtr is the only owner
\end{lstlisting}

因此,当对象移动时,具有共享指针成员的对象将失去这些成员所指向的资源的所有权。这对性能有好处,但会创建无效的从状态移动。因此,应该对拥有std::shared\_ptr<>类型成员的已移动对象的状态进行双重检查。\par

\hspace*{\fill} \par %插入空行
\textbf{15.4.2 std::unique\_ptr<>的移动语义}

类模板std::unique\_ptr<>实现了独占所有权的概念。类型系统确保一个对象在任何时候只能有一个所有者。技巧是使用类型系统来禁用复制唯一指针。因为这个检查是在编译时完成的,所以这种方法不会在运行时引入任何显著的性能开销。\par

可以在函数中创建unique指针,并将其返回给调用者:\par

{\color{red}{lib/uniqueptr1.cpp}}\par

\begin{lstlisting}[caption={}]
#include <iostream>
#include <string>
#include <memory>

std::unique_ptr<std::string> source()
{
	static long id{0};
	// create string with new and let ptr own it:
	auto ptr = std::make_unique<std::string>("obj" + std::to_string(++id));
	...
	return ptr; // transfer ownership to caller
}

int main()
{
	std::unique_ptr<std::string> p;
	for (int i = 0; i < 10; ++i) {
		p = source(); // p gets ownership of the returned object
		              //(previously returned object of source() is deleted)
		std::cout << *p << '\n';
		...
	}
} // last-owned object of p is deleted
\end{lstlisting}

和已移动类型一样,要将所有权传递给接收器函数,只能通过\textit{std::move()}传递:\par

\begin{lstlisting}[caption={}]
std::vector<std::unique_ptr<std::string>> coll;
std::unique_ptr<std::string> up;
...
coll.push_back(up); // ERROR: copying disabled
coll.push_back(std::move(up); // OK, moves ownership into new element of coll
\end{lstlisting}

如果传递unique指针给一个通过引用接受参数的潜在接收器函数,则不知道所有权是否已经移动。所有权是否转移取决于功能的实现。这种情况下,可以用bool()操作符再次检查状态:\par

\begin{lstlisting}[caption={}]
std::unique_ptr<std::string> up;
...
sink(std::move(up)); // might move ownership to sink()
if (up) { // does it still have the ownership?
	...
}
\end{lstlisting}

或者,可以确保所有权消失(资源释放):\par

\begin{lstlisting}[caption={}]
std::unique_ptr<std::string> up;
...
sink(std::move(up)); // might move ownership to sink()
up.reset(); // ensure ownership is gone (resource deleted)
\end{lstlisting}













