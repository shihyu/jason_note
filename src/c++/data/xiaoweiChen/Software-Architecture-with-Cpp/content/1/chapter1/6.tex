
现在,进一步了解在本书中使用最多的编程语言。C++是一种已经存在了几十年的多范式语言。从创立至今,已经发生了很大的变化。当C++11问世时,该语言的创造者Bjarne Stroustrup说,\textit{感觉像是一门全新的语言}。C++20的发布标志着这头巨兽进化的又一个里程碑,也给编码方式带来了一场革命。然而,语言的哲学始终如一,这些年来从未改变。

简而言之,可以总结为三条规则:

\begin{itemize}
\item C++中不应该有其他语言(汇编除外)。
\item 只为所使的付费。
\item 以低成本提供高级抽象(零成本是终极的目标)。
\end{itemize}

\textit{为使用的东西付费}意味着,可以在堆栈上创建数据成员(许多语言都在堆上分配对象,但C++却不需要这样做),在堆上分配是有代价的——分配器可能需要锁定互斥量,这在某些类型的应用程序中可能是一个负担。好的方面是,可以方便地分配变量,而不必每次都动态地分配内存。

高级抽象是C++与C或汇编等低级语言的区别,其允许在源码中直接表达思想和意图,这对语言的类型安全性非常有利。看下下面的代码段:

\begin{lstlisting}[style=styleCXX]
struct Duration {
	int millis_;
};

void example() {
	auto d = Duration{};
	d.millis_ = 100;
	
	auto timeout = 1; // second
	d.millis_ = timeout; // ouch, we meant 1000 millis but assigned just 1
}

\end{lstlisting}

更好的做法是利用该语言提供的安全类型:

\begin{lstlisting}[style=styleCXX]
#include <chrono>

using namespace std::literals::chrono_literals;

struct Duration {
	std::chrono::milliseconds millis_;
};

void example() {
	auto d = Duration{};
	// d.millis_ = 100; // compilation error, as 100 could mean anything
	d.millis_ = 100ms; // okay
	auto timeout = 1s; // or std::chrono::seconds(1);
	d.millis_ =
		timeout; // okay, converted automatically to milliseconds
}
\end{lstlisting}

前面的抽象可以避免犯错误,并且在这样做的时候不会付出任何代价,生成的汇编与第一个示例相同。这就是C++中为什么有零成本抽象的原因。有时,C++允许使用抽象,从而产生比更好的代码。C++20中的协程就是一个语言特性的例子,在使用时通常会带来这样的好处。

标准库提供的另一组很棒的抽象是算法。认为下列哪一段代码更容易阅读,更容易证明没错?哪个更好地表达意图?

\begin{lstlisting}[style=styleCXX]
// Approach #1
int count_dots(const char *str, std::size_t len) {
	int count = 0;
	for (std::size_t i = 0; i < len; ++i) {
		if (str[i] == '.') count++;
	}
	return count;
}

// Approach #2
int count_dots(std::string_view str) {
	return std::count(std::begin(str), std::end(str), '.');
}

\end{lstlisting}

OK,第二个函数有一个不同的接口,但即使它保持不变,也可以通过指针和长度创建\texttt{std::string\_view}。由于它是轻量级的类型,编译器可以对其进行优化。

使用更高层次的抽象可以得到更简单、更易于维护的代码。C++从一开始就致力于提供零成本的抽象,所以可以在此基础上进行构建,而不是重新设计轮子。

谈到简单和可维护的代码,下一节将介绍一些在编写此类代码的过程中的规则和方法。















