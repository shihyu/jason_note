可以通过使用现代的C++结构,来减少代码中常见安全漏洞的数量,而不是使用较老的C代码。然而,总是有更安全的抽象也证明是脆弱的。仅仅选择更安全的实现,并认为已经尽了最大努力是不够的。大多数情况下,有一些方法可以加强代码。

什么是加强代码?根据定义,是减少系统表面脆弱性的过程。通常,这意味着关闭不使用的功能,并着眼于更简单的系统而不是复杂的系统。这还可能使用工具来增加现有功能的健壮性。

当应用于操作系统级别时,这些工具可能意味着内核补丁、防火墙和入侵检测系统(IDS)。在应用程序级别,这可能意味着各种缓冲区溢出和下溢保护机制,使用容器和虚拟机(VM)进行特权隔离和进程隔离,或强制加密通信和存储。

本节中,将关注应用程序级别的一些示例,而下一节将关注操作系统级别。

\subsubsubsection{10.4.1\hspace{0.2cm}面向安全性的内存分配器}

如果非常重视保护应用程序免受堆相关的攻击,例如:堆溢出、释放后使用或双重释放,可以考虑用面向安全的版本替换标准内存分配器。以下是两个可能会引起兴趣的项目:

\begin{itemize}
\item 
FreeGuard,可在\url{https://github.com/UTSASRG/FreeGuard}获得,并在\url{https://arxiv.org/abs/1709.02746}的一篇论文中描述

\item 
GrapheneOS项目的hardened\_malloc,可在\url{https://github.com/GrapheneOS/hardened\_malloc}处获得
\end{itemize}

FreeGuard于2017年发布,自那以后除了零星的bug修复外,并没有太大的变化。另一方面,hardened\_malloc在积极地开发。这两个分配器都设计为作为标准malloc()的替代品。可以在不修改应用程序的情况下使用它们,只需设置LD\_PRELOAD环境变量或将库添加到/etc/preload.so配置文件中。FreeGuard的目标是Linux和64位x86系统上的Clang编译器,而hardened\_malloc的目标是更广泛的兼容性,尽管目前主要支持Android的Bionic,musl和glibc。hardened\_malloc也是基于OpenBSD的alloc,OpenBSD本身就是以安全为中心的项目。

可以将所使用的集合替换为更安全的等价物,而不是替换内存分配器。SaferCPlusPlus(\url{https://duneroadrunner.github.io/SaferCPlusPlus/})项目提供了\texttt{std::vector<>}、\texttt{std::array<>}和\texttt{std::string}的替代品,这些可以作为现有代码中的临时替换。该项目还包括防止未初始化使用或符号不匹配的基本类型的替代品、并发数据类型,以及指针和引用的替代品。

\subsubsubsection{10.4.2\hspace{0.2cm}自动检查}

一些工具可以有助于确保正在构建的系统的安全性。

\hspace*{\fill} \\ %插入空行
\noindent
\textbf{编译器警告}

虽然编译器警告本身不是一种工具,但可以使用并调整编译器警告,以获得每个C++开发人员都将使用的工具——C++编译器——更好的输出。

因为编译器可以做一些比标准要求的更深入的检查,所以建议利用这种可能性。当使用GCC或Clang等编译器时,推荐的设置包括\texttt{-Wall -Wextra}标志。这将产生更多的诊断信息,并在代码没有遵循诊断时产生警告。如果希望更严格一些,还可以启用\texttt{-Werror},它会把所有警告转换为错误,并防止编译没有通过增强诊断的代码。如果想严格遵守标准,有\texttt{-pedantic}和\texttt{-pedanticerrors}标志,将检查代码是否符合标准。

当使用CMake进行构建时,可以使用以下函数在编译期间启用这些标志:

\begin{lstlisting}[style=styleCMake]
add_library(customer ${SOURCES_GO_HERE})
target_include_directories(customer PUBLIC include)
target_compile_options(customer PRIVATE -Werror -Wall -Wextra)
\end{lstlisting}

这样,除非修复编译器报告的所有警告(已转换的错误),否则编译将失败。

也可以在OWASP(\url{https://www.owasp.org/index.php/C-Based\_Toolchain\_Hardening})和Red Hat(\url{https://developers.redhat.com/blog/2018/03/21/compiler-and-linker-flags-gcc/})的这些文章中找到工具链的建议设置。

\hspace*{\fill} \\ %插入空行
\noindent
\textbf{静态分析}

有一类工具可以提高代码的安全性,这就是所谓的静态应用程序安全测试(SAST)工具,只关注安全方面的静态分析工具。

SAST工具可以很好地集成到CI/CD流水中,它们只是读取源代码。输出通常也适用于CI/CD,因为它突出了源代码中特定位置发现的问题。另一方面,静态分析可能会忽略许多类型的问题,这些问题无法自动找到,或者仅通过静态分析无法找到。这些工具也不关心与配置相关的问题,因为配置文件没有在源代码本身中表示。

C++ SAST工具的例子包括以下开源解决方案:

\begin{itemize}
\item 
Cppcheck(\url{http://cppcheck.sourceforge.net/}),是一个通用的静态分析工具,专注于消除误报

\item 
Flawfinder(\url{https://dwheeler.com/flawfinder/}),似乎没有积极地维护

\item 
LGTM(\url{https://lgtm.com/help/lgtm/about-lgtm}),支持几种不同的语言,具有自动分析拉请求的功能

\item 
SonarQube(\url{https://www.sonarqube.org/}),可以很好的CI/CD集成和语言覆盖,并提供了一个商业版本

\item 
Checkmarx CxSAST(\url{https://www.checkmarx.com/products/staticapplication-security-testing/}),可以零配置和语言覆盖

\item 
CodeSonar(\url{https://www.grammatech.com/products/codesonar}),专注于深度分析和发现缺陷

\item 
Klocwork(\url{https://www.perforce.com/products/klocwork}),专注于准确性

\item 
Micro Focus Fortify(\url{https://www.microfocus.com/en-us/products/staticcode-analysis-sast/overview}),具有广泛的语言支持和集成的其他工具(由同一制造商)

\item 
Parasoft C/C++测试(\url{https://www.parasoft.com/products/ctest}),这是一个静态和动态分析、单元测试、跟踪等的(集成)解决方案

\item 
Polyspace Bug Finder来自MathWorks(\url{https://www.mathworks.com/products/polyspace-bug-finder.html}),可以与Simulink模型的集成

\item 
Veracode静态分析(\url{https://www.veracode.com/products/binarystatic-analysis-sast}),是一个用于静态分析的SaaS解决方案

\item 
白帽哨兵源(\url{https://www.whitehatsec.com/platform/staticapplication-security-testing/}),也专注于消除误报
\end{itemize}

\hspace*{\fill} \\ %插入空行
\noindent
\textbf{动态分析}

就像对源代码执行静态分析一样,对生成的二进制文件执行动态分析,名称中的“动态”是指观察代码在动作中处理的实际数据。当关注安全性时,这类工具也可以称为动态应用程序安全测试(DAST)。

与SAST相比,主要优势是可以找到许多从源代码分析角度看不到的流。当然,这也有缺点,必须运行应用程序才能执行分析,这可能既耗时又耗内存。

DAST工具通常专注于Web相关的漏洞,如XSS、SQL(和其他)注入,或公开的敏感信息。下一小节中,将更多地关注更通用的动态分析工具Valgrind。

\hspace*{\fill} \\ %插入空行
\noindent
\textit{Valgrind和应用验证器}

Valgrind通常称为内存泄漏调试工具。事实上,它是一个帮助构建动态分析工具的工具框架,而不一定与内存问题相关。除了内存错误检测器之外,该工具套件目前还包括一个线程错误检测器、一个缓存和分支预测分析器以及一个堆分析器。它支持类unix操作系统(包括Android)上的各种平台。

本质上Valgrind作为VM,将二进制转换为一种更简单的形式,称为中间表示。它不是在一个实际的处理器上运行程序,而是在这个VM下执行,这样就可以分析和验证每个调用。

若在Windows上开发,可以使用应用程序验证器(AppVerifier),而不是Valgrind。AppVerifier可以检测稳定性和安全性问题,可以监视正在运行的应用程序和用户模式驱动程序,以查找内存问题,如泄漏和堆损坏、线程和锁问题、无效句柄等。

\hspace*{\fill} \\ %插入空行
\noindent
\textit{消杀工具}

消杀程序是基于代码编译时插装的动态测试工具,可以提高系统的整体稳定性和安全性,以及避免未定义行为。在\url{https://github.com/google/sanitizers},可以找到LLVM(Clang是基于LLVM)和GCC的实现。它们解决了内存访问、内存泄漏、数据竞争和死锁、未初始化的内存使用和未定义行为。

AddressSanitizer(ASan)保护代码免受与内存寻址相关的问题,例如全局缓冲区溢出、释放后使用或返回后堆栈使用。尽管这是同类中最快的解决方案之一,但仍然会将这个过程减慢两倍左右。最好在运行测试和进行开发时使用,但在生产版本中关闭。其可以通过\texttt{-fsanitize=address}标志开启。

AddressSanitizerLeakSanitizer(LSan)集成了ASan来查找内存泄漏,默认在x86\_64 Linux和x86\_64 macOS上启用。需要设置一个环境变量,\texttt{ASAN\_OPTIONS=detect\_leaks=1},这样LSan在过程结束时就可以执行泄漏检测。LSan也可以作为一个独立的库使用,而不需要AddressSanitizer,但是这种模式的测试要少得多。

Threadsantizer(TSan)可以检测并发性问题,如数据竞争和死锁。可以使用\texttt{-fsanitize=thread}标志来启用。

MemorySanitizer(MSan)专注于与访问未初始化内存相关的bug,实现了在前一小节中介绍的Valgrind的一些特性。MSan支持64位x86、ARM、PowerPC和MIPS平台。可以使用Clang的\texttt{-fsanitize=memory -fPIE -pie}标志来启用(还会打开独立于位置的可执行文件,这个稍后会讨论)。

硬件辅助地址消杀程序(HWASAN)类似于常规的ASan。主要区别是在可能的情况下使用硬件辅助,该特性目前仅在64位ARM架构上可用。

UndefinedBehaviorSanitizer(UBSan)查找未定义行为的其他可能原因,如整数溢出、被零除或不当的位移位操作。可以使用Clang的\texttt{-fsanitize=undefined}标志来启用它。

尽管消杀程序可以发现许多潜在的问题,但它们的性能仅与运行它们的测试一样。在使用消杀程序时,请保持测试的代码高覆盖率。否则,可能会对安全性产生错觉。

\hspace*{\fill} \\ %插入空行
\noindent
\textit{模糊测试}

模糊测试是DAST工具的子类,其检查应用程序在遇到无效、意外、随机或恶意格式的数据时的行为。当对跨信任边界的接口(如最终用户文件上传表单或输入)使用这种检查时,会特别有用。

这个类别中一些有趣的工具包括:

\begin{itemize}
\item 
Peach Fuzzer: \url{https://www.peach.tech/products/peach-fuzzer/}

\item 
PortSwigger Burp: \url{https://portswigger.net/burp}

\item 
OWASP Zed攻击代理项目: \url{https://www.owasp.org/index.php/OWAS\_Zed\_Attack\_Proxy\_Project}

\item 
Google的ClusterFuzz: \url{https://github.com/google/clusterfuzz} (和OSSFuzz: \url{https://github.com/google/oss-fuzz})
\end{itemize}

\subsubsubsection{10.4.3\hspace{0.2cm}处理隔离和沙盒}

如果想在自己的环境中运行未经验证的软件,可能希望将其与系统的其余部分隔离开来。一些使用沙箱的方法是通过虚拟机、容器或微虚拟机,如AWS Lambda使用的是Firecracker(\url{https://firecracker-microvm.github.io/})。

这样,应用程序的崩溃、泄漏和安全问题就不会传播到整个系统,从而使它变得无用或受损。由于每个进程都有自己的沙箱,最坏的情况是只丢失这一个服务。

对于C和C++代码,也有沙盒API(SAPI:\url{https://github.com/google/sandboxed-api})是一个由Google领导的开源项目,允许为库而不是整个过程构建沙盒。它在Google的Chrome和Chromium网络浏览器中就有使用。

尽管VM和容器可以是过程隔离策略的一部分,但不要将它们与微服务混淆,后者通常使用类似的构建块。微服务是一种架构设计模式,它们并不自动等于更好的安全性。

















