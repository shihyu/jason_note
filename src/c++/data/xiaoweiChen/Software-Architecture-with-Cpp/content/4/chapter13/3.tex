

关于单应用有很多观点。一些架构师认为,单应用就是邪恶的,因为它们不能很好地扩展,是紧密耦合的,而且很难维护。还有一些人声称,来自单应用的性能优势抵消了缺点。事实上,紧密耦合的组件在网络、处理能力和内存方面所需的开销比松散耦合的组件要小得多。

当涉及到相关方时,每个应用程序都有独特的业务需求,并在独特的环境中操作,因此没有关于哪种方法更适合的通用规则。更令人困惑的是,在从单服务到微观服务的最初迁移之后,一些公司开始将微观服务整合为宏观服务。这是因为维护数千个独立软件实例的负担太大,有时可能无法处理。

架构的选择应该总是来自于业务需求和对不同备选方案的仔细分析,将意识形态置于实用主义之前通常会导致组织内耗。当一个团队不考虑不同的解决方案或不同的外部意见,而试图不惜一切代价坚持给定的方法时,这个团队就不再履行为正确的工作交付正确的工具的义务。

如果正在开发或维护一个单应用,可以考虑提高扩展性。本节介绍的技术旨在解决这个问题,同时也使应用程序更容易迁移到微服务(如果决定这样做的话)。

瓶颈的主要原因有三:

\begin{itemize}
\item 
内存

\item 
存储

\item 
计算
\end{itemize}

这里将展示如何使用它们来开发基于微服务的可扩展解决方案。

\subsubsubsection{13.3.1\hspace{0.2cm}外包的内存管理}

帮助微服务规模扩大的方法是外包他们的任务。内存管理和缓存数据可能会阻碍扩展工作。

对于单应用,将缓存的数据直接存储在进程内存中没有问题,因为进程将是唯一访问缓存的进程。但是由于一个过程有多个副本,这种方法会出现一些问题。

如果一个副本已经计算了工作负载的一部分,并将其存储在本地缓存中,该怎么办?另一个副本不知道这一事实,必须再次计算。这样,应用程序既浪费了计算时间(因为必须多次执行相同的任务),又浪费了内存(因为结果也分别存储在每个副本中)。

为了避免这种情况,可以考虑切换到外部内存存储,而不是在应用程序内部管理缓存。使用外部解决方案的另一个好处是,缓存的生命周期不再与应用程序的生命周期绑定。可以重新启动并部署新版本的应用程序,并保留已经存储在缓存中的值。

这也可能导致更短的启动时间,因为应用程序不再需要在启动期间执行计算。两种主流行的内存缓存解决方案是Memcached和Redis。

\hspace*{\fill} \\ %插入空行
\noindent
\textbf{Memcached}

2003年发布的Memcached是两者中较老的产品。它是一个通用的、分布式的键值存储。该项目最初的目标是通过将缓存的值存储在内存中来卸载Web应用程序中使用的数据库。Memcached是按设计分发的,从1.5.18版本开始,可以在不丢失缓存内容的情况下重新启动Memcached服务器。这可以通过使用RAM磁盘作为临时存储空间来实现。

它使用了一个简单的API,可以通过telnet或netcat进行操作,也可以使用许多流行编程语言的绑定。没有针对C++的绑定,但可以使用C/C++的libmemcached库。

\hspace*{\fill} \\ %插入空行
\noindent
\textbf{Redis}

Redis是一个比Memcached更新的项目,最初版本发布于2009年。从那以后,Redis在很多情况下都取代了Memcached。就像Memcached一样,它是一个分布式的、通用的、内存中的键值存储。

与Memcached不同的是,Redis还有可选的数据持久性。虽然Memcached操作的键和值是简单的字符串,Redis也支持其他数据类型,例如:

\begin{itemize}
\item 
字符串列表

\item 
字符串集合

\item 
已排序的字符串集合

\item 
其中键和值为字符串的哈希表

\item 
地理空间数据(从Redis 3.2开始)

\item 
HyperLog日志
\end{itemize}

Redis的设计使它成为缓存会话数据、缓存网页和实现排行榜的绝佳选择。除此之外,还可以用于消息队列。Python中流行的分布式任务队列库Celery,使用Redis作为可能的代理之一,还有RabbitMQ和Apache SQS。

Microsoft,Amazon,Google和Alibaba都将基于redis的管理服务作为其云平台的一部分。

Redis客户端有很多C++的实现。两个有趣的是使用C++17编写的redis-cpp库(\url{https://github.com/tdv/redis-cpp})和使用Qt工具包编写的QRedisClient(\url{https://github.com/uglide/qredisclient})。

以下使用redis-cpp的示例取自官方文档,演示了如何来设置和获取存储中的一些数据:

\begin{lstlisting}[style=styleCXX]
#include <cstdlib>
#include <iostream>
#include <redis-cpp/execute.h>
#include <redis-cpp/stream.h>
int main() {
	try {
		auto stream = rediscpp::make_stream("localhost", "6379");
		
		auto const key = "my_key";
		
		auto response = rediscpp::execute(*stream, "set", key,
										"Some value for 'my_key'", "ex",
										"60");
		
		std::cout << "Set key '" << key << "': "
				  << response.as<std::string>()
				  << std::endl;
		
		response = rediscpp::execute(*stream, "get", key);
		std::cout << "Get key '" << key << "': "
				  << response.as<std::string>()
				  << std::endl;
	} catch (std::exception const &e) {
		std::cerr << "Error: " << e.what() << std::endl;
		return EXIT_FAILURE;
	}
	return EXIT_SUCCESS;
}
\end{lstlisting}

该库处理不同的数据类型,该示例将该值设置为字符串列表。

\hspace*{\fill} \\ %插入空行
\noindent
\textbf{内存中的哪个缓存更好?}

对于大多数应用程序来说,Redis是一个更好的选择。它有更好的用户社区,许多不同的实现,并且得到很好的支持。除此之外,还具有快照、复制、事务和发布/订阅模型。在Redis中嵌入Lua脚本是可能的,对地理空间数据的支持使它成为地理支持的Web和移动应用程序的一个很好的选择。

但若主要目标是在Web应用程序中缓存数据库查询的结果,Memcached是一个更简单的解决方案,开销要小得多。这意味着其可以更好地使用资源,因为它不必存储类型元数据或执行不同类型之间的转换。

\subsubsubsection{13.3.2\hspace{0.2cm}外包存储}

引入和扩展微服务时,另一个可能的限制是存储。通常,本地块设备用于存储不属于数据库的对象(如静态PDF文件、文档或图像)。即使在今天,块存储仍然非常受本地块设备和网络文件系统(如NFS或CIFS)的欢迎。

虽然NFS和CIFS是网络附加存储(NAS)的领域,但还有一些协议与在不同级别上运行的一个概念相关:存储区域网络(SAN)。一些流行的是iSCSI,网络块设备(NBD),以太网ATA,光纤通道协议和以太网光纤通道。

不同的方法为分布式计算设计了集群文件系统:GlusterFS、CephFS或Lustre。然而,所有这些都作为块设备向用户公开相同的POSIX文件API。

Amazon网络服务提出了一个关于存储的新观点。S3(Amazon Simple Storage Service)是对象存储服务,API提供了对存储在bucket中的对象的访问。这与传统的文件系统不同,因为文件、目录或索引节点之间没有区别。有指向对象的桶和键,对象是服务存储的二进制数据。

\subsubsubsection{13.3.3\hspace{0.2cm}外包计算}

微服务的原则之一,是流程应该只负责完成工作流的单个部分。从单服务迁移到微服务的一个自然步骤是定义可能的长时间运行的任务,并将它们拆分为单独的进程。

这是任务队列背后的概念,任务队列处理管理任务的整个生命周期。使用任务队列,不需要自己实现线程化或多处理,而是委托要执行的任务,然后由任务队列异步处理。该任务可以在与原始进程相同的机器上执行,但也可以在具有特殊需求的机器上运行。

任务及其结果是异步的,因此在主进程中不存在阻塞。Web开发中流行的任务队列例子有Python的Celery,Ruby的Sidekiq, Node.js的Kue和Go的Machinery。所有这些都可以用Redis作为代理。不幸的是,对于C++还没有任何成熟的解决方案。

如果正在认真考虑采用这种方式,可能的方法是直接在Redis中实现一个任务队列。Redis和它的API提供了必要的原语来支持这样的行为。另一种可能的方法是使用现有的任务队列之一(如Celery),并通过直接调用Redis来进行调用。但不建议这样做,因为取决于任务队列的实现细节,而不是文档化的公共API。另一种方法是使用SWIG或类似方法提供的绑定来对接任务队列。



















