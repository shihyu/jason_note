

虽然微服务不依赖于特定的编程语言或技术,但实现微服务的一个常见选择是Go。这并不意味着其他语言不适合微服务的开发,C++的低计算和内存开销使其成为微服务的理想选择。 

但首先,将详细介绍微服务的一些优点和缺点。之后,将重点关注与微服务相关的设计模式(与第4章中涉及的一般设计模式相反)。

\subsubsubsection{13.2.1\hspace{0.2cm}微服务的好处}

可能经常听到“微服务”的高级说法,其确实能带来一些好处,以下是其中一些。

\hspace*{\fill} \\ %插入空行
\noindent
\textbf{模块化}

由于整个应用程序可分割成许多相对较小的模块,所以更容易理解每个微服务的功能。这种理解的自然结果是,测试单个微服务也更容易。每个微服务通常都有一个有限的范围,这一事实也有助于测试。毕竟,只测试日历应用程序要比测试整个Personal Information Management(PIM)套件容易得多。

然而,这种模块化是有代价的。开发团队可能对单个微服务有更好的理解,但与此同时,会发现很难理解整个应用程序是如何组成的。虽然不需要了解构成应用程序的微服务的内部细节,但组件之间的关系数量之多对认知构成了挑战。在使用这种架构方法时,应用微服务契约是一个良好的实践。

\hspace*{\fill} \\ %插入空行
\noindent
\textbf{扩展性}

扩展范围有限的应用程序更容易,这样做的一个原因是瓶颈更少。

扩展更小的工作流也更划算。想象一个负责管理展会的应用程序,当系统开始出现性能问题,实现规模化的唯一方法就是引入更大的机器来运行整个系统,这叫做垂直扩展。

使用微服务的第一个优势是可以横向扩展,可以引入更多的机器,而不是更大的机器(通常更便宜)。第二个优势来自于这样一个事实:只需要扩展应用程序中存在性能问题的那些部分。这也有助于节省基础设施的资金。

\hspace*{\fill} \\ %插入空行
\noindent
\textbf{灵活性}

如果设计得当,微服务不太容易受到供应商的限制。当决定要切换一个第三方组件时,不必一次完成整个痛苦的迁移。微服务的设计考虑到需要使用接口,因此唯一需要修改的部分是微服务和第三方组件之间的接口。

组件也可能一个接一个地迁移,有些仍然使用来自旧供应商的软件。这样,可以将一次在多个地方引入破坏性更改的风险分离开来。更重要的是,可以将此与金丝雀部署模式结合起来,以更细的粒度管理风险。

这种灵活性不仅仅与单个服务相关。还可能意味着不同的数据库、不同的队列和消息传递解决方案,甚至完全不同的云平台。虽然不同的云平台通常提供不同的服务和API来使用,但通过微服务架构,可以逐个迁移工作负载,并在一个新的平台上独立测试。

由于性能问题、扩展性或可用的依赖关系而需要重写时,重写微服务要比重写整体服务快得多。

\hspace*{\fill} \\ %插入空行
\noindent
\textbf{与旧系统集成}

微服务不一定是全有或全无的方法。如果应用程序经过了良好的测试,并且迁移到微服务可能会产生很多风险,就没有压力去废除可用的解决方案。最好只拆分需要进一步开发的部分,并将它们作为原始服务使用的微服务引入。

通过这种方法,将获得与微服务相关的敏捷发布周期的好处,同时避免从零开始创建新的架构和基本上重新构建整个应用程序。如果某些东西已经运行得很好了,那最好是专注于如何在不破坏好的部分的情况下添加新功能,而不是从头开始。

\hspace*{\fill} \\ %插入空行
\noindent
\textbf{分布式开发}

开发团队规模小、位置集中的时代已经过去了。即使在传统的基于办公室的公司中,远程工作和分布式开发也是一个事实。像IBM、Microsoft和Intel这样的巨头会让来自不同地方的人在一个项目上工作。

微服务支持更小、更敏捷的团队,这使得分布式开发更加容易。当不再需要促进20人或20人以上的团队之间的沟通时,构建需要较少外部管理的自组织团队也更容易。

\subsubsubsection{13.2.2\hspace{0.2cm}微服务的缺点}

即使认为可能需要微服务,请记住它们的一些缺点。简而言之,它们不适合每个人。大公司通常可以抵消这些缺点,但小公司通常不能有这种奢求。

\hspace*{\fill} \\ %插入空行
\noindent
\textbf{依赖成熟的DevOps}

构建和测试微服务应该比在大型单应用程序上执行类似操作要快得多。但是为了实现敏捷开发,这种构建和测试将需要更频繁地执行。 

当处理一个庞然大物时,手动部署应用程序可能是明智的,但是如果将同样的方法应用于微服务,则会导致许多问题。

为了在开发中拥抱微服务,必须确保团队有DevOps的思维,理解构建和运行微服务的需求。仅仅把代码交给别人,然后忘记是不够的。

DevOps的思维模式将帮助团队尽可能地实现自动化,在没有持续集成/持续交付流水的情况下开发微服务可能是软件架构中最糟糕的想法,这种方法将带来微服务的所有其他缺点。

\hspace*{\fill} \\ %插入空行
\noindent
\textbf{难以调试}

微服务需要引入可观察性。当出现了问题,永远不知道从哪里开始寻找根本原因。可观察性是一种推断应用程序状态的方法,无需运行调试器或将日志记录到运行工作负载的机器上。

日志聚合、应用程序指标、监视和分布式跟踪的组合是管理基于微服务架构的先决条件。当考虑到自动扩展和自修复可能阻止访问单个服务(开始崩溃)时,这一点尤为正确。

\hspace*{\fill} \\ %插入空行
\noindent
\textbf{额外的开销}

微服务应该是精简和敏捷的,这通常是正确的。然而,基于微服务的架构通常需要额外的开销,第一层开销与用于微服务通信的附加接口有关。RPC库、API提供者和使用者不仅要乘以微服务的数量,还要乘以副本的数量。然后还有辅助服务,如数据库、消息队列等。这些服务还包括可观察性设施,通常包括存储设施和收集数据的个人收集器。

以更好的扩展性进行优化的成本可能会超过运行整个服务所需的成本,这些服务不能立即带来业务价值。更重要的是,可能很难向相关方证明这些成本(在基础设施和开发开销方面)的合理性。

\subsubsubsection{13.2.3\hspace{0.2cm}微服务的设计模式}

很多通用的设计模式也适用于微服务,还有一些通常与微服务相关的设计模式。这里提供的模式对于新建项目和从单一应用程序迁移都很有用。

\hspace*{\fill} \\ %插入空行
\noindent
\textbf{分解模式}

这些模式与分解微服务的方式有关。我们希望确保架构是稳定的,服务是松散耦合的。还希望确保服务具有内聚性和可测试性。最后,希望团队完全拥有一个或多个服务。

\hspace*{\fill} \\ %插入空行
\noindent
\textbf{按业务能力分解}

其中一种分解模式需要根据业务能力进行分解,业务能力与企业为了产生价值所做的事情有关。业务功能的例子有商人管理和客户管理,通常按层次结构组织。

应用此模式时的主要挑战是正确识别业务功能。这需要理解业务本身,并可能需要与业务分析师合作。

\hspace*{\fill} \\ %插入空行
\noindent
\textbf{分解子域名}

另一种分解模式与领域驱动设计(DDD)方法有关。为了定义服务,需要识别DDD子域。就像业务能力一样,标识子域需要业务方面的知识。

这两种方法的主要区别在于,按业务能力分解时,重点更多地放在业务组织(其结构)上。而按子域分解时,重点放在业务试图解决的问题上。

\hspace*{\fill} \\ %插入空行
\noindent
\textbf{数据库/服务模式}

存储和处理数据在每个软件架构中都是一个复杂的问题。错误的选择可能会影响扩展性、性能或维护成本。对于微服务,由于希望微服务是松散耦合的,这就增加了复杂性。

这导致了这样一种设计模式:每个微服务都连接到自己的数据库,因此不受其他服务引入的影响。虽然这种模式会增加一些开销,但好处是可以为每个微服务分别优化模式和索引。

由于数据库往往是相当庞大的基础设施,所以这种方法可能不可行,在微服务之间共享数据库是一种可以理解的权衡方式。

\hspace*{\fill} \\ %插入空行
\noindent
\textbf{部署策略}

对于在多个主机上运行的微服务,可能想知道哪种分配资源的方式更好。让我们比较一下这两种方法。

\hspace*{\fill} \\ %插入空行
\noindent
\textbf{单主机单服务}

使用这个模式,允许每个主机只服务特定类型的微服务。好处是可以调整机器以更好地适应所需的工作负载,并且很好地隔离了服务。当提供超大内存或快速存储时,可以确保其只用于需要的微服务。服务也无法消耗超过所提供的资源。

这种方法的缺点是有些主机可能没有得到充分利用。一种可能的解决方法是使用尽可能小的机器来满足微服务需求,并在必要时扩展。但这种解决方法不能解决主机本身额外开销的问题。

\hspace*{\fill} \\ %插入空行
\noindent
\textbf{单主机多服务}

另一种相反的方法是在每个主机上托管多个服务。这有助于优化机器的利用率,但也有一些缺点。首先,不同的微服务可能需要不同的优化,因此将它们托管在单个主机上仍然不可能。此外,这种方法将失去对主机分配的控制,因此一个微服务中的问题可能会导致一个托管微服务的中断,即使后者在其他方面不会受到影响。

另一个问题是微服务之间的依赖冲突。当微服务没有彼此隔离时,部署必须考虑不同的依赖关系。这个模型也不太安全。

\hspace*{\fill} \\ %插入空行
\noindent
\textbf{可观测性模式}

前一节中,提到微服务是有代价的。这个代价包括引入可观察性的需求,或者冒着失去调试应用程序能力的风险。这里有一些与可观察性相关的模式。

\hspace*{\fill} \\ %插入空行
\noindent
\textbf{日志收集}

微服务就像单应用程序一样使用日志。日志不是存储在本地,而是聚合并转发到一个中央设施。这样,即使服务本身宕机,日志也可用。以集中的方式存储日志还有助于关联来自不同微服务的数据。

\hspace*{\fill} \\ %插入空行
\noindent
\textbf{应用指标}

要根据数据做出决策,首先需要一些数据来采取行动。收集应用程序度量有助于理解实际用户使用的应用程序行为,而不是在合成测试中使用的行为。收集这些指标的方法是推(应用程序在其中主动调用性能监视服务)和拉(性能监视服务定期检查配置的端点)。

\hspace*{\fill} \\ %插入空行
\noindent
\textbf{分布式追踪}

分布式跟踪不仅有助于研究性能问题,还有助于更好地了解真实流量下的应用程序行为。与从单个点收集信息片段的日志记录不同,跟踪关注单个事务的整个生命周期,从源自用户操作的点开始。

\hspace*{\fill} \\ %插入空行
\noindent
\textbf{健康检查API}

由于微服务通常是自动化的目标,因此需要能够传达内部状态。即使进程存在于系统中,也不意味着应用程序是可操作的。这同样适用于开放的网络端口,应用程序可能正在监听,但还不能响应。运行状况检查API为外部服务提供了一种确定应用程序是否已准备好处理工作负载的方法。自修复和自动扩展使用运行状况检查来确定何时需要干预。基本前提是,当应用程序的行为符合预期时,给定端点(例如\texttt{/health})返回HTTP代码200,如果发现问题,则返回不同的代码(或者根本不返回)。

了解了所有的优点、缺点和模式,接下来将展示如何将单个应用程序分割并将其部分地转换为微服务。所提出的方法不仅限于微服务;它们在其他情况下也可能有用,包括单应用程序。















