
构建的每个微服务都需要遵循一般的架构设计模式。微服务和传统应用程序的主要区别在于前者需要实现可观察性。

本节主要讨论一些可观察性的方法。在这里描述了几种开源解决方案,可能会发现它们在设计系统时很有用。

\subsubsubsection{13.4.1\hspace{0.2cm}记录日志}

即使从未设计过微服务,也应该熟悉日志记录这个主题。日志(或日志文件)保存了系统中发生的事件的信息。系统可能意味着应用程序、应用程序运行的操作系统或用于部署的云平台。这些组件中的每一个都可以提供日志。

日志作为单独的文件存储,因为提供了发生的所有事件的永久记录。当系统变得无响应时,可以查询日志并找出故障的原因。

这意味着日志还提供了审计跟踪。因为事件是按时间顺序记录的,所以能够通过检查记录的历史状态来了解系统的状态。

为了协助调试,日志通常需要提供好的可读性。日志有二进制格式,但在使用文件存储日志时,这种格式非常罕见。

\hspace*{\fill} \\ %插入空行
\noindent
\textbf{微服务中的日志记录}

这种日志记录方法本身与传统方法没有太大区别。微服务通常将日志打印到标准输出,而不是使用文本文件在本地存储日志。然后使用统一的日志记录层检索和处理日志。要实现日志记录,需要一个可以配置的日志库。

\hspace*{\fill} \\ %插入空行
\noindent
\textbf{使用spdlog在C++中进行日志记录}

spdlog是C++中一种流行且快速的日志库。用C++11构建,既可以作为头文件库使用,也可以作为静态库使用(这样可以减少编译时间)。

spdlog的一些有趣特性包括:

\begin{itemize}
\item 
格式化

\item 
多种接收方式:
\begin{itemize}
\item 
文件

\item 
控制台

\item 
系统记录

\item 
自定义(作为单个函数实现)
\end{itemize}

\item 
多线程和单线程版本

\item 
可选的异步模式
\end{itemize}

spdlog可能缺少的一个特性是对Logstash或Fluentd的直接支持。如果希望使用这些聚合器,仍然可以使用文件接收输出配置spdlog,并使用Filebeat或Fluent Bit将文件内容转发到适当的聚合器。

\hspace*{\fill} \\ %插入空行
\noindent
\textbf{统一的日志记录层}

大多数时候,无法控制使用的所有微服务。其中一些将使用同一个日志库,而其他将使用不同的日志库。最重要的是,格式将完全不同,轮换政策也将完全不同。更糟糕的是,仍然希望将操作系统事件与应用程序事件关联起来。这就是统一日志记录层发挥作用的地方。

统一日志记录层的目的是收集来自不同来源的日志。这种统一的日志记录层工具提供许多集成,并理解不同的日志记录格式和传输(如文件、HTTP和TCP)。

统一日志记录层还可以对日志进行过滤。可能需要过滤以满足合规性,匿名化客户个人详细信息,或保护服务的实施细节。

为了便于以后查询日志,统一日志层还可以执行格式之间的转换。即使使用的不同服务以JSON、CSV和Apache格式存储日志,统一的日志层解决方案也能够将它们全部转换为JSON。

统一日志记录层的最后一项任务是将日志转发到下一个目的地。根据系统的复杂性,下一个目的地可能是存储设施或另一个过滤、转换和转发设施。

以下是一些有趣的组件,允许构建统一的日志记录层。

\hspace*{\fill} \\ %插入空行
\noindent
\textbf{Logstash}

Logstash是最受欢迎的统一日志层解决方案之一。目前,它属于Elasticsearch背后的公司Elastic。若听说过ELK堆栈(现在称为弹性堆栈),Logstash是首字母缩写中的“L”。

Logstash是用Ruby编写的,然后移植到JRuby上。不幸的是,这意味着它是相当资源密集型的。由于这个原因,不建议在每台机器上运行Logstash。相反,它主要用作日志转发,在每台机器上部署轻量级的Filebeat,只进行收集。

\hspace*{\fill} \\ %插入空行
\noindent
\textbf{Filebeat}

Filebeat是Beats系列产品的一部分,目标是为Logstash提供一个轻量级的替代方案,可以直接与应用程序一起使用。

通过这种方式,Beats提供了低开销和良好的扩展性,而集中的Logstash安装执行所有繁重的工作,包括转译、过滤和转发。

除了Filebeat, Beats家族的其他产品如下:

\begin{itemize}
\item 
Metricbeat用于性能

\item 
Packetbeat用于网络数据

\item 
Auditbeat用于审计数据

\item 
Heartbeat用于运行时监控
\end{itemize}

\hspace*{\fill} \\ %插入空行
\noindent
\textbf{Fluentd}

Fluentd是Logstash的主要竞争对手,也是一些云提供商选择的工具。

由于使用插件的模块化方法,可以找到用于数据源(如Ruby应用程序、Docker容器、SNMP或MQTT协议)、数据输出(如Elastic Stack、SQL Database、Sentry、Datadog或Slack)以及其他几种过滤器和中间件的插件。

Fluentd应该比Logstash更少的资源,但不是大规模运行的完美解决方案。与使用Fluentd的Filebeat对应的称为Fluent Bit。

\hspace*{\fill} \\ %插入空行
\noindent
\textbf{Fluent Bit}

Fluent Bit用C编写,提供了一种插入Fluentd的更快、更轻的解决方案。作为一个日志处理器和转发器,还具有许多输入和输出的集成功能。

除了日志收集,Fluent Bit还可以监控Linux系统上的CPU和内存指标。可以与Fluentd一起使用,也可以直接转发给Elasticsearch或InfluxDB。

\hspace*{\fill} \\ %插入空行
\noindent
\textbf{Vector}

虽然Logstash和Fluentd是稳定、成熟和经过尝试的解决方案,但在统一日志记录层领域也有新的主张。

其中一个是Vector,目标是在一个工具中处理所有的可观测数据。为了区别于竞争,其关注性能和正确性。这也体现在技术的选择上。Vector使用Rust作为引擎,使用Lua编写脚本(与Logstash和Fluentd使用的定制领域特定语言相反)。

在撰写本书的时候,它还没有达到稳定的1.0版本,所以不应该将其应用于生产。

\hspace*{\fill} \\ %插入空行
\noindent
\textbf{日志聚合}

日志聚合解决了由于数据过多而产生的另一个问题:如何存储和访问日志。虽然统一的日志记录层使日志即使在机器停机的情况下也可用,但日志聚合的任务是快速找到正在寻找的信息。

可以存储、索引和查询大量数据的两种产品是Elasticsearch和Loki。

\hspace*{\fill} \\ %插入空行
\noindent
\textbf{Elasticsearch}

Elasticsearch是自托管日志聚合的最流行的解决方案。这是(前)ELK堆栈中的“E”。它有一个基于Apache Lucene的强大搜索引擎。

作为其领域内的事实上的标准,Elasticsearch有很多集成,并且有来自社区和商业服务的强大支持。一些云提供商将Elasticsearch作为托管服务提供,这使得在应用程序中引入Elasticsearch更加容易。除此之外,制造Elasticsearch的公司Elastic提供了一个托管解决方案,不绑定任何特定的云提供商。

\hspace*{\fill} \\ %插入空行
\noindent
\textbf{Loki}

Loki的目标是解决Elasticsearch的一些缺点。Loki关注的领域是水平可扩展性和高可用性。它是作为云本机解决方案从头构建的。

Loki的设计灵感来自Prometheus和Grafana。这并不奇怪,因为它是由负责Grafana的团队开发的。

虽然Loki应该是一个稳定的解决方案,但它没有Elasticsearch那么受欢迎,这意味着可能会缺少一些集成,文档和社区支持也不会像Elasticsearch一样,但Fluentd和Vector都有支持Loki日志聚合的插件。

\hspace*{\fill} \\ %插入空行
\noindent
\textbf{日志可视化}

想要考虑的日志堆栈的最后一部分是日志可视化。便于对日志进行查询和分析。可以以一种可访问的方式显示数据,以便所有相关方(如运营商、开发人员、QA或业务人员)都可以检查数据。

日志可视化工具允许创建仪表板,更容易读取感兴趣的数据。就能够探索事件,搜索相关性,并从一个简单的用户界面中找到离群的数据。

有两个主要的产品专门用于日志可视化。

\hspace*{\fill} \\ %插入空行
\noindent
\textbf{Kibana}

Kibana是ELK Stack的最后一个元素,在Elasticsearch之上提供了一种更简单的查询语言。尽管可以使用Kibana查询和可视化不同类型的数据,但它主要集中在日志上。

与ELK Stack的其他部分一样,它目前是可视化日志的实际标准。

\hspace*{\fill} \\ %插入空行
\noindent
\textbf{Grafana}

Grafana是另一个数据可视化工具。最近,它还主要关注性能指标的时间序列数据。然而,随着Loki的引入,也可以用于日志。

它的优点是,它在构建时考虑到了可插拔后端,因此很容易切换存储以满足需求。

\subsubsubsection{13.4.2\hspace{0.2cm}监控}

监视是从系统中收集与性能相关的指标的过程。当与警报结合使用时,监视可以了解系统何时按预期运行,以及何时发生事件。

最感兴趣的三种参数类型如下:

\begin{itemize}
\item 
可用性,知道哪些资源已经启动并运行,以及哪些资源已经崩溃或失去响应。

\item 
资源利用率,了解工作负载如何适应系统。

\item 
性能,告诉我们在哪里以及如何提高服务质量。
\end{itemize}

监测的两种模型是推和拉。前者,每个监视对象(一台机器、一个应用程序和一个网络设备)周期性地将数据推送到中心点。在后一种情况下,对象在配置的端点上显示数据,监视代理定期抓取数据。

拉模型使它更容易扩展,多个对象就不会阻塞监视代理连接。相反,多个代理可以随时收集数据,从而更好地利用可用资源。

提供C++客户端库特性的两种监控解决方案是Prometheus和InfluxDB。Prometheus是一个基于拉模型的例子,专注于收集和存储时间序列数据。默认情况下,fluxdb使用推送模型。除了监控,物联网、传感器网络和家庭自动化也很受欢迎。

Prometheus和InfluxDB通常与Grafana一起用于可视化数据和管理仪表板。两者都有内置警报,也可以通过Grafana与外部警报系统集成。

\subsubsubsection{13.4.3\hspace{0.2cm}跟踪}

跟踪提供的信息通常比事件日志的信息级别低。另一个重要的区别是,跟踪存储每个事务的ID,因此很容易可视化整个工作流。这个ID通常称为跟踪ID、事务ID或关联ID。

与事件日志不同,跟踪并不意味着良好的可读性,其由示踪器处理。在实现跟踪时,需要使用与系统的所有可能元素集成的跟踪解决方案:前端应用程序、后端应用程序和数据库。通过这种方式,跟踪有助于查明性能滞后的确切原因。

\hspace*{\fill} \\ %插入空行
\noindent
\textbf{OpenTracing}

分布式跟踪的标准之一是OpenTracing。这个标准是由开源追踪器之一的Jaeger的作者提出的。

除了Jaeger之外,OpenTracing还支持许多不同的跟踪器,并且支持许多不同的编程语言。最重要的包括以下几点:

\begin{itemize}
\item 
Go

\item 
C++

\item 
C\#

\item 
Java

\item 
JavaScript

\item 
Objective-C

\item 
PHP

\item 
Python

\item 
Ruby
\end{itemize}

OpenTracing最重要的特性是它是厂商中立的。在测试应用程序时,将不需要修改整个代码库来切换到不同的跟踪程序,这就可以防止厂商锁定。

\hspace*{\fill} \\ %插入空行
\noindent
\textbf{Jaeger}

Jaeger是一个跟踪器,可以用于各种后端,包括Elasticsearch,Cassandra和Kafka。

它与OpenTracing天生兼容,由于它是一个本地云计算基础毕业的项目,因此它拥有强大的社区支持,这也转化为与其他服务和框架的良好集成。

\hspace*{\fill} \\ %插入空行
\noindent
\textbf{OpenZipkin}

OpenZipkin是Jaeger的主要竞争对手,它在市场上已经有很长时间了。虽然这意味着它是一个更成熟的解决方案,但与Jaeger相比,它的受欢迎程度正在下降。特别是由于OpenZipkin中的C++没有积极地维护,这可能会导致未来的维护问题。

\subsubsubsection{13.4.4\hspace{0.2cm}集成的可观测性解决方案}

如果不想自己构建可观察层,可以考虑一些流行的商业解决方案。它们都以软件即服务的模式运作,不会在这里进行详细的比较,因为在本书写完之后,相应的东西可能会发生巨大的变化。

这些服务如下:

\begin{itemize}
\item 
Datadog

\item 
Splunk

\item 
Honeycomb
\end{itemize}

本节中,了解到了在微服务中实现可观察性。接下来,将继续了解如何连接微服务。













