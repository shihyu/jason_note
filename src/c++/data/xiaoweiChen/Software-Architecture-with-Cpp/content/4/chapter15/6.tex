
本章中要讨论的最后一个主题是GitOps。尽管这个词听起来新颖时髦,但它背后的理念并不完全新颖。它是著名的持续集成/持续部署(CI/CD)模式的扩展。或者可能扩展不是一个好的描述。

CI/CD系统的目标通常是非常灵活,而GitOps则寻求尽可能减少可能的集成数量。两个主要常量是Git和Kubernetes。Git主要用于版本控制、发布管理和环境分离。Kubernetes用作标准化和可编程的部署平台。

这样,CI/CD流水就变得几乎透明了,它与命令式代码处理构建的所有阶段的方法相反。为了允许这样的抽象级别,通常需要了解以下几点:

\begin{itemize}
\item 
将基础设施作为代码,以允许所有必要的自动化部署

\item 
Git工作流,带有特性分支和拉请求或合并请求

\item 
声明式工作流配置,Kubernetes中已经有了
\end{itemize}

\subsubsubsection{15.6.1\hspace{0.2cm}GitOps的原则}

由于GitOps是已建立的CI/CD模式的扩展,因此无法很清楚地区分两者。以下是将这种方法与通用CI/CD区别开来的一些GitOps原则。

\hspace*{\fill} \\ %插入空行
\noindent
\textbf{声明性描述}

经典CI/CD系统与GitOps的主要区别在于操作方式。大多数CI/CD系统都是必要的:它们包含一系列步骤,以便流水成功。

甚至连流水的概念都是必要的,因为它是一个具有条目、一组连接和一个接收器的对象。有些步骤可以并行执行,但是只要存在依赖项,流程就必须停止并等待依赖项步骤完成。

在GitOps中,配置是声明性的。这指的是系统的整个状态——应用程序、配置、监视和指示盘。其会视为代码,具有与常规应用程序代码相同的特性。

\hspace*{\fill} \\ %插入空行
\noindent
\textbf{使用Git版本控制系统的状态}

由于系统的状态是用代码编写的,因此可以从中获得一些好处。更容易的审核、代码审查和版本控制等特性现在不仅适用于应用程序代码。结果是若出现错误,恢复到工作状态需要使用\texttt{git revert}。

可以使用Git的签名提交以及SSH和GPG密钥的强大功能来控制不同的环境。通过添加限制机制来确保,只有符合所需标准的提交才能推送到存储库,还消除了许多由于使用ssh或kubectl手动运行命令而导致的意外错误。

\hspace*{\fill} \\ %插入空行
\noindent
\textbf{可审计}

存储在版本控制系统中的所有内容都变得可审计。在引入新代码之前,要进行代码评审。当注意到一个bug时,可以还原导致bug的更改,或者返回到上一个可工作版本。存储库成为关于整个系统的关键。

当应用于应用程序代码时,可以将功能扩展到审计配置、助手服务、指标、仪表板,甚至部署策略,使其更加强大。不再需要问自己:“好吧,为什么这种配置最终会用于生产呢?”所要做的就是检查Git日志。

\hspace*{\fill} \\ %插入空行
\noindent
\textbf{与已存在的组件进行集成}

大多数CI/CD工具引入专有的配置语法。Jenkins使用Jenkins DSL。每个流行的SaaS解决方案都使用YAML,但是YAML文件彼此不兼容。在必须重写流水的情况下从Travis切换到CircleCI,或从CircleCI切换到GitLab CI。

这有两个缺点。其一是明显的供应商锁定。另一个是需要学习使用给定工具的配置语法。即使大多数流水已经在其他地方定义(shell脚本、Dockerfiles或Kubernetes清单),仍然需要编写一些粘合代码来指示CI/CD工具使用它。

这与GitOps不同,不需要编写显式指令或使用专有语法。相反,可以重用其他通用标准,如Helm或Kustomize。需要学习的东西更少,迁移过程也更舒适。此外,GitOps工具通常与CNCF生态系统中的其他组件集成得很好,因此可以将部署指标存储在Prometheus中,并使用Grafana进行审计。

\hspace*{\fill} \\ %插入空行
\noindent
\textbf{防漂配置}

当给定系统的当前状态与存储库中描述的期望状态不同时,就会发生配置漂移。多种原因导致了配置漂移。

例如一个配置管理工具,它具有基于VM的工作负载,所有虚拟机以相同的状态启动。当CM第一次运行时,将机器运行至所需的状态。但是,如果自动更新代理在默认情况下运行在这些机器上,那么该代理可能会自己更新一些包,而不考虑CM期望的状态。此外,由于网络连接可能很脆弱,一些机器可能会更新到一个软件包的新版本,而另一些则不会。

极端情况下,某个更新的包可能与应用程序所需的固定包不兼容。这种情况将破坏整个CM工作流,使机器处于不可用的状态。

使用GitOps,始终在系统中运行代理,跟踪系统的当前状态和期望状态。如果当前状态与期望的状态突然不同,代理可以修复它或发出有关配置漂移的警报。

防止配置漂移为系统增加了另一层自修复功能。如果在运行Kubernetes,就已经有了自愈功能。每当一个Pod失效时,另一个Pod就会在其位置上重新创建。如果在底层使用可编程的基础设施(例如云提供商或OpenStack本地部署),那么节点还具有自修复功能。通过GitOps,可以自愈工作负载及其配置。

\subsubsubsection{15.6.2\hspace{0.2cm}GitOps的好处}

GitOps的上述特性提供了几个好处,以下是其中一些。

\hspace*{\fill} \\ %插入空行
\noindent
\textbf{提高生产率}

CI/CD流水已经自动化了许多常规任务,通过帮助获得更多部署来缩短交付时间。GitOps添加了一个反馈循环,防止配置漂移,并允许自修复。开发者和其团队可以更快地交付,因为它们很容易恢复,不用担心引入问题。反过来,开发吞吐量增加,则可以更快地、更有信心引入新特性。

\hspace*{\fill} \\ %插入空行
\noindent
\textbf{更好的开发体验}

使用GitOps,开发人员不必担心构建容器或使用kubectl来控制集群。部署新特性只需要使用Git,在大多数环境中已经是一个熟悉的工具。

新员工的上手速度更快,因为不需要为了提高效率而学习很多新工具。GitOps使用标准和一致的组件,因此在操作端引入更改不会影响开发人员。

\hspace*{\fill} \\ %插入空行
\noindent
\textbf{更高的稳定性和可靠性}

使用Git存储系统状态意味着可以访问审计日志,该日志包含所有引入的更改的描述。如果任务跟踪系统与Git集成(这是一个很好的实践),通常可以分辨出哪个业务特性与系统的更改相关。

使用GitOps,不太需要允许手动访问节点或整个集群,这减少了由于运行无效命令而产生意外错误的机会。通过使用Git强大的恢复特性,可以很容易地修复进入系统的那些随机错误。

从严重的灾难(如失去整个控制太)中恢复也容易得多,所需要的是建立一个新的干净的集群,在那里安装一个GitOps操作符,并将其指向带有配置的存储库。过一会儿,就会得到了之前生产系统的副本,这一切都不需要人工干预。

\hspace*{\fill} \\ %插入空行
\noindent
\textbf{安全性提高}

减少了对集群和节点的访问,这意味着提高了安全性。就秘钥丢失或被盗而言,不用太担心了。为了避免了这样一种情况:即使某人不再在团队(或公司)工作,仍然保留对生产环境的访问权。

当涉及到对系统的访问时,单点是由Git存储库处理的。即使恶意的参与者决定在系统中引入后门,所需要的更改也将经过代码审查。当存储库使用带有强验证的GPG签名提交时,模仿其他开发人员的难度更大。

目前为止,主要讨论了开发和操作的好处,GitOps也给企业带来了好处。它提供了系统中的业务可观察性,以前这是很难实现的。

跟踪给定版本中的特性很容易,因为都存储在Git中。由于Git提交了任务跟踪器的链接,业务人员可以获得预览链接,以查看应用程序在不同开发阶段的表现。

还提供了回答以下常见问题的答案:

\begin{itemize}
\item 
生产中运行的是什么?

\item 
上一个版本解决了哪些问题?

\item 
哪些更改可能导致服务降级?
\end{itemize}

所有这些答案的问题可以在一个友好的仪表盘中显示。当然,仪表盘本身也可以存储在Git中。

\subsubsubsection{15.6.3\hspace{0.2cm}GitOps工具}

GitOps是一个新兴且不断增长的领域,有些工具可以认为是稳定和成熟的。以下是一些最受欢迎的工具。

\hspace*{\fill} \\ %插入空行
\noindent
\textbf{FluxCD}

FluxCD是Kubernetes的GitOps运营商,选定的集成提供核心功能。它使用Helm chart和Kustomize来描述资源。

它与Prometheus的集成为部署过程增加了可观察性。为了帮助维护,FluxCD提供了一个CLI。

\hspace*{\fill} \\ %插入空行
\noindent
\textbf{ArgoCD}

与FluxCD不同的是,提供了更广泛的工具选择。如果已经在配置中使用Jsonnet或Ksonnet,这可能会有用。与FluxCD一样,它集成了Prometheus,并具有CLI功能。

撰写本书时,ArgoCD是比FluxCD更受欢迎的解决方案。

\hspace*{\fill} \\ %插入空行
\noindent
\textbf{Jenkins X}

与名字所暗示的相反,Jenkins X与著名的Jenkins CI系统并没有太多的共同之处。由同一家公司支持,但Jenkins和Jenkins X的整个概念完全不同。

其他两个工具都是小而独立的,而Jenkins X是一个复杂的解决方案,有很多集成和更广泛的范围。支持自定义构建任务的触发,使它看起来像一个经典的CI/CD系统和GitOps之间的桥梁。








