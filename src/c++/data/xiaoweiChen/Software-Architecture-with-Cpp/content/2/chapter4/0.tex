
模式有助于处理复杂的东西。在单组件的级别上,可以使用软件模式,比如著名的\textit{设计模式:可复用面向对象软件的基础}所描述的模式。可以向更高的方向发展,了解不同组件之间的架构,了解何时以及如何应用架构模式将大有裨益。

有无数这样的模式,适用于不同的场景。要想了解所有,需要阅读可不止一本书,本书只选择了几个适合于实现架构目标的模式。

这章中,将介绍一些与架构设计相关的概念和谬误。将展示何时使用上述模式,以及如何设计易于部署的高质量组件。

本章将讨论以下内容:

\begin{itemize}
\item 不同的服务模型
\item 避免分布式计算的错误
\item CAP定理的结果和一致性
\item 使系统具有容错性和可用性
\item 集成系统
\item 分级性能
\item 部署系统
\item 管理API
\end{itemize}

本章结束时,将了解如何设计架构中的几个重要的特性,例如容错性、可扩展性和可部署性。在此之前,首先了解分布式架构的两个方面。



























