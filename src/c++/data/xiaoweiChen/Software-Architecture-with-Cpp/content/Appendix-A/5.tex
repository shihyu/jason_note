本书中多次使用了DevOps(和DevSecOps)这个术语。在我们看来,这个话题值得一说。DevOps是一种构建软件产品的方法,打破了传统的基于塔式的开发。

瀑布模型中,团队相互独立地操作单个工作方面。开发团队会编写代码,QA会测试并验证代码,然后是安全性和合规性。最终,运营团队将负责维护工作。团队很少沟通。即使这样,也是一个非常正规的过程。

关于特定领域的专业知识,只对负责工作流中给定部分的团队有效。开发者对QA知之甚少,对运营更是一无所知。虽然这种设置非常方便,但现代开发方式需要的灵活性,要比瀑布模型高的多。

这就是为什么提出了一种新的工作模式,它鼓励在软件产品的不同涉众之间进行更多的协作、更好的交流和大量的知识共享。虽然DevOps指的是把开发人员和运营人员聚集在一起,但其本意是让每个人都更了解他人的工作。

开发人员甚至在编写第一行代码之前,就开始从事QA和安全性工作。操作工程师更熟悉代码库。企业可以轻松地跟踪票据的进度。某些情况下,甚至可以以自助方式进行部署和预览。

DevOps已经成为使用Terraform或Kubernetes等特定工具的代名词。但DevOps绝不等同于使用特定的工具。你的组织可以在不使用Terraform或Kubernetes的情况下遵循DevOps原则,也可以在不使用DevOps的情况下使用Terraform和Kubernetes。

DevOps的原则之一是,鼓励改进产品利益相关者之间的信息流。有了这些,就有可能实现另一个原则:减少对最终产品没有价值的浪费活动。

当构建现代系统时,使用现代方法是值得的。将现有的组织迁移到DevOps可能需要大量的思维转变,所以这并不总可行。当开始一个可以控制的新项目时,这时是可以的。