虽然与云原生设计有关,但无服务器架构本身是一个热门话题。自从引入了AWS Lambda、AWS Fargate、Google Cloud Run和Azure Functions等FaaS或CaaS产品后,就变得非常受欢迎。

无服务主要是像Heroku这样的PaaS产品的发展。它抽象了底层基础设施,这样开发人员就可以专注于应用程序,而不是基础设施的选择。

与旧的PaaS解决方案相比,无服务器的另一个好处是不必为不使用的东西付费。通常不是为给定的服务水平付费,而是为部署的无服务器工作负载的实际执行时间付费。如果只想每天运行一段特定的代码,则不需要每月为底层服务器支付费用。

虽然没有详细介绍关于无服务器的内容,但它很少与C++一起使用。谈到FaaS,目前只有AWS Lambda可能支持C++。由于容器是语言无关的,所以可以使用C++应用程序和函数与CaaS产品(如AWS Fargate、Azure容器实例或Google Cloud Run)一起使用。

若想运行与C++应用程序一起使用的非C++的辅助代码,那么无服务器函数可能是一个不错的选择。维护任务和计划作业非常适合无服务器环境,它们通常不依赖C++二进制的性能或效率。



