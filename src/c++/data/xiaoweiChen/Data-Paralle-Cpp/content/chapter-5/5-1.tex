不处理检测到(抛出)的错误,那么应用程序将终止,并显示发生的错误。这种行为在编写应用程序时不关注错误处理,并且相信错误会以某种方式告知开发人员或用户。当然,我们并不是建议忽略错误处理!开发应用程序应该将错误处理作为体系结构的核心,但是应用程序在开发时通常不觉得。C++的目标是即使没有显式处理错误,那些无法处理错误的代码依旧能够观察到错误。\par

SYCL属于数据并行,原理相同:如果代码中不做任何管理错误的工作,并且检测到错误,程序将会出现异常终止。开发应用程序应该将错误处理作为体系结构的核心,不仅要报告错误,还要从错误状态中进行恢复。\par

\begin{tcolorbox}[colback=red!5!white,colframe=red!75!black]
因为不添加任何错误处理代码,在出现错误时程序会异常终止,所以需要对错误进行处理。
\end{tcolorbox}


























