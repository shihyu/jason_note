还有一些参数可以作为内核的微调参数。这些可以忽略,而不会影响程序的正确性。可以使我们的内核能够真正利用硬件的特性来提高性能。\par

\begin{tcolorbox}[colback=red!5!white,colframe=red!75!black]
在调优缓存(如果存在的话)时,这些查询的结果会有所帮助。
\end{tcolorbox}

\hspace*{\fill} \par %插入空行
\textbf{设备查询}

global\_mem\_cache\_line\_size: 全局内存缓存行的大小(字节)。\par

global\_mem\_cache\_size: 全局内存缓存大小(以字节为单位)。\par

local\_mem\_type: 支持的本地内存类型。可以是info::local\_mem\_
type::local表示专用的本地内存存储,如SRAM或info::local\_mem\_type::global。后一种类型意味着本地内存只是作为全局内存之上的抽象实现的,没有任何性能提高。本地内存类型也可以是info::local\_mem\_type::none,表示不支持本地内存。\par

\hspace*{\fill} \par %插入空行
\textbf{内核查询}

preferred\_work\_group\_size: 在特定设备上执行内核的首选工作组大小。\par

preferred\_work\_group\_size\_multiple: 在特定设备上执行内核的首选工作组大小。\par
















































