为了充分利用CPU上的线程级并行性和SIMD向量级并行性,需要牢记以下目标:\par

\begin{itemize}
	\item 熟悉所有类型的DPC++并行性和相应CPU的底层架构。
	\item 在最匹配硬件资源的线程级别上,不要增加或减少正确的并行度。使用供应商工具,如调试器和分析器,来帮助指导我们的调优工作,以实现这一点。
	\item 首先要注意线程关联性和内存对程序性能的影响。
	\item 使用数据布局、对齐和数据宽度设计数据结构,以便最频繁执行的计算能以SIMD友好的方式访问内存,并具有最大的SIMD并行性。
	\item 要注意平衡掩码和代码分支的成本。
	\item 使用清晰的编程风格,最大限度地减少潜在的内存混叠和副作用。
	\item 注意使用向量类型和接口的可扩展性限制。如果编译器将它们映射到SIMD指令,那么跨多代CPU和来自不同供应商的CPU的固定向量大小,可能无法很好地与SIMD寄存器宽度适配。
\end{itemize}

\newpage