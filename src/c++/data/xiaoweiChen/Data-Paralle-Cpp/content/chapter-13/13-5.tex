本节提供一些调试建议,以减少调试并行程序(特别是针对异构机器的并行程序)的挑战。\par

当应用程序在主机设备上运行时,可以选择调试,这个调试技巧在第2章中描述为方法2。因为设备通常很少包含调试方式,所以调试可以在主机上的完成。主机上运行的另一个好处是,与同步相关的许多错误将消失,包括在主机和设备之间移动内存导致的问题。虽然最终需要调试所有错误,但这可以允许增量调试,可以先解决一些错误。\par

\begin{tcolorbox}[colback=red!5!white,colframe=red!75!black]
\textbf{调试技巧} 功能强大的调试工具可以运行在主机上。
\end{tcolorbox}

在主机上运行所有代码时,工具通常更容易检测和消除并行编程错误,特别是数据竞争和死锁。当在主机和设备的组合上运行时,经常会看到这种并行编程的错误导致程序运行失败。出现这样的问题时,记住使用host-only的方式更有助于调试。SYCL和DPC++可以让我们使用这个选项,访问相关的内存。\par

\begin{tcolorbox}[colback=red!5!white,colframe=red!75!black]
\textbf{调试技巧} 如果程序出现死锁,请检查主机访问器是否正确销毁。
\end{tcolorbox}

开始调试时,可以使用下面的DPC++编译器选项:\par

\begin{itemize}
	\item -g: 在输出中输入调试信息。
	\item -ferror-limit=1: 当使用带有SYCL/DPC++等C++模板库时,可以保持完整性。
	\item -Werror -Wall -Wpedantic: 让编译器强制执行编码,以避免生成运行时可调试的错误代码。
\end{itemize}

不需要仅仅为了使用DPC++而去修复迂腐的警告,所以可以选择不使用-Wpedantic。\par

在运行时让代码及时编译时,我们可以检查代码。这高度依赖于编译器,因此查看编译器手册以获得建议是一个好主意。\par

\hspace*{\fill} \par %插入空行
\textbf{调试内核代码}

调试内核代码时,首先要在主机设备上运行(如第2章所述)。第2章中的设备选择器,可以很容易地修改运行时选项或编译时选项,以便在调试时将工作重定向到主机。\par

调试内核代码时,SYCL定义了一个C++风格的输出流,可以在内核中使用(图13-4)。DPC++还提供了一个C风格printf的实验性实现,功能有一些限制。更多的细节可以参考在在线oneAPI DPC++手册。\par

\hspace*{\fill} \par %插入空行
图13-4 sycl::stream
\begin{lstlisting}[caption={}]
Q.submit([&](handler &h){
	stream out(1024, 256, h);
	h.parallel_for(range{8}, [=](id<1> idx){
		out << "Testing my sycl stream (this is work-item ID:" << idx << ")\n";
	});
});
\end{lstlisting}

调试内核代码时,可以在parallel\_for之前或在parallel\_for内部设置断点。即使是在执行下一个操作之后,放置在parallel\_for的断点也可以多次触发。C++调试方式适用于许多模板展开,其中模板调用上的断点将在编译器展开时,转换为一组复杂的断点。可能有方法可以缓解这种情况,但这里的关键是,可以通过在parallel\_for上设置断点来避免混淆。\par

\hspace*{\fill} \par %插入空行
\textbf{调试运行时的错误}

发生运行时错误时,要么是在处理编译器/运行时错误,要么是无意中编写了废代码。深入研究这些bug可能有点令人生畏,但可以让更好地了解导致特定问题的原因。可能会产生一些额外的信息,指导我们避免这个问题,或者向编译器团队提交错误报告。不管怎样,了解一些能够提供帮助的工具很重要。\par

运行时失败的程序输出如下所示:\par

\begin{tcolorbox}[colback=white,colframe=black]
origin>: error: Invalid record (Producer: 'LLVM9.0.0' Reader: 'LLVM 9.0.0')\\
terminate called after throwing an instance of 'cl::sycl::compile\_program\_error'
\end{tcolorbox}

看到这里的throw,可以知道主程序可以捕捉这个错误。虽然这不能解决问题,但确实意味着运行时编译器失败不需要中止应用程序。第5章会深入探讨这个主题。\par

当看到运行时失败并难以快速调试时,不妨尝试使用提前编译进行重新构建。如果所使用的设备具有提前编译选项,这可能是很容易的事情,可能会产生更容易理解的诊断信息。如果错误是在编译时,而不是在JIT或运行时,通常会在编译器的错误消息中找到有用的信息,而不是在JIT或运行时看到的少量错误信息。对于特定选项,请查看在线oneAPI DPC++手册,了解提前编译。\par

SYCL程序运行在OpenCL运行时上,并使用OpenCL后端时,可以使用OpenCL拦截:\\github.com/intel/opencl-intercept-layer运行程序。可以检查、记录和修改应用程序(或更高级别运行时)生成OpenCL命令的工具。它支持很多控件,但是最好的初始设置是ErrorLogging、BuildLogging和CallLogging(会生成很多输出)。DumpProgramSPIRV可以提供有用的转储。OpenCL拦截是一个独立的程序,不是OpenCL实现的一部分,所以可以与SYCL编译器一起工作。\par

对于Intel GPU在Linux系统上的编译器问题,可以从Intel图形编译器转储中间编译器输出。通过将环境变量\\IGC\_ShaderDumpEnable设置为1(用于某些输出)或将环境变量IGC\_ShaderDumpEnableAll设置为1(用于大量输出)来做到这一点,转储的输出在/tmp/IntelIGC目录下。这种技术可能不适用于所有图形驱动程序的构建,但值得一试。\par

图13-5列出了这些变量以及编译器或运行时支持的一些附加环境变量(在Windows和Linux上支持),以帮助进行高级调试。这些是DPC++实现相关的高级调试选项,用于检查和控制编译模型。\par

这些选项在本书中没有详细描述,这里提到是为了在需要时打开高级调试的通道。这些选项可以让我们深入了解如何解决问题。使用这些选项是为了对编译器本身进行调试。因此,选项更多地与编译器开发人员联系在一起,而不是编译器用户。一些高级用户认为这些选项很有用,因此在这里提及。更多信息可以参考GitHub for DPC++在llvm/sycl/doc/EnvironmentVariables.md下关于所有环境变量的文档。\par

\begin{tcolorbox}[colback=red!5!white,colframe=red!75!black]
\textbf{调试技巧} 当其他选项都用尽,且无法解决需要调试运行时问题时,可能寻找为提供相关原因的工具。
\end{tcolorbox}

\hspace*{\fill} \par %插入空行
图13-5 DPC++的高级调试选项
\begin{table}[H]
	\begin{tabular}{|l|l|l|}
		\hline
		环境变量         & 值                                                                                                                                                                                                                              & 描述                                                                                                                                                                                                                                  \\ \hline
		SYCL\_PI\_TRACE               & \begin{tabular}[c]{@{}l@{}}1(basic),\\ 2(advanced),\\ -1(all)\end{tabular}                                                                                                                                                         & \begin{tabular}[c]{@{}l@{}}运行时:\\值为1启用运行时插件接口(PI)\\跟踪插件,并寻找设备;\\值为2表示跟踪所有PI呼叫;\\值为-1释放所有级别的跟踪。\end{tabular} \\ \hline
		SYCL\_PRINT\_EXECUTION\_GRAPH & \begin{tabular}[c]{@{}l@{}}always(\\或者通过指定\\只转储选定文件;\\before\_addcg、\\after\_addgc、\\before\_addcopyback、\\after\_addcopyback、\\before\_addhostacc、\\或after\_addhostacc)\end{tabular} & \begin{tabular}[c]{@{}l@{}}运行时:\\创建跟踪执行图的文本文件\\(DOT扩展名),这样浏览运行时\\发生的跟踪会更容易。\end{tabular}                                                \\ \hline
		CL\_CONFIG\_USE\_VBECTORIZER  & true 或 false                                                                                                                                                                                                                      & \begin{tabular}[c]{@{}l@{}}运行时:\\请求CPU编译器启用\\或禁用向量。\end{tabular}                                                                                                                                \\ \hline
		CL\_CONFIG\_CPU\_TARGET\_ARCH & skx, core-avx2                                                                                                                                                                                                                     & \begin{tabular}[c]{@{}l@{}}运行时:\\要求Intel CPU编译器支持\\Inter Advanced Vector Extensions\\(AVX512和AVX2)的代码。\end{tabular}                                                      \\ \hline
		CL\_CONFIG\_DUMP\_ASM         & true 或 false                                                                                                                                                                                                                      & \begin{tabular}[c]{@{}l@{}}运行时:\\通过英特尔CPU编译器转\\储出CPU汇编代码。\end{tabular}                                                                                                                                      \\ \hline
		IGC\_ShaderDumpEnable         & 0 或 1                                                                                                                                                                                                                             & \begin{tabular}[c]{@{}l@{}}只限定于Linux。\\运行时:\\通过Intel图形编译器(JIT)\\转储一些信息。\end{tabular}                                                                                                                  \\ \hline
		IGC\_ShaderDumpEnableALL      & 0 或 1                                                                                                                                                                                                                             & \begin{tabular}[c]{@{}l@{}}只限定于Linux。\\运行时:\\通过Intel图形编译器(JIT)\\转储大量信息。\end{tabular}                                                                                                               \\ \hline
		SYCL\_DUMP\_IMAGES            & true 或 false                                                                                                                                                                                                                      & \begin{tabular}[c]{@{}l@{}}编译时:\\通过SYCL\_DUMP\_IMAGES=1\\请求编译器转储出SPV文件,\\其中包含在执行期间传递给\\JIT编译器的中间码。\end{tabular}                             \\ \hline
		SYCL\_USE\_KERNEL\_SPV        & \textless{}device binary\textgreater{}                                                                                                                                                                                             & \begin{tabular}[c]{@{}l@{}}运行时:\\从指定文件加载设备镜像。\\如果运行时无法读取该文件,则\\抛出cl::sycl::runtime\_error异常。\end{tabular}                                                         \\ \hline
	\end{tabular}
\end{table}




















