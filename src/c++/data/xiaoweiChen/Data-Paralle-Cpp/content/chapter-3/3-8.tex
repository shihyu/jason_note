为数据管理提供的另一个方法的是缓冲区对象。缓冲区是一种数据抽象,表示给定C++类型的一个或多个对象。缓冲区对象可以是标量数据类型(如int、float或double)、向量数据类型(第11章)或用户定义的类或结构。缓冲区中的数据结构必须可复制,这意味着可以安全地逐个字节地复制对象,而不需要调用复制构造函数。\par

虽然缓冲区本身是单个实例,但缓冲区封装的C++类型可以是包含多个对象的数组。缓冲区代表的是数据对象,而不是特定的内存地址,所以不能像常规C++数组那样直接访问。实际上,出于性能原因,缓冲区对象可能映射到多个不同设备上的多个不同内存位置,甚至是同一个设备上的多个内存位置。而我们只能使用访问器,来读取和写入缓冲区。\par

第7章有更多关于缓冲区的描述。\par

\hspace*{\fill} \par %插入空行
\textbf{创建缓冲区}

可以通过多种方式创建缓冲区。最简单的方法是构造新的缓冲区,指定缓冲区的大小。然而,以这种方式创建的缓冲区并不会初始化数据,试图从缓冲区中读取数据之前,必须通过其他方式初始化缓冲区。\par

还可以以主机上的现有数据创建缓冲区。通过调用构造函数来实现,构造函数接受指向主机的内存指针、一组输入迭代器或具有某些属性的容器。构造缓冲区的过程中,数据从现有的主机内存复制到缓冲区对象的主机内存中。如果在OpenCL中使用SYCL互操作特性,也可以使用现有的cl\_mem实例创建缓冲区。\par

\hspace*{\fill} \par %插入空行
\textbf{访问缓存}

主机和设备不能直接访问缓冲区(除非通过这里没有描述的高级和不经常使用的机制),而必须创建访问器来读取和写入缓冲区。访问器向运行时提供如何使用缓冲区中的数据信息,从而允许运行时正确地进行数据移动。\par

\hspace*{\fill} \par %插入空行
图3-6 缓冲区和访问器
\begin{lstlisting}[caption={}]
#include <CL/sycl.hpp>
#include <array>
using namespace sycl;
constexpr int N = 42;

int main() {
	std::array<int,N> my_data;
	for (int i = 0; i < N; i++)
		my_data[i] = 0;
	
	{
		queue q;
		buffer my_buffer(my_data);
		
		q.submit([&](handler &h) {
			// create an accessor to update
			// the buffer on the device
			accessor my_accessor(my_buffer, h);
			
			h.parallel_for(N, [=](id<1> i) {
				my_accessor[i]++;
			});
		});
	
		// create host accessor
		host_accessor host_accessor(my_buffer);
		for (int i = 0; i < N; i++) {
			// access myBuffer on host
			std::cout << host_accessor[i] << " ";
		}
		std::cout << "\n";
	}

	// myData is updated when myBuffer is
	// destroyed upon exiting scope
	for (int i = 0; i < N; i++) {
		std::cout << my_data[i] << " ";
	}
	std::cout << "\n";
}
\end{lstlisting}

\hspace*{\fill} \par %插入空行
图3-7 缓冲区的访问模式
\begin{table}[H]
	\begin{tabular}{|l|l|}
		\hline
		\textbf{访问模式} & \textbf{描述}                                \\ \hline
		\textbf{read}        & 只能读                                   \\ \hline
		\textbf{write}       & 只写访问(以前的内容不丢弃) \\ \hline
		\textbf{read\_write} & 可读写              \\ \hline
	\end{tabular}
\end{table}

\hspace*{\fill} \par %插入空行
\textbf{访问模式}

创建访问器时,可以通知运行时来提供更多的优化信息,可以通过指定访问模式来实现这一点。图3-7中的Access::mode enum中定义了访问模式。图3-6所示的代码中,访问器myAccessor以默认的访问模式创建,access::mode::read\_write,这让运行时知道我们打算通过myAccessor对缓冲区进行读写操作。访问模式是运行时优化隐式数据移动的方式,例如:使用access::mode::read会在内核开始执行之前,需要数据在设备上可用。如果内核只通过访问器读取数据,在内核完成后就不需要将数据复制回主机。同样,access::mode::write让运行时知道我们将修改缓冲区的内容,并且需要在计算结束后将结果复制回来。\par

创建适当模式的访问器,可以向运行时提供更多关于如何在程序中使用数据的信息。运行时使用访问器对数据的使用进行排序,也可以使用这些数据来优化内核调度和数据移动。第7章中,将继续讨论访问模式和优化标记。\par










