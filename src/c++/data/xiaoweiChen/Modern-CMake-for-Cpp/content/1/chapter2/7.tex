本章开启了使用CMake进行实际编程的大门——现在能够编写出色的、信息丰富的注释和调用内置指令,并且了解如何正确地为它们提供各种参数。仅这些知识就可以帮助您理解在其他项目中可能见过的CMake列表文件的不寻常语法。

接下来,介绍了CMake中的变量——如何引用、设置和取消设置普通变量、缓存变量和环境变量。我们深入研究了目录和函数作用域的工作方式,并讨论了与嵌套作用域相关的问题(及其解决方法)。

还讨论了列表和控制结构,讨论了条件的语法、逻辑操作、无引号求值,以及字符串和变量。学习了如何比较值,进行简单的检查,以及检查系统中文件的状态。这样就编写条件块和while循环。当我们讨论循环的时候,也掌握了foreach循环的语法。

我相信,知道如何用宏和函数语句定义自己的命令,将有助于您以更过程化的风格编写更清晰的代码。我们还分享了一些关于如何更好地构造我们的代码,并提出更可读的名称的想法。

最后,正式介绍了message(),及其多个日志级别。还研究了如何划分和包含列表文件,并发现了一些其他有用的命令。了解了这些,就可以进入下一章了,在CMake中编写我们的第一个项目。