
CMake中的变量是一个复杂的主题。有三类变量——普通变量、缓存变量和环境变量——而且它们还在不同的作用域中,并对一个作用域如何影响另一个作用域具有特定的规则。通常情况下,对所有这些规则的不理解会成为bug和头痛的根源。我建议仔细学习这一节,确保在继续学习之前理解了这些概念。

先从CMake中关于变量的关键点开始:

\begin{itemize}
\item 
变量名区分大小写,可以包含任何字符。

\item 
变量都在内部作为字符串存储,有些智力可以解释为其他数据类型的值(甚至是列表!)

\item 
基本的变量操作指令是set()和unset(),但还有其他指令可以影响变量,如string()和list()。
\end{itemize}

要设置变量,只需使用set(),提供名称和值:

\begin{lstlisting}[style=styleCMake]
# chapter02/02-variables/set.cmake

set(MyString1 "Text1")
set([[My String2]] "Text2")
set("My String 3" "Text3")
message(${MyString1})
message(${My\ String2})
message(${My\ String\ 3})
\end{lstlisting}

使用括号和引号参数允许在变量名中包含空格。但当以后引用时,必须使用反斜杠来转义空格(\verb|\|)。因此,建议在变量名中只使用字母数字字符、减号(-)和下划线(\_)。

还要避免使用任何字符开头的保留名称(上、下或混合大小写):CMAKE\_、\_CMAKE\_或下划线(\_),后面跟着CMake指令的名称。

\begin{tcolorbox}[colback=blue!5!white,colframe=blue!75!black,title=Note]
set()指令接受纯文本变量名作为第一个参数,但message()使用包装在\$\{\}语法中的变量引用。若向set()提供包装在\$\{\}语法中的变量,会发生什么?要回答这个问题,需要更好地理解引用变量。
\end{tcolorbox}

要取消变量的设置,可以使用:unset(MyString1)。

\subsubsubsection{2.3.1\hspace{0.2cm}引用变量}

已经在指令参数部分提到了引用,这是为引号和非引号参数进行计算的。创建对已定义变量的引用,需要使用\$\{\}语法:message(\$\{MyString1\})。

求值时CMake将遍历作用域堆栈,并将\$\{MyString1\}替换为值,若没有找到变量则替换为空字符串(CMake不会产生错误)。这个过程称为变量求值、展开或插值。

这样的插值是由内而外的方式执行的:

\begin{itemize}
\item 
若遇到以下引用——\$\{MyOuter\$\{MyInner\}\}——CMake将首先尝试求值MyInner,而不是搜索名为MyOuter\$\{MyInner\}的变量。

\item 
若MyInner变量成功展开,CMake将重复展开过程,直到不能再展开为止。
\end{itemize}

考虑包含以下变量的例子:

\begin{itemize}
\item 
MyInner的值是Hello

\item 
MyOuter的值是\$\{My
\end{itemize}

若使用message("\$\{MyOuter\}Inner\} World"),输出将是Hello World,这是因为\$\{My替换了\$\{MyOuter\},当与顶层值Inner\}结合时,会创建另一个变量引用——\$\{MyInner\}。

CMake执行这个扩展,只有这样它才会将结果值作为参数传递给命令。这就是为什么使用set(\$\{MyInner\} "Hi")时,实际上不会更改MyInner变量,而是更改Hello变量。CMake将\$\{MyInner\}展开为Hello,并将该字符串作为第一个参数传递给set(),以及一个新值Hi。很多时候,这并不是我们想要的。

当涉及到变量类别时,变量引用的工作方式有点奇怪。以下是通常情况适用的方式:

\begin{itemize}
\item 
\$\{\}用于引用普通变量或缓存变量。

\item 
\$ENV\{\}用于引用环境变量。

\item 
\$CACHE\{\}用于引用缓存变量。
\end{itemize}

没错,使用\$\{\}以从一个类别或另一个类别获得值,后续章节会有详解。先介绍一些其他类型的变量,以便了解它们是什么。

\begin{tcolorbox}[colback=blue!5!white,colframe=blue!75!black,title=Note]
可以通过命令行在-{}-标记之后将参数传递给脚本。值将存储在CMAKE\_ARGV<n>变量中,传递的参数的计数将存储在CMAKE\_ARGC变量中。
\end{tcolorbox}

\subsubsubsection{2.3.2\hspace{0.2cm}环境变量}

这是最简单的一种变量。CMake生成环境中用于启动CMake进程的变量的副本,并使它们在单一的全局作用域中可用。要引用这些变量,使用\$ENV\{<name>\}。

CMake还允许设置变量(set())和取消设置变量(unset()),但更改只会在运行的CMake进程中对本地副本进行,而不会对实际的系统环境进行更改;此外,这些更改对于构建或测试的后续运行不可见。

要修改或创建一个变量,使用set(ENV\{<variable>\} <value>)指令:

\begin{lstlisting}[style=styleCMake]
set(ENV{CXX} "clang++")
\end{lstlisting}

要清除环境变量,使用unset(ENV\{<variable>\}):

\begin{lstlisting}[style=styleCMake]
unset(ENV{VERBOSE})
\end{lstlisting}

注意,有几个环境变量会影响CMake行为的不同方面。CXX变量就是其中之一——指定编译C++文件时,使用什么可执行文件。我们将讨论其他环境变量,因为它们与本书相关。完整的列表可以在文档中找到:

\url{https://cmake.org/cmake/help/latest/manual/cmake-envvariables.7.html}

若使用ENV变量作为指令的参数,这些值将在生成构建系统期间插入,并且会将其嵌入到构建树中。

例子:

\begin{lstlisting}[style=styleCMake]
# chapter02/03-environment/CMakeLists.txt

cmake_minimum_required(VERSION 3.20.0)
project(Environment)
message("generated with " $ENV{myenv})
add_custom_target(EchoEnv ALL COMMAND echo "myenv in build
	is" $ENV{myenv})
\end{lstlisting}

前面的示例有两个步骤:将在配置期间打印myenv环境变量,并通过add\_custom\_target()添加一个构建阶段,该阶段作为构建过程的一部分响应相同的变量。可以测试bash脚本会发生什么,在配置阶段使用一个值,在构建阶段使用另一个值:

\begin{lstlisting}[style=stylePython]
# chapter02/03-environment/build.sh
	
#!/bin/bash
export myenv=first
echo myenv is now $myenv
cmake -B build .
cd build
export myenv=second
echo myenv is now $myenv
cmake --build .
\end{lstlisting}

运行上面的代码,可以清楚地看到在配置过程中,设置的值会保留在生成的构建系统中:

\begin{tcblisting}{commandshell={}}
$ ./build.sh | grep -v "\-\-"
myenv is now first
generated with first
myenv is now second
Scanning dependencies of target EchoEnv
myenv in build is first
Built target EchoEnv
\end{tcblisting}

\subsubsubsection{2.3.3\hspace{0.2cm}缓存变量}

它们是存储在构建树中的CMakeCache.txt文件中的变量,包含在项目配置阶段收集的信息,包括从系统(到编译器、链接器、工具和其他的路径)和通过GUI从用户收集的信息。缓存变量在脚本中不可用(因为没有CMakeCache.txt文件)——其只存在于项目中。

缓存变量可以通过\$CACHE\{<name>\}语法来引用。

要设置一个缓存变量,使用set():

\begin{lstlisting}[style=styleCMake]
set(<variable> <value> CACHE <type> <docstring> [FORCE]) 
\end{lstlisting}

这里有一些必需参数(与用于普通变量的set()指令相比),还引入了第一个关键字:CACHE和FORCE。

将CACHE指定为set()参数可以更改在配置阶段提供的内容,要求提供变量<type>和<docstring>值。这是因为用户可以配置这些变量,GUI需要知道如何显示它们。通常接受以下类型:

\begin{itemize}
\item 
BOOL: 一个布尔的开/关值。GUI将显示一个复选框。

\item 
FILEPATH: 磁盘上文件的路径。GUI将打开一个文件对话框。

\item 
STRING: 一行字符串。GUI提供了一个要填充的文本字段,可以通过下拉控件选择替换。

\begin{lstlisting}[style=styleCMake]
set_property(CACHE <variable> STRINGS <values>)
\end{lstlisting}

\item 
INTERNAL: 一行字符串。GUI跳过内部条目。内部条目可以用于跨运行持久化存储变量。使用此类型会隐式添加FORCE关键字。
\end{itemize}

<doctring>值只是一个标签,将由GUI显示在字段旁边,以便向用户提供关于该设置的更多详细信息。即使是INTERNAL类型也需要它。

设置缓存变量遵循与环境变量相同的规则——仅在当前执行CMake时覆盖值:

\begin{lstlisting}[style=styleCMake]
set(FOO "BAR" CACHE STRING "interesting value")
\end{lstlisting}

若变量存在于缓存中,上述调用不会产生永久影响。然而,若该值在缓存中不存在或指定了可选的FORCE参数,则该值将固定:

\begin{lstlisting}[style=styleCMake]
set(FOO "BAR" CACHE STRING "interesting value" FORCE)
\end{lstlisting}

设置缓存变量有一些不明显的含义,具有相同名称的普通变量都将删除。我们将在下一节中找出原因。

需要提醒的是,缓存变量也可以在命令行端进行管理。

\subsubsubsection{2.3.4\hspace{0.2cm}如何正确使用变量作用域}

变量作用域可能是CMake语言整个概念中最难的部分。这可能是因为我们太习惯于在更高级语言中做事情的方式。CMake没有这些机制,所以它用自己的方式处理这个问题。

需要说明的是,变量作用域作为一个通用概念是为了分离不同的抽象层,以便在调用用户定义的函数时,该函数中设置的变量是局部的。这些局部变量不会影响全局作用域,即使局部变量的名称与全局变量的名称完全相同。若显式需要,函数也应该具有对全局变量的读/写访问权。这种变量(或作用域)分离必须在多个层面上工作——当一个函数调用另一个函数时,分离规则同样适用。

CMake有两个作用域:

\begin{itemize}
\item 
函数作用域:用于执行用function()定义的自定义函数

\item 
目录作用域:当从add\_subdirectory()指令执行嵌套目录中的CMakeLists.txt文件时
\end{itemize}

将在本书后面介绍上述指令。首先,需要知道变量作用域的概念如何实现。当创建嵌套作用域时,CMake只需用来自当前作用域的所有变量的副本填充。后续命令将影响这些副本。但若完成了嵌套作用域的执行,所有的副本都会删除,而原始的父作用域将恢复。

考虑以下场景:

\begin{enumerate}
\item 
父作用域将VAR变量设置为ONE。

\item 
开始嵌套作用域,并将VAR输出到控制台。

\item 
VAR变量设置为TWO,并且VAR输出到控制台。

\item 
嵌套的作用域结束,VAR再输出到控制台。
\end{enumerate}

控制台的输出应该是:ONE, TWO, ONE。这是因为在嵌套作用域结束后,复制的VAR变量将丢弃。

CMake中作用域的概念如何工作,若在嵌套作用域中执行时取消设置(unset())父作用域中创建的变量,只会在嵌套作用域中消失。当嵌套作用域完成时,变量将恢复到以前的值。

这就引出了变量引用的行为和\$\{\}语法。每当尝试访问普通变量时,CMake将从当前作用域搜索变量,若定义了具有这样一个名称的变量,它将返回其值,但当CMake找不到具有该名称的变量时(例如,若它不存在或未设置(unset())),将搜索缓存变量,若找到匹配则从那里返回一个值。

若有一个使用unset()的嵌套作用域,这可能是个问题。根据引用该变量的位置——内部还是外部作用域——将访问缓存或原始值。

若真的需要更改调用(父)作用域中的变量,该怎么办呢?

CMake有一个PARENT\_SCOPE标志,可以在set()和unset()指令的末尾添加:

\begin{lstlisting}[style=styleCMake]
set(MyVariable "New Value" PARENT_SCOPE)
unset(MyVariable PARENT_SCOPE)
\end{lstlisting}

这种解决方法有一定的局限性,因为其不允许访问超过一个级别的变量。另一件值得注意的事情是,使用PARENT\_SCOPE不会改变当前作用域中的变量。

来看看变量作用域在实践中是如何工作的:

\begin{lstlisting}[style=styleCMake]
# chapter02/04-scope/CMakeLists.txt

function(Inner)
	message(" > Inner: ${V}")
	set(V 3)
	message(" < Inner: ${V}")
endfunction()

function(Outer)
	message(" > Outer: ${V}")
	set(V 2)
	Inner()
	message(" < Outer: ${V}")
endfunction()

set(V 1)
message("> Global: ${V}")
Outer()
message("< Global: ${V}")
\end{lstlisting}

将全局变量V设为1,然后调用Outer函数;然后将V设置为2并调用Inner函数,然后将V设置为3。每一步后面,都将变量打印到控制台:

\begin{tcblisting}{commandshell={}}
> Global: 1
  > Outer: 1
    > Inner: 2
    < Inner: 3
  < Outer: 2
< Global: 1
\end{tcblisting}

深入函数时,变量值会复制到嵌套作用域,但当退出作用域时,原始值将恢复。

若改变Inner函数的set()来操作父作用域:set(V 3 PARENT\_SCOPE),输出会是什么?

\begin{tcblisting}{commandshell={}}
> Global: 1
  > Outer: 1
    > Inner: 2
    < Inner: 2
  < Outer: 3
< Global: 1
\end{tcblisting}

我们影响了外部函数的作用域,但没有影响内部函数的作用域或全局作用域!

CMake文档还提到CMake脚本在一个目录范围内绑定变量(这有点多余,因为唯一有效创建目录范围的命令add\_subdirectory()在脚本中是不允许的)。

由于所有变量都存储为字符串,CMake必须采用更有效的方法来处理更复杂的数据结构,如列表。



























