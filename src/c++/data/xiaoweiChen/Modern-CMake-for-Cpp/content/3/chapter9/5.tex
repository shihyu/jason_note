“你将花更多的时间研究代码而不是创建代码——因此,应该优化阅读,而不是仅仅去编写代码。”

这句话出现在不止一本讨论干净代码实践的书中,像咒语一样反复出现。这也不奇怪,因为这是非常正确的,正如许多软件开发人员在实践中所测试的那样——以至于对于诸如空格的数量、换行符和\#import语句的顺序等非常微小的事情的规则都已经成了规范。这样做不是因为小气,而是为了节省成本。通过遵循本章概述的实践,不需要担心手工正确格式化代码。作为构建的副作用,将自动格式化——我们无论如何都必须执行的步骤,以检查代码是否正常工作。通过引入ClangFormat,还可以确保其看起来正确。

当然,我们想要的不仅仅是简单的空格校正,代码必须符合几十个其他的小规则。这是通过添加Clang-Tidy并配置来实现的,以适配所选择的编码风格。详细讨论了这个静态检查器,也提到了其他选项:Cpplint、Cppcheck、Include-what-you-use和Link-what-you-use。由于静态链接器相对较快,可以用很少精力将它们添加到构建中,而且这通常是物超所值的。

最后,研究了Valgrind实用程序,特别是Memcheck,其允许调试与内存管理相关的问题:错误的读取、写入、释放等。这是一个非常方便的工具,可以节省数小时的手工检查,并防止错误潜入生产环境。不过,其执行速度可能有点慢,这就是为什么需要创建了单独的目标,以便在提交代码之前显式地运行它。还学习了如何使用Memcheck-Cover(HTML报告生成器),以更友好的形式表示Valgrind的输出。这在不支持运行IDE的环境中非常有用(例如CI流水)。

当然,我们并不局限于这些工具,还有很多免费和开源项目,以及相应的商业产品。这里,只是对这门学科的介绍而已,一定要探索什么是适合你自己的。下一章中,将会更深入地研究如何生成文档。
