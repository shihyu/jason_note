
理解目标对于编写干净、现代的CMake项目至关重要。本章中,不仅讨论了目标的构成,以及目标之间如何相互依赖,还讨论了如何使用Graphviz模块在图表中显示这些信息。

有了这些基础,就能够了解目标的关键特征——属性(各种属性)。不仅使用了一些命令在目标上设置常规属性,还解决了传递使用需求或传播属性的问题。这是一个很难解决的问题,因为不仅需要控制传播哪些属性,还需要控制如何将它们可靠地传播到选定的进一步目标。此外们还发现了当这些传播的属性来自多个源时,如何保证兼容。

然后简要地讨论了伪目标——导入目标、别名目标和接口库。其在项目中都会派上用场,特别是当知道如何将它们与传播的属性连接起来时。然后,讨论了生成的构建目标,以及如何在配置阶段对操作产生直接影响。之后,了解了自定义命令(如何生成用于其他目标的文件、编译、翻译等)和钩子函数——在构建目标时执行额外的步骤。

本章的最后一部分专门讨论了生成器表达式的概念,或者简称为genex。我们解释了它的语法、嵌套以及它的条件表达式如何工作。然后,进行了两种类型的求值——对布尔值和对字符串值。每一种都有自己的一套表达,对此进行了详细的探讨和评论。此外,还介绍了一些使用示例,并阐明了其在实践中的工作原理。

有了这样一个坚实的基础,就为下一个主题做好了准备——将C++源代码编译为可执行程序和库。

\subsubsubsection{4.5.1\hspace{0.2cm}扩展阅读}

有关更多信息,请浏览以下连接:

\begin{itemize}
\item 
Graphviz模块文档:

\url{https://gitlab.kitware.com/cmake/community/-/wikis/doc/cmake/Graphviz}

\url{https://cmake.org/cmake/help/latest/module/CMakeGraphVizOptions.html}
	
\item 
Graphviz软件:

\url{https://graphviz.org}
	
\item 
CMake目标属性:

\url{https://cmake.org/cmake/help/latest/manual/cmakeproperties.7.html\#properties-on-targets}

\item 
传递性使用的要求:
 
\url{https://cmake.org/cmake/help/latest/manual/cmakebuildsystem.7.html\#transitive-usage-requirements}
\end{itemize}










