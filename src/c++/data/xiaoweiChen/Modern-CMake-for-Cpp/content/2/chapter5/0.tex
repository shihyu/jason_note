简单的编译通常由工具链的默认配置处理,或者由IDE直接提供。但在专业环境中,业务需求通常需要更高级的东西。这可能是对更高的性能、更小的二进制文件、更强的可移植性、需要支持测试或调试等。以简单的方式管理所有这些,很快就让工程会变得复杂而混乱(特别是需要支持多个平台时)。

C++中对编译过程的解释往往不够充分(像虚基类这样的主题似乎更有趣)。这一章中,将了解编译是如何工作的,其内部阶段是什么,以及如何影响二进制的输出。

之后,将关注先决条件——将讨论可以使用什么命令来调整编译,如何要求编译器提供特定的特性,以及如何向编译器提供它必须处理的输入文件。

然后,关注编译的第一个阶段——预处理器。我们将为包含的头文件提供路径,并将研究如何使用预处理器定义插入来自CMake和环境的变量。我们将介绍一些有趣的用例,并学习如何将CMake变量批量公开给C++代码。

之后讨论优化器,以及不同的标志如何影响性能。可能会痛苦地意识到优化的成本——调试损坏的代码有多么困难。

最后,将解释如何管理编译过程,使用预编译头文件和统一构建减少编译时间,为发现错误做准备,调试构建,并在最终的二进制文件中存储调试信息。

本章中,我们将讨论以下主题:

\begin{itemize}
\item 
编译基础

\item 
预处理配置

\item 
配置优化器

\item 
编译过程
\end{itemize}