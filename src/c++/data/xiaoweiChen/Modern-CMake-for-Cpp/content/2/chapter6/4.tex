
Phil Karlton说得很对:

\texttt{在计算机科学中有两件难事:缓存失效和命名。}

起名字很难有几个原因——必须精确、简单、简短,同时还要有表现力。这使它们变得有意义,并允许程序员理解原始实现背后的概念。C++和许多其他语言强加了另外一个要求——许多名称必须唯一。这体现在几个不同的方面,例如:开发者需要遵循ODR。

在单个翻译单元(单个.cpp文件)的范围内,即使多次声明相同的名称(变量、函数、类类型、枚举、概念或模板),也需要只定义一次。

对于在代码和非内联函数中有效使用的所有变量,该规则扩展到整个程序的作用域。考虑下面的例子:

\begin{lstlisting}[style=styleCXX]
// chapter06/02-odr-fail/shared.h

int i;
\end{lstlisting}

\begin{lstlisting}[style=styleCXX]
// chapter06/02-odr-fail/one.cpp

#include <iostream>
#include "shared.h"
int main() {
	std::cout << i << std::endl;
}
\end{lstlisting}

\begin{lstlisting}[style=styleCXX]
// chapter06/02-odr-fail/two.cpp
	
#include "shared.h"
\end{lstlisting}

\begin{lstlisting}[style=styleCMake]
# chapter06/02-odr-fail/two.cpp
	
cmake_minimum_required(VERSION 3.20.0)
project(ODR CXX)
set(CMAKE_CXX_STANDARD 20)
add_executable(odr one.cpp two.cpp)
\end{lstlisting}

我们创建了一个shared.h头文件,用于两个独立的翻译单元:

\begin{itemize}
\item 
one.cpp,只是把i打印到屏幕上

\item 
two.cpp,除了包含头文件外什么都不做
\end{itemize}

然后将两者链接到一个可执行文件中,会看到以下错误:

\begin{tcblisting}{commandshell={}}
[100%] Linking CXX executable odr
/usr/bin/ld: CMakeFiles/odr.dir/two.cpp.o:(.bss+0x0): multiple
definition of 'i'
; CMakeFiles/odr.dir/one.cpp.o:(.bss+0x0): first defined here
collect2: error: ld returned 1 exit status
\end{tcblisting}

这些东西不能定义两次。但有一个明显的例外——若类型、模板和extern内联函数完全相同(具有相同的标记序列),则可以在多个翻译单元中重复其定义。可以通过用类的定义替换简单的定义int i;来证明这一点:

\begin{lstlisting}[style=styleCXX]
// chapter06/03-odr-success/shared.h

struct shared {
	static inline int i = 1;
};
\end{lstlisting}

然后,可以这样使用:

\begin{lstlisting}[style=styleCXX]
// chapter06/03-odr-success/one.cpp

#include <iostream>
#include "shared.h"
int main() {
	std::cout << shared::i << std::endl;
}
\end{lstlisting}

其他两个文件two.cpp和CMakeLists.txt与02odrfail示例中一样。这样的更改将链接成功:

\begin{tcblisting}{commandshell={}}
-- Build files have been written to: /root/examples/
chapter06/03-odr-success/b
[ 33%] Building CXX object CMakeFiles/odr.dir/one.cpp.o
[ 66%] Building CXX object CMakeFiles/odr.dir/two.cpp.o
[100%] Linking CXX executable odr
[100%] Built target odr
\end{tcblisting}

或者,可以将变量标记为翻译单元的局部变量(不会被导出到目标文件之外)。为此,可以使用static关键字,如下所示:

\begin{lstlisting}[style=styleCXX]
// chapter06/04-odr-success/shared.h

static int i;
\end{lstlisting}

所有其他文件将保持相同,就像在原始示例中一样,并且链接仍然会成功。当然,前面代码中的变量存储在每个翻译单元的单独内存中,对其中一个的更改不会影响另一个。

\subsubsubsection{6.4.1\hspace{0.2cm}动态链接的重复符号}

ODR规则对静态库的作用和对目标文件的作用完全相同,但是当用动态库构建代码时,事情就不同了。链接器允许在这里复制符号。下面的示例中,将创建两个动态库A和B,其中有一个duplicated()函数和两个A()和B()函数:

\begin{lstlisting}[style=styleCXX]
//chapter06/05-dynamic/a.cpp

#include <iostream>
void a() {
	std::cout << "A" << std::endl;
}
void duplicated() {
	std::cout << "duplicated A" << std::endl;
}
\end{lstlisting}

第二个实现文件几乎完全复制了第一个:

\begin{lstlisting}[style=styleCXX]
// chapter06/05-dynamic/b.cpp

#include <iostream>
void b() {
	std::cout << "B" << std::endl;
}
void duplicated() {
	std::cout << "duplicated B" << std::endl;
}
\end{lstlisting}

现在,使用每个函数来看看发生了什么(为了简单起见,将使用extern在本地声明它们):

\begin{lstlisting}[style=styleCXX]
// chapter06/05-dynamic/main.cpp

extern void a();
extern void b();
extern void duplicated();

int main() {
	a();
	b();
	duplicated();
}
\end{lstlisting}

前面的代码将运行来自每个库的函数,然后使用在两个动态库中使用相同签名定义的函数。你觉得会发生什么?这种情况下,连接顺序重要吗?来针对两种情况进行测试:

\begin{itemize}
\item 
main\_1与a库链接

\item 
main\_2与b库链接
\end{itemize}

下面是代码:

\begin{lstlisting}[style=styleCMake]
// chapter06/05-dynamic/CMakeLists.txt
	
cmake_minimum_required(VERSION 3.20.0)
project(Dynamic CXX)

add_library(a SHARED a.cpp)
add_library(b SHARED b.cpp)

add_executable(main_1 main.cpp)
target_link_libraries(main_1 a b)

add_executable(main_2 main.cpp)
target_link_libraries(main_2 b a)
\end{lstlisting}

构建并运行这两个可执行程序之后,将看到以下输出:

\begin{tcblisting}{commandshell={}}
root@ce492a7cd64b:/root/examples/chapter06/05-dynamic# b/main_1
A B
duplicated A
root@ce492a7cd64b:/root/examples/chapter06/05-dynamic# b/main_2
A
B
duplicated B
\end{tcblisting}

啊哈!因此,链接器确实关心链接库的顺序。若不小心的话,这可能会造成一些混乱。实际上,命名冲突并不像看上去那么罕见。

这种行为也有一些例外。若定义本地可见的符号,将优先于从动态链接库中可用的符号。向main.cpp中添加以下函数会将两个二进制文件的最后一行输出更改为“duplicated MAIN”,如下所示:

\begin{lstlisting}[style=styleCXX]
#include <iostream>
void duplicated() {
	std::cout << "duplicated MAIN" << std::endl;
}
\end{lstlisting}

从库导出名称时一定要非常小心,可能会遇到名称冲突。

\subsubsubsection{6.4.2\hspace{0.2cm}使用命名空间——不要依赖链接器}

命名空间的概念是为了避免这些奇怪的问题,并以可管理的方式处理ODR。建议将库代码包装在以库命名的命名空间中,这一点并不奇怪。这样,就可以避免所有符号重复的问题。

项目中可能会遇到这样的情况:一个动态库连接另一个动态库,然后连接另一个动态库,形成一个长长的依赖链。这并不罕见,特别是在更复杂的设置中。重要的是要记住,简单地将一个库链接到另一个库并不意味着类型的命名空间继承。这个链的每个链接中的符号仍然不受保护,保存在最初编译它们的命名空间中。

某些情况下,链接器的怪癖是有趣和有用的。接下来,让我们讨论一个不常见的问题——当正确定义的符号丢失而没有解释时,该怎么办?













